\documentclass{yann}
\usepackage{tcolorbox}

\newcommand{\Par}{\mathcal{P}}

\begin{document}
\title{Formule de Vandermonde}
\maketitle

On veut montrer le résultat suivant:
\begin{tcolorbox}
  Si $(p,q,n)∈ℕ^3$, on a
  \[ \binom{p+q}{n} = ∑_{k≥0} \binom{p}{k} \binom{q}{n-k}. \]
\end{tcolorbox}

\section{Preuve informelle}

On dispose de $p$~chaussettes noires et de $q$~chaussettes rouges.
De combien de façons peut-on choisir $n$~chaussettes?

D'abord, on a $N = \binom{p+q}{n}$ car il s'agit de choisir $n$~chaussettes parmi les $p+q$.

Par ailleurs, on peut diviser cette somme en plusieurs parties, selon le nombre~$k$ de chaussettes noires choisies.
Pour $k$~fixé, il y a $\binom{p}{k}$ façons de choisir les $k$~chaussettes noires
et $\binom{q}{n-k}$ façons de choisir les $(n-k)$~chaussettes rouges.
Ainsi, pour $k$~fixé, il y a exactement $\binom{p}{k} \binom{q}{n-k}$ façons de choisir les chaussettes.
Comme ces possibilités sont disjointes, on en déduit $N = ∑_k \binom{p}{k} \binom{q}{n-k}$, d'où le résultat.

\section{Preuve formelle}

\begin{enumerate}
\item
  \textbf{Rappel:} si $A$ est un ensemble fini de cardinal $n$ et $k∈ℤ$,
  on note $\Par_k(A)$ l'ensemble des parties de $A$ de cardinal $k$.
  On sait que $\Card \Par_k(A) = \binom{n}{k}$.

\item
  Soit $X$ et $Y$ deux ensembles \emph{disjoints} et $Z = X∪Y$.
  On définit deux applications
  \[ \Fonction{f}{\Par(Z)}{\Par(X)×\Par(Y)}{A}{(A∩X, A∩Y)} \]
  \[ \Fonction{g}{\Par(X)×\Par(Y)}{\Par(Z)}{(B,C)}{B∪C} \]

  Si $A∈\Par(Z)$, on a $g(f(A)) = (A∩X)∪(A∩Y) = A∩(X∪Y) = A∩Z = A$.
  Ceci montre que $g◦f = \Id_{\Par(Z)}$.

  Si $B∈\Par(X)$ et $C∈\Par(Y)$, on a $(B∪C)∩X = (B∩X)∪(C∩X) = B∪∅ = B$
  et $(B∪C)∩Y = (B∩Y)∪(C∩Y) = ∅∪C = C$ donc $f(g((B,C))) = (B,C)$.
  Ceci montre que $f◦g = \Id_{\Par(X)×\Par(Y)}$.

  Ainsi, $f$ et $g$ sont des bijections réciproques l'une de l'autre.

\item
  Supposons désormais que $X$ (respectivement $Y$) est un ensemble fini de cardinal $p$ (resp. $q$).
  Notons $\mathcal{A} = \Par_n(Z)$
  et $\mathcal{B} = f(\mathcal{A}) = \Ensemble{f(A)}{A∈\mathcal{A}}$.

  Comme $\Card(Z) = p+q$, on a directement que $\Card \mathcal{A} = \binom{p+q}{n}$.
  De plus, comme $f$ est injective, $\mathcal{A}$ et $\mathcal{B}$ ont le même cardinal.

\item
  On a $\mathcal{B} = \Ensemble{(B,C)∈\Par(X)×\Par(Y)}{\Card(B∪Y) = n}$.
  Posons, pour $k∈ℕ$,
  $\mathcal{B}_k = \Ensemble{(B,C)∈\Par(X)×\Par(Y)}
  {\Card B = k \text{ et } \Card C = n-k}$.

  Les $(\mathcal{B}_k)_{k∈ℕ}$ sont deux à deux disjoints, car si $(B,C) ∈ \mathcal{B}_k ∩ \mathcal{B}_l$,
  on a $k = \Card B = l$.  De plus,
  \[ ⋃_{k≥0} \mathcal{B}_k = \mathcal{B}. \]
  On le prouve par double inclusion.

  $\cro⊂$
  Soit $(B,C)∈⋃_{k≥0} \mathcal{B}_k$.
  Il existe un $k_0∈ℕ$ tel que $x∈\mathcal{B}_{k_0}$.
  Par définition, $B⊂X$, $C⊂Y$, $\Card(B)=k_0$ et $\Card(C)=n-k_0$.
  De plus $B∩C ⊂ X∩Y = ∅$, donc $B$ et $C$ sont disjoints,
  donc $\Card(B∪C) = \Card(B)+\Card(C) = n$.
  Ainsi $(B,C)∈\mathcal{B}$.

  $\cro⊃$
  Soit $(B,C)∈\mathcal{B}$.
  Par définition, $B⊂X$, $C⊂Y$ et $\Card(B∪C) = n$.
  Notons $k_0 = \Card(B)$.
  De plus $B∩C ⊂ X∩Y = ∅$, donc $B$ et $C$ sont disjoints,
  donc $\Card(C) = \Card(B∪C) - \Card(B) = n - k_0$.
  Ainsi $(B,C)∈\mathcal{B}_{k_0}$,
  donc $(B,C)∈⋃_{k≥0} \mathcal{B}_k$.

\item
  Le point précédent permet donc d'affirmer
  \[ ∑_{k≥0} \Card(\mathcal{B}_k) = \Card(B). \]
  Notons qu'il s'agit d'une somme finie
  car $\Card(\mathcal{B}_k) = 0$ si $k>n$.

\item
  Enfin, $\mathcal{B}_k = \Par_k(X)×\Par_{n-k}(Y)$
  donc $\Card(\mathcal{B}_k) = \binom{p}{k} \binom{q}{n-k}$.
  De plus, $\Card(\mathcal{B}) = \Card(\mathcal{A}) = \binom{p+q}{n}$.
\end{enumerate}
CQFD.

\section{Généralisation}

Soit $n, k, p_1, \dots, p_n$ des entiers naturels et $p = p_1 + \dots + p_n$.
On a
\[ \binom pk  = ∑_{a∈S} \,\, ∏_{i=1}^n \binom{p_i}{a_i} \]
où $S = \Ensemble{ a = \nUplet a1n ∈ℕ^n} { a_1 + \dots + a_n = k }$.

\end{document}
