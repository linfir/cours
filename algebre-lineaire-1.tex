\documentclass{yann}
%\yann{layout=onecolumn}
\newcommand{\myop}{\mathop□}

\begin{document}
\title{Algèbre linéaire 1}
\maketitle

\section{Espace vectoriel}

\subsection{Loi de composition interne}

\Para{Définitions}

Soit $E$ un ensemble. Une \emph{loi de composition interne} $\myop$ sur $E$ est une application
$$ \Fonction{\myop}{E×E}{E}{(x,y)}{x \myop y} $$
\begin{enumerate}
\item
  On dit que $\myop$ est \emph{associative} si $∀(x,y,z)∈E^3$, $x \myop (y \myop z) = (x \myop y) \myop z$.
  On ne considèrera que des loi associatives.
\item
  On dit que $\myop$ est \emph{commutative} si $∀(x,y)∈E^2$, $x \myop y = y \myop x$;
\item
  On dit que $\myop$ \emph{admet un neutre} si $∃e∈E$ tel que $∀x∈E$, $x \myop e = e \myop x = x$.
  Il existe au plus un élément $e$ vérifiant cette propriété, et on l'appelle \emph{le} neutre de la loi $\myop$.
\item
  Si $\myop$ est une loi associative qui admet un neutre $e$, et si $x∈E$, on appelle \emph{symétrique} (ou \emph{inverse}) de $x$ pour la loi $\myop$ tout élément $x'∈E$ tel que $x \myop x' = x' \myop x = e$.
  Si $\myop$ est également associative, il existe au plus un élément $x'$ vérifiant cette propriété, et on l'appelle \emph{le} symétrique de $x$ pour la loi $\myop$.
\end{enumerate}

\Para{Proposition}

Soit $E$ un ensemble muni d'une loi de composition interne associative $\myop$.
Soit $(x,y)∈E^2$.
Si $x$ et $y$ sont symétrisables pour $\myop$, de symétriques $x'$ et $y'$ respectivement,
alors $x \myop y$ est également symétrisable et son symétrique est $y' \myop x'$.

\subsection{Groupe}

\Para{Définitions}

Un \emph{groupe} est un couple $(G,\myop)$ où $G$ est un ensemble et $\myop$ une loi de composition interne sur $G$ associative, admettant un neutre et pour laquelle tout élément de $G$ admet un symétrique pour la loi $\myop$.
Un groupe est dit \emph{abélien} ou \emph{commutatif} si la loi $\myop$ est de plus commutative.

\subsection{Corps}

\Para{Définition}

Un \emph{corps} est un triplet $(𝕂,+,×)$ où
\begin{enumerate}
\item
  $𝕂$ est un ensemble;
\item
  $+$ et $×$ sont des lois de composition internes sur $𝕂$;
\item
  $(𝕂,+)$ est un groupe abélien, £cad.
  $+$ est associative, commutative, admet un neutre et tout élément de $𝕂$ admet un symétrique pour $+$.
  On note $0_𝕂$ le neutre de $+$;
\item
  $×$ est associative, commutative, admet un neutre et tout élément de $𝕂∖\acco{0_𝕂}$ admet un symétrique pour $×$.
  On note $1_𝕂$ le neutre de $×$;
\item
  La loi $×$ est distributive sur la loi $+$, £cad.
  $∀(x,y,z)∈𝕂^3$, $x×(y+z) = (x×y)+(x×z)$;
\item
  $0_𝕂 ≠ 1_𝕂$.
\end{enumerate}

\Para{Remarque}

Lorsque le contexte est clair, on écrit souvent $𝕂$ au lieu de $(𝕂,+,×)$.

\Para{Exemples}
\begin{itemize}
\item le corps des réels $ℝ$, des complexes $ℂ$, des rationnels $ℚ$;
\item si $𝕂$ est un corps, le corps des fractions rationnelles à coefficients dans $𝕂$, noté $𝕂(X)$;
\item $ℚ[i] = \Ensemble{a+ib}{(a,b)∈ℚ^2}$;
\item le corps des entiers modulo un nombre premier $p$, noté $ℤ/pℤ$.
\end{itemize}
Tous ces corps sont munis des \enquote{lois usuelles}.

\subsection{Espace vectoriel}

\Para{Définition}

Soit $𝕂$ un corps.
Un $𝕂$-espace vectoriel est un triplet $(E,+,\bullet)$ où
$+$ est une loi de composition interne sur $E$ et
$\bullet$ est une loi de composition externe sur $E$, £cad. une application
\[ \Fonction{\bullet}{𝕂×E}{E}{(λ,x)}{λ\bullet x} \]
vérifiant les axiomes suivants:
\begin{enumerate}
\item $(E, +)$ est un groupe abélien;
\item la loi $\bullet$ est compatible avec la structure de groupe $(E, +)$, £cad.
  \begin{enumerate}
  \item $∀(λ,μ)∈𝕂^2$, $∀x∈E$, $(λ+μ) \bullet x = (λ\bullet x) + (μ\bullet x)$;
  \item $∀λ∈𝕂$, $∀(x,y)∈E^2$, $λ\bullet (x+y) = (λ\bullet x) + (λ\bullet y)$;
  \item $∀x∈E$, $1_𝕂\bullet x = x$;
  \item $∀(λ,μ)∈𝕂^2$, $∀x∈E$, $λ\bullet (μ\bullet x) = (λμ) \bullet x$.
  \end{enumerate}
\end{enumerate}

\Para{Exemples}

\begin{itemize}
\item
  $(𝕂^n, \tilde +, \bullet)$
  où $\nUplet x1n \tilde + \nUplet y1n = \pa{x_1+y_1, x_2+y_2, \dots, x_n+y_n}$
  et $λ\bullet \nUplet x1n = \pa{λx_1, λx_2, \dots, λx_n}$.
\item
  les matrices $\M{M}{n,p}{𝕂}$ muni des lois usuelles (les expliciter!),
\item
  si $X$ est un ensemble et $E$ un $𝕂$-espace vectoriel, l'ensemble des fonctions $\mathcal{F}(X,E)$ muni des lois usuelles.
\end{itemize}

\subsection{Espace vectoriel produit}

\Para{Définition}

Soit $(E_1, +_1, \bullet_1)$, $(E_2, +_2, \bullet_2)$, ..., $(E_p, +_p, \bullet_p)$ des $𝕂$-espaces vectoriels.
On pose \[ E = ∏_{k=1}^p E_k = \Ensemble{\nUplet x1p}{ x_1∈E_1, \dots, x_p∈E_p }. \]
$E$ est le produit cartésien des ensembles $E_1, \dots, E_n$.
On définit la loi de composition interne $+$ sur $E$ par:
$∀\nUplet x1p∈E$, $∀\nUplet y1p∈E$,
\[ \nUplet x1p + \nUplet y1p = \nUplet z1p \]
où $z_k = x_k +_k y_k$ pour $k∈\Dcro{1,p}$.
De même, on définit la loi de composition externe $\bullet$ sur $E$ par:
$∀λ∈𝕂\+∀\nUplet x1p∈E$
\[ λ\bullet \nUplet x1p = \nUplet y1p \]
où $y_k = λ\bullet_k x_k$ pour $k∈\Dcro{1,p}$.

Alors $(E, +, \bullet)$ est un espace vectoriel, appelé espace vectoriel produit des espaces vectoriels $(E_1, +_1, \bullet_1)$, ..., $(E_p, +_p, \bullet_p)$ et noté \[ E = ∏_{k=1}^p E_k. \]

\Para{Remarque}

Pour $p∈\Ns$ et $E$ un $𝕂$-espace vectoriel, on définit $E^p$ par \[ E^p = ∏_{k=1}^p E. \]
Notez le cas particulier $E = 𝕂$.

\subsection{Famille finie}

\Para{Définition}

Soit $E$ un $𝕂$-espace vectoriel.
Une \emph{famille finie} de vecteurs de $E$ est un $p$-uplet $\mathcal{F} = \nUplet x1p$ formée d'éléments de $E$, où $p∈ℕ$.
Dans le cas $p=0$, on notera $∅$ l'unique famille de $0$~élément.

\Para{Définitions}
Avec les mêmes notations, une \emph{combinaison linéaire} de la famille $\mathcal{F}$ est un vecteur $x ∈E$ de la forme $x = ∑_{i=1}^p α_i x_i$ où $\nUplet α1p ∈ 𝕂^p$.
On note $\Vect \mathcal{F}$ l'ensemble des combinaison linéaires de la famille $\nUplet x1p$.
Par convention, $\Vect ∅= \acco{0_E}$.

\Para{Définitions}

Avec les mêmes notations, on dit que la famille $\mathcal{F}$ est \emph{libre} \ssi{} la seule combinaison linéaire de $\mathcal{F}$ nulle est triviale, \cad{} \ssi
\begin{multline*}
  ∀\nUplet λ1p ∈𝕂^p, \\
  \Pa{ ∑_{i=1}^p λ_i x_i = 0_E \implies ∀i ∈\ccro{1,p} \+ λ_i = 0_𝕂 }.
\end{multline*}

On dit que la famille $\mathcal{F}$ est \emph{génératrice} £ssi. tout vecteur de $E$ est combinaison linéaire de $\mathcal{F}$, \cad{} \ssi{} $\Vect \mathcal{F} = E$, \cad{} \ssi
\[ ∀x ∈𝔼\+ ∃\nUplet α1p ∈𝕂^p \+ x = ∑_{i=1}^p α_i x_i. \]

On dit que la famille $\mathcal{F}$ est une \emph{base} de $E$ £ssi. elle est libre et génératrice.
De façon équivalente, la famille $\nUplet x1n$ est une base de $E$ £ssi.
\[ ∀x∈E \+ ∃!\nUpletα1p \+ x = ∑_{i=1}^p α_i x_i. \]

\Para{Proposition}
Avec les mêmes notations, les conditions suivantes sont équivalentes:
\begin{enumerate}
\item
  $\mathcal{F}$ est une base de $E$;
\item
  $\mathcal{F}$ est libre et maximale:
  $\mathcal{F}$ est libre et pour tout $x ∈E$, la famille $\mathcal{F}$ augmentée du vecteur $x$ n'est pas libre;
\item
  $\mathcal{F}$ est génératrice et minimale:
  $\mathcal{F}$ est génératrice, mais si l'on enlève un vecteur quelconque de $\mathcal{F}$, la famille résultante n'est pas génératrice.
\end{enumerate}

\subsection{Sous-espace vectoriel}

\Para{Définition}

Soit $(E,+,⋅)$ un $𝕂$-espace vectoriel et $F$ une partie de $E$.
On dit que $F$ est un \emph{£sev.} de $E$ £ssi.
\begin{enumerate}
\item
  $F$ est non vide;
\item
  $F$ est stable par $+$, £cad. $∀(x,y) ∈F^2$, $x+y∈F$;
\item
  $F$ est stable par $⋅$, £cad. $∀λ∈𝕂$, $∀x ∈F$, $λx∈F$.
\end{enumerate}

\Para{Critère}

Soit $E$ un $𝕂$-espace vectoriel et $F⊂E$.
$F$ est un £sev. de $E$ £ssi.
\begin{enumerate}
\item
  $0_E∈F$;
\item
  $∀(λ,μ)∈𝕂^2$, $∀(x,y) ∈F^2$, $λx + μy ∈ F$.
\end{enumerate}

\Para{Proposition}
Soit $E$ un $𝕂$-espace vectoriel, $F_1, \dots, F_p$ des £sevs..
Alors l'intersection $⋂_{i=1}^p F_i$ est également un £sev..

\Para{Attention}

En revanche, l'union $⋃_{i=1}^p F_i$ n'est presque jamais un £sev..

\Para{Proposition-Définition}

Soit $E$ un $𝕂$-£ev. et $\nUplet x1p$ une famille de vecteurs de $E$.
L'ensemble $\Vect\nUplet x1p$ est un £sev. de $E$, appelé \emph{espace engendré} par la famille $\nUplet x1p$.
Il s'agit du plus petit (pour l'inclusion) £sev. de $E$ contenant $\Uplet{x_1}{x_p}$.

\subsection{Somme de sous-espaces vectoriels}

\Para{Proposition-Définition}

Soit $E$ un $𝕂$-espace vectoriel et $\Uplet{F_1}{F_p}$ des sous-espaces vectoriels de $E$.
On appelle \emph{somme} de $\Uplet{F_1}{F_p}$ l'ensemble $S$ des vecteurs de la forme $f_1 + \dots + f_p$
où $f_1∈F_1$, $f_2∈F_2$, ..., $F_p∈F_p$;
autrement dit,
\[ S = \BiggEnsemble{x ∈E}{ ∃\nUplet f1p ∈ ∏_{i=1}^p F_i \+ x = ∑_{i=1}^p f_i }. \]
On le note $S = ∑_{k=1}^p F_k$.
Il s'agit d'un £sev. de $E$;
plus précisément $S$ est le plus petit (au sens de l'inclusion) £sev. de $E$ contenant $\Uplet{F_1}{F_p}$.

\Para{Définition}
Avec les mêmes notations, on dit que la somme $∑_{i=1}^p F_i$ est \emph{directe} £ssi. tout vecteur de la somme se décompose \emph{de façon unique} sous la forme $f_1 + \dots + f_p$ où $f_1∈F_1$, $f_2∈F_2$, ..., $F_p∈F_p$.
On note alors la somme $⨁_{k=1}^p F_k = ∑_{k=1}^p F_k$.

\Para{Cas de deux sous-espaces vectoriels}
Soit $E$ un $𝕂$-espace vectoriel, $F$ et $G$ deux sous-espaces vectoriels.
Alors la somme $F+G$ est directe £ssi. $F∩G = \acco{0_E}$.

\Para{Attention}

Ce critère ne se généralise pas (simplement) pour $n > 2$.

Contre-exemple: $E = ℝ^2$, $e_1 = (1,0)$, $e_2 = (0,1)$ et $e_3 = (1,1)$.
Vérifiez que $ℝe_1 ∩ ℝe_2 = ℝe_1 ∩ ℝe_3 = ℝe_2 ∩ ℝe_3 = \Acco{(0,0)}$ mais que la somme $ℝe_1 + ℝe_2 + ℝe_3$ n'est pas directe.

\Para{Critère}

Soit $E$ un $𝕂$-espace vectoriel et $\Uplet{F_1}{F_p}$ des sous-espaces vectoriels de $E$. La somme $∑_{i=1}^p F_i$ est directe £ssi.
\begin{multline*}
  ∀x_1 ∈F_1 \+ ∀x_2 ∈F_2 \+ \dots \+ ∀x_p ∈F_p, \\
  ∑_{k=1}^p x_k = 0 \implies x_1 = \dots = x_p = 0.
\end{multline*}

\Para{Définition}

Soit $E$ un $𝕂$-espace vectoriel et $\Uplet{F_1}{F_p}$ des sous-espaces vectoriels de $E$.
Si la somme $∑_{k=1}^p F_k$ est directe et égale à $E$,
on dit que $\Uplet{F_1}{F_p}$ sont \emph{supplémentaires}, et on note
\[ E = ⨁_{k=1}^p F_k. \]

\subsection{Application linéaire}

\Para{Définition}

Soit $E$ et $F$ deux $𝕂$-espaces vectoriels.
Une application $\Fn{f}{E}{F}$ est dite \emph{linéaire} £ssi.
\begin{enumerate}
\item
  $∀(x,y)∈E^2$, $f(x+y) = f(x) + f(y)$;
\item
  $∀λ∈𝕂$, $∀x∈E$, $f(λx) = λf(x)$.
\end{enumerate}
Ces deux conditions sont équivalentes à
\begin{enumerate}[resume]
\item $∀(λ,μ)∈𝕂^2$, $∀(x,y)∈E^2$, $f(λx + μy) = λf(x) + μf(y)$.
\end{enumerate}

\Para{Définitions}

\begin{itemize}
\item
  Une application linéaire est appelée également \emph{morphisme} d'espaces vectoriels.
\item
  Une application linéaire dont l'espace de départ est le même que celui d'arrivée est un \emph{endomorphisme}.
\item
  Un \emph{isomorphisme} est une application linéaire bijective.
\item
  Un \emph{automorphisme} est un endomorphisme bijectif.
\end{itemize}

\Para{Proposition-Définition}

Soit $E$ et $F$ deux $𝕂$-espaces vectoriels, $E'$ un £sev. de $E$, $F'$ un £sev. de $F$ et $\Fn uEF$ une application linéaire.
\begin{itemize}
\item
  L'ensemble $\Ensemble{u(x)}{x ∈E'}$ est un £sev. de $F$, appelé \emph{image directe} de $E'$ par $u$ et noté $u(E')$.
\item
  L'ensemble $\Ensemble{x ∈E}{u(x) ∈F'}$ est un £sev. de $E$, appelé \emph{image réciproque} de $F'$ par $u$ et noté $u^{-1}(F')$. Notez que ceci est bien défini que $u$ soit bijectif ou non.
\end{itemize}

\Para{Définitions}

Soit $E$ et $F$ deux $𝕂$-espaces vectoriels, et $\Fn uEF$ une application linéaire.
\begin{itemize}
\item
  Le \emph{noyau} de $u$ est le £sev. $f^{-1}\pa{\acco{0_F}} = \Ensemble{x∈E}{f(x) = 0_F}$;
  on le note $\Ker u$.
\item
  L'\emph{image} de $u$ est le £sev. $f(E) = \Ensemble{f(x)}{x∈E}$;
  on le note $\Ima u$.
\end{itemize}

\Para{Proposition-Définition}

Soit $E$ et $F$ deux $𝕂$-espaces vectoriels.
L'ensemble des applications linéaires de $E$ dans $F$ est noté $\mathscr{L}(E,F)$;
il s'agit d'un £sev. de l'espace des fonctions de $E$ dans $F$ muni des lois usuelles.
L'£ev. des endomorphismes de $E$ se note $\LE = \mathscr{L}\pa{E,E}$.

\Para{Proposition}
Soit $E$ et $F$ deux $𝕂$-£evs. et $\Fn fEF$ une application linéaire.
\begin{itemize}
\item
  $f$ est injective £ssi. $\Ker f = \acco{0_E}$;
\item
  $f$ est surjective £ssi. $\Ima f = F$;
\item
  si $f$ est bijective, alors $f^{-1}$ est également linéaire.
\end{itemize}

\Para{Proposition}

Soit $E$ un $𝕂$-£evdf., $\nUplet e1n$ une base de $E$.
Soit $F$ un $𝕂$-£ev. et $\Fn uEF$ une application linéaire.
Alors $u$ est bijective £ssi. la famille $\bigPa{u(e_1),\dots,u(e_n)}$ est une base de $F$.

% -----------------------------------------------------------------------------
\section{Dimension}

\subsection{Espace vectoriel de dimension finie}

\Para{Définition}

Soit $E$ un $𝕂$-£ev..
On dit que $E$ est \emph{de dimension finie} s'il existe une famille finie génératrice de $E$,
et \emph{de dimension infinie} sinon.

\Para{Théorème de la base incomplète}

Soit $E$ un £evdf., et $\nUplet e1p$ une famille libre de $E$.
Alors il existe un entier $n≥p$ et des vecteurs $\Uplet{e_{p+1}}{e_n}$ tels que $\nUplet e1n$ soit une base de $E$.

\Para{Corollaire}

Tout £evdf. admet une base.

\Para{Théorème de la base extraite}

Soit $E$ un £ev. et $\nUplet e1q$ une famille génératrice.
Alors il existe $I ⊂\ccro{1,q}$ tel que la famille $(e_i)_{i∈I}$ soit une base de $E$.

\Para{Lemme de Steinitz}

Soit $E$ un £ev., $\nUplet e1n$ une base de $E$ et $\nUplet v1p$ une famille libre de $E$.
Alors $p ≤n$ et il existe une permutation $\pa{\Uplet{e'_1}{e'_n}}$ de $\nUplet e1n$ telle que
la famille $(v_1,\dots,v_p,e'_{p+1},\dots,e'_n)$ soit une base de $E$.

\Para{Corollaire}

Soit $E$ un £evdf. et $\nUplet e1p$ et $\nUplet f1q$ deux bases de $E$.
Alors $p=q$.

\Para{Définition}

La \emph{dimension} d'un £evdf. est égale au cardinal d'une base quelconque de $E$.

\Para{Corollaire}

Si $E$ est un £evdf. de dimension $n$, alors toute famille de $n+1$ vecteurs est liée.

\subsection{Applications}

\Para{Proposition}

Soit $E$ un £evdf. et $F$ un £sev. de $E$.
Alors $F$ est également de dimension finie et $\dim F ≤ \dim E$, avec égalité £ssi. $F = E$.

\Para{Proposition}
Dans un £evdf., tout £sev. admet un supplémentaire.

Autrement dit, si $E$ est un £evdf. et $F$ un £sev. de $E$,
alors il existe un £sev. $G$ de $E$ tel que $E = F ⊕G$.

\Para{Proposition}

Soit $\Uplet{E_1}{E_p}$ des $𝕂$-espaces vectoriels.
L'espace vectoriel produit $∏_{k=1}^p E_k$ est de dimension finie £ssi. $∀k∈\Dcro{1,p}$, $E_k$ est de dimension finie.
De plus, dans ce cas, \[ \dim \Pa{ ∏_{k=1}^p E_k } = ∑_{k=1}^p \dim E_k. \]

\Para{Corollaire}

Pour $p∈\Ns$ et $E$ un $𝕂$-espace vectoriel.
L'£ev. $E^p$ est de dimension finie £ssi. $E$ l'est;
dans ce cas, on a $\dim(E^p) = p\dim E$.

\Para{Proposition}

Soit $E$ et $F$ deux $𝕂$-£evdf..
Alors $\mathscr{L}(E,F)$ est également de dimension finie
et $\dim \mathscr{L}(E,F) = \dim(E) ×\dim(F)$.

\subsection{Théorème du rang}

\Para{Théorème}

Soit $E$ et $F$ deux $𝕂$-£evs. et $\Fn uEF$ une application linéaire.
Si $S$ est un supplémentaire de $\Ker u$, alors $u$ induit un isomorphisme de $S$ sur $\Ima u$,
£cad. que l'application linéaire
\[ \Fonction{v}{S}{\Ima u}{x}{u(x)} \]
est bijective.

\Para{Corollaire}[théorème du rang]

Soit $E$ et $F$ deux $𝕂$-£evs. et $\Fn uEF$ une application linéaire.
On suppose que $E$ est de dimension finie.
Alors $\Ima u$ est de dimension finie et
\[ \dim E = \Rang u + \dim \Ker u. \]

\Para{Corollaire important}

Soit $E$ et $F$ deux $𝕂$-£evdfs. et $\Fn uEF$ une application linéaire.
On suppose que $\dim E = \dim F$.
Les conditions suivantes sont équivalentes:
\begin{enumerate}
\item
  $u$ est injective;
\item
  $u$ est surjective;
\item
  $u$ est bijective.
\end{enumerate}
Notez que ceci s'applique en particulier au cas des endomorphismes.

\subsection{Somme directe}

\Para{Proposition}

Soit $E$ un $𝕂$-espace vectoriel et $\Uplet{F_1}{F_p}$ des sous-espaces vectoriels de dimension finie de $E$.
Alors la somme $∑_{k=1}^p F_k$ est également de dimension finie, et
\[ \dim\Pa{ ∑_{k=1}^p F_k } ≤ ∑_{k=1}^p \dim(F_k). \]
De plus, il y a égalité si et seulement si la somme $∑_{k=1}^p F_k$ est directe.

\Para{Proposition}

Soit $E$ un $𝕂$-£evdf. et $F_1, \dots, F_p$ des £sevs. de $E$.
On suppose que \[ ∑_{k=1}^p \dim F_k = \dim E. \]
Les conditions suivantes sont alors équivalentes:
\begin{enumerate}
\item
  les £sevs. $F_1, \dots, F_p$ sont supplémentaires;
\item
  les £sevs. $F_1, \dots, F_p$ sont en somme directe;
\item
  $∑_{k=1}^p F_k = E$;
\item
  tout vecteur $x∈E$ se décompose sous la forme
  $x = ∑_{k=1}^p f_k$
  où $f_1∈F_1$, $\dots$, $f_p∈F_p$,
\item pour tous $f_1∈F_1$, $f_2∈F_2$, $\dots$, $f_p∈F_p$,
  si $∑_{k=1}^p f_k = 0$, alors $f_1 = \dots = f_p = 0$.
\end{enumerate}


% -----------------------------------------------------------------------------
\section{Forme linéaire et hyperplan}

\Para{Définitions}

Soit $E$ un $𝕂$-espace vectoriel.
\begin{itemize}
\item
  Une \emph{forme linéaire} sur $E$ est une application linéaire $\FnφE𝕂$.
\item
  L'ensemble des formes linéaires sur $E$ s'appelle le \emph{dual} de $E$
  et se note $E^* = \mathscr{L}(E,𝕂)$.
\item
  Une \emph{droite} de $E$ est un £sev. de dimension $1$.
  Autrement dit, $D$ est une droite (vectorielle) £ssi. $∃x∈E∖\Acco{0_E}$, $D =𝕂x$.
\item
  Un \emph{hyperplan} de $E$ est un £sev. $H$
  qui admet une droite comme supplémentaire.
  Autrement dit, si $H$ est un £sev. de $E$,
  $H$ est un hyperplan £ssi. $∃x∈E∖\Acco{0_E}$, $E = H⊕𝕂x$.
\end{itemize}

\Para{Proposition}

Si $E$ est un $𝕂$-£evdf., alors $\dim E^* = \dim E$.

\Para{Théorème}

Soit $E$ un $𝕂$-£ev. de dimension $n∈\Ns$ et $H$ un £sev. de $E$.
Les conditions suivantes sont équivalentes:
\begin{enumerate}
\item $H$ est un hyperplan, £cad. qu'il existe une droite $D$ telle que $E = H ⊕ D$;
\item $\dim H = n - 1$;
\item il existe une forme linéaire non nulle $φ$ telle que $H = \Ker φ$.
\end{enumerate}

\Para{Remarque}

En dimension quelconque, on a encore l'équivalence entre~1 et~3.

\Para{Exemple}

On se place dans $E = ℝ^3$.
\begin{itemize}
\item
  Soit $P = \Ensemble{(x,y,z)∈ℝ^3}{x+y+z=0}$.
  $P$ est un £sev. de $E$ de dimension $2$, donc un hyperplan.
\item
  Soit $x = (1,0,0)$ et $D = 𝕂x$. On a $E = P ⊕D$.
\item Soit \[ \Fonction{φ}{E}{ℝ}{(x,y,z)}{x+y+z.} \]
  $φ$ est une forme linéaire.
\item On a $P = \Ker φ$.
\end{itemize}

\Para{Proposition}

Soit $E$ un $𝕂$-espace vectoriel, $φ$ et $ψ$ deux formes linéaires non nulles sur $E$.
Alors \[ \Ker φ = \Ker ψ \iff ∃λ∈\Ks \+ φ = λψ. \]
Autrement dit, deux formes linéaires non nulles définissent le même hyperplan £ssi. elles sont proportionnelles.

\Para{Remarque}

Soit $E$ un $𝕂$-£evdf..  On a une correspondance bijective entre:
\begin{itemize}
\item les hyperplans sur $E$, et
\item les formes linéaires non nulles sur $E$, à multiplication par un scalaire non nul près.
\end{itemize}

% -----------------------------------------------------------------------------
\section{Divers}

\subsection{Sous-espace vectoriel stable}

\Para{Définitions}

Soit $F$ un £sev. de $E$ et $u∈\LE$.
\begin{itemize}
\item
  On dit que $F$ est \emph{$u$-stable}, ou \emph{stable par $u$}, si et seulement si $u(F)⊂F$.
\item
  Si $F$ est $u$-stable, il existe un unique endomorphisme $v∈\mathscr{L}(F)$ tel que $∀x∈F$, $u(x) = v(x)$.
  $v$ s'appelle l'\emph{endomorphisme induit} par $u$ sur $F$.
\end{itemize}

\Para{Lemme}

Si $f$ et $g$ sont des endomorphismes de $E$ \emph{qui commutent},
alors $\Ker g$ et $\Ima g$ sont stables par $f$.

\subsection{Polynôme d'endomorphisme}

\Para{Définitions}

Soit $E$ un $𝕂$-espace vectoriel et $u∈\LE$.
On définit par récurrence $u^n$ pour $n∈ℕ$ par $u^0 = \Id_E$ et $∀n∈ℕ$, $u^{n+1} = u◦u^n$.
Pour $P = ∑_{k=0}^d a_k X^k$, on pose $P(u) = ∑_{k=0}^d a_k u^k$.

\Para{Théorème}

Soit $E$ un $𝕂$-espace vectoriel et $u∈\LE$.
L'application \[ \Fonction{ϕ_u}{𝕂[X]}{\LE}{P}{P(u)} \]
est un morphisme d'algèbre.
Autrement dit, si $P$ et $Q$ sont deux polynômes et $λ$ un scalaire, on a
\begin{enumerate}
\item $(λP)(u) = λP(u)$
\item $(P+Q)(u) = P(u) + Q(u)$
\item $(PQ)(u) = P(u)◦Q(u)$
\end{enumerate}


\subsection{Base adaptée}

\Para{Définition}

Soit $E$ un $𝕂$-£evdf. $n$ et $F$ un £sev. de $E$.
La base $\B = \nUplet e1n$ est dite \emph{adaptée} à $F$
si et seulement s'il existe $p∈\Dcro{0,n}$
tel que $\nUplet e1p$ soit une base de $F$.

\Para{Définition}

Soit $E$ un $𝕂$-£evdf., $\Uplet{F_1}{F_p}$ des sous-espaces vectoriels supplémentaires de $E$ et $\B$ une base de $E$.
La base $\B$ est dite \emph{adaptée} à la décomposition $E =⨁_{k=1}^p F_k$ si et seulement si $\B$ peut s'écrire comme la concaténation de $\Uplet{\B_1}{\B_p}$ où $\B_k$ est une base de $F_k$ pour tout $k∈\Dcro{1,p}$.

\Para{Proposition}

Soit $E$ un $𝕂$-£evdf. et $\B$ une base de $E$.
On suppose que $\B$ s'écrit comme la concaténation de $\Uplet{\B_1}{\B_p}$.
Pour $k∈\Dcro{1,p}$, notons $F_k$ le £sev. engendré par $\B_k$.
Alors les sous-espaces vectoriels $\Uplet{F_1}{F_p}$ sont supplémentaires.

\Para{Proposition}

Soit $E$, $F$ deux $𝕂$-espaces vectoriels.
Soit $\Uplet{E_1}{E_p}$ des sous-espaces vectoriels de $E$ qui sont supplémentaires.
L'application \[ \Fonction{φ}{\mathscr{L}(E,F)}{∏_{k=1}^p \mathscr{L}(E_k,F)}{u}{(u_{\vert E_1}, ..., u_{\vert E_p})} \] est alors un isomorphisme.

Autrement dit, si l'on se donne des applications linéaires $\Fn{u_k}{E_k}{F}$ pour $1≤k≤p$, il existe une unique application $\Fn{u}{E}{F}$ telle que sa restriction à $E_k$ soit $u_k$ pour tout $1≤k≤p$.

Autrement dit, la donnée d'une application linéaire $E \to F$
est équivalente à la donnée d'applications linéaires $E_k \to F$ pour tout $1≤k≤p$.

% -----------------------------------------------------------------------------
\section{Exercices}

\Exercice

Soit $(𝕂,+,⋅)$ un corps.
Pour tout $x∈𝕂$, on note $-x$ le symétrique de $x$ pour la loi $+$.
Montrer que les relations suivantes vraies pour tous $(x,y,z)∈𝕂^3$:
\begin{enumerate}
\item $0 x = 0$;
\item $(-1) x = -x$;
\item $(-x)y = x(-y) = -(xy)$;
\item $(x-y)z = xz - yz$.
\end{enumerate}

\Exercice

Soit $(E,+,⋅)$ un $𝕂$-espace vectoriel.
Pour tout $x∈E$, on note $-x$ le symétrique de $x$ pour la loi $+$.
Pour $(λ,μ)∈𝕂^2$ et $(x,y)∈E^2$, montrer que:
\begin{enumerate}
\item $λx = 0_E$ £ssi. $λ= 0_𝕂$ ou $x = 0_E$;
\item $(-λ) x = λ(-x) = -(λx)$;
\item $(λ- μ)x = λx - μx$;
\item $λ(x-y) = λx - λy$.
\end{enumerate}

\Exercice

Les ensembles suivants sont-ils des $ℝ$-espaces vectoriels?
\begin{enumerate}
\item L'ensemble des suites réelles convergentes.
\item L'ensemble des suites réelles convergentes vers $0$.
\item L'ensemble des suites réelles convergentes vers $1$.
\item L'ensemble des suites réelles bornées.
\item L'ensemble des suites réelles croissantes.
\item L'ensemble des suites réelles monotones.
\item L'ensemble des suites réelles non convergentes.
\item L'ensemble des suites réelles périodiques à partir d'un certain rang.
\item L'ensemble des fonctions lipschitziennes de $ℝ$ dans $ℝ$.
\item L'ensemble des fonctions paires de $ℝ$ dans $ℝ$.
\item L'ensemble des fonctions de $ℝ$ dans $ℝ$ qui prennent la valeur $β$ en $α$.
\end{enumerate}

\Exercice

Soit $E$, $F$, $G$ trois $𝕂$-espaces vectoriels, $f∈\mathscr{L}(E,F)$ et $g∈\mathscr{L}(F,G)$.

Montrer que $g◦f = \tilde 0$ £ssi. $\Ima f⊂\Ker g$.

\Exercice

Soit $E$ un $𝕂$-espace vectoriel et $(f,g)∈\LE^2$.

Montrer que $f \bigl( \Ker(g◦f) \bigr) = \Ker g∩\Ima f$.

\Exercice

Soit $E = \mathcal{F}(ℝ,ℝ)$.
Pour $n∈\Ns$, on définit $\Fn{f_n}{ℝ}{ℝ}$ par $f_n(x) = \sin(nx)$.
\begin{enumerate}
\item Les trois fonctions $f_1$, $f_2$ et $f_3$ sont-elles linéairement indépendantes?
\item Généraliser.
\end{enumerate}

\Exercice

Soit $E$ le $ℝ$-espace vectoriel des suites réelles convergentes.
On note $F$ le £sev. des suites qui convergent vers $0$ et $G$ le £sev. des suites constantes.

Montrer que $F$ et $G$ sont supplémentaires dans $E$.

\Exercice

\begin{enumerate}
\item Déterminer une base des espaces vectoriels suivants:
  \begin{enumerate}
  \item $\Vect(e_1,e_2,e_3,e_4) ⊂ℝ^3$
    où $e_1 = (5,-3,-2)$, $e_2 = (17,-12,-5)$, $e_3 = (8,-12,4)$, $e_4 = (-91,46,45)$.
  \item $\Ensemble{P∈ℝ_3[X]}{P(1) = P'(1) = P''(1) = 0}$
  \item $\Ensemble{(x,y,z,t)∈ℝ^4}{ x+y+z+t = 2x+3y+4z+5t = 0 }$.
  \end{enumerate}
\item Déterminer un système d'équations cartésiennes des sous-espaces vectoriels suivants:
  \begin{enumerate}
  \item $\Vect(e_1,e_2,e_3,e_4) ⊂ℝ^3$
    où $e_1 = (5,4,9)$, $e_2 = (17,-12,5)$, $e_3 = (8,-12,-4)$, $e_4 = (-91,46,-45)$.
  \item $\Vect\bigl( (1,2,3,4), (5,6,7,8) \bigr)⊂ℝ^4$.
  \end{enumerate}
\end{enumerate}

\Exercice

On pose $Ω=ℝ^ℕ$ l'espace vectoriel des suites réelles et
\begin{align*}
  E   &= \Ensemble{u∈Ω}{∀n∈ℕ\+ u_{n+3} - u_{n+2} - u_{n+1} + u_n = 0}, \\
  E_1 &= \Ensemble{u∈Ω}{∀n∈ℕ\+ u_{n+1} + u_n = 0}, \\
  E_2 &= \Ensemble{u∈Ω}{∀n∈ℕ\+ u_{n+2} - 2u_{n+1} + u_n = 0}.
\end{align*}
\begin{enumerate}
\item Montrer que $E$ est un $ℝ$-espace vectoriel et que $E_1$ et $E_2$ sont deux sous-espaces vectoriels de $E$.
\item Montrer que $E = E_1⊕E_2$. \emph{Indication:} pour montrer que tout $u∈E$ se décompose en $u = v + w$ où $v∈E_1$ et $w∈E_2$, vérifier que l'on peut prendre $v_n = \frac14 \pa{ u_{n+2} - 2u_{n+1} + u_n }$.
\item Donner une base de $E_1$ et de $E_2$ et en déduire une base de $E$.
\end{enumerate}

\Exercice

Soit $E$ un $𝕂$-espace vectoriel de dimension $n∈\Ns$ et $f∈\LE$ un endomorphisme nilpotent.
On rappelle qu'un endomorphisme $f$ est dit \emph{nilpotent}
s'il existe $m∈\Ns$ tel que $f^m = \tilde0$.

On note $q$ \emph{l'indice de nilpotence} de $f$, £cad.
\[ q = \min \Ensemble{k∈\Ns}{f^k = \tilde 0}. \]
\begin{enumerate}
\item Montrer que $\Ker(f^{q-1})≠E$.
\item Soit $x∈E∖\Ker(f^{q-1})$.
  Montrer que la famille
  \[ \bigl( x,f(x),\dots,f^{q-1}(x) \bigr) \] est libre.
\item En déduire $q≤n$.
\end{enumerate}

\Exercice

Soit $E$ un $𝕂$-£evdf. et $\nUplet e1n$ une base de $E$.
On pose:
\[ ∀i∈\Dcro{1,n}\+ f_i = ∑_{\substack{j=1 \\ j≠i}}^n e_j. \]
Que peut-on dire de la famille $\nUplet f1n$?

\Exercice

Soit $E$ un $𝕂$-espace vectoriel et $u∈\LE$.
On suppose que $u^2 - 2 u - 15 \Id_E = 0$.
\begin{enumerate}
\item Montrer que $E = \Ker(u+3\Id_E)⊕\Ker(u-5\Id_E)$.
\item \emph{(pour les 5/2)}
  Reprendre l'exercice en supposant $E$ de dimension finie et en utilisant les résultats du cours sur la réduction.
\end{enumerate}

\Exercice

Soit $E$ un $𝕂$-espace vectoriel et $F$, $G$, $H$ des sous-espaces vectoriels de $E$.
\begin{enumerate}
\item Comparer (au sens de l'inclusion) $F+(G∩H)$ et $(F+G)∩(F+H)$.
\item Comparer de même $F∩(G+H)$ et $(F∩G)+(F∩H)$.
\item Montrer que, si $F⊂G$, on a $F+(G∩H) = (F+G)∩(F+H)$.
  Contre-exemple si $F \not⊂G$?
\item Montrer que, si $F⊂G$, $F+H=G+H$ et $F∩H=G∩H$, alors $F=G$.
\end{enumerate}

\Exercice

Soit $E$ le $ℝ$-espace vectoriel des fonctions de classe $\CC∞$ et $2π$-périodiques de $ℝ$ dans $ℝ$.
Soit $\FonctionφE E f {f''}$
\begin{enumerate}
\item Montrer que $\Kerφ= \Ensemble{f∈E}{f \text{ est constante}}$.
\item Montrer que $\Imaφ= \Ensemble{f∈E}{∫_0^{2π} f = 0 }$.
\item En déduire que $E = \Kerφ⊕\Imaφ$.
\end{enumerate}

\Exercice

Soit $E$ un $𝕂$-£evdf. et $u$ un endomorphisme de $E$.
On note $u^2$ l'endomorphisme $u◦u$.
Montrer que les conditions suivantes sont équivalentes:
\begin{enumerate}
\item $\Ima u = \Ima u^2$
\item $\Rang u = \Rang u^2$
\item $\dim \Ker u = \dim \Ker u^2$
\item $\Ker u = \Ker u^2$
\item $\Ima u∩\Ker u = \Acco{0}$
\item $E = \Ima u⊕\Ker u$
\item $E = \Ima u + \Ker u$
\end{enumerate}

\Exercice

Soit $E$ un $𝕂$-espace vectoriel et $f∈\LE$ tel que $∀x∈E$, la famille $(x, f(x))$ est liée.

Montrer que $f$ est une homothétie.

\Exercice

Soit $E$ un $𝕂$-espace vectoriel et $A⊂\LE$.
On appelle \emph{centre} de $A$ l'ensemble
\[ \mathcal{Z}(A) = \Ensemble{f∈\LE} {∀g∈A \+ f◦g = g◦f}. \]
Déterminer $\mathcal{Z}\bigl(\LE\bigr)$.

\Exercice[projecteur et projection]

Soit $E$ un $𝕂$-espace vectoriel.
\begin{itemize}
\item On appelle \emph{projecteur} de $E$ tout endomorphisme $u∈E$ tel que $u◦u = u$.
\item Soit $F$ et $G$ sont deux sous-espaces vectoriels supplémentaires de $E$.
  On appelle \emph{projection} sur $F$ parallèlement à $G$ l'application $\Fn vEE$ qui à tout vecteur $x∈E$ associe le vecteur $y$ tel que $x = y + z$ où $y∈F$ et $z∈G$.
\end{itemize}

On va montrer l'équivalence entre ces deux notions.
\begin{enumerate}
\item Montrer que la projection $v$ sur $F$ parallèlement à $G$ est linéaire, puis que c'est un projecteur.
\item Soit $u$ un projecteur.
  Montrer que $\Ima(u)$ et $\Ker(u)$ sont supplémentaires,
  puis que $u$ est la projection sur $\Ima(u)$ parallèlement à $\Ker(u)$.
\end{enumerate}

\Exercice[à connaître]

Soit $E$ un $𝕂$-espace vectoriel et $p$ un projecteur de $E$.
\begin{enumerate}
\item Montrer que $\Ima p = \Ensemble{x∈E}{p(x) = x} = \Ker(p - \Id_E)$.
\item Montrer que $E = \Ker p⊕\Ima p$.
\end{enumerate}

\Exercice

Soit $E$ un $𝕂$-£evdf. et $p$ un projecteur de $E$.
Montrer que $\Tr p = \Rang p$.

\Exercice

Soit $E$ un $𝕂$-espace vectoriel, $p$ un projecteur de $E$ et $u∈\LE$.
Montrer que $p$ et $u$ commutent £ssi. $\Ima p$ et $\Ker p$ sont stables par $u$.

\Exercice

Soit $E$ un $𝕂$-espace vectoriel, $p$ et $q$ deux projecteurs de $E$.
\begin{enumerate}
\item Montrer que $p+q$ est un projecteur de $E$ £ssi. $p◦q = q◦p = \tilde 0$.
\item Montrer que, dans ce cas, $\Ima(p+q) = \Ima p⊕\Ima q$ et que $\Ker(p+q) = \Ker(p)∩\Ker q$.
\end{enumerate}

\Exercice

Soit $E$ un $𝕂$-espace vectoriel, $p$ et $q$ deux projecteurs de $E$ qui commutent.
\begin{enumerate}
\item Montrer que $p◦q$ est un projecteur.
\item Montrer que, dans ce cas, $\Ima(p◦q) = \Ima p∩\Ima q$ et $\Ker(p◦q) = \Ker p + \Ker q$.
\end{enumerate}

\Exercice

Soit $E$ un $𝕂$-espace vectoriel, $p$ et $q$ deux projecteurs de $E$ tels que $p◦q = \tilde 0$.
Soit $r = p + q - q◦p$.
\begin{enumerate}
\item Montrer que $r$ est un projecteur.
\item Montrer que $\Ima r = \Ima p⊕\Ima q$ et $\Ker r = \Ker p∩\Ker q$.
\end{enumerate}

\Exercice

Soit $E$ le $ℝ$-espace vectoriel des applications de $ℝ$ dans $ℝ$.

Étudier l'indépendance des familles de fonctions suivantes:
\begin{enumerate}
\item $\bigl( x \mapsto \sin^n x \bigr)_{n∈\Dcro{0,N}}$ où $N∈ℕ$ fixé
\item $\bigl( x \mapsto \ch^n x \bigr)_{n∈\Dcro{0,N}}$ où $N∈ℕ$ fixé
\item $\bigl( x \mapsto \Abs{x-a} \bigr)_{a∈A}$ où $A⊂ℝ$ est une partie finie de $ℝ$.
\end{enumerate}

\Exercice

Soit $E$ l'espace vectoriel des applications de $ℝ$ dans $ℝ$
et $F = \Ensemble{f∈E}{f(0)=f(1)=0}$.

Montrer que $F$ est un £sev. de $E$
et en donner un supplémentaire.

\Exercice

Soit $E$ le $ℝ$-espace vectoriel des applications de classe $\CC∞$ de $ℝ$ dans $ℝ$.

On considère la famille de fonctions $(f_n)_{n∈ℕ}$
définie par
\[ ∀n∈ℕ\+ f_{2n}(x) = \cos(nx) \text{ et } f_{2n+1}(x) = \sin\bigl( (n+1)x \bigr). \]

On pose alors $F_0 = \Vect(f_0)$, et $∀p∈\Ns$, $F_p = \Vect(f_{2p-1}, f_{2p})$.
Soit $\Fonction ΦE E f {f''}$
\begin{enumerate}
\item Montrer que $∀p∈ℕ^*$, $F_p = \Ker(Φ+ p^2 \Id_E)$
\item En déduire que la somme $∑_{k=0}^p F_k$ est directe.
\item En déduire que la famille $(f_n)_{0≤n≤N}$ est libre.
\end{enumerate}

\Exercice[£sev. engendré par une partie]

Soit $E$ un $𝕂$-espace vectoriel et $X⊂E$ une partie non vide de $E$.
On note $V$ l'ensemble des vecteurs $v∈E$ tels qu'il existe $n∈ℕ^*$, $\nUplet x1n ∈X^n$ et $\nUpletα1n ∈𝕂^n$ tels que
\[ v = ∑_{i=1}^n α_i x_i. \]
\begin{enumerate}
\item Dans cette question seulement, on suppose que l'ensemble $X = \{y_1,\dots,y_p\}$ est fini. Montrer que $X = \Vect(y_1,\dots,y_p)$.
\item Montrer que $V$ est un £sev. de $E$ et que $X⊂V$.
\item Soit $F$ un £sev. de $E$ tel que $X⊂V$.
  Montrer que $V⊂F$.
\item En déduire que $V$ est, au sens de l'inclusion, le plus petit £sev. de $E$ contenant la partie $X$.
\end{enumerate}

Le £sev. $V$ s'appelle le £sev. engendré par $X$ et noté $V = \Vect(X)$.

\Exercice[l'union de sous-espaces vectoriels n'est pas un £sev.]

Soit $E$ un $𝕂$-espace vectoriel.
\begin{enumerate}
\item On suppose que $F$ et $G$ sont deux sous-espaces vectoriels de $E$ tels que $H = F∪G$ est également un sous-espace vectoriel. Montrer que $F⊂G$ ou $G⊂F$.
\item Plus généralement, si $𝕂$ est infini, en supposant que $\Uplet{F_1}{F_n}$ sont des sous-espaces vectoriels de $E$ tels que $H = ⋃_{i=1}^n F_i$ est également un sous-espace vectoriel, montrer qu'il existe un $i_0∈\Dcro{1,n}$ tel que $∀i∈\Dcro{1,n}$, $F_i⊂F_{i_0}$.
\end{enumerate}

\Exercice[existence d'un supplémentaire commun]

Soit $E$ un $𝕂$-£evdf., $F$ et $G$ deux sous-espaces vectoriels de $E$ de même dimension $r$.
Montrer qu'il existe un sous-espace vectoriel $H$ tel que $E = F⊕H = G⊕H$.
On pourra procéder par récurrence descendante sur $r$.

\Exercice[polynômes de Lagrange]

Soit $\nUplet a0n ∈𝕂^{n+1}$ des nombres deux à deux distincts.
\begin{enumerate}
\item Soit $\nUplet b0n ∈𝕂^{n+1}$.
  Montrer qu'il existe un unique polynôme $P∈𝕂_n[X]$ tel que
  \[ ∀i∈\Dcro{0,n} \+ P(a_i) = b_i. \]
  Il s'agit du polynôme interpolateur de Lagrange.
  On pourra faire apparaître une matrice de Vandermonde.
\item On définit l'application
  \[ \Fonction{φ}{𝕂_n[X]}{𝕂^{n+1}}{P}{\bigl(P(a_0),\dots,P(a_n)\bigr).} \]
  \begin{enumerate}
  \item Montrer que $φ$ est un isomorphisme.
  \item On note $(e_0,\dots,e_n)$ la base canonique de $𝕂^{n+1}$.
    Déterminer explicitement l'unique polynôme $L_i$ tel que $φ(L_i) = e_i$.
  \item En déduire une expression du polynôme interpolateur de Lagrange.
  \end{enumerate}
\end{enumerate}

\Exercice

Soit $E$, $F$ et $G$ des $𝕂$-espaces vectoriels de dimensions finies, $\Fn uEF$ et $\Fn vFG$ des applications linéaires.
\begin{enumerate}
\item Montrer que $\Ima(v◦u)⊂\Ima(v)$. En déduire que $\Rang(v◦u)≤\Rang(v)$.
\item Montrer que $\Ker(v◦u)⊃\Ker(u)$. En déduire que $\Rang(v◦u)≤\Rang(u)$.
\item On appelle $w$ la restriction de $v$ à $\Ima(u)$.
  \begin{enumerate}
  \item Déterminer l'image et le noyau de $w$.
  \item Appliquer le théorème du rang à $w$.
  \item En remarquant que $\Ker(w) ⊂ \Ker(v)$, montrer que
    $ \Rang(u) + \Rang(v)≤\Rang(v◦u) + \dim(F) $.
  \end{enumerate}
\item Montrer que
  $ \Rang(u) + \Rang(v) - \dim(F) ≤ \Rang(v◦u) ≤ \min(\Rang u, \Rang v)$,
  et $ \dim \Ker(v◦u)≤\dim \Ker(u) + \dim \Ker(v) $.
\end{enumerate}


\Exercice

\begin{enumerate}
\item
  Soit $E$ et $F$ deux $𝕂$-£evdfs., $\Fn uEF$ et $\Fn vFE$ deux applications linéaires telles que $u◦v = \Id_F$.
  \begin{enumerate}
  \item
    On suppose $\dim E = \dim F$; montrer que $v◦u = \Id_E$.
  \item
    Trouver un contre-exemple dans le cas où $\dim E ≠\dim F$.
  \end{enumerate}
\item Soit $E$ un $𝕂$-£ev., $u$ et $v$ deux endomorphismes de $E$ tels que $u◦v = \Id_E$.
  A-t-on nécessairement $v◦u = \Id_E$?
\end{enumerate}


\Exercice

Soit $E$ un $𝕂$-£evdf., $f$ et $g$ deux endomorphismes de $E$ tels que $f^2 + f◦g = \Id_E$. Montrer que $f$ et $g$ commutent.

\end{document}
