\documentclass{yann}
\usepackage{tcolorbox}

\newcommand{\Par}{\mathcal{P}}
\newcommand{\BW}{Bolzano-Weierstraß}

\begin{document}
\title{Équivalence des normes en dimension finie}
\maketitle

On veut montrer le résultat suivant:
\begin{tcolorbox}
    Soit $\K$ le corps $\R$ ou $\C$.
    Si $E$ est un £Kevdf., alors toutes les normes sur $E$ sont équivalentes.
\end{tcolorbox}

\section{Préliminaires}

\Para{\BW{} sur $\R$}

Toute suite réelle bornée admet une sous-suite convergente.

\Para{\BW{} sur $\C$}

Toute suite réelle bornée admet une sous-suite convergente.

\emph{Démonstration:}
si $(z_n)$ est une suite bornée, notons $z_n = x_n + i y_n$ où $(x_n, y_n) \in \R^2$.
Les suites $(x_n)$ et $(y_n)$ sont également bornées.
D'après \BW, il existe une extractice $\phi$ telle que la suite $(x_{\phi(n)})$ converge.
On considère alors la suite de terme général $y'_n = y_{\phi(n)}$.
Elle est bornée comme sous-suite d'une suite bornée, donc on peut à nouveau utiliser \BW:
il existe une extractrice $\psi$ telle que la suite $y'_{\psi(n)}$ converge.
Notons $\sigma = \phi \circ \psi$.
La suite $(x_{\sigma(n)})$ converge car c'est une sous-suite de $(x_{\phi(n)})$
et la suite $(y_{\sigma(n)})$ converge par construction,
donc la suite $(z_{\sigma(n)})$ converge.

\Para{Bolzano-Weierstraß sur $\K^p$}

Soit $(u_n^1)_{n\in\N}$, $(u_n^2)_{n\in\N}$, $\dots$, $(u_n^p)_{n\in\N}$ des suites numériques bornées.
Alors il existe une extractrice $\sigma$ telle que
pour tout $i\in\ccro{1,p}$, la suite $(u_{\sigma(n)}^i)_{n\in\N}$ converge.

\emph{Démonstration:}
On fait une récurrence sur $p$; l'hérédité se démontre avec la même technique que ci-dessus.

\section{Preuve}

Soit $\B = \nUplet e1p$ une base de $E$, fixée dans toute la preuve.
Pour tout $x\in\E$, on note $(x^1, \dots, x^p)$ les coordonnées de $x$ dans la base $\B$, £cad.
\[ x = \sum_{i=1}^p x^i e_i. \]
On note $\DS N(x) = \max_{1\leq i\leq p} \Abs{x^i}$.
On vérifie sans peine que $N$ est bien une norme sur $E$.

Soit $\Norme$ une norme quelconque sur $E$; on va chercher à montrer que $N$ et $\Norme$ sont équivalentes.

Pour tout $x\in E$, on a
\begin{align*}
    \Norm{x} &= \left\| \sum_{i=1}^p x^i e_i \right\|
    \leq \sum_{i=1}^p \Norm{x^i e_i}
    = \sum_{i=1}^p \Abs{x^i} \Norm{e_i} \\
    &\leq \sum_{i=1}^p N(x) \Norm{e_i} = \beta N(x)
\end{align*}
où l'on pose $\beta = \sum_{i=1}^p \Norm{e_i} > 0$.
On pose également $S = \Ensemble{x\in E}{N(x) = 1}$ et $\DS \alpha = \inf_{x\in S} \Norm{x}$.

Si $x\in E$, $x\neq0$, on pose $y=x/N(x)$ de sorte que $y\in S$.
Par définition de $\alpha$, on a donc $\Norm{y} \geq \alpha$
et donc par homogénéité de la norme $\Norme$, on a $\Norm{x} \geq \alpha N(x)$.
Cette dernière inégalité est encore vraie si $x=0$.
Ainsi, on a montré que
\[ \forall x \in E \+ \alpha N(x) \leq \Norm{x} \leq \beta N(x). \]

Il ne reste plus qu'à prouver que $\alpha > 0$.
Supposons donc par l'absurde que $\alpha = 0$.

Pour tout $n\in\N$, il existe par définition de $\alpha$ un vecteur $u_n$ de $S$ tel que $\Norm{u_n} \leq \frac{1}{n+1}$.
D'après \BW{} sur $\K^p$, il existe une sous-suite $(v_n)$ de $(u_n)$ telle que
les suites $(v_n^i)_{n\in\N}$ convergent.
Notons $\ell = \sum_{i=1}^p \ell^i e_i$ où $\ell^i = \lim_\ninf v_n^i$.

On a $\Norm{\ell} = \Norm{\ell - v_n + v_n} \leq \Norm{\ell - v_n} + \Norm{v_n} \leq \beta N(v_n - \ell) + \Norm{v_n}$.
Or pour tout $i\in\ccro{1,p}$, on a $\Abs{v_n^i - \ell^i} \Toninf 0$ et donc $N(v_n - \ell) \Toninf 0$.
De plus, la suite $(\Norm{v_n})$ étant une sous-suite de $(\Norm{u_n})$, on a $\Norm{v_n} \Toninf 0$.
Ainsi, $\Norm{\ell} \Toninf 0$ donc $\Norm{\ell} = 0$ donc $\ell = 0$.

Pour tout $i\in\ccro{1,p}$, on a $v_n^i \to \ell^i = 0$, donc $\exists n_i\in\N$, $\forall n\geq n_i$, $\Abs{v_n^i} \leq 1/2$.
Soit $m = \max(n_1,\dots,n_p)$ de sorte que $N(v_m) \leq 1/2$. Ce qui est absurde car $v_m \in S$.

Ainsi $\alpha > 0$ donc les normes $N$ et $\Norme$ sont bien équivalentes.
Bref, toute norme sur $E$ était équivalente à $N$,
donc toutes les normes sont équivalentes.

\end{document}
