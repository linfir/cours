\documentclass{yann}
\usepackage{tcolorbox}

\newcommand{\Par}{\mathcal{P}}
\newcommand{\BW}{Bolzano-Weierstraß}

\begin{document}
\title{Équivalence des normes en dimension finie}
\maketitle

On veut montrer le résultat suivant:
\begin{tcolorbox}
  Soit $𝕂$ le corps $ℝ$ ou $ℂ$.
  Si $E$ est un £Kevdf., alors toutes les normes sur $E$ sont équivalentes.
\end{tcolorbox}

\section{Préliminaires}

\Para{\BW{} sur $ℝ$}

Toute suite réelle bornée admet une sous-suite convergente.

\Para{\BW{} sur $ℂ$}

Toute suite réelle bornée admet une sous-suite convergente.

\emph{Démonstration:}
si $(z_n)$ est une suite bornée, notons $z_n = x_n + i y_n$ où $(x_n, y_n) ∈ ℝ^2$.
Les suites $(x_n)$ et $(y_n)$ sont également bornées.
D'après \BW, il existe une extractice $ϕ$ telle que la suite $(x_{ϕ(n)})$ converge.
On considère alors la suite de terme général $y'_n = y_{ϕ(n)}$.
Elle est bornée comme sous-suite d'une suite bornée, donc on peut à nouveau utiliser \BW:
il existe une extractrice $ψ$ telle que la suite $y'_{ψ(n)}$ converge.
Notons $σ = ϕ ◦ ψ$.
La suite $(x_{σ(n)})$ converge car c'est une sous-suite de $(x_{ϕ(n)})$
et la suite $(y_{σ(n)})$ converge par construction,
donc la suite $(z_{σ(n)})$ converge.

\Para{Bolzano-Weierstraß sur $𝕂^p$}

Soit $(u_n^1)_{n∈ℕ}$, $(u_n^2)_{n∈ℕ}$, $\dots$, $(u_n^p)_{n∈ℕ}$ des suites numériques bornées.
Alors il existe une extractrice $σ$ telle que
pour tout $i∈\ccro{1,p}$, la suite $(u_{σ(n)}^i)_{n∈ℕ}$ converge.

\emph{Démonstration:}
On fait une récurrence sur $p$; l'hérédité se démontre avec la même technique que ci-dessus.

\section{Preuve}

Soit $\B = \nUplet e1p$ une base de $E$, fixée dans toute la preuve.
Pour tout $x∈𝔼$, on note $(x^1, \dots, x^p)$ les coordonnées de $x$ dans la base $\B$, £cad.
\[ x = ∑_{i=1}^p x^i e_i. \]
On note $\DS N(x) = \max_{1≤i≤p} \Abs{x^i}$.
On vérifie sans peine que $N$ est bien une norme sur $E$.

Soit $\Norme$ une norme quelconque sur $E$; on va chercher à montrer que $N$ et $\Norme$ sont équivalentes.

Pour tout $x∈E$, on a
\begin{align*}
  \Norm{x} &= \left\| ∑_{i=1}^p x^i e_i \right\|
  ≤ ∑_{i=1}^p \Norm{x^i e_i}
  = ∑_{i=1}^p \Abs{x^i} \Norm{e_i} \\
  &≤ ∑_{i=1}^p N(x) \Norm{e_i} = βN(x)
\end{align*}
où l'on pose $β = ∑_{i=1}^p \Norm{e_i} > 0$.
On pose également $S = \Ensemble{x∈E}{N(x) = 1}$ et $\DS α = \inf_{x∈S} \Norm{x}$.

Si $x∈E$, $x≠0$, on pose $y=x/N(x)$ de sorte que $y∈S$.
Par définition de $α$, on a donc $\Norm{y} ≥ α$
et donc par homogénéité de la norme $\Norme$, on a $\Norm{x} ≥ αN(x)$.
Cette dernière inégalité est encore vraie si $x=0$.
Ainsi, on a montré que
\[ ∀x ∈E \+ αN(x) ≤ \Norm{x} ≤ βN(x). \]

Il ne reste plus qu'à prouver que $α > 0$.
Supposons donc par l'absurde que $α = 0$.

Pour tout $n∈ℕ$, il existe par définition de $α$ un vecteur $u_n$ de $S$ tel que $\Norm{u_n} ≤ \frac{1}{n+1}$.
D'après \BW{} sur $𝕂^p$, il existe une sous-suite $(v_n)$ de $(u_n)$ telle que
les suites $(v_n^i)_{n∈ℕ}$ convergent.
Notons $ℓ = ∑_{i=1}^p ℓ^i e_i$ où $ℓ^i = \lim_\ninf v_n^i$.

On a $\Normℓ = \Norm{ℓ - v_n + v_n} ≤ \Norm{ℓ - v_n} + \Norm{v_n} ≤ βN(v_n - ℓ) + \Norm{v_n}$.
Or pour tout $i∈\ccro{1,p}$, on a $\Abs{v_n^i - ℓ^i} \Toninf 0$ et donc $N(v_n - ℓ) \Toninf 0$.
De plus, la suite $(\Norm{v_n})$ étant une sous-suite de $(\Norm{u_n})$, on a $\Norm{v_n} \Toninf 0$.
Ainsi, $\Normℓ \Toninf 0$ donc $\Normℓ = 0$ donc $ℓ = 0$.

Pour tout $i∈\ccro{1,p}$, on a $v_n^i \to ℓ^i = 0$, donc $∃n_i∈ℕ$, $∀n≥n_i$, $\Abs{v_n^i} ≤ 1/2$.
Soit $m = \max(n_1,\dots,n_p)$ de sorte que $N(v_m) ≤ 1/2$. Ce qui est absurde car $v_m ∈S$.

Ainsi $α > 0$ donc les normes $N$ et $\Norme$ sont bien équivalentes.
Bref, toute norme sur $E$ était équivalente à $N$,
donc toutes les normes sont équivalentes.

\end{document}
