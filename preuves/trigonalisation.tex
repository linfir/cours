% autogenerated by ytex.rs

\documentclass{scrartcl}

\usepackage[francais]{babel}
\usepackage{geometry}
\usepackage{scrpage2}
\usepackage{lastpage}
\usepackage{ragged2e}
\usepackage{multicol}
\usepackage{etoolbox}
\usepackage{xparse}
\usepackage{enumitem}
\usepackage{csquotes}
\usepackage{amsmath}
\usepackage{amsfonts}
\usepackage{amssymb}
\usepackage{mathrsfs}
\usepackage{stmaryrd}
\usepackage{dsfont}
\usepackage{eurosym}
\usepackage{numprint}
\usepackage[most]{tcolorbox}
\usepackage{tikz}
\usepackage{tkz-tab}
\usepackage[unicode]{hyperref}
\usepackage[ocgcolorlinks]{ocgx2}

\let\ifTwoColumns\iffalse
\def\Classe{$\Psi$2019--2020}

% reproducible builds
% LuaTeX: \pdfvariable suppressoptionalinfo 1023 \relax
\pdfinfoomitdate=1
\pdftrailerid{}

\newif\ifDisplaystyle
\everymath\expandafter{\the\everymath\ifDisplaystyle\displaystyle\fi}
\newcommand\DS{\displaystyle}

\clearscrheadfoot
\pagestyle{scrheadings}
\thispagestyle{empty}
\ohead{\Classe}
\ihead{\thepage/\pageref*{LastPage}}

\setlist[itemize,1]{label=\textbullet}
\setlist[itemize,2]{label=\textbullet}

\ifTwoColumns
  \geometry{margin=1cm,top=2cm,bottom=3cm,foot=1cm}
  \setlist[enumerate]{leftmargin=*}
  \setlist[itemize]{leftmargin=*}
\else
  \geometry{margin=3cm}
\fi

\makeatletter
\let\@author=\relax
\let\@date=\relax
\renewcommand\maketitle{%
    \begin{center}%
        {\sffamily\huge\bfseries\@title}%
        \ifx\@author\relax\else\par\medskip{\itshape\Large\@author}\fi
        \ifx\@date\relax\else\par\bigskip{\large\@date}\fi
    \end{center}\bigskip
    \ifTwoColumns
        \par\begin{multicols*}{2}%
        \AtEndDocument{\end{multicols*}}%
        \setlength{\columnsep}{5mm}
    \fi
}
\makeatother

\newcounter{ParaNum}
\NewDocumentCommand\Para{smo}{%
  \IfBooleanF{#1}{\refstepcounter{ParaNum}}%
  \paragraph{\IfBooleanF{#1}{{\tiny\arabic{ParaNum}~}}#2\IfNoValueF{#3}{ (#3)}}}

\newcommand\I{i}
\newcommand\mi{i}
\def\me{e}

\def\do#1{\expandafter\undef\csname #1\endcsname}
\docsvlist{Ker,sec,csc,cot,sinh,cosh,tanh,coth,th}
\undef\do

\DeclareMathOperator\ch{ch}
\DeclareMathOperator\sh{sh}
\DeclareMathOperator\th{th}
\DeclareMathOperator\coth{coth}
\DeclareMathOperator\cotan{cotan}
\DeclareMathOperator\argch{argch}
\DeclareMathOperator\argsh{argsh}
\DeclareMathOperator\argth{argth}

\let\epsilon=\varepsilon
\let\phi=\varphi
\let\leq=\leqslant
\let\geq=\geqslant
\let\subsetneq=\varsubsetneq
\let\emptyset=\varnothing

\newcommand{\+}{,\;}

\undef\C
\newcommand\ninf{{n\infty}}
\newcommand\N{\mathbb{N}}
\newcommand\Z{\mathbb{Z}}
\newcommand\Q{\mathbb{Q}}
\newcommand\R{\mathbb{R}}
\newcommand\C{\mathbb{C}}
\newcommand\K{\mathbb{K}}
\newcommand\Ns{\N^*}
\newcommand\Zs{\Z^*}
\newcommand\Qs{\Q^*}
\newcommand\Rs{\R^*}
\newcommand\Cs{\C^*}
\newcommand\Ks{\K^*}
\newcommand\Rp{\R^+}
\newcommand\Rps{\R^+_*}
\newcommand\Rms{\R^-_*}
\newcommand{\Rpinf}{\Rp\cup\Acco{+\infty}}

\undef\B
\newcommand\B{\mathscr{B}}

\undef\P
\DeclareMathOperator\P{\mathbb{P}}
\DeclareMathOperator\E{\mathbb{E}}
\DeclareMathOperator\Var{\mathbb{V}}

\DeclareMathOperator*\PetitO{o}
\DeclareMathOperator*\GrandO{O}
\DeclareMathOperator*\Sim{\sim}
\DeclareMathOperator\Tr{tr}
\DeclareMathOperator\Ima{Im}
\DeclareMathOperator\Ker{Ker}
\DeclareMathOperator\Sp{Sp}
\DeclareMathOperator\Diag{diag}
\DeclareMathOperator\Rang{rang}
\DeclareMathOperator*\Coords{Coords}
\DeclareMathOperator*\Mat{Mat}
\DeclareMathOperator\Pass{Pass}
\DeclareMathOperator\Com{Com}
\DeclareMathOperator\Card{Card}
\DeclareMathOperator\Racines{Racines}
\DeclareMathOperator\Vect{Vect}
\DeclareMathOperator\Id{Id}

\newcommand\DerPart[2]{\frac{\partial #1}{\partial #2}}

\def\T#1{{#1}^T}

\def\pa#1{({#1})}
\def\Pa#1{\left({#1}\right)}
\def\bigPa#1{\bigl({#1}\bigr)}
\def\BigPa#1{\Bigl({#1}\Bigr)}
\def\biggPa#1{\biggl({#1}\biggr)}
\def\BiggPa#1{\Biggl({#1}\Biggr)}

\def\pafrac#1#2{\pa{\frac{#1}{#2}}}
\def\Pafrac#1#2{\Pa{\frac{#1}{#2}}}
\def\bigPafrac#1#2{\bigPa{\frac{#1}{#2}}}
\def\BigPafrac#1#2{\BigPa{\frac{#1}{#2}}}
\def\biggPafrac#1#2{\biggPa{\frac{#1}{#2}}}
\def\BiggPafrac#1#2{\BiggPa{\frac{#1}{#2}}}

\def\cro#1{[{#1}]}
\def\Cro#1{\left[{#1}\right]}
\def\bigCro#1{\bigl[{#1}\bigr]}
\def\BigCro#1{\Bigl[{#1}\Bigr]}
\def\biggCro#1{\biggl[{#1}\biggr]}
\def\BiggCro#1{\Biggl[{#1}\Biggr]}

\def\abs#1{\mathopen|{#1}\mathclose|}
\def\Abs#1{\left|{#1}\right|}
\def\bigAbs#1{\bigl|{#1}\bigr|}
\def\BigAbs#1{\Bigl|{#1}\Bigr|}
\def\biggAbs#1{\biggl|{#1}\biggr|}
\def\BiggAbs#1{\Biggl|{#1}\Biggr|}

\def\acco#1{\{{#1}\}}
\def\Acco#1{\left\{{#1}\right\}}
\def\bigAcco#1{\bigl\{{#1}\bigr\}}
\def\BigAcco#1{\Bigl\{{#1}\Bigr\}}
\def\biggAcco#1{\biggl\{{#1}\biggr\}}
\def\BiggAcco#1{\Biggl\{{#1}\Biggr\}}

\def\ccro#1{\llbracket{#1}\rrbracket}
\def\Dcro#1{\llbracket{#1}\rrbracket}

\def\floor#1{\lfloor#1\rfloor}
\def\Floor#1{\left\lfloor{#1}\right\rfloor}

\def\sEnsemble#1#2{\mathopen\{#1\mid#2\mathclose\}}
\def\bigEnsemble#1#2{\bigl\{#1\bigm|#2\bigr\}}
\def\BigEnsemble#1#2{\Bigl\{#1\Bigm|#2\Bigr\}}
\def\biggEnsemble#1#2{\biggl\{#1\biggm|#2\biggr\}}
\def\BiggEnsemble#1#2{\Biggl\{#1\Biggm|#2\Biggr\}}
\let\Ensemble=\bigEnsemble

\newcommand\IntO[1]{\left]#1\right[}
\newcommand\IntF[1]{\left[#1\right]}
\newcommand\IntOF[1]{\left]#1\right]}
\newcommand\IntFO[1]{\left[#1\right[}

\newcommand\intO[1]{\mathopen]#1\mathclose[}
\newcommand\intF[1]{\mathopen[#1\mathclose]}
\newcommand\intOF[1]{\mathopen]#1\mathclose]}
\newcommand\intFO[1]{\mathopen[#1\mathclose[}

\newcommand\Fn[3]{#1\colon#2\to#3}
\newcommand\CC[1]{\mathscr{C}^{#1}}
\newcommand\D{\mathop{}\!\mathrm{d}}

\newcommand\longto{\longrightarrow}

\undef\M
\newcommand\M[3]{\mathrm{#1}_{#2}\pa{#3}}
\newcommand\MnR{\M{M}{n}{\R}}
\newcommand\MnC{\M{M}{n}{\C}}
\newcommand\MnK{\M{M}{n}{\K}}
\newcommand\GLnR{\M{GL}{n}{\R}}
\newcommand\GLnC{\M{GL}{n}{\C}}
\newcommand\GLnK{\M{GL}{n}{\K}}
\newcommand\DnR{\M{D}{n}{\R}}
\newcommand\DnC{\M{D}{n}{\C}}
\newcommand\DnK{\M{D}{n}{\K}}
\newcommand\SnR{\M{S}{n}{\R}}
\newcommand\AnR{\M{A}{n}{\R}}
\newcommand\OnR{\M{O}{n}{\R}}
\newcommand\SnRp{\mathrm{S}_n^+(\R)}
\newcommand\SnRpp{\mathrm{S}_n^{++}(\R)}

\newcommand\LE{\mathscr{L}(E)}
\newcommand\GLE{\mathscr{GL}(E)}
\newcommand\SE{\mathscr{S}(E)}
\renewcommand\OE{\mathscr{O}(E)}

\newcommand\ImplD{$\Cro\Rightarrow$}
\newcommand\ImplR{$\Cro\Leftarrow$}
\newcommand\InclD{$\Cro\subset$}
\newcommand\InclR{$\Cro\supset$}
\newcommand\notInclD{$\Cro{\not\subset}$}
\newcommand\notInclR{$\Cro{\not\supset}$}

\newcommand\To[1]{\xrightarrow[#1]{}}
\newcommand\Toninf{\To{\ninf}}

\newcommand\Norm[1]{\|#1\|}
\newcommand\Norme{{\Norm{\cdot}}}

\newcommand\Int[1]{\mathring{#1}}
\newcommand\Adh[1]{\overline{#1}}

\newcommand\Uplet[2]{{#1},\dots,{#2}}
\newcommand\nUplet[3]{(\Uplet{{#1}_{#2}}{{#1}_{#3}})}

\newcommand\Fonction[5]{{#1}\left|\begin{aligned}{#2}&\;\longto\;{#3}\\{#4}&\;\longmapsto\;{#5}\end{aligned}\right.}

\DeclareMathOperator\orth{\bot}
\newcommand\Orth[1]{{#1}^\bot}
\newcommand\PS[2]{\langle#1,#2\rangle}

\newcommand{\Tribu}{\mathscr{T}}
\newcommand{\Part}{\mathcal{P}}
\newcommand{\Pro}{\bigPa{\Omega,\Tribu}}
\newcommand{\Prob}{\bigPa{\Omega,\Tribu,\P}}

\newcommand\DEMO{$\spadesuit$}
\newcommand\DUR{$\spadesuit$}

\newenvironment{psmallmatrix}{\left(\begin{smallmatrix}}{\end{smallmatrix}\right)}


% -----------------------------------------------------------------------------


\newcommand\HR[1]{\mathcal{H}_{#1}}
\newcommand\demo{\par\medskip\emph{D\'emonstration.}\par}
\newcommand\Ia{\mathcal{I}}

\begin{document}
\title{Trigonalisation}
\maketitle

\Para{Th\'eor\`eme de B\'ezout}

Soit $P$ et $Q$ deux polyn\^omes de $\K[X]$ premiers entre eux.
Alors il existe deux polyn\^omes $A$ et $B$ tels que
\[ AP + BQ = 1. \]

\demo

Soit $\Ia = \Ensemble{AP+BP}{(A,B) \in{} \K[X]^2}$.
On doit montrer que $1 \in{} \Ia$.

On montre imm\'ediatement $\Ia$ est un \emph{id\'eal} de $\K[X]$, c.-\`a-d.:
\begin{itemize}
\item
  $\Ia \subset{} \K[X]$;
\item
  si $R \in{} \Ia$ et $S \in{} \Ia$, alors $R+S \in{} \Ia$;
\item
  si $R \in{} \K[X]$ et $S \in{} \Ia$, alors $RS \in{} \Ia$.
\end{itemize}

Comme $P$ et $Q$ sont dans $\Ia$, on doit avoir $\Ia \neq{} \Acco{0}$.
Soit $d = \min \Ensemble{\deg P}{P \in{} \Ia\setminus\Acco{0}}$
et $D \in{} \Ia$ tel que $\deg D = d$.

On effectue la division euclidienne de $P$ par $D$:
il existe deux polyn\^omes $U$ et $V$ de $\K[X]$ tels que $P = DU + V$
et $\deg V < \deg D$.
Comme $D \in{} \Ia$, on a $(-U)D \in{} \Ia$.
Comme $P \in{} \Ia$, on a $V = P + (-U)D \in{} \Ia$.
Or $\deg V < d$, donc par d\'efinition de $d$, on doit avoir $V = 0$.
Ainsi $P = UD$ est un multiple de $D$.

Les r\^oles de $P$ et de $Q$ \'etant sym\'etriques, $Q$ est \'egalement un multiple de $D$.
Comme $P$ et $Q$ sont premiers entre eux,
on en d\'eduit que le diviseur commun $D$ est un polyn\^ome constant non nul.
Notons $D = \delta{} \in{} \Ks$. Ainsi $\frac{1}{\delta} D \in{} \Ia$, d'o\`u $1 \in{} \Ia$, d'o\`u le r\'esultat.

\Para{Lemme des noyaux}

Soit $E$ un $\K$-espace vectoriel,
$u\in\LE$,
$P$ et $Q$ deux polyn\^omes de $\K[X]$ premiers entre eux.
Alors
\[ \Ker\bigCro{PQ(u)} = \Ker\bigCro{P(u)} \oplus{} \Ker\bigCro{Q(u)}. \]

\demo

Notons $F = \Ker\bigCro{P(u)}$, $G = \Ker\bigCro{Q(u)}$ et $H = \Ker\bigCro{PQ(u)}$.

\begin{itemize}
\item
  Comme $P$ et $Q$ sont premiers entre eux, le th\'eor\`eme de B\'ezout assure
  l'existence de deux polyn\^omes $A$ et $B$ de $\K[X]$ tels que
  $AP+BQ = 1$.
  En \'evaluant en $u$, il vient
  \[ A(u) \circ{} P(u) + B(u) \circ{} Q(u) = \Id_E. \]

  Si $x \in{} F \cap{} G$, on a alors
  \[ x = \bigCro{A(u) \circ{} P(u)} (x) + \bigCro{B(u) \circ{} Q(u)} (x) = 0_E + 0_E = 0_E. \]

  Ainsi, la somme $F \oplus{} G$ est directe.

\item
  On rappelle que pour tous endomorphismes $f$ et $g$, on a $\Ker(g) \subset{} \Ker(f\circ g)$.
  Or $PQ(u) = Q(u) \circ{} P(u)$, donc $F \subset{} H$.
  De m\^eme, $PQ(u) = P(u) \circ{} Q(u)$, donc $G \subset{} H$.
  Ainsi $F \oplus{} G \subset{} H$.

\item
  Soit $x \in{} H$, posons $y = BQ(u)(x)$ et $z = AP(u)(x)$.
  Alors \[ y+z = (BQ+AP)(u)(x) = 1_{\K[X]}(u)(x) = \Id_E(x) = x \]
  et \[ P(u)(y) = PBQ(u)(x) = \bigCro{ B(u) \circ{} PQ(u) }(x) = B(u)(0_E) = 0_E, \]
  de sorte que $y \in{} F$.
  De m\^eme,
  \[ Q(u)(z) = (QAP)(u)(x) = \bigCro{ A(u) \circ{} PQ(u) }(x) = A(u)(0_E) = 0_E, \]
  de sorte que $z \in{} G$.
  On a montr\'e que $H \subset{} F \oplus{} G$
\end{itemize}

D'o\`u le r\'esultat.

\Para{Corollaire}

Soit $E$ un $\K$-espace vectoriel,
$u\in\LE$.
On suppose que le polyn\^ome
\[ P = \prod_{i=1}^p (X-\lambda_i)^{n_i} \]
est un polyn\^ome annulateur de $u$,
o\`u $\nUplet\lambda1n$ sont deux \`a deux distincts.
Alors
\[ E = \bigoplus_{i=1}^p \Ker\bigCro{ \pa{u-\lambda_i \Id_E}^{n_i} }. \]

\demo

Pour $1 \leq{} i \leq{} p$, notons $P_i = (X-\lambda_i)^{n_i}$.
Les polyn\^omes $\nUplet{P}{1}{n}$ sont deux \`a deux premiers entre eux.
Une r\'ecurrence facile \`a partir du lemme des noyaux prouve que
\[ \Ker P(u) = \bigoplus_{i=1}^n \Ker P_i(u). \]

Comme $P$ est un polyn\^ome annulateur de $u$, on a $\Ker P(u) = E$, d'o\`u le r\'esultat.

\Para{Proposition}

Soit $E$ un $\K$-espace vectoriel de dimension finie et $u\in\LE$.
On suppose que $u$ un endomorphisme nilpotent.
Alors $u$ est trigonalisable.

\demo

Notons, pour tout $k\in \N$, $K_k = \Ker(u^k)$.
Pour tout $k \in{} \Ns$, on v\'erifie imm\'ediatement que $K_{k-1} \subset{} K_k$.
On peut donc choisir un suppl\'ementaire $F_k$ de $K_{k-1}$ dans $K_k$,
de sorte que
\[ \forall{} k \in{} \Ns \+ F_k \oplus{} K_{k-1} = K_k. \]

On v\'erifie alors que
\[ \forall{} k \in{} \Ns \+ K_k = \bigoplus_{i=1}^k F_i. \]

Soit $p \in{} \N$ tel que $u^p = \tilde 0$.
On a donc $K_p = E$, d'o\`u
\[ E = \bigoplus_{i=1}^p F_i. \]

Soit $\B$ une base adapt\'ee \`a cette d\'ecomposition.
Comme $u(K_k) \subset{} K_{k-1}$, la matrice de $u$ dans cette base
est triangulaire par blocs, \`a blocs diagonaux nuls.
Elle est donc triangulaire sup\'erieure \`a diagonale nulle,
d'o\`u le r\'esultat.

\Para{Th\'eor\`eme}

Soit $E$ un $\K$-espace vectoriel de dimension finie et $u\in\LE$.
On suppose que $u$ admet un polyn\^ome annulateur scind\'e (non nul).
Alors $u$ est trigonalisable.

\demo

Quitte \`a diviser par le coefficient dominant,
on peut supposer que le polyn\^ome
\[ P = \prod_{i=1}^p (X-\lambda_i)^{n_i} \]
est un polyn\^ome annulateur de $u$, o\`u $\nUplet\lambda1n$ sont deux \`a deux distincts.
D'apr\`es le corollaire du lemme des noyaux, on a alors
\[ E = \bigoplus_{i=1}^p F_i \quad\text{o\`u}\quad F_i = \Ker\Cro{(u-\lambda_i \Id_E)^{n_i}}. \]

Comme $u$ commute avec $(u-\lambda_i \Id_E)^{n_i}$, le sous-espace vectoriel $F_i$ est stable par $u$.
Notons $u_i$ l'endomorphisme induit par $u$ sur $F_i$
et $v_i = u_i - \lambda_i \Id_{F_i}$.
On v\'erifie facilement que $v_i^{n_i} = \tilde 0$,
donc $v_i$ est trigonalisable d'apr\`es le proposition pr\'ec\'edente,
donc $u_i = v_i + \lambda_i \Id_E$ est \'egalement trigonalisable.
Soit $\B_i$ une base de $F_i$ dans laquelle $\Mat_{\B_i}(u_i)$ est triangulaire.
La concat\'enation $\B$ de $\B_1, \dots, \B_p$ est alors une base de $E$
dans laquelle la matrice de $u$
est diagonale par blocs:
\[ \begin{pmatrix} T_1 & & \\ & \ddots & \\ & & T_p \end{pmatrix} \quad\text{o\`u}\quad T_i = \Mat_{\B_i}(u_i) \text{ est triangulaire sup\'erieure.} \]

Cette matrice est triangulaire sup\'erieure, d'o\`u le r\'esultat.
\end{document}
