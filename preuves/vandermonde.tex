% autogenerated by ytex.rs

\documentclass{scrartcl}

\usepackage[francais]{babel}
\usepackage{geometry}
\usepackage{scrpage2}
\usepackage{lastpage}
\usepackage{ragged2e}
\usepackage{multicol}
\usepackage{etoolbox}
\usepackage{xparse}
\usepackage{enumitem}
% \usepackage{csquotes}
\usepackage{amsmath}
\usepackage{amsfonts}
\usepackage{amssymb}
\usepackage{mathrsfs}
\usepackage{stmaryrd}
\usepackage{dsfont}
\usepackage{eurosym}
% \usepackage{numprint}
\usepackage[most]{tcolorbox}
% \usepackage{tikz}
% \usepackage{tkz-tab}
\usepackage[unicode]{hyperref}
\usepackage[ocgcolorlinks]{ocgx2}

\let\ifTwoColumns\iftrue
\def\Classe{$\Psi$2019--2020}

% reproducible builds
% LuaTeX: \pdfvariable suppressoptionalinfo 1023 \relax
\pdfinfoomitdate=1
\pdftrailerid{}

\newif\ifDisplaystyle
\everymath\expandafter{\the\everymath\ifDisplaystyle\displaystyle\fi}
\newcommand\DS{\displaystyle}

\clearscrheadfoot
\pagestyle{scrheadings}
\thispagestyle{empty}
\ohead{\Classe}
\ihead{\thepage/\pageref*{LastPage}}

\setlist[itemize,1]{label=\textbullet}
\setlist[itemize,2]{label=\textbullet}

\ifTwoColumns
  \geometry{margin=1cm,top=2cm,bottom=3cm,foot=1cm}
  \setlist[enumerate]{leftmargin=*}
  \setlist[itemize]{leftmargin=*}
\else
  \geometry{margin=3cm}
\fi

\makeatletter
\let\@author=\relax
\let\@date=\relax
\renewcommand\maketitle{%
    \begin{center}%
        {\sffamily\huge\bfseries\@title}%
        \ifx\@author\relax\else\par\medskip{\itshape\Large\@author}\fi
        \ifx\@date\relax\else\par\bigskip{\large\@date}\fi
    \end{center}\bigskip
    \ifTwoColumns
        \par\begin{multicols*}{2}%
        \AtEndDocument{\end{multicols*}}%
        \setlength{\columnsep}{5mm}
    \fi
}
\makeatother

\newcounter{ParaNum}
\NewDocumentCommand\Para{smo}{%
  \IfBooleanF{#1}{\refstepcounter{ParaNum}}%
  \paragraph{\IfBooleanF{#1}{{\tiny\arabic{ParaNum}~}}#2\IfNoValueF{#3}{ (#3)}}}

\newcommand\I{i}
\newcommand\mi{i}
\def\me{e}

\def\do#1{\expandafter\undef\csname #1\endcsname}
\docsvlist{Ker,sec,csc,cot,sinh,cosh,tanh,coth,th}
\undef\do

\DeclareMathOperator\ch{ch}
\DeclareMathOperator\sh{sh}
\DeclareMathOperator\th{th}
\DeclareMathOperator\coth{coth}
\DeclareMathOperator\cotan{cotan}
\DeclareMathOperator\argch{argch}
\DeclareMathOperator\argsh{argsh}
\DeclareMathOperator\argth{argth}

\let\epsilon=\varepsilon
\let\phi=\varphi
\let\leq=\leqslant
\let\geq=\geqslant
\let\subsetneq=\varsubsetneq
\let\emptyset=\varnothing

\newcommand{\+}{,\;}

\undef\C
\newcommand\ninf{{n\infty}}
\newcommand\N{\mathbb{N}}
\newcommand\Z{\mathbb{Z}}
\newcommand\Q{\mathbb{Q}}
\newcommand\R{\mathbb{R}}
\newcommand\C{\mathbb{C}}
\newcommand\K{\mathbb{K}}
\newcommand\Ns{\N^*}
\newcommand\Zs{\Z^*}
\newcommand\Qs{\Q^*}
\newcommand\Rs{\R^*}
\newcommand\Cs{\C^*}
\newcommand\Ks{\K^*}
\newcommand\Rp{\R^+}
\newcommand\Rps{\R^+_*}
\newcommand\Rms{\R^-_*}
\newcommand{\Rpinf}{\Rp\cup\Acco{+\infty}}

\undef\B
\newcommand\B{\mathscr{B}}

\undef\P
\DeclareMathOperator\P{\mathbb{P}}
\DeclareMathOperator\E{\mathbb{E}}
\DeclareMathOperator\Var{\mathbb{V}}

\DeclareMathOperator*\PetitO{o}
\DeclareMathOperator*\GrandO{O}
\DeclareMathOperator*\Sim{\sim}
\DeclareMathOperator\Tr{tr}
\DeclareMathOperator\Ima{Im}
\DeclareMathOperator\Ker{Ker}
\DeclareMathOperator\Sp{Sp}
\DeclareMathOperator\Diag{diag}
\DeclareMathOperator\Rang{rang}
\DeclareMathOperator*\Coords{Coords}
\DeclareMathOperator*\Mat{Mat}
\DeclareMathOperator\Pass{Pass}
\DeclareMathOperator\Com{Com}
\DeclareMathOperator\Card{Card}
\DeclareMathOperator\Racines{Racines}
\DeclareMathOperator\Vect{Vect}
\DeclareMathOperator\Id{Id}

\newcommand\DerPart[2]{\frac{\partial #1}{\partial #2}}

\def\T#1{{#1}^T}

\def\pa#1{({#1})}
\def\Pa#1{\left({#1}\right)}
\def\bigPa#1{\bigl({#1}\bigr)}
\def\BigPa#1{\Bigl({#1}\Bigr)}
\def\biggPa#1{\biggl({#1}\biggr)}
\def\BiggPa#1{\Biggl({#1}\Biggr)}

\def\pafrac#1#2{\pa{\frac{#1}{#2}}}
\def\Pafrac#1#2{\Pa{\frac{#1}{#2}}}
\def\bigPafrac#1#2{\bigPa{\frac{#1}{#2}}}
\def\BigPafrac#1#2{\BigPa{\frac{#1}{#2}}}
\def\biggPafrac#1#2{\biggPa{\frac{#1}{#2}}}
\def\BiggPafrac#1#2{\BiggPa{\frac{#1}{#2}}}

\def\cro#1{[{#1}]}
\def\Cro#1{\left[{#1}\right]}
\def\bigCro#1{\bigl[{#1}\bigr]}
\def\BigCro#1{\Bigl[{#1}\Bigr]}
\def\biggCro#1{\biggl[{#1}\biggr]}
\def\BiggCro#1{\Biggl[{#1}\Biggr]}

\def\abs#1{\mathopen|{#1}\mathclose|}
\def\Abs#1{\left|{#1}\right|}
\def\bigAbs#1{\bigl|{#1}\bigr|}
\def\BigAbs#1{\Bigl|{#1}\Bigr|}
\def\biggAbs#1{\biggl|{#1}\biggr|}
\def\BiggAbs#1{\Biggl|{#1}\Biggr|}

\def\acco#1{\{{#1}\}}
\def\Acco#1{\left\{{#1}\right\}}
\def\bigAcco#1{\bigl\{{#1}\bigr\}}
\def\BigAcco#1{\Bigl\{{#1}\Bigr\}}
\def\biggAcco#1{\biggl\{{#1}\biggr\}}
\def\BiggAcco#1{\Biggl\{{#1}\Biggr\}}

\def\ccro#1{\llbracket{#1}\rrbracket}
\def\Dcro#1{\llbracket{#1}\rrbracket}

\def\floor#1{\lfloor#1\rfloor}
\def\Floor#1{\left\lfloor{#1}\right\rfloor}

\def\sEnsemble#1#2{\mathopen\{#1\mid#2\mathclose\}}
\def\bigEnsemble#1#2{\bigl\{#1\bigm|#2\bigr\}}
\def\BigEnsemble#1#2{\Bigl\{#1\Bigm|#2\Bigr\}}
\def\biggEnsemble#1#2{\biggl\{#1\biggm|#2\biggr\}}
\def\BiggEnsemble#1#2{\Biggl\{#1\Biggm|#2\Biggr\}}
\let\Ensemble=\bigEnsemble

\newcommand\IntO[1]{\left]#1\right[}
\newcommand\IntF[1]{\left[#1\right]}
\newcommand\IntOF[1]{\left]#1\right]}
\newcommand\IntFO[1]{\left[#1\right[}

\newcommand\intO[1]{\mathopen]#1\mathclose[}
\newcommand\intF[1]{\mathopen[#1\mathclose]}
\newcommand\intOF[1]{\mathopen]#1\mathclose]}
\newcommand\intFO[1]{\mathopen[#1\mathclose[}

\newcommand\Fn[3]{#1\colon#2\to#3}
\newcommand\CC[1]{\mathscr{C}^{#1}}
\newcommand\D{\mathop{}\!\mathrm{d}}

\newcommand\longto{\longrightarrow}

\undef\M
\newcommand\M[3]{\mathrm{#1}_{#2}\pa{#3}}
\newcommand\MnR{\M{M}{n}{\R}}
\newcommand\MnC{\M{M}{n}{\C}}
\newcommand\MnK{\M{M}{n}{\K}}
\newcommand\GLnR{\M{GL}{n}{\R}}
\newcommand\GLnC{\M{GL}{n}{\C}}
\newcommand\GLnK{\M{GL}{n}{\K}}
\newcommand\DnR{\M{D}{n}{\R}}
\newcommand\DnC{\M{D}{n}{\C}}
\newcommand\DnK{\M{D}{n}{\K}}
\newcommand\SnR{\M{S}{n}{\R}}
\newcommand\AnR{\M{A}{n}{\R}}
\newcommand\OnR{\M{O}{n}{\R}}
\newcommand\SnRp{\mathrm{S}_n^+(\R)}
\newcommand\SnRpp{\mathrm{S}_n^{++}(\R)}

\newcommand\LE{\mathscr{L}(E)}
\newcommand\GLE{\mathscr{GL}(E)}
\newcommand\SE{\mathscr{S}(E)}
\renewcommand\OE{\mathscr{O}(E)}

\newcommand\ImplD{$\Cro\Rightarrow$}
\newcommand\ImplR{$\Cro\Leftarrow$}
\newcommand\InclD{$\Cro\subset$}
\newcommand\InclR{$\Cro\supset$}
\newcommand\notInclD{$\Cro{\not\subset}$}
\newcommand\notInclR{$\Cro{\not\supset}$}

\newcommand\To[1]{\xrightarrow[#1]{}}
\newcommand\Toninf{\To{\ninf}}

\newcommand\Norm[1]{\|#1\|}
\newcommand\Norme{{\Norm{\cdot}}}

\newcommand\Int[1]{\mathring{#1}}
\newcommand\Adh[1]{\overline{#1}}

\newcommand\Uplet[2]{{#1},\dots,{#2}}
\newcommand\nUplet[3]{(\Uplet{{#1}_{#2}}{{#1}_{#3}})}

\newcommand\Fonction[5]{{#1}\left|\begin{aligned}{#2}&\;\longto\;{#3}\\{#4}&\;\longmapsto\;{#5}\end{aligned}\right.}

\DeclareMathOperator\orth{\bot}
\newcommand\Orth[1]{{#1}^\bot}
\newcommand\PS[2]{\langle#1,#2\rangle}

\newcommand{\Tribu}{\mathscr{T}}
\newcommand{\Part}{\mathcal{P}}
\newcommand{\Pro}{\bigPa{\Omega,\Tribu}}
\newcommand{\Prob}{\bigPa{\Omega,\Tribu,\P}}

\newcommand\DEMO{$\spadesuit$}
\newcommand\DUR{$\spadesuit$}

\newenvironment{psmallmatrix}{\left(\begin{smallmatrix}}{\end{smallmatrix}\right)}

% -----------------------------------------------------------------------------

\newcommand{\Par}{\mathcal{P}}

\begin{document}
\title{Formule de Vandermonde}
\maketitle

On veut montrer le r\'esultat suivant:
\begin{tcolorbox}
  Si $(p,q,n)\in \N^3$, on a
  \[ \binom{p+q}{n} = \sum_{k\geq0} \binom{p}{k} \binom{q}{n-k}. \]
\end{tcolorbox}

\section{Preuve informelle}

On dispose de $p$~chaussettes noires et de $q$~chaussettes rouges.
De combien de fa\c cons peut-on choisir $n$~chaussettes?

D'abord, on a $N = \binom{p+q}{n}$ car il s'agit de choisir $n$~chaussettes parmi les $p+q$.

Par ailleurs, on peut diviser cette somme en plusieurs parties, selon le nombre~$k$ de chaussettes noires choisies.
Pour $k$~fix\'e, il y a $\binom{p}{k}$ fa\c cons de choisir les $k$~chaussettes noires
et $\binom{q}{n-k}$ fa\c cons de choisir les $(n-k)$~chaussettes rouges.
Ainsi, pour $k$~fix\'e, il y a exactement $\binom{p}{k} \binom{q}{n-k}$ fa\c cons de choisir les chaussettes.
Comme ces possibilit\'es sont disjointes, on en d\'eduit $N = \sum_k \binom{p}{k} \binom{q}{n-k}$, d'o\`u le r\'esultat.

\section{Preuve formelle}

\begin{enumerate}
\item
  \textbf{Rappel:} si $A$ est un ensemble fini de cardinal $n$ et $k\in \Z$,
  on note $\Par_k(A)$ l'ensemble des parties de $A$ de cardinal $k$.
  On sait que $\Card \Par_k(A) = \binom{n}{k}$.

\item
  Soit $X$ et $Y$ deux ensembles \emph{disjoints} et $Z = X\cup Y$.
  On d\'efinit deux applications
  \[ \Fonction{f}{\Par(Z)}{\Par(X)\times\Par(Y)}{A}{(A\cap X, A\cap Y)} \]
  \[ \Fonction{g}{\Par(X)\times\Par(Y)}{\Par(Z)}{(B,C)}{B\cup C} \]

  Si $A\in\Par(Z)$, on a $g(f(A)) = (A\cap X)\cup(A\cap Y) = A\cap(X\cup Y) = A\cap Z = A$.
  Ceci montre que $g\circ f = \Id_{\Par(Z)}$.

  Si $B\in\Par(X)$ et $C\in\Par(Y)$, on a $(B\cup C)\cap X = (B\cap X)\cup(C\cap X) = B\cup\emptyset{} = B$
  et $(B\cup C)\cap Y = (B\cap Y)\cup(C\cap Y) = \emptyset\cup C = C$ donc $f(g((B,C))) = (B,C)$.
  Ceci montre que $f\circ g = \Id_{\Par(X)\times\Par(Y)}$.

  Ainsi, $f$ et $g$ sont des bijections r\'eciproques l'une de l'autre.

\item
  Supposons d\'esormais que $X$ (respectivement $Y$) est un ensemble fini de cardinal $p$ (resp. $q$).
  Notons $\mathcal{A} = \Par_n(Z)$
  et $\mathcal{B} = f(\mathcal{A}) = \Ensemble{f(A)}{A\in\mathcal{A}}$.

  Comme $\Card(Z) = p+q$, on a directement que $\Card \mathcal{A} = \binom{p+q}{n}$.
  De plus, comme $f$ est injective, $\mathcal{A}$ et $\mathcal{B}$ ont le m\^eme cardinal.

\item
  On a $\mathcal{B} = \Ensemble{(B,C)\in\Par(X)\times\Par(Y)}{\Card(B\cup Y) = n}$.
  Posons, pour $k\in \N$,
  $\mathcal{B}_k = \Ensemble{(B,C)\in\Par(X)\times\Par(Y)}
  {\Card B = k \text{ et } \Card C = n-k}$.

  Les $(\mathcal{B}_k)_{k\in \N}$ sont deux \`a deux disjoints, car si $(B,C) \in{} \mathcal{B}_k \cap{} \mathcal{B}_l$,
  on a $k = \Card B = l$.  De plus,
  \[ \bigcup_{k\geq0} \mathcal{B}_k = \mathcal{B}. \]
  On le prouve par double inclusion.

  $\cro\subset$
  Soit $(B,C)\in\bigcup_{k\geq0} \mathcal{B}_k$.
  Il existe un $k_0\in \N$ tel que $x\in\mathcal{B}_{k_0}$.
  Par d\'efinition, $B\subset X$, $C\subset Y$, $\Card(B)=k_0$ et $\Card(C)=n-k_0$.
  De plus $B\cap C \subset{} X\cap Y = \emptyset$, donc $B$ et $C$ sont disjoints,
  donc $\Card(B\cup C) = \Card(B)+\Card(C) = n$.
  Ainsi $(B,C)\in\mathcal{B}$.

  $\cro\supset$
  Soit $(B,C)\in\mathcal{B}$.
  Par d\'efinition, $B\subset X$, $C\subset Y$ et $\Card(B\cup C) = n$.
  Notons $k_0 = \Card(B)$.
  De plus $B\cap C \subset{} X\cap Y = \emptyset$, donc $B$ et $C$ sont disjoints,
  donc $\Card(C) = \Card(B\cup C) - \Card(B) = n - k_0$.
  Ainsi $(B,C)\in\mathcal{B}_{k_0}$,
  donc $(B,C)\in\bigcup_{k\geq0} \mathcal{B}_k$.

\item
  Le point pr\'ec\'edent permet donc d'affirmer
  \[ \sum_{k\geq0} \Card(\mathcal{B}_k) = \Card(B). \]
  Notons qu'il s'agit d'une somme finie
  car $\Card(\mathcal{B}_k) = 0$ si $k>n$.

\item
  Enfin, $\mathcal{B}_k = \Par_k(X)\times\Par_{n-k}(Y)$
  donc $\Card(\mathcal{B}_k) = \binom{p}{k} \binom{q}{n-k}$.
  De plus, $\Card(\mathcal{B}) = \Card(\mathcal{A}) = \binom{p+q}{n}$.
\end{enumerate}
CQFD.

\section{G\'en\'eralisation}

Soit $n, k, p_1, \dots, p_n$ des entiers naturels et $p = p_1 + \dots + p_n$.
On a
\[ \binom pk  = \sum_{a\in S} \,\, \prod_{i=1}^n \binom{p_i}{a_i} \]
o\`u $S = \Ensemble{ a = \nUplet a1n \in \N^n} { a_1 + \dots + a_n = k }$.

\end{document}
