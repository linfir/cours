% autogenerated by ytex.rs

\documentclass{scrartcl}

\usepackage[francais]{babel}
\usepackage{geometry}
\usepackage{scrpage2}
\usepackage{lastpage}
\usepackage{ragged2e}
\usepackage{multicol}
\usepackage{etoolbox}
\usepackage{xparse}
\usepackage{enumitem}
\usepackage{csquotes}
\usepackage{amsmath}
\usepackage{amsfonts}
\usepackage{amssymb}
\usepackage{mathrsfs}
\usepackage{stmaryrd}
\usepackage{dsfont}
\usepackage{eurosym}
\usepackage{numprint}
\usepackage[most]{tcolorbox}
\usepackage{tikz}
\usepackage{tkz-tab}
\usepackage[unicode]{hyperref}
\usepackage[ocgcolorlinks]{ocgx2}

\let\ifTwoColumns\iffalse
\def\Classe{$\Psi$2019--2020}

% reproducible builds
% LuaTeX: \pdfvariable suppressoptionalinfo 1023 \relax
\pdfinfoomitdate=1
\pdftrailerid{}

\newif\ifDisplaystyle
\everymath\expandafter{\the\everymath\ifDisplaystyle\displaystyle\fi}
\newcommand\DS{\displaystyle}

\clearscrheadfoot
\pagestyle{scrheadings}
\thispagestyle{empty}
\ohead{\Classe}
\ihead{\thepage/\pageref*{LastPage}}

\setlist[itemize,1]{label=\textbullet}
\setlist[itemize,2]{label=\textbullet}

\ifTwoColumns
  \geometry{margin=1cm,top=2cm,bottom=3cm,foot=1cm}
  \setlist[enumerate]{leftmargin=*}
  \setlist[itemize]{leftmargin=*}
\else
  \geometry{margin=3cm}
\fi

\makeatletter
\let\@author=\relax
\let\@date=\relax
\renewcommand\maketitle{%
    \begin{center}%
        {\sffamily\huge\bfseries\@title}%
        \ifx\@author\relax\else\par\medskip{\itshape\Large\@author}\fi
        \ifx\@date\relax\else\par\bigskip{\large\@date}\fi
    \end{center}\bigskip
    \ifTwoColumns
        \par\begin{multicols*}{2}%
        \AtEndDocument{\end{multicols*}}%
        \setlength{\columnsep}{5mm}
    \fi
}
\makeatother

\newcounter{ParaNum}
\NewDocumentCommand\Para{smo}{%
  \IfBooleanF{#1}{\refstepcounter{ParaNum}}%
  \paragraph{\IfBooleanF{#1}{{\tiny\arabic{ParaNum}~}}#2\IfNoValueF{#3}{ (#3)}}}

\newcommand\I{i}
\newcommand\mi{i}
\def\me{e}

\def\do#1{\expandafter\undef\csname #1\endcsname}
\docsvlist{Ker,sec,csc,cot,sinh,cosh,tanh,coth,th}
\undef\do

\DeclareMathOperator\ch{ch}
\DeclareMathOperator\sh{sh}
\DeclareMathOperator\th{th}
\DeclareMathOperator\coth{coth}
\DeclareMathOperator\cotan{cotan}
\DeclareMathOperator\argch{argch}
\DeclareMathOperator\argsh{argsh}
\DeclareMathOperator\argth{argth}

\let\epsilon=\varepsilon
\let\phi=\varphi
\let\leq=\leqslant
\let\geq=\geqslant
\let\subsetneq=\varsubsetneq
\let\emptyset=\varnothing

\newcommand{\+}{,\;}

\undef\C
\newcommand\ninf{{n\infty}}
\newcommand\N{\mathbb{N}}
\newcommand\Z{\mathbb{Z}}
\newcommand\Q{\mathbb{Q}}
\newcommand\R{\mathbb{R}}
\newcommand\C{\mathbb{C}}
\newcommand\K{\mathbb{K}}
\newcommand\Ns{\N^*}
\newcommand\Zs{\Z^*}
\newcommand\Qs{\Q^*}
\newcommand\Rs{\R^*}
\newcommand\Cs{\C^*}
\newcommand\Ks{\K^*}
\newcommand\Rp{\R^+}
\newcommand\Rps{\R^+_*}
\newcommand\Rms{\R^-_*}
\newcommand{\Rpinf}{\Rp\cup\Acco{+\infty}}

\undef\B
\newcommand\B{\mathscr{B}}

\undef\P
\DeclareMathOperator\P{\mathbb{P}}
\DeclareMathOperator\E{\mathbb{E}}
\DeclareMathOperator\Var{\mathbb{V}}

\DeclareMathOperator*\PetitO{o}
\DeclareMathOperator*\GrandO{O}
\DeclareMathOperator*\Sim{\sim}
\DeclareMathOperator\Tr{tr}
\DeclareMathOperator\Ima{Im}
\DeclareMathOperator\Ker{Ker}
\DeclareMathOperator\Sp{Sp}
\DeclareMathOperator\Diag{diag}
\DeclareMathOperator\Rang{rang}
\DeclareMathOperator*\Coords{Coords}
\DeclareMathOperator*\Mat{Mat}
\DeclareMathOperator\Pass{Pass}
\DeclareMathOperator\Com{Com}
\DeclareMathOperator\Card{Card}
\DeclareMathOperator\Racines{Racines}
\DeclareMathOperator\Vect{Vect}
\DeclareMathOperator\Id{Id}

\newcommand\DerPart[2]{\frac{\partial #1}{\partial #2}}

\def\T#1{{#1}^T}

\def\pa#1{({#1})}
\def\Pa#1{\left({#1}\right)}
\def\bigPa#1{\bigl({#1}\bigr)}
\def\BigPa#1{\Bigl({#1}\Bigr)}
\def\biggPa#1{\biggl({#1}\biggr)}
\def\BiggPa#1{\Biggl({#1}\Biggr)}

\def\pafrac#1#2{\pa{\frac{#1}{#2}}}
\def\Pafrac#1#2{\Pa{\frac{#1}{#2}}}
\def\bigPafrac#1#2{\bigPa{\frac{#1}{#2}}}
\def\BigPafrac#1#2{\BigPa{\frac{#1}{#2}}}
\def\biggPafrac#1#2{\biggPa{\frac{#1}{#2}}}
\def\BiggPafrac#1#2{\BiggPa{\frac{#1}{#2}}}

\def\cro#1{[{#1}]}
\def\Cro#1{\left[{#1}\right]}
\def\bigCro#1{\bigl[{#1}\bigr]}
\def\BigCro#1{\Bigl[{#1}\Bigr]}
\def\biggCro#1{\biggl[{#1}\biggr]}
\def\BiggCro#1{\Biggl[{#1}\Biggr]}

\def\abs#1{\mathopen|{#1}\mathclose|}
\def\Abs#1{\left|{#1}\right|}
\def\bigAbs#1{\bigl|{#1}\bigr|}
\def\BigAbs#1{\Bigl|{#1}\Bigr|}
\def\biggAbs#1{\biggl|{#1}\biggr|}
\def\BiggAbs#1{\Biggl|{#1}\Biggr|}

\def\acco#1{\{{#1}\}}
\def\Acco#1{\left\{{#1}\right\}}
\def\bigAcco#1{\bigl\{{#1}\bigr\}}
\def\BigAcco#1{\Bigl\{{#1}\Bigr\}}
\def\biggAcco#1{\biggl\{{#1}\biggr\}}
\def\BiggAcco#1{\Biggl\{{#1}\Biggr\}}

\def\ccro#1{\llbracket{#1}\rrbracket}
\def\Dcro#1{\llbracket{#1}\rrbracket}

\def\floor#1{\lfloor#1\rfloor}
\def\Floor#1{\left\lfloor{#1}\right\rfloor}

\def\sEnsemble#1#2{\mathopen\{#1\mid#2\mathclose\}}
\def\bigEnsemble#1#2{\bigl\{#1\bigm|#2\bigr\}}
\def\BigEnsemble#1#2{\Bigl\{#1\Bigm|#2\Bigr\}}
\def\biggEnsemble#1#2{\biggl\{#1\biggm|#2\biggr\}}
\def\BiggEnsemble#1#2{\Biggl\{#1\Biggm|#2\Biggr\}}
\let\Ensemble=\bigEnsemble

\newcommand\IntO[1]{\left]#1\right[}
\newcommand\IntF[1]{\left[#1\right]}
\newcommand\IntOF[1]{\left]#1\right]}
\newcommand\IntFO[1]{\left[#1\right[}

\newcommand\intO[1]{\mathopen]#1\mathclose[}
\newcommand\intF[1]{\mathopen[#1\mathclose]}
\newcommand\intOF[1]{\mathopen]#1\mathclose]}
\newcommand\intFO[1]{\mathopen[#1\mathclose[}

\newcommand\Fn[3]{#1\colon#2\to#3}
\newcommand\CC[1]{\mathscr{C}^{#1}}
\newcommand\D{\mathop{}\!\mathrm{d}}

\newcommand\longto{\longrightarrow}

\undef\M
\newcommand\M[3]{\mathrm{#1}_{#2}\pa{#3}}
\newcommand\MnR{\M{M}{n}{\R}}
\newcommand\MnC{\M{M}{n}{\C}}
\newcommand\MnK{\M{M}{n}{\K}}
\newcommand\GLnR{\M{GL}{n}{\R}}
\newcommand\GLnC{\M{GL}{n}{\C}}
\newcommand\GLnK{\M{GL}{n}{\K}}
\newcommand\DnR{\M{D}{n}{\R}}
\newcommand\DnC{\M{D}{n}{\C}}
\newcommand\DnK{\M{D}{n}{\K}}
\newcommand\SnR{\M{S}{n}{\R}}
\newcommand\AnR{\M{A}{n}{\R}}
\newcommand\OnR{\M{O}{n}{\R}}
\newcommand\SnRp{\mathrm{S}_n^+(\R)}
\newcommand\SnRpp{\mathrm{S}_n^{++}(\R)}

\newcommand\LE{\mathscr{L}(E)}
\newcommand\GLE{\mathscr{GL}(E)}
\newcommand\SE{\mathscr{S}(E)}
\renewcommand\OE{\mathscr{O}(E)}

\newcommand\ImplD{$\Cro\Rightarrow$}
\newcommand\ImplR{$\Cro\Leftarrow$}
\newcommand\InclD{$\Cro\subset$}
\newcommand\InclR{$\Cro\supset$}
\newcommand\notInclD{$\Cro{\not\subset}$}
\newcommand\notInclR{$\Cro{\not\supset}$}

\newcommand\To[1]{\xrightarrow[#1]{}}
\newcommand\Toninf{\To{\ninf}}

\newcommand\Norm[1]{\|#1\|}
\newcommand\Norme{{\Norm{\cdot}}}

\newcommand\Int[1]{\mathring{#1}}
\newcommand\Adh[1]{\overline{#1}}

\newcommand\Uplet[2]{{#1},\dots,{#2}}
\newcommand\nUplet[3]{(\Uplet{{#1}_{#2}}{{#1}_{#3}})}

\newcommand\Fonction[5]{{#1}\left|\begin{aligned}{#2}&\;\longto\;{#3}\\{#4}&\;\longmapsto\;{#5}\end{aligned}\right.}

\DeclareMathOperator\orth{\bot}
\newcommand\Orth[1]{{#1}^\bot}
\newcommand\PS[2]{\langle#1,#2\rangle}

\newcommand{\Tribu}{\mathscr{T}}
\newcommand{\Part}{\mathcal{P}}
\newcommand{\Pro}{\bigPa{\Omega,\Tribu}}
\newcommand{\Prob}{\bigPa{\Omega,\Tribu,\P}}

\newcommand\DEMO{$\spadesuit$}
\newcommand\DUR{$\spadesuit$}

\newenvironment{psmallmatrix}{\left(\begin{smallmatrix}}{\end{smallmatrix}\right)}


% -----------------------------------------------------------------------------

\usepackage{tikz}
\Displaystyletrue

\begin{document}
\title{Preuve: produit de Cauchy}
\maketitle

Soit $(a_n)_{n\in \N}$ et $(b_n)_{n\in \N}$ deux suites num\'eriques telles que
les s\'eries $\sum_n a_n$ et $\sum_n b_n$ sont absolument convergentes.
On note $(c_n)_{n\in \N}$ la suite d\'efinie par
$\forall n\in \N$, $c_n = \sum_{k=0}^n a_k b_{n-k}$.
On cherche \`a montrer que $\sum_n c_n$ est absolument convergente et que
\[ \sum_{n=0}^{+\infty} c_n = \Pa{ \sum_{n=0}^{+\infty} a_n } \Pa{ \sum_{n=0}^{+\infty} b_n }. \]

\section{Cas positif}

On commence par traiter le cas o\`u $(a_n)$ et $(b_n)$ sont \`a valeurs positives.
Introduisons quelques notations:
\begin{itemize}
\item
  $A_n = \sum_{k=0}^n a_k$, $B_n = \sum_{k=0}^n b_k$ et $C_n = \sum_{k=0}^n c_k$;
\item
  $A = \sum_{n=0}^{+\infty} a_n$ et $B = \sum_{n=0}^{+\infty} b_n$;
\item
  $\Delta_n = \Ensemble{(i,j)\in \N^2}{i+j\leq n}$.
\item
  $\square_n = \Ensemble{(i,j)\in \N^2}{i\leq n \text{ et } j\leq n}$
\end{itemize}

On v\'erifie que, pour tout entier $n\in \N$, on a:
\begin{gather}
  \Delta_n \subset{} \square_n \subset{} \Delta_{2n} \label{eq:subset} \\
  A_n B_n = \sum_{(i,j)\in\square_n}{a_i b_j} \label{eq:square} \\
  C_n = \sum_{(i,j)\in \Delta_n}{a_i b_j} \label{eq:triangle}
\end{gather}

\begin{center}
  \begin{tikzpicture}
    \def\fig{
      \draw[->,thick,>=stealth] (0,0) -- (5.5,0) ;
      \draw[->,thick,>=stealth] (0,0) -- (0,5.5) ;
      \foreach \i in {0,...,5} {
        \foreach \j in {0,...,5} {
          \filldraw (\i, \j) circle (1pt) ;
        }
      }
    }
    \begin{scope}
      \fill[gray!50] (0,0) -- (4.05,0) -- (0,4.05);
      \fig
      \node[above] at (2,0) {$\Delta_4$} ;
    \end{scope}
    \begin{scope}[xshift=8cm]
      \fill[gray!50] (0,0) rectangle (4.05,4.05) ;
      \fig
      \node[above] at (2,0) {$\square_4$} ;
    \end{scope}
  \end{tikzpicture}
\end{center}

Si $(i,j)\in \Delta_n$, alors $i+j\leq n$, donc $i\leq i+j\leq n$ et de m\^eme $j\leq i+j\leq n$, donc $(i,j)\in\square_n$.
De plus, si $(i,j)\in\square_n$, alors $i+j\leq n+n=2n$, donc $(i,j)\in \Delta_{2n}$.
Cela prouve~(\ref{eq:subset}).

La relation~(\ref{eq:square}) vient du fait que
\[ A_n B_n = \Pa{ \sum_{i=0}^n a_i } \Pa{ \sum_{j=0}^n b_j } = \sum_{0\leq i,j\leq n} a_i b_j. \]

Enfin, commme \( c_k = \sum_{\substack{(i,j)\in \N^2 \\ i+j=k}} a_i b_j \),
on a la relation~(\ref{eq:triangle}) car
\[ C_n = \sum_{k=0}^n c_k = \sum_{k\leq n} \, \sum_{i+j=k} a_i b_j
  = \sum_{i+j\leq n} a_i b_j
= \sum_{(i,j)\in \Delta_n} a_i b_j. \]

Pour $n\in \N$, notons $p=\floor{n/2}$.
De~(\ref{eq:subset}) on d\'eduit que $\square_p \subset{} \Delta_n \subset{} \square_n$, et il vient
\[ \sum_{(i,j)\in\square_p} a_i b_j \leq{} \sum_{(i,j)\in \Delta_n} a_i b_j \leq{} \sum_{(i,j)\in\square_n} a_i b_j \]
car les $a_i b_j \geq{} 0$.
Autrement dit, on a $A_p B_p \leq{} C_n \leq{} A_n B_n$.
En faisant tendre $n\to+\infty$, le th\'eor\`eme des gendarmes assure que $C_n \to AB$, ce qui est exactement le r\'esultat souhait\'e.
L'absolue convergence de $\sum{} c_n$ est imm\'ediate car $c_n\geq0$ pour tout $n\in \N$.

\section{Cas g\'en\'eral}

On garde les m\^emes notations, en ajoutant:
\begin{itemize}
\item
  $\tilde a_n = \Abs{a_n}$, $\tilde b_n = \Abs{b_n}$
  et $\tilde c_n = \sum_{k=0}^n \tilde a_k \tilde b_{n-k}$.
  Notons que $\tilde c_n \neq{} \Abs{c_n}$ en g\'en\'eral;
\item
  $\tilde A_n = \sum_{k=0}^n \tilde a_k$ et de m\^eme pour $\tilde B_n$ et $\tilde C_n$.
\end{itemize}

Comme les s\'eries $\sum{} \Abs{a_n}$ et $\sum{} \Abs{b_n}$ sont absolument convergentes et positives,
le cas positif montre que la s\'erie de terme g\'en\'eral
$\tilde c_n$ est \'egalement absolument convergente.
Or d'apr\`es l'in\'egalit\'e triangulaire, $\Abs{c_n} \leq{} \tilde c_n$,
donc la s\'erie $\sum{} c_n$ est \'egalement absolument convergente.
De plus,
\[ A_n B_n - C_n
  = \sum_{(i,j)\in\square_n} a_i b_j - \sum_{(i,j)\in \Delta_n} a_i b_j
= \sum_{(i,j)\in\square_n\setminus \Delta_n} a_i b_j \]
donc
\[ \Abs{A_n B_n - C_n}
  \leq{} \sum_{(i,j)\in\square_n\setminus \Delta_n} \Abs{a_i} \cdot{} \Abs{b_j}
  = \sum_{(i,j)\in\square_n} \Abs{a_i} \cdot{} \Abs{b_j} - \sum_{(i,j)\in \Delta_n} \Abs{a_i} \cdot{} \Abs{b_j}
= \tilde A_n \tilde B_n - \tilde C_n \]

Or d'apr\`es le cas positif, $\tilde A_n \tilde B_n - \tilde C_n \to 0$,
d'o\`u $A_n B_n - C_n \to 0$, ce qui est exactement le r\'esultat cherch\'e.

CQFD

\end{document}
