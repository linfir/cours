% autogenerated by ytex.rs

\documentclass{scrartcl}

\usepackage[francais]{babel}
\usepackage{geometry}
\usepackage{scrpage2}
\usepackage{lastpage}
\usepackage{ragged2e}
\usepackage{multicol}
\usepackage{etoolbox}
\usepackage{xparse}
\usepackage{enumitem}
\usepackage{csquotes}
\usepackage{amsmath}
\usepackage{amsfonts}
\usepackage{amssymb}
\usepackage{mathrsfs}
\usepackage{stmaryrd}
\usepackage{dsfont}
\usepackage{eurosym}
\usepackage{numprint}
\usepackage[most]{tcolorbox}
\usepackage{tikz}
\usepackage{tkz-tab}
\usepackage[unicode]{hyperref}
\usepackage[ocgcolorlinks]{ocgx2}

\let\ifTwoColumns\iftrue
\def\Classe{$\Psi$2019--2020}

% reproducible builds
% LuaTeX: \pdfvariable suppressoptionalinfo 1023 \relax
\pdfinfoomitdate=1
\pdftrailerid{}

\newif\ifDisplaystyle
\everymath\expandafter{\the\everymath\ifDisplaystyle\displaystyle\fi}
\newcommand\DS{\displaystyle}

\clearscrheadfoot
\pagestyle{scrheadings}
\thispagestyle{empty}
\ohead{\Classe}
\ihead{\thepage/\pageref*{LastPage}}

\setlist[itemize,1]{label=\textbullet}
\setlist[itemize,2]{label=\textbullet}

\ifTwoColumns
  \geometry{margin=1cm,top=2cm,bottom=3cm,foot=1cm}
  \setlist[enumerate]{leftmargin=*}
  \setlist[itemize]{leftmargin=*}
\else
  \geometry{margin=3cm}
\fi

\makeatletter
\let\@author=\relax
\let\@date=\relax
\renewcommand\maketitle{%
    \begin{center}%
        {\sffamily\huge\bfseries\@title}%
        \ifx\@author\relax\else\par\medskip{\itshape\Large\@author}\fi
        \ifx\@date\relax\else\par\bigskip{\large\@date}\fi
    \end{center}\bigskip
    \ifTwoColumns
        \par\begin{multicols*}{2}%
        \AtEndDocument{\end{multicols*}}%
        \setlength{\columnsep}{5mm}
    \fi
}
\makeatother

\newcounter{ParaNum}
\NewDocumentCommand\Para{smo}{%
  \IfBooleanF{#1}{\refstepcounter{ParaNum}}%
  \paragraph{\IfBooleanF{#1}{{\tiny\arabic{ParaNum}~}}#2\IfNoValueF{#3}{ (#3)}}}

\newcommand\I{i}
\newcommand\mi{i}
\def\me{e}

\def\do#1{\expandafter\undef\csname #1\endcsname}
\docsvlist{Ker,sec,csc,cot,sinh,cosh,tanh,coth,th}
\undef\do

\DeclareMathOperator\ch{ch}
\DeclareMathOperator\sh{sh}
\DeclareMathOperator\th{th}
\DeclareMathOperator\coth{coth}
\DeclareMathOperator\cotan{cotan}
\DeclareMathOperator\argch{argch}
\DeclareMathOperator\argsh{argsh}
\DeclareMathOperator\argth{argth}

\let\epsilon=\varepsilon
\let\phi=\varphi
\let\leq=\leqslant
\let\geq=\geqslant
\let\subsetneq=\varsubsetneq
\let\emptyset=\varnothing

\newcommand{\+}{,\;}

\undef\C
\newcommand\ninf{{n\infty}}
\newcommand\N{\mathbb{N}}
\newcommand\Z{\mathbb{Z}}
\newcommand\Q{\mathbb{Q}}
\newcommand\R{\mathbb{R}}
\newcommand\C{\mathbb{C}}
\newcommand\K{\mathbb{K}}
\newcommand\Ns{\N^*}
\newcommand\Zs{\Z^*}
\newcommand\Qs{\Q^*}
\newcommand\Rs{\R^*}
\newcommand\Cs{\C^*}
\newcommand\Ks{\K^*}
\newcommand\Rp{\R^+}
\newcommand\Rps{\R^+_*}
\newcommand\Rms{\R^-_*}
\newcommand{\Rpinf}{\Rp\cup\Acco{+\infty}}

\undef\B
\newcommand\B{\mathscr{B}}

\undef\P
\DeclareMathOperator\P{\mathbb{P}}
\DeclareMathOperator\E{\mathbb{E}}
\DeclareMathOperator\Var{\mathbb{V}}

\DeclareMathOperator*\PetitO{o}
\DeclareMathOperator*\GrandO{O}
\DeclareMathOperator*\Sim{\sim}
\DeclareMathOperator\Tr{tr}
\DeclareMathOperator\Ima{Im}
\DeclareMathOperator\Ker{Ker}
\DeclareMathOperator\Sp{Sp}
\DeclareMathOperator\Diag{diag}
\DeclareMathOperator\Rang{rang}
\DeclareMathOperator*\Coords{Coords}
\DeclareMathOperator*\Mat{Mat}
\DeclareMathOperator\Pass{Pass}
\DeclareMathOperator\Com{Com}
\DeclareMathOperator\Card{Card}
\DeclareMathOperator\Racines{Racines}
\DeclareMathOperator\Vect{Vect}
\DeclareMathOperator\Id{Id}

\newcommand\DerPart[2]{\frac{\partial #1}{\partial #2}}

\def\T#1{{#1}^T}

\def\pa#1{({#1})}
\def\Pa#1{\left({#1}\right)}
\def\bigPa#1{\bigl({#1}\bigr)}
\def\BigPa#1{\Bigl({#1}\Bigr)}
\def\biggPa#1{\biggl({#1}\biggr)}
\def\BiggPa#1{\Biggl({#1}\Biggr)}

\def\pafrac#1#2{\pa{\frac{#1}{#2}}}
\def\Pafrac#1#2{\Pa{\frac{#1}{#2}}}
\def\bigPafrac#1#2{\bigPa{\frac{#1}{#2}}}
\def\BigPafrac#1#2{\BigPa{\frac{#1}{#2}}}
\def\biggPafrac#1#2{\biggPa{\frac{#1}{#2}}}
\def\BiggPafrac#1#2{\BiggPa{\frac{#1}{#2}}}

\def\cro#1{[{#1}]}
\def\Cro#1{\left[{#1}\right]}
\def\bigCro#1{\bigl[{#1}\bigr]}
\def\BigCro#1{\Bigl[{#1}\Bigr]}
\def\biggCro#1{\biggl[{#1}\biggr]}
\def\BiggCro#1{\Biggl[{#1}\Biggr]}

\def\abs#1{\mathopen|{#1}\mathclose|}
\def\Abs#1{\left|{#1}\right|}
\def\bigAbs#1{\bigl|{#1}\bigr|}
\def\BigAbs#1{\Bigl|{#1}\Bigr|}
\def\biggAbs#1{\biggl|{#1}\biggr|}
\def\BiggAbs#1{\Biggl|{#1}\Biggr|}

\def\acco#1{\{{#1}\}}
\def\Acco#1{\left\{{#1}\right\}}
\def\bigAcco#1{\bigl\{{#1}\bigr\}}
\def\BigAcco#1{\Bigl\{{#1}\Bigr\}}
\def\biggAcco#1{\biggl\{{#1}\biggr\}}
\def\BiggAcco#1{\Biggl\{{#1}\Biggr\}}

\def\ccro#1{\llbracket{#1}\rrbracket}
\def\Dcro#1{\llbracket{#1}\rrbracket}

\def\floor#1{\lfloor#1\rfloor}
\def\Floor#1{\left\lfloor{#1}\right\rfloor}

\def\sEnsemble#1#2{\mathopen\{#1\mid#2\mathclose\}}
\def\bigEnsemble#1#2{\bigl\{#1\bigm|#2\bigr\}}
\def\BigEnsemble#1#2{\Bigl\{#1\Bigm|#2\Bigr\}}
\def\biggEnsemble#1#2{\biggl\{#1\biggm|#2\biggr\}}
\def\BiggEnsemble#1#2{\Biggl\{#1\Biggm|#2\Biggr\}}
\let\Ensemble=\bigEnsemble

\newcommand\IntO[1]{\left]#1\right[}
\newcommand\IntF[1]{\left[#1\right]}
\newcommand\IntOF[1]{\left]#1\right]}
\newcommand\IntFO[1]{\left[#1\right[}

\newcommand\intO[1]{\mathopen]#1\mathclose[}
\newcommand\intF[1]{\mathopen[#1\mathclose]}
\newcommand\intOF[1]{\mathopen]#1\mathclose]}
\newcommand\intFO[1]{\mathopen[#1\mathclose[}

\newcommand\Fn[3]{#1\colon#2\to#3}
\newcommand\CC[1]{\mathscr{C}^{#1}}
\newcommand\D{\mathop{}\!\mathrm{d}}

\newcommand\longto{\longrightarrow}

\undef\M
\newcommand\M[3]{\mathrm{#1}_{#2}\pa{#3}}
\newcommand\MnR{\M{M}{n}{\R}}
\newcommand\MnC{\M{M}{n}{\C}}
\newcommand\MnK{\M{M}{n}{\K}}
\newcommand\GLnR{\M{GL}{n}{\R}}
\newcommand\GLnC{\M{GL}{n}{\C}}
\newcommand\GLnK{\M{GL}{n}{\K}}
\newcommand\DnR{\M{D}{n}{\R}}
\newcommand\DnC{\M{D}{n}{\C}}
\newcommand\DnK{\M{D}{n}{\K}}
\newcommand\SnR{\M{S}{n}{\R}}
\newcommand\AnR{\M{A}{n}{\R}}
\newcommand\OnR{\M{O}{n}{\R}}
\newcommand\SnRp{\mathrm{S}_n^+(\R)}
\newcommand\SnRpp{\mathrm{S}_n^{++}(\R)}

\newcommand\LE{\mathscr{L}(E)}
\newcommand\GLE{\mathscr{GL}(E)}
\newcommand\SE{\mathscr{S}(E)}
\renewcommand\OE{\mathscr{O}(E)}

\newcommand\ImplD{$\Cro\Rightarrow$}
\newcommand\ImplR{$\Cro\Leftarrow$}
\newcommand\InclD{$\Cro\subset$}
\newcommand\InclR{$\Cro\supset$}
\newcommand\notInclD{$\Cro{\not\subset}$}
\newcommand\notInclR{$\Cro{\not\supset}$}

\newcommand\To[1]{\xrightarrow[#1]{}}
\newcommand\Toninf{\To{\ninf}}

\newcommand\Norm[1]{\|#1\|}
\newcommand\Norme{{\Norm{\cdot}}}

\newcommand\Int[1]{\mathring{#1}}
\newcommand\Adh[1]{\overline{#1}}

\newcommand\Uplet[2]{{#1},\dots,{#2}}
\newcommand\nUplet[3]{(\Uplet{{#1}_{#2}}{{#1}_{#3}})}

\newcommand\Fonction[5]{{#1}\left|\begin{aligned}{#2}&\;\longto\;{#3}\\{#4}&\;\longmapsto\;{#5}\end{aligned}\right.}

\DeclareMathOperator\orth{\bot}
\newcommand\Orth[1]{{#1}^\bot}
\newcommand\PS[2]{\langle#1,#2\rangle}

\newcommand{\Tribu}{\mathscr{T}}
\newcommand{\Part}{\mathcal{P}}
\newcommand{\Pro}{\bigPa{\Omega,\Tribu}}
\newcommand{\Prob}{\bigPa{\Omega,\Tribu,\P}}

\newcommand\DEMO{$\spadesuit$}
\newcommand\DUR{$\spadesuit$}

\newenvironment{psmallmatrix}{\left(\begin{smallmatrix}}{\end{smallmatrix}\right)}


% -----------------------------------------------------------------------------


\newcommand{\Par}{\mathcal{P}}
\newcommand{\BW}{Bolzano-Weierstrass}

\begin{document}
\title{\'Equivalence des normes en dimension finie}
\maketitle

On veut montrer le r\'esultat suivant:
\begin{tcolorbox}
  Soit $\K$ le corps $\R$ ou $\C$.
  Si $E$ est un $\K$-espace vectoriel de dimension finie, alors toutes les normes sur $E$ sont \'equivalentes.
\end{tcolorbox}

\section{Pr\'eliminaires}

\Para{\BW{} sur $\R$}

Toute suite r\'eelle born\'ee admet une sous-suite convergente.

\Para{\BW{} sur $\C$}

Toute suite r\'eelle born\'ee admet une sous-suite convergente.

\emph{D\'emonstration:}
si $(z_n)$ est une suite born\'ee, notons $z_n = x_n + i y_n$ o\`u $(x_n, y_n) \in{} \R^2$.
Les suites $(x_n)$ et $(y_n)$ sont \'egalement born\'ees.
D'apr\`es \BW, il existe une extractice $\phi$ telle que la suite $(x_{\phi(n)})$ converge.
On consid\`ere alors la suite de terme g\'en\'eral $y'_n = y_{\phi(n)}$.
Elle est born\'ee comme sous-suite d'une suite born\'ee, donc on peut \`a nouveau utiliser \BW:
il existe une extractrice $\psi$ telle que la suite $y'_{\psi(n)}$ converge.
Notons $\sigma{} = \phi{} \circ{} \psi$.
La suite $(x_{\sigma(n)})$ converge car c'est une sous-suite de $(x_{\phi(n)})$
et la suite $(y_{\sigma(n)})$ converge par construction,
donc la suite $(z_{\sigma(n)})$ converge.

\Para{\BW{} sur $\K^p$}

Soit $(u_n^1)_{n\in \N}$, $(u_n^2)_{n\in \N}$, $\dots$, $(u_n^p)_{n\in \N}$ des suites num\'eriques born\'ees.
Alors il existe une extractrice $\sigma$ telle que
pour tout $i\in\ccro{1,p}$, la suite $(u_{\sigma(n)}^i)_{n\in \N}$ converge.

\emph{D\'emonstration:}
On fait une r\'ecurrence sur $p$; l'h\'er\'edit\'e se d\'emontre avec la m\^eme technique que ci-dessus.

\section{Preuve}

Soit $\B = \nUplet e1p$ une base de $E$, fix\'ee dans toute la preuve.
Pour tout $x\in \E$, on note $(x^1, \dots, x^p)$ les coordonn\'ees de $x$ dans la base $\B$, c.-\`a-d.
\[ x = \sum_{i=1}^p x^i e_i. \]
On note $\DS N(x) = \max_{1\leq i\leq p} \Abs{x^i}$.
On v\'erifie sans peine que $N$ est bien une norme sur $E$.

Soit $\Norme$ une norme quelconque sur $E$; on va chercher \`a montrer que $N$ et $\Norme$ sont \'equivalentes.

Pour tout $x\in E$, on a
\begin{align*}
  \Norm{x} &= \left\| \sum_{i=1}^p x^i e_i \right\|
  \leq{} \sum_{i=1}^p \Norm{x^i e_i}
  = \sum_{i=1}^p \Abs{x^i} \Norm{e_i} \\
  &\leq{} \sum_{i=1}^p N(x) \Norm{e_i} = \beta N(x)
\end{align*}
o\`u l'on pose $\beta{} = \sum_{i=1}^p \Norm{e_i} > 0$.
On pose \'egalement $S = \Ensemble{x\in E}{N(x) = 1}$ et $\DS \alpha{} = \inf_{x\in S} \Norm{x}$.

Si $x\in E$, $x\neq0$, on pose $y=x/N(x)$ de sorte que $y\in S$.
Par d\'efinition de $\alpha$, on a donc $\Norm{y} \geq{} \alpha$
et donc par homog\'en\'eit\'e de la norme $\Norme$, on a $\Norm{x} \geq{} \alpha N(x)$.
Cette derni\`ere in\'egalit\'e est encore vraie si $x=0$.
Ainsi, on a montr\'e que
\[ \forall x \in E \+ \alpha N(x) \leq{} \Norm{x} \leq{} \beta N(x). \]

Il ne reste plus qu'\`a prouver que $\alpha{} > 0$.
Supposons donc par l'absurde que $\alpha{} = 0$.

Pour tout $n\in \N$, il existe par d\'efinition de $\alpha$ un vecteur $u_n$ de $S$ tel que $\Norm{u_n} \leq{} \frac{1}{n+1}$.
D'apr\`es \BW{} sur $\K^p$, il existe une sous-suite $(v_n)$ de $(u_n)$ telle que
les suites $(v_n^i)_{n\in \N}$ convergent.
Notons $\ell{} = \sum_{i=1}^p \ell^i e_i$ o\`u $\ell^i = \lim_\ninf v_n^i$.

On a $\Norm\ell{} = \Norm{\ell{} - v_n + v_n} \leq{} \Norm{\ell{} - v_n} + \Norm{v_n} \leq{} \beta N(v_n - \ell) + \Norm{v_n}$.
Or pour tout $i\in\ccro{1,p}$, on a $\Abs{v_n^i - \ell^i} \Toninf 0$ et donc $N(v_n - \ell) \Toninf 0$.
De plus, la suite $(\Norm{v_n})$ \'etant une sous-suite de $(\Norm{u_n})$, on a $\Norm{v_n} \Toninf 0$.
Ainsi, $\Norm\ell{} \Toninf 0$ donc $\Norm\ell{} = 0$ donc $\ell{} = 0$.

Pour tout $i\in\ccro{1,p}$, on a $v_n^i \to \ell^i = 0$, donc $\exists n_i\in \N$, $\forall n\geq n_i$, $\Abs{v_n^i} \leq{} 1/2$.
Soit $m = \max(n_1,\dots,n_p)$ de sorte que $N(v_m) \leq{} 1/2$. Ce qui est absurde car $v_m \in S$.

Ainsi $\alpha{} > 0$ donc les normes $N$ et $\Norme$ sont bien \'equivalentes.
Bref, toute norme sur $E$ est \'equivalente \`a $N$,
donc toutes les normes sont \'equivalentes.

\end{document}
