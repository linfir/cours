% Time-stamp: <2017-12-07 11:48:24 yann>
\documentclass{yann}

\newcommand\Corrige[1]{\setcounter{ExoNum}{#1}\addtocounter{ExoNum}{-1}\relax\Exercice\relax}

\begin{document}
\title{Variables aléatoires discrètes: corrigés}
\maketitle

% -----------------------------------------------------------------------------
\Corrige{9}

Soit $k∈\ccro{0,20}$ fixé.

\begin{enumerate}
\item
  On peut modéliser l'expérience par $Ω = \ccro{0,20}^n$
  où $ω=\nUplet x1n$ signifie que le $i$-ème élève a obtenu la note $x_i$.
  Chaque événement élémentaire est équiprobable donc $ℙ$ est la probabilité uniforme.

  L'événement \enquote{toutes les notes sont inférieures ou égales à $k$}
  est égal à $(X_n ≤ k)$
  et aussi à $\ccro{0,k}^n$.
  Ainsi,
  \[ ℙ(X_n≤k) = \frac{\Card(\ccro{0,k}^n)}{\Card(\ccro{0,20}^n)} = \frac{(k+1)^n}{21^n}. \]
  Remarquons que cette formule est encore valable pour $k=-1$, ce qui sera utile à la question suivante.
\item
  \begin{enumerate}
  \item
    \begin{align*}
      ℙ(X_n<k \mid X_n≤k)
      &= \frac{ℙ(X_n<k, X_n≤k)}{ℙ(X_n≤k)} \\
      &= \frac{ℙ(X_n≤k-1)}{ℙ(X_n≤k)} \\
      &= \frac{k^n}{(k+1)^n}.
    \end{align*}
    La dernière égalité provenant de la question précédente.
  \item
    Notons $ℚ$ la probabilité sachant $(X_n≤k)$.
    On a
    \begin{align*}
      ℚ(X_n=k)
      &= 1 - ℚ(X_n≠k)
      = 1 - ℚ(X_n<k) \\
      &= 1 - \frac{k^n}{(k+1)^n}.
    \end{align*}
  \end{enumerate}
\item
  L'événement $(X_n ≤ k)$ est l'union disjointe des événements $(X_n = k)$ et $(X_n ≤ k-1)$.
  Ainsi
  \begin{align*}
    ℙ(X_n=k)
    &= ℙ(X_n≤k) - ℙ(X_n≤k-1) \\
    &= \frac{(k+1)^n-k^n}{21^n}.
  \end{align*}
  Si $k∈\ccro{0,19}$, le numérateur est majoré en valeur absolue par $20^n$, donc $ℙ(X_n=k) \to 0$ quand $n\to+∞$.
  Si $k=20$, $ℙ(X_n=k) \to 1$.

  L'interprétation est immédiate: si $n$ est très grand, il est très probable que $X_n$ vaille~20,
  puisqu'il suffit qu'un seul élève aie~20.
\end{enumerate}

% -----------------------------------------------------------------------------
\Corrige{24}

\begin{enumerate}
\item
  \begin{enumerate}
  \item
    On prend $a = f'(μ)$ et $b = f(μ)-aμ$.
    Soit $g(x) = f(x) - ax - b$.
    Comme $f$ est convexe, $f'$ est croissante donc $g'$ l'est aussi.
    Or $g'$ s'annule en $μ$ (on a choisi $a$ exactement pour cette raison!),
    donc en faisant un tableau de variations, on voit que $g$ admet un minimum en $μ$.
    Or $g(μ)=0$ (on a choisi $b$ exactement pour cette raison!),
    donc $g$ est positive sur $I$. CQFD.

  \item
    Traitons le cas où $I$ est de la forme $\intFO{a,b}$ où $a$ et $b$ sont deux réels tels que $a<b$.
    Comme $X≥a$, on a par croissance de l'espérance, $𝔼(X) ≥ 𝔼(a) = a$.
    De plus, on a $X<b$ donc a fortiori $X≤b$ et donc de même $𝔼(X) ≤ b$.
    Supposons par l'absurde $𝔼(X) = b$.
    Alors la variable aléatoire $Y = b-X$ est positive et d'espérance nulle.
    Elle est donc nulle presque sûrement, £cad. $X = b$ presque sûrement, ce qui est absurde car $X<b$.
    Ainsi $𝔼(X)∈\intFO{a,b} = I$.

    Les autre cas où $I$ est de la forme
    $\intF{a,b}$,
    $\intO{a,b}$,
    $\intOF{a,b}$,
    $\intO{-∞,b}$,
    $\intOF{-∞,b}$.
    $\intO{a,+∞}$,
    $\intFO{a,+∞}$,
    $ℝ$
    se traitent de la même façon.

  \item
    On applique la première question à $f$ et $μ = 𝔼(X)$.
    On a bien $μ∈I$ d'après la seconde question.
    On a $f(X) ≥ aX + b$ d'où en prenant l'espérance
    $𝔼(f(X)) ≥ a𝔼(X)+b = aμ+b = f(μ) = f(𝔼(X))$.
    CQFD.
  \end{enumerate}

\item
  \emph{Indication:}
  Notons $g(x) = \frac{f(x)-f(μ)}{x-μ}$ définie sur $I∖\acco{μ}$.
  La convexité de $f$ entraîne la croissance de $g$.
  On prend $a$ tel que $\lim_{μ^-} g ≤ a ≤ \lim_{μ^+} g$
  et $b = f(μ) - aμ$.
  Il reste à vérifier que cela marche.
\end{enumerate}

% -----------------------------------------------------------------------------
\Corrige{25}

Comme toutes les variables aléatoires $(X_i)$ ont la même loi,
on notera $G_X = G_{X_i}$.

\begin{enumerate}
\item
  Pour tout entier $n∈ℕ$, posons $T_n = X_1 + \dots + X_n$.
  On a par indépendance, $G_{T_n} = \pa{G_X}^n$.
  Ainsi,
  \begin{align*}
    G_N(G_X(t))
    &= ∑_{n=0}^{+∞} ℙ(N=n) G_X(t)^n \\
    &= ∑_{n=0}^{+∞} ℙ(N=n) G_{T_n}(t) \\
    &= ∑_{n=0}^{+∞} ℙ(N=n) \Pa{ ∑_{p=0}^{+∞} ℙ(T_n=p) t^p } \\
    &= ∑_{n=0}^{+∞} ∑_{p=0}^{+∞} ℙ(N=n) ℙ(T_n=p) t^p
  \end{align*}
  Remarquons qu'il n'y a pas de problème de convergence,
  puisque $G_X$ et $G_N$ convergent au moins sur $\intF{-1,1}$
  et que pour $\abs{t}≤1$,
  on a $\Abs{G_X(t)} = \Abs{𝔼(t^X)} ≤ 𝔼(\abs{t}^X) ≤ 𝔼(1) = 1$.

  En admettant que l'on peut permuter les sommes si $\abs{t}<1$, il vient
  \begin{align*}
    G_N(G_X(t))
    &= ∑_{p=0}^{+∞} ∑_{n=0}^{+∞} ℙ(N=n) ℙ(T_n=p) t^p \\
    &= ∑_{p=0}^{+∞} \underbrace{\Pa{ ∑_{n=0}^{+∞} ℙ(N=n) ℙ(T_n=p) }}_{α_p} t^p
  \end{align*}

  Or $ℙ(N=n)ℙ(T_n=p) = ℙ((N=n) ∩ (T_n = p))$
  car $T_n$ et $N$ sont indépendants.
  De plus, l'événement $(N=n)∩(T_n=p)$ est égal à $(N=n)∩(S=p)$ par définition de $S$.
  Ainsi
  \[ α_p = ∑_{n=0}^{+∞} ℙ((N=n)∩(S=p)) = ℙ(S=p) \]
  d'après la formule des probabilités totales avec le système complet d'événements $(N=n)_{n∈ℕ}$.

  Finalement, on obtient bien
  \[ G_N(G_X(t)) = ∑_{p=0}^{+∞} ℙ(S=p) t^p = G_S(t). \]

\item
  $X_1$ et $N$ étant d'espérances finies, les fonctions $G_N$ et $G_X$ sont dérivables sur $\intF{-1,1}$.
  De plus, comme on l'a vu précédemment, $\intF{-1,1}$ est stable par $G_X$,
  donc $G_S = G_N ◦ G_X$ est également dérivable sur $\intF{-1,1}$.
  Ainsi, $S$ est d'espérance finie et
  \[ 𝔼(S) = G_S'(1) = G_X'(1) G_N'(\underbrace{G_X(1)}_{=1}) = 𝔼(X_1) 𝔼(N). \]

\end{enumerate}

\end{document}
