% Time-stamp: <2018-06-07 08:54>

\documentclass{yann}
\usepackage{tikz}
\usepackage{tkz-tab}

\newcommand{\gammaI}{\Fn{γ}{I}}
\newcommand{\gammaIE}{\gammaI\EE}
\newcommand{\deltaJ}{\Fn{δ}{J}}
\newcommand{\deltaJE}{\deltaJ\EE}
\newcommand{\EE}{\mathscr{E}}
\renewcommand\Vec{\overrightarrow}
\newcommand\AccoDeux[2]{\begin{cases} #1 \\ #2 \end{cases}}
\newcommand\AccoTrois[3]{\begin{cases} #1 \\ #2 \\ #3 \end{cases}}

\tikzset{
  Axe/.style = {},
  Vecs/.style = { thick, <->, >=stealth },
  Vec/.style = { thick, ->, >=stealth },
  Curve/.style = { very thick },
  Curve name/.style = {},
  gamma p pos/.style = { below },
  gamma q pos/.style = { left },
}

\begin{document}
\title{Courbes paramétrées}
\maketitle

\Para{Notations}
\begin{itemize}
\item $I$ et $J$ désignent des intervalles non vides et non réduits à un point de $ℝ$;
\item $\EE$ désigne un espace euclidien de dimension~2.
\end{itemize}

% -----------------------------------------------------------------------------
\section{Généralités}

\Para{Définition}
On appelle \emph{arc paramétré} de $\EE$ toute application $\gammaIE$ de classe $\CC1$.
On parle parfois de \emph{courbe paramétrée} ou d'\emph{arc orienté} à la place d'arc paramétré.

\Para{Définition}[Interprétation cinématique]
Un \emph{mouvement ponctuel} est un arc paramétré $γ$ dans lequel la variable $t$ est le temps.
\begin{itemize}
\item $γ(t)$ est la \emph{position} du point mobile à l'instant~$t$;
\item $γ'(t)$ est la \emph{vitesse} du point mobile à l'instant~$t$;
\item $γ''(t)$ est l'\emph{accélération} du point mobile à l'instant~$t$;
\item $Γ=γ(I)$ s'appelle la \emph{trajectoire} du point mobile.
\end{itemize}

\begin{center}
  \begin{tikzpicture}
    \draw[Curve] (-1.5,-1) .. controls +(60:1) and +(-25:-1) .. (1,1) .. controls +(-25:1) and +(0,0) .. (2,-1) ;
    \draw[Vec] (1,1) -- +(-25:2) ;
    \filldraw (1,1) circle (1pt) node [below,xshift=-5pt] { $γ(t)$ } ;
    \draw (2.25,0.8) node { $γ'(t)$ } ;
    \draw[Curve name] (-1.5,-0.5) node { $Γ$ } ;
  \end{tikzpicture}
\end{center}

% -----------------------------------------------------------------------------
\section{Symétries}

\Para{Proposition}
Soit $\Fn{γ}{I}{\EE}$ une courbe paramétrée.
Soit $\Fn{θ}{I}{I}$ et $\Fn{s}{\EE}{\EE}$ deux applications.
Si $γ◦θ=s◦γ$, alors la trajectoire $Γ$ de $γ$ est stable par~$s$.

\Para{Corollaire}
En particulier, si $I$ est symétrique par rapport à zéro et si
\[ γ(-t) = s\bigPa{γ(t)}, \]
Alors la trajectoire de $γ$ est stable par $s$; on peut alors réduire l'étude à $I∩ℝ^+$.

\Para{Exemples de symétries}
Soit $\Fn{γ}{ℝ}{ℝ^2}$ un arc paramétré.
On note $γ(t) = \bigPa{x(t),y(t)}$.
\begin{enumerate}
\item Si $x(-t) = x(t)$ et $y(-t) = y(t)$,
  alors il suffit d'étudier $γ$ sur $ℝ^+$.
\item Si $x(-t) = x(t)$ et $y(-t) = -y(t)$,
  alors la trajectoire de $γ$ est symétrique
  par rapport à l'axe des abscisses et
  il suffit d'étudier $γ$ sur $ℝ^+$.
\item Si $x(θ(t)) = -y(t)$ et $y(θ(t)) = x(t)$,
  alors la trajectoire de $γ$ est invariante
  par rotation d'angle $\fracπ2$.
\item Si
  \[ \left\{ \begin{aligned}
      x(t+17) &= -\frac12 x(t) - \frac{√3}{2} y(t) \\
      y(t+17) &= \frac{√3}2 x(t) - \frac1{2} y(t)
  \end{aligned} \right. \]
  alors la trajectoire de $γ$ est stable
  par la rotation d'angle $2π/3$ et il suffit d'étudier
  $γ$ sur $\intF{0,17}$.
\item etc.
\end{enumerate}

% -----------------------------------------------------------------------------
\section{Variations}

L'outil principal pour le tracé de l'allure de $γ$ est le tableau regroupant les variations de $x$ et de $y$.
Par exemple, pour la strophoïde droite (cf. exercice~1):
\begin{center}
  \begin{tikzpicture}
    \tkzTabInit
    {$t$/1, $x'$/1, $x$/1.5, $y$/1.5, $y'$/1}
    {$0$, $√{√{5}-2}$, $+∞$}
    \tkzTabLine{z,,-}
    \tkzTabVar{+/$1$,R,-/$-1$}
    \tkzTabVar{-/$0$,+/ ,-/$-∞$}
    \tkzTabLine{$1$, +, z, -}
  \end{tikzpicture}
\end{center}

% -----------------------------------------------------------------------------
\section{Étude locale}

% -----------------------------------------------------------------------------
\subsection{Tangentes}

\Para{Définition}[demi-tangente]
Soit $\gammaIE$ un arc paramétré.
Pour $t ∈I$, on note $M(t) = γ(t)$.
Soit $t_0 ∈I$. On note $A = A(t_0) = M(t_0)$.
On dit que \emph{$γ$ admet une demi-tangente en $A(t_0^+)$} £ssi. la limite
\[
  \vec u = \lim_{t \to t_0^+} \frac{\Vec{AM(t)}}{\Norm{AM(t)}}
  = \lim_{t \to t_0^+} \frac{1}{\Norm{γ(t)-γ(t_0)}} \BigPa{ γ(t) - γ(t_0) }
\]
existe.
Dans ce cas, la \emph{demi-tangente à $γ$ en $A(t_0^+)$} est par définition la droite passant par $A$ de vecteur directeur $\vec u ≠0$.

On définit de façon analogue la demi-tangente en $A(t_0^-)$.

\Para{Définition}[tangente]
Avec les mêmes notations que précédemment, si deux limites sont \emph{égales ou opposées}, alors les demi-tangentes sont égales et l'on dit que $γ$ admet une tangente en $A(t_0)$.

% -----------------------------------------------------------------------------
\subsection{Au voisinage d'un point régulier}

\Para{Définitions}
Soit $\gammaIE$ un arc paramétré.
\begin{itemize}
\item On dit que $t_0 ∈I$ est un \emph{point régulier} de $γ$ £ssi. $γ'(t_0)≠0$.
  On dit que $t_0$ est un \emph{point stationnaire} dans le cas contraire.
\item On dit que $γ$ est un \emph{arc régulier} £ssi. tous les points $t ∈I$ sont des points réguliers de $γ$.
\end{itemize}

\Para{Proposition}
Soit $\gammaIE$ un arc paramétré et $t_0 ∈I$ un point régulier de $γ$.
Alors $γ$ admet une tangente en $t_0$;
il s'agit de la droite passant par $γ(t_0)$ et de vecteur directeur $γ'(t_0)$.

% -----------------------------------------------------------------------------
\subsection{Au voisinage d'un point stationnaire}

Dans ce cas, l'allure locale est plus complexe.
La proposition suivante permet d'étudier l'allure au voisinage d'un point quelconque.

\Para{Proposition}
Soit $\Fn{γ}{I}{\EE}$ un arc paramétré de classe $\CC k$ et $t∈I$.
On note
\begin{itemize}
\item $p$ le plus petit entier tel que
  \[ \left\{ \begin{array}{l} 1≤p<k, \\ γ^{(p)}(t)≠0; \end{array} \right. \]
\item $q$ le plus petit entier tel que
  \[ \left\{ \begin{array}{l} p<q≤k, \\ \text{la famille $\bigPa{ γ^{(p)}(t), γ^{(q)}(t)}$ est libre}.  \end{array} \right. \]
\item $\vec u = γ^{(p)(t)}$ et $\vec v = γ^{(q)}(t)$.
\end{itemize}
On suppose que $p$ et $q$ existent.

La famille $(\vec u,\vec v)$ est alors une base du plan $\EE$, et d'après Taylor-Young, quand $h\to0$, on a
\[ γ(t+h) = γ(t) + \frac{h^p}{p!} \bigPa{1+o(1)} \, \vec u + \frac{h^q}{q!} \bigPa{1+o(1)} \, \vec v. \]

Cette formule montre que l'allure locale de $γ$ au voisinage de $t$ dépend essentiellement de la parité de $p$ et~$q$.

% <noindent>
\newenvironment{myfig}[2]{
  \begin{center}
    \begin{tikzpicture}[scale=1.5]
      %\draw[step=0.25,gray!30,very thin] (-1,-1) grid (2.25,2);
      \useasboundingbox (-1,-1) -- (2.25,2);
      \draw[Axe] (-1,0) -- (2,0);
      \draw (60:-1) -- (60:2);
      \draw[Vecs] (60:1) -- (0,0) -- (1,0);

      \draw (1,0) node[gamma p pos] {$\vec u$};
      \draw (60:1) node[gamma q pos] {$\vec v$};
      \draw (0,1.5) node[align=center] {$p$ #1\\$q$ #2};
}{\end{tikzpicture}\end{center}}
% </noindent>

\begin{enumerate}
\item Point à allure normale:
  \begin{myfig}{impair}{pair}
    \draw[Curve] (2,0.5) .. controls +(-1,-0.5) and +(0.5,0) .. (0,0) .. controls +(-1,0) and +(0,0) .. (-1,1);
    \draw[Curve name] (2,0.5) node[above] {$Γ$};
  \end{myfig}

\item Point d'inflexion:
  \begin{myfig}{impair}{impair}
    \draw[Curve] (2,0.5) .. controls +(-1,-0.5) and +(0.5,0) .. (0,0) .. controls +(-1,0) and +(60:0.1) .. (-1,-0.5);
    \draw[Curve name] (2,0.5) node[above] {$Γ$};
  \end{myfig}

  \pagebreak[3]
\item Point de rebroussement de première espèce:
  \begingroup
    \tikzset{gamma p pos/.style = {below left=1ex}}
    \begin{myfig}{pair}{impair}
      \draw[Curve] (2,0.5) .. controls +(-1,-0.5) and +(0.5,0) .. (0,0) .. controls +(1,0) and +(-0.5,0.5) .. (2,-0.75);
      \draw[Curve name] (2,0.5) node[above] {$Γ$};
    \end{myfig}
  \endgroup

\item Point de rebroussement de seconde espèce:
  \begin{myfig}{pair}{pair}
    \draw[Curve] (2,0.5) .. controls +(-1,-0.5) and +(0.5,0) .. (0,0) .. controls +(1,0) and +(0,0) .. (2,1);
    \draw[Curve name] (2,1) node[above] {$Γ$};
  \end{myfig}
\end{enumerate}


% -----------------------------------------------------------------------------
\section{Branches infinies}

\Para{Définition}
Soit $\gammaI{\EE}$ un arc paramétré.
On dit que $γ$ admet une \emph{branche infinie} lorsque $t$ tend vers $t_0$ £ssi.
\[ \Norm{γ(t)} \To{t\to t_0} +∞. \]

\Para{Définition}
Soit $\gammaI{\EE}$ un arc paramétré admettant une \emph{branche infinie} lorsque $t$ tend vers $t_0$ £ssi.
On dit que $γ$ admet une \emph{direction asymptotique}
lorsque $t$ tend vers $t_0$ £ssi. la limite
\[ \vec u = \lim_{t \to t_0} \frac{γ(t)}{\Norm{γ(t)}} \]
existe.
On dit que $\vec u$ est la \emph{direction asymptotique} de $γ$ lorsque $t$ tend vers $t_0$.

\Para{Définition}[asymtote et branche parabolique]
Avec les mêmes notations, supposons que $γ$ admette $\vec u$ comme direction asymptotique lorsque $t$ tend vers $t_0$.
Soit $\vec v$ un vecteur tel que $\B=(\vec u, \vec v)$ soit libre.
Notons $\bigPa{X(t), Y(t)}$ les coordonnées de $γ(t)$ dans la base $\B$.

\begin{itemize}
\item Si \[ \left\{ \begin{array}{l} X(t) \To{t\to t_0} ±∞\\ Y(t) \To{t\to t_0} b∈ℝ, \end{array} \right. \]
  on dit que la droite d'équation $Y=b$ est \emph{asymptote} à $γ$ lorsque $t$ tend vers $t_0$.
\item Dans le cas contraire, on dit que $γ$ admet une \emph{branche parabolique} de direction $\vec u$ lorsque $t$ tend vers $t_0$.
\end{itemize}

\Para{Remarque}
Ceci est indépendant du choix du vecteur $\vec v$ non colinéaire à $\vec u$.

% -----------------------------------------------------------------------------
\section{Changement de paramétrage}

Cette partie n'est pas clairement au programme.

\Para{Définition}
Soit $\Fn{ϕ}{I}{J}$ une application où $I$ et $J$ sont
des intervalles de $ℝ$.
On dit que $ϕ$ est un \emph{$\CC k$-difféomorphisme} ($k ≥1$)
lorsque les conditions suivantes sont vérifiées:
\begin{enumerate}
\item $ϕ$ est bijective;
\item $ϕ$ est de classe $\CC k$ sur~$I$;
\item $ϕ^{-1}$ est de classe $\CC k$ sur~$J$.
\end{enumerate}

\Para{Théorème}[caractérisation des difféomorphismes]
Soit $\Fn{ϕ}{I}{J}$ une application où $I$ et $J$ sont
des intervalles de $ℝ$.
Alors $ϕ$ est un $\CC k$-difféomorphisme ($k ≥1$) \ssi:
\begin{enumerate}
\item $ϕ$ est surjective;
\item $ϕ$ est de classe $\CC k$ sur $I$;
\item $∀t ∈I \+ ϕ'(t) ≠0$.
\end{enumerate}

\Para{Définition}
Soit $\gammaIE$ un arc paramétré de classe $\CC k$, $k ≥1$.
On appelle \emph{changement de paramétrage} (de classe $\CC k$) de $γ$ tout $\CC k$-difféomorphisme $\Fn{ϕ}{J}{I}$ où $J$ est un intervalle de $I$.

\Para{Définition}
On dit que deux arcs paramétrés $\gammaIE$ et $\deltaJE$ de classe $\CC k$ ($k ≥1$) sont \emph{équivalents} £ssil. existe un changement de paramétrage (de classe $\CC k$) $ϕ$ de $γ$ tel que $δ=γ◦ϕ$.

\Para{Exemple}
Soit $\EE = ℝ^2$. On définit les arcs $γ$ et $δ$ par
\[ \Fonction{γ} {ℝ} {ℝ^2} {t} {\Pa{\frac{1-t^2}{1+t^2},\frac{2t}{1+t^2}},} \]
\[ \Fonction{δ} {\intO{-π,π}} {ℝ^2} {θ} {\pa{\cosθ,\sinθ}.} \]
Alors les arcs $γ$ et $δ$ sont équivalents via le changement de paramétrage $t = \tan(θ/2)$.

\Para{Proposition}
Il s'agit d'une relation d'équivalence, \cad
\begin{itemize}
\item tout arc paramétré est équivalent à lui-même;
\item si $γ$ est équivalent à $δ$, alors $δ$ est équivalent à $γ$;
\item deux arcs paramétrés équivalents à un même troisième sont équivalents.
\end{itemize}

\Para{Définition}
Soit $\gammaIE$ un arc paramétré de classe $\CC k$, $k ≥1$.
On appelle \emph{paramétrage admissible} de $γ$ toute arc paramétré $\deltaJE$ équivalent à $γ$.

\Para{Définition}
Soit $\gammaIE$ et $\deltaJE$ deux arcs paramétrés équivalents et $\Fn{ϕ}{J}{I}$ un difféomorphisme tel que $δ=γ◦ϕ$.
Comme $ϕ$ est une application injective entre deux intervalles de $ℝ$, elle est soit strictement croissante, soit strictement décroissante.
\begin{itemize}
\item Si $ϕ$ est croissante, on dit que $γ$ et $δ$ sont \emph{orientés dans le même sens}.
\item Si $ϕ$ est décroissante, on dit que $γ$ et $δ$ sont \emph{orientés dans le sens contraire}.
\end{itemize}

% -----------------------------------------------------------------------------
\subsection{Longueur d'une courbe paramétrée}

\Para{Définition}
Soit $\Fn{γ}{\intF{a,b}}{\EE}$ un arc paramétré régulier.
La \emph{longueur} de $γ$ est par définition
\[ L = ∫_a^b \Norm{γ'(t)} \D t. \]
Si $\B$ est une £bon. de $\EE$ et si
\[ \Coords_\B γ(t) = \bigPa{ x(t), y(t) }, \]
alors
\[ L = ∫_a^b √{x'(t)^2 + y'(t)^2} \D t. \]

% -----------------------------------------------------------------------------
\section{Exercices}

\subsection{Études de courbes paramétrées}

\Exercice
Tracer les courbes suivantes.
On fera l'étude la plus complète possible des courbes suivantes:
les points réguliers, doubles, les asymptotes, les points d'inflexion;
et pour les plus motivés: la courbure, la longueur de la courbe, l'aire des boucles, etc.
\begin{enumerate}
\item Astroïde:
  $\AccoDeux{x = a\cos^3(t)}{y=a\sin^3(t)}$
\item Strophoïde droite:
  $\AccoDeux{x = \frac{1-t^2}{1+t^2}}{y = t\frac{1-t^2}{1+t^2}}$
\item Lemniscate de Benoulli:
  $\AccoDeux{x = \frac{t}{1+t^4}}{y = \frac{t^3}{1+t^4}}$
\item Folium De Descartes:
  $\AccoDeux{x = \frac{t}{1+t^3}}{y = \frac{t^2}{1+t^3}}$
\item Deltoïde:
  $\AccoDeux{x=2\cos t+\cos2t}{y=2\sin t-\sin2t}$
\item Cœur de Rémi:
  $\AccoDeux{x=\sin^3 t}{y=\cos t - \cos^4 t}$
\end{enumerate}

% -----------------------------------------------------------------------------
\subsection{Autres exercices}

\Exercice
Soit $\mathscr{P}$ la parabole d'équation $y^2 = 2px$.
Trouver la longueur minimale d'une corde de $\mathscr{P}$ normale à $\mathscr{P}$ en l'une de ses extrémités.

\Exercice
Reconnaître la transformation
\[ \left\{ \begin{alignedat}{5}
    x' &{}={}&  3x &{}+{}& 4y &{}+{}& 2z &{}-{}& 4 \\
    y' &{}={}& -2x &{}-{}& 3y &{}-{}& 2z &{}+{}& 4 \\
    z' &{}={}&  4x &{}+{}& 8y &{}+{}& 5z &{}-{}& 8
\end{alignedat} \right. \]


\Exercice
Montrer que les droites
\[ \mathcal{D}  \AccoDeux{x = 2z + 1}{y = z - 1}
  \qquad \text{et} \qquad
\mathcal{D}' \AccoDeux{x = z + 2}{y = 3z - 3} \]
sont coplanaires, et déterminer leur plan.


\Exercice
Soit $\Fn{f}{ℝ^2}{ℝ^2}$ une fonction telle que pour tous points $A$ et $B$ du plan $ℝ^2$,
la famille $(\Vec{AB}, \Vec{f(A)f(B)})$ soit liée.

Montrer que $f$ est soit une homothétie, soit une translation, soit une application constante.

\Exercice
Soit $A$, $B$, $C$ et $D$ quatre points de l'espace, et $λ$, $μ$ deux réels.
On définit les points $M$, $N$, $P$ et $Q$ par les relations:
$\Vec{AM} = λ\Vec{AB}$,
$\Vec{DN} = λ\Vec{DC}$,
$\Vec{AP} = μ\Vec{AD}$ et
$\Vec{BQ} = μ\Vec{BC}$.
Montrer que les droites $(MN)$ et $(PQ)$ sont coplanaires.


\Exercice[distance d'un point à une droite]
Soit $\mathcal{D}$ la droite du plan d'équation $ax+by+c=0$ et $M_0$ le point de coordonnées $(x_0,y_0)$.
Montrer que la distance de $M_0$ à $\mathcal{D}$ est donnée par
\[ d(M_0,\mathcal{D}) = \frac{\Abs{ax_0+by_0+c}}{√{a^2+b^2}}. \]

Même question dans l'espace avec le plan $\mathcal{P}$ d'équation $ax+by+cz+d = 0$
et le point $M_0(x_0,y_0,z_0)$.


\Exercice
Soit $\mathcal{D}$ et $\mathcal{D}'$ deux droites de l'espace non parallèles.
Montrer qu'il existe une unique droite $Δ$ qui soit perpendiculaire $\mathcal{D}$ et à $\mathcal{D}'$;
$Δ$ s'appelle la \emph{perpendiculaire commune} à $\mathcal{D}$ et $\mathcal{D}'$.

La calculer pour $\mathcal{D}  \AccoDeux{x=a}{y=b}$ et $\mathcal{D}' \AccoDeux{x+cy+z=0}{cx-y+z=0}$.


\Exercice
Former un système d'équations cartésiennes de la symétrique $\mathcal{D}'$
de la droite $\mathcal{D} \AccoDeux{x = 3z+1}{y = -z+2}$
par rapport à la droite $Δ\AccoDeux{x = y}{y = z}$


\Exercice
Déterminer l'image du cercle unité de $ℂ$ par l'application $f$ définie par $f(z) = 1/(1-z)^2$.

\end{document}
