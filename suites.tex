\documentclass{yann}

\undef\U \undef\V
\newcommand\I{i}
\newcommand\U{(u_n)_{n\in\N}}
\newcommand\V{(v_n)_{n\in\N}}

\begin{document}
\title{Suites numériques}
\maketitle

% -----------------------------------------------------------------------------
\section{Généralités}

\Para{Définitions}
\begin{itemize}
\item
Une \emph{suite} à valeurs dans un ensemble $X$ est une application $\Fn{u}{\N}{X}$.
  On note fréquemment $u_n$ au lieu de $u(n)$ et $\U$ au lieu de $u$.
\item
On dit que $u_n$ est le \emph{terme général} de la suite $\U$.
\item
Une \emph{suite numérique} est une suite à valeurs dans $\R$ ou $\C$.
\item
Une \emph{suite réelle} est une suite à valeurs dans $\R$.
\item
Une suite réelle $\U$ est dite \emph{croissante} si et seulement si
  $\forall n\in\N$, $u_n\leq u_{n+1}$.
\item
Une suite réelle $\U$ est dite \emph{décroissante} si et seulement si
  $\forall n\in\N$, $u_n\geq u_{n+1}$.
\item
Une suite numérique $\U$ est dite \emph{stationnaire} £ssil. existe
  $c \in \K$ et $n_0 \in \N$ tels que $\forall n \geq n_0$, $u_n = c$.
\end{itemize}

\Para{Définitions}
\begin{itemize}
\item
Soit $\U$ une suite numérique et $\ell\in\K$.
  On dit que la suite $\U$ \emph{converge} vers $\ell$ si et seulement si
  \[ \forall\epsilon> 0 \+\exists N\in\N\+\forall n\geq N\+
  \Abs{u_n -\ell}\leq\epsilon. \]
  $\ell$ est nécessairement unique et s'appelle \emph{la limite} de la suite $\U$.
  On écrit $\lim\limits_{\ninf} u_n = \ell$ ou encore $u_n \Toninf\ell$.
\item
Une suite numérique $\U$ est dite \emph{convergente} si et seulement s'il
  existe $\ell\in\K$ tel que $\U$ converge vers $\ell$.
  Elle est dite \emph{divergente} dans le cas contraire.
\end{itemize}

\Para{Définitions}

Soit $\U$ une suite réelle.
\begin{itemize}
\item
On dit que \emph{$u_n \Toninf +\infty$} si et seulement si
  \[ \forall M\in\R\+\exists N\in\N\+\forall n\geq N\+ u_n\geq M. \]
\item
On dit que \emph{$u_n \Toninf -\infty$} si et seulement si
  \[ \forall M\in\R\+\exists N\in\N\+\forall n\geq N\+ u_n\leq M. \]
\end{itemize}

\Para{Remarque}

Soit $\U$ une suite réelle telle que $u_n \Toninf\pm\infty$.
Alors $\U$ est une suite \emph{divergente}.

\Para{Définitions}
\begin{itemize}
\item
Une suite réelle $\U$ est dite \emph{majorée} si et seulement si
  \[ \exists A\in\R\+\forall n\in\N\+ u_n\leq A. \]
\item
Une suite réelle $\U$ est dite \emph{minorée} si et seulement si
  \[ \exists B\in\R\+\forall n\in\N\+ u_n\geq B. \]
\item
Une suite numérique $\U$ est dite \emph{bornée} si et seulement si
  \[ \exists M\in\R\+\forall n\in\N\+ \Abs{u_n}\leq M. \]
\end{itemize}

\Para{Définitions}
\begin{itemize}
\item
Une \emph{extractrice} est une fonction $\N\to\N$ strictement croissante.
\item
Une \emph{sous-suite} ou \emph{suite extraite} de la suite $\U$
  est une suite de la forme $(u_{\sigma(n)})_{n\in\N}$ où $\sigma$
  est une extractrice.
\end{itemize}

\Para{Proposition}

Une sous-suite d'une suite convergente est également convergente.

Plus précisément:
Soit $\U$ une suite numérique et $\ell\in\K$ tels que $u_n \Toninf\ell$.
Soit $\sigma$ une extractice.
Alors $u_{\sigma(n)} \Toninf\ell$.

\subsection{Théorèmes}

\Para{Théorème}

Toute suite réelle croissante majorée converge.

Autrement dit:
Soit $\U$ une suite réelle croissante et majorée.
Alors $\U$ est convergente.

\Para{Corollaire}

Toute suite réelle décroissante minorée converge.

\Para{Proposition}

Toute suite réelle croissante non majorée tend vers $+\infty$.

\Para{Théorème des gendarmes}

Soit $(u_n)_{n\in\N}$, $(v_n)_{n\in\N}$ et $(w_n)_{n\in\N}$ trois suites réelles et $\ell\in\R$.
Si:
\begin{itemize}
\item
$u_n \Toninf\ell$;
\item
$w_n \Toninf\ell$;
\item
$\exists N\in\N \+ \forall n\geq N \+ u_n\leq v_n\leq w_n$.
\end{itemize}

Alors $(v_n)_{n\in\N}$ est une suite convergente et $v_n \Toninf\ell$.

\Para{Théorème}

Soit $\U$ une suite réelle.
Soit $I$ un intervalle \emph{ouvert} de $\R$ et $\Fn fI\R$.
Si:
\begin{itemize}
\item
$u_n \Toninf\ell$;
\item
$\ell\in I$;
\item
$f$ continue en $\ell$.
\end{itemize}

Alors la suite $\bigl( f(u_n) \bigr)_n$ est définie à partir d'un certain rang et
$f(u_n) \Toninf f(\ell)$.

\Para{Théorème de Bolzano-Weierstraß}[hors-programme]

Toute suite réelle bornée admet une sous-suite convergente.

Autrement dit:

Soit $\U$ une suite réelle \emph{bornée}.
Alors il existe une extractrice $\sigma$
et un réel $\ell$ tels que $u_{\sigma(n)} \Toninf\ell$.

\Para{Définition}

Soit $\U$ et $\V$ deux suites réelles.
On dit qu'elles sont \emph{adjacentes} si et seulement si:
\begin{itemize}
\item
$\U$ est croissante;
\item
$\V$ est décroissante;
\item
$v_n - u_n \Toninf 0$.
\end{itemize}

\Para{Théorème}

Soit $\U$ et $\V$ deux suites adjacentes.
Alors elles convergent toutes les deux vers la même limite $\ell\in\R$.
De plus:
\begin{multline*}
u_0\leq\cdots\leq u_n\leq u_{n+1}\leq\cdots\leq\ell \\
\ell\leq\cdots\leq v_{n+1}\leq v_n\leq\cdots\leq v_0
\end{multline*}

\Para{Théorème de Cesàro}

Soit $\U$ une suite numérique telle que $u_n \Toninf\ell\in\K$.
On pose $m_n = \frac{1}{n}\sum_{k=0}^{n-1} u_k$;
il s'agit de la moyenne arithmétique des $n$ premiers termes de la suite $\U$.

Alors $m_n \Toninf\ell$.

\Para{Remarque}

Si $\U$ est une suite réelle qui tend vers $\pm\infty$, le résultat est encore vrai.

\subsection{Comparaison}

\Para{Définitions}

Soit $\U$ et $\V$ deux suites numériques.
\begin{itemize}
\item
On dit que \emph{$u_n = \GrandO_\ninf(v_n)$} si et seulement s'il existe
  une suite numérique $(e_n)_{n\in\N}$ \emph{bornée} telle que
  $\exists N\in\N,\forall n\geq N$, $u_n = e_n v_n$.
\item
On dit que \emph{$u_n = \PetitO_\ninf(v_n)$} si et seulement s'il existe
  une suite numérique $(e_n)_{n\in\N}$ \emph{qui tend vers 0} telle que
  $\exists N\in\N,\forall n\geq N$, $u_n = e_n v_n$.
\item
On dit que \emph{$u_n \Sim_\ninf v_n$} si et seulement s'il existe
  une suite numérique $(e_n)_{n\in\N}$ \emph{qui tend vers 1} telle que
  $\exists N\in\N,\forall n\geq N$, $u_n = e_n v_n$.
\end{itemize}

\Para{Proposition}

Soit $\U$ et $\V$ deux suites numériques.
Alors $u_n \Sim_\ninf v_n$ si et seulement si $u_n = v_n + \PetitO_\ninf(v_n)$.

\Para{Proposition}

Soit $\U$ et $\V$ deux suites numériques.
On suppose de plus que:
\[ \exists N\in\N\+\forall n\geq N\+ v_n\neq0. \]
Alors:
\begin{itemize}
\item
$u_n = \GrandO_\ninf(v_n)$ si et seulement si la suite $\left(\frac{u_n}{v_n}\right)_{n\geq N}$ est bornée.
\item
$u_n = \PetitO_\ninf(v_n)$ si et seulement si $\frac{u_n}{v_n} \Toninf 0$.
\item
$u_n \Sim_\ninf v_n$ si et seulement si $\frac{u_n}{v_n} \Toninf 1$.
\end{itemize}

% -----------------------------------------------------------------------------
\section{Suites récurrentes du premier ordre}

\Para{Définition}

Soit $\U$ une suite à valeurs dans un ensemble $X$
et $\Fn fXX$.

On dit que $\U$ est une \emph{suite récurrente (du premier ordre)}
associée à la fonction $f$
si et seulement si \[ \forall n\in\N, u_{n+1} = f(u_n). \]

\Para{Proposition}

Soit $\U$ une suite récurrente associée à $f$.
Si:
\begin{itemize}
\item
$u_n \Toninf\ell$;
\item
$\ell\in X$;
\item
$f$ continue en $\ell$.
\end{itemize}

Alors $f(\ell) =\ell$.

\Para{Remarque}

Cela sert à trouver les limites \emph{potentielles} de $\U$.

\Para{Définition}

Soit $\Fn fXX$ et $Y\subset X$.
On dit que $Y$ est \emph{stable} par $f$ si et seulement si $f(Y)\subset Y$.

\Para{Proposition}

Soit $\U$ une suite récurrente réelle associée à $f$
et $I$ un intervalle stable par $f$ tel que $u_{n_0}\in I$.
Alors $\forall n\geq n_0$, $u_n\in I$.

\Para{Proposition}

Soit $\Fn{f}{I}{I}$ où $I$ est un intervalle de $\R$
et $\U$ une suite récurrente associée à $f$.
\begin{enumerate}
\item
Si $f$ est croissante sur $I$, alors $\U$ est monotone;
\item
Si $f$ est décroissante sur $I$, alors les suites extraites
  $(u_{2n})_{n\in\N}$ et $(u_{2n+1})_{n\in\N}$ sont monotones;
\item
Dans le cas général, on ne peut rien dire: $\U$ peut avoir un comportement très complexe.
\end{enumerate}

\Para{Exemples}
\begin{itemize}
\item
\emph{Suites arithmétiques:}
  $\forall n\in\N\+ u_{n+1} = u_n + b$.
  On a alors $\forall n\in\N\+ u_n = u_0 + nb$.
\item
\emph{Suites géométriques:}
  $\forall n\in\N\+ u_{n+1} = a u_n$.
  On a alors $\forall n\in\N\+ u_n = a^n u_0$.
\item
\emph{Suites arithmético-géométriques:}
  $\forall n\in\N\+ u_{n+1} = au_n + b$ où $a\neq1$.
  On pose $\ell$ tel que $\ell= a\ell+ b$, puis $v_n = u_n -\ell$.
  On vérifie alors que $\V$ est une suite géométrique.
\item
\emph{Récurrence homographique:}
  $\forall n\in\N\+ u_{n+1} = \frac{a u_n + b}{c u_n + d}$.
  Soit $A = \begin{pmatrix} a & b \\ c & d \end{pmatrix}$.
  On calcule $A^n = \begin{pmatrix} \alpha_n & \beta_n \\ \gamma_n & \delta_n \end{pmatrix}$
  et l'on a:
  \[ \forall n\in\N\+
  u_n = \frac{\alpha_n u_0 +\beta_n}{\gamma_n u_0 +\delta_n}. \]
\end{itemize}

\subsection{Suites récurrentes du second ordre}

\Para{Définition}

Soit $\U$ une suite à valeurs dans un ensemble $X$,
et $\Fn{f}{X^2}{X}$.
On dit que $\U$ est une \emph{suite récurrente du second ordre} associée à la fonction $f$
si et seulement si \[ \forall n\in\N, u_{n+2} = f(u_n, u_{n+1}). \]

\Para{Proposition}

Soit $\U$ une suite récurrente du second ordre associée à $f$.
Si:
\begin{itemize}
\item
$u_n \Toninf\ell$;
\item
$\ell\in X$;
\item
$f$ continue en $(\ell,\ell)$.
\end{itemize}

Alors $f(\ell,\ell) =\ell$.

\subsection{Cas des suites récurrentes linéaires du second ordre à coefficients constants sans second membre}

\Para{Proposition}

Soit $\U$ une suite numérique et $(a,b,c)\in\K^3$ avec
$a\neq0$ et $(b,c)\neq(0,0)$
tels que \[ \forall n\in\N\+ au_{n+2} + bu_{n+1} + c u_n = 0. \]

On forme l'\emph{équation caractéristique}:
\[ ar^2 + br + c = 0. \tag{EC} \]

Si \emph{$\K=\C$}, il y a deux cas:
\begin{enumerate}
\item
Soit $\Delta\neq0$:
  (EC) admet deux racines distinctes $r_1$ et $r_2$, et
  \[ \exists(A,B)\in\C^2\+\forall n\in\N\+ u_n = A r_1^n + B r_2^n \]
\item
Soit $\Delta= 0$:
  (EC) admet une racine double $r_0\neq0$, et
  \[ \exists(A,B)\in\C^2\+\forall n\in\N\+ u_n = (An + B) r_0^n \]
\end{enumerate}

Si \emph{$\K=\R$}, il y a trois cas:
\begin{enumerate}
\item
Soit $\Delta> 0$:
  (EC) admet deux racines distinctes $r_1$ et $r_2$, et
  \[ \exists(A,B)\in\R^2\+\forall n\in\N\+ u_n = A r_1^n + B r_2^n \]
\item
Soit $\Delta= 0$:
  (EC) admet une racine double $r_0\neq0$, et
  \[ \exists(A,B)\in\R^2\+\forall n\in\N\+ u_n = (An + B) r_0^n \]
\item
Soit $\Delta< 0$:
  (EC) admet deux racines complexes conjuguées $\rho e^{\pm\I\theta}$, et
  \[ \exists(A,B)\in\R^2\+\forall n\in\N\+ u_n =\rho^n \Big( A \cos(n\theta) + B \sin(n\theta) \Big) \]
\end{enumerate}

% -----------------------------------------------------------------------------
\section{Exercices}

\Exercice

Déterminer la limite, ou montrer la divergence des suites $\U$ définies par
\begin{enumerate}
  \item
$u_n = \frac{3^n-(-2)^n}{3^n+(-2)^n}$
  \item
$u_n = \sqrt{n^2+n+1} - \sqrt{n^2-n+1}$
  \item
$u_n = \frac{n-\sqrt{n^2+1}}{n+\sqrt{n^2+1}}$
  \item
$u_n = \frac{1}{n^2} \sum_{k=1}^n k$
  \item
$u_n = \bigPa{1+\frac1n}^n$
  \item
$u_n = \sqrt[n]{n^2}$
  \item
$u_n = \bigPa{\sin\frac1n}^{1/n}$
  \item
$u_n = \bigPa{\frac{n-1}{n+1}}^n$
  \item
$u_n = \frac{\sin n}{n+(-1)^{n+1}}$
  \item
$u_n = \frac{n!}{n^n}$
  \item
$u_n = \frac{n-(-1)^n}{n+(-1)^n}$
  \item
$u_n = \frac{e^n}{n^n}$
  \item
$u_n = \sqrt[n]{2+(-1)^n}$
  \item
$u_n = \sqrt[n]{n}$
  \item
$u_n = \bigPa{1+\frac xn}^n$
  \item
$u_n = \bigPa{\frac{n-1}{n+1}}^{n+2}$
  \item
$u_n = n^2 \bigPa{\cos\frac{1}{n} - \cos\frac{1}{n+1}}$
  \item
$u_n = \bigPa{\tan\pa{\frac\pi4 + \frac{\alpha}{n}}}^n$
  \item
$u_n = \bigPa{\frac{\ln(n+1)}{\ln(n)}}^{n\ln n}$
  \item
$u_n = \bigPa{\frac{\arctan(n+1)}{\arctan(n)}}^{n^2}$
  \item
$u_n = \cos\bigPa{\pi n^2 \ln(1-1/n)}$
\end{enumerate}

\Exercice

Montrer que l'ensemble des entiers naturels $n$ tels que $2^{n^2} < (4n)!$ est fini.

\Exercice

Soit $\U$ une suite réelle convergeant vers $\ell\in\R$.
Si $\floor\cdot$ désigne la fonction partie entière,
la suite $\bigPa{\floor{u_n}}_{n\in\N}$ est-elle convergente?

\Exercice

Trouver un exemple de suite qui diverge mais dont la moyenne de Cesàro converge.

\Exercice

Étudier la convergence des suites réelles $\U$ définies par:
\begin{enumerate}
\item
$u_0 = a > 0, u_{n+1} = \frac12\left(u_n + \frac{a}{u_n}\right)$
\item
$u_0 = 1$, $u_{n+1} = \frac{u_n}{u_n^2+1}$
\item
$u_0\geq0$, $u_{n+1} = \frac16(u_n^2+8)$
\item
$u_0 = 1$, $u_{n+1} = \frac{1}{2+u_n}$
\item
$u_{n+1} = \cos u_n$
\item
$0 < u_0 < \frac{\sqrt5-1}{2}, u_{n+1} = 1 - u_n^2$
\item
$u_{n+1} = u_n - u_n^2$
\item
$u_{n+1} = u_n + \frac{1+u_n}{1+2u_n}$
\item
$0\leq u_0\leq1$, $u_{n+1} = \frac{\sqrt{u_n}}{\sqrt{u_n}+\sqrt{1-u_n}}$
\item
$u_{n+1} =\sqrt{2-u_n}$
\end{enumerate}

\Exercice
\begin{enumerate}
\item
\begin{enumerate}
\item
Résoudre $u_0 = a$, $u_1 = b$ et $u_{n+2} = \frac{u_{n+1} + u_n}{2}$.
\item
Déterminer $\lim_\ninf u_n$.
\end{enumerate}
\item
\begin{enumerate}
\item
Résoudre $v_0 =\alpha> 0$, $v_1 =\beta> 0$ et $v_{n+2} =\sqrt{ v_{n+1} v_n }$
\item
Déterminer $\lim_\ninf v_n$.
\end{enumerate}
\end{enumerate}

\Exercice

Soit $\U$ et $\V$ deux suites réelles vérifiant
\[ \left\{ \begin{array}{l}
  0\leq u_n\leq1, \\
  0\leq u_n\leq1, \\
  u_n v_n \Toninf 1.
\end{array} \right. \]
Que peut-on dire de ces deux suites?

\Exercice

Pour $n\in\N$, on pose
$u_n =\sum_{k=0}^n \frac{1}{k!}$ et
$v_n = u_n + \frac{1}{n\cdot n!}$.
\begin{enumerate}
\item
Montrer que $\U$ et $\V$ sont deux suites adjacentes;
  on note $\ell$ leur limite commune.
\item
Déterminer un $n$ tel que $v_n - u_n\leq10^{-7}$.
\item
En déduire une valeur approchée de $\ell$ à $10^{-7}$ près.
\end{enumerate}

\Exercice

Trouver $\lim_\ninf\sqrt{1+\sqrt{1+ \cdots +\sqrt{1}}}$ (avec $n$ radicaux).

\Exercice

Étudier la convergence des suites $\U$ et $\V$ définies par
$0 < u_0 < v_0$,
$u_{n+1} = \frac{2u_n+v_n}{3}$ et
$v_{n+1} = \frac{2v_n+u_n}{3}$.

\Exercice

Soit $\U$ et $\V$ les suites réelles définies par
$u_0 = a > 0$,
$v_0 = b > 0$,
$u_{n+1} =\sqrt{u_n v_n}$ et
$v_{n+1} = \frac{u_n+v_n}{2}$.

Montrer que $\U$ et $\V$ sont des suites adjacentes.

\emph{Remarque:}
leur limite commune $M(a,b)$ s'appelle la \emph{moyenne arithmético-géométrique} de $a$ et $b$.
On peut prouver que:
\[ \frac{\pi}{M(a,b)} =\int_{-\infty}^{+\infty} \frac{\D x}{\sqrt{(x^2+a^2)(x^2+b^2)}} \]

\Exercice

Soit $\U$ et $\V$ deux suites réelles telles que $u_n^2 + u_n v_n + v_n^2 \to 0$.
Que dire de $\U$ et $\V$?

\Exercice

Soit $\U$ une suite à valeurs positives et $\lambda \in \intO{0,1}$.
\begin{enumerate}
  \item
On suppose que $u_{n+1} \leq \lambda u_n$ pour tout entier~$n$.
  Montrer que $u_n \to 0$.
  \item
On suppose qu'il existe une suite réelle $\V$ tendant vers~0 telle que $u_{n+1} \leq \lambda (u_n + v_n)$ pour tout entier~$n$.
  A-t-on $u_n \to 0$?
\end{enumerate}

\Exercice

Soit $\U$ une suite à valeurs strictement positives
telle que $\frac{u_{n+1}}{u_n} \to \ell$.
\begin{enumerate}
  \item
Si $\ell < 1$, montrer que $u_n \to 0$.
  \item
Si $\ell > 1$, montrer que $u_n \to +\infty$.
  \item
Montrer qu'on ne peut pas conclure dans le cas $\ell = 1$.
\end{enumerate}

\Exercice

Soit $\U$ une suite injective à valeurs dans $\N$.
Montrer que $u_n \to +\infty$.

\Exercice[lemme des petits pas]

Soit $\U$ une suite numérique telle que $u_{n+1} - u_n \to \ell$.
Montrer que $u_n = n \ell + o(n)$ quand $n \to +\infty$.

\end{document}
