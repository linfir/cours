\documentclass{yann}

\newcommand{\DerPart}[2]{\frac{∂#1}{∂#2}}
\newcommand{\Div}{\mathop{\mathrm{div}}}
\newcommand{\DF}[2]{\D{#1} ({#2})}
\newcommand{\DIF}[3]{\DF{#1}{#2}⋅{#3}}

\begin{document}
\title{Fonctions de plusieurs variables}
\maketitle

\Para{Contexte}

On s'intéresse à l'étude des fonctions qui dépendent de $p$ variables, c.-à-d. aux fonctions $\Fn{f}{U}{ℝ^n}$ où $U$ est un ouvert de $ℝ^p$.

\Para{Définition}

On dit que $f$ est une \emph{fonction numérique} si et seulement si $n=1$.

\Para{Notation}

En général, on notera
\[f(x) = \bigl(\Uplet{f_1(x)}{f_n(x)}\bigr) \quad \text{où} \quad x = \nUplet x1p,\]
de sorte que $\Fn{f_k}{U}{ℝ}$ pour $k∈\Dcro{1,n}$ est une fonction numérique.
Les $f_k$ s'appellent les \emph{composantes} de $f$.

% -----------------------------------------------------------------------------
\section{Applications continues (rappels)}

\Para{Définition}

Soit $\Fn{f}{U}{ℝ^n}$ où $U$ est un ouvert de $ℝ^p$.
Soit $a∈U$.
Les conditions suivantes sont équivalentes:
\begin{enumerate}
\item la fonction $f$ est continue en $a$,
\item $\lim\limits_{x \to a} f(x) = f(a)$,
\item $\lim\limits_{\Norm{x-a} \to 0} \Norm{f(x)-f(a)} = 0$,
\item $∀ε>0$, $∃η>0$, $\Norm{x-a}<η\implies \Norm{f(x)-f(a)}<ε$.
\end{enumerate}

\Para{Remarque}

Puisque les normes sont équivalentes en dimension finie,
la continuité de $f$ ne dépend pas du choix des normes sur $ℝ^p$ et sur $ℝ^n$.

% -----------------------------------------------------------------------------
\section{Applications de classe $\CC1$}

\subsection{Dérivées}

\Para{Définition}[dérivée selon un vecteur]

Soit $\Fn{f}{U}{ℝ^n}$ où $U$ est un ouvert de $ℝ^p$.
Soit $a∈U$ et $h∈ℝ^p$.
On pose $φ(t) = f(a+th)$.
Alors $φ$ est définie au voisinage de 0.

Si $φ$ est dérivable en $0$,
on dit que $f$ est \emph{dérivable en $a$ selon le vecteur $h$},
et l'on note
\[\DIF fah = φ'(0)
= \lim_{\substack{t \to 0 \\ t∈ℝ^*}} \left(\frac{f(a+th)-f(a)}{t}\right).\]
$\DIF fah$ s'appelle la \emph{dérivée de $f$ en $a$ selon le vecteur $h$}.

\Para{Notation}

On note $\nUplet e1p$ la base canonique de $ℝ^p$.

\Para{Définition}[dérivées partielles]

Soit $\Fn{f}{U}{ℝ^n}$ où $U$ est un ouvert de $ℝ^p$.
Soit $a∈U$.
On appelle \emph{$k$-ième dérivée partielle de $f$}
la dérivée de $f$ selon le vecteur $e_k$, si elle existe.
On la note alors $∂_k f(a)$ ou encore $\DerPart{f}{x_k}(a)$.
Ainsi,
\[\begin{aligned}
\DerPart{f}{x_k}(a) = \lim_{\substack{h\to 0 \\ h≠0}} \frac{1}{h} \Bigl[
  & f(a_1, \ldots, a_{k-1}, a_k + h, a_{k+1}, \ldots, a_p) \\
  & - f(a_1, \ldots, a_p) \Bigr].
\end{aligned}\]

\Para{Attention}

Une fonction peut admettre des dérivées partielles en tout point sans pour autant être continue.
cf. exercice 12.

% -----------------------------------------------------------------------------
\subsection{Applications de classe $\CC1$}

\Para{Définition}

Soit $\Fn{f}{U}{ℝ^n}$ où $U$ est un ouvert de $ℝ^p$.
On dit que \emph{$f$ est de classe $\CC1$ sur $U$}
si
\begin{itemize}
\item $f$ admet des dérivées partielles en tout point de $U$;
\item les dérivées partielles $∂_k f$ sont \emph{continues} sur $U$ pour tout $k∈\Dcro{1,p}$.
\end{itemize}

\Para{Théorème}

Soit $\Fn{f}{U}{ℝ^n}$ une fonction de classe $\CC1$ où $U$ est un ouvert de $ℝ^p$.
Alors $f$ est continue sur $U$,
et admet en tout point de $U$ des dérivées selon tout vecteur.
De plus, on a $∀a∈U$, $∀h = \nUplet h1p∈ℝ^p$,
\[\DIF fah = ∑_{k=1}^p h_k \DerPart{f}{x_k}(a).\]

\Para{Proposition}

Soit $\Fn{f}{U}{ℝ^n}$ une fonction de classe $\CC1$ où $U$ est un ouvert de $ℝ^p$.
Soit $a∈U$.
On a
\[f(a+h) = f(a) + \DIF fah + \PetitO(h) \quad \text{quand } h\to0.\]

\Para{Remarque}

Soit $U$ un ouvert de $ℝ^p$ contenant $0$ et $\Fn{f}{U}{ℝ^n}$.
Les conditions suivantes sont équivalentes:
\begin{enumerate}
\item $f(x) = \PetitO(x)$ quand $x \to 0$;
\item $\Norm{f(x)} = \PetitO\left(\Norm x\right)$ quand $x \to 0$;
\item $\lim\limits_{x\to\vec0} \left(\frac{\Norm{f(x)}}{\Norm x}\right) = 0$.
\end{enumerate}

% -----------------------------------------------------------------------------
\subsection{Différentielle}

\Para{Définition}

Soit $\Fn{f}{U}{ℝ^n}$ une fonction de classe $\CC1$ où $U$ est un ouvert de $ℝ^p$.
Soit $a∈U$.
L'application \[\Fonction{\DF fa}{ℝ^p}{ℝ^n}{h}{\DIF fah}\]
est une application \emph{linéaire},
appelée \emph{différentielle de $f$ en $a$}
et notée $\DF fa$.

\Para{Remarque}

Notons $\D{x_k}$ l'application coordonnée
\[\Fonction{\D{x_k}}{ℝ^p}{ℝ}{\nUplet x1p}{x_k.}\]
On a alors
\[\DF fa = ∑_{k=1}^p \DerPart{f}{x_k}(a) \D{x_k},\]
ou encore, en omettant le point $a$,
\[\D f = \DerPart{f}{x_1} \D{x_1} + \cdots + \DerPart{f}{x_p} \D{x_p}.\]

\Para{Proposition}[réduction à des fonctions numériques]

Soit $\Fn{f}{U}{ℝ^n}$ où $U$ est un ouvert de $ℝ^p$.
On note $f(x) = \left( \Uplet{f_1(x)}{f_n(x)} \right)$ où $f_k \colon U \toℝ$.
Alors
\begin{enumerate}
\item $f$ est continue sur $U$ si et seulement si les fonctions $\Uplet{f_1}{f_n}$ le sont.
\item $f$ est de classe $\CC1$ si et seulement si les fonctions $\Uplet{f_1}{f_n}$ le sont.
\item Plus généralement, avec les définitions que l'on verra plus loin, on a pour tout $k∈ℕ∪\Acco{∞}$:
  $f$ est de classe $\CC k$ si et seulement si les fonctions $\Uplet{f_1}{f_n}$ le sont.
\end{enumerate}

\subsection{Opérations sur les fonctions de classe $\CC1$}

\Para{Proposition}[somme et produit]

$U$ désigne un ouvert de $ℝ^p$.
\begin{enumerate}
\item Si $f \colon U \toℝ^n$ et $g \colon U \toℝ^n$ sont de classe $\CC1$, alors $f+g$ l'est.
  De plus, $\DerPart{(f+g)}{x_k} = \DerPart{f}{x_k} + \DerPart{g}{x_k}$.
\item Si $f \colon U \toℝ$ et $g \colon U \toℝ^n$ sont de classe $\CC1$, alors $fg$ l'est.
  De plus, $\DerPart{(fg)}{x_k} = \DerPart{f}{x_k} g + f \DerPart{g}{x_k}$.
\end{enumerate}

\Para{Théorème}[composition]

Soit $\Fn{f}{U}{ℝ^n}$ et $\Fn{g}{V}{ℝ^m}$ où $U$ est un ouvert de $ℝ^p$ et $V$ un ouvert de $ℝ^n$.
On suppose que $f$ et $g$ sont de classe $\CC1$ et que $f(U)⊂V$.
Alors $\Fn{g◦f}{U}{ℝ^m}$ est de classe $\CC1$.
De plus, pour tout $a∈U$,
\[\DF{(g◦f)}{a} = \Big[ \D{g}\bigl(f(a)\bigr) \Big] ◦\Big[ \DF fa \Big].\]
Autrement dit,
\[\DerPart{(g◦f)_i}{x_j}(a) = ∑_{k=1}^{n} \DerPart{g_i}{y_k}\bigl(f(a)\bigr) \DerPart{f_k}{x_j}(a).\]

\Para{Corollaire}

Soit $I$ un intervalle de $ℝ$, $U$ un ouvert de $ℝ^p$.
Soit $φ\colon I \to U$ et $\Fn{f}{U}{ℝ^n}$ deux applications de classe $\CC1$.
Alors $f◦φ\colon I \toℝ^n$ est de classe $\CC1$ et
\[\begin{aligned}
(f◦φ)'(t)
&= \DF{f}{φ(t)} (φ'(t)) \\
&= ∑_{k=1}^p \DerPart{f}{x_k} \bigl(φ_k(t)\bigr) \, φ_k'(t).
\end{aligned}\]

Cela peut s'interpréter géométriquement:
\begin{itemize}
\item $φ$ correspond à un arc paramétré,
\item $φ'$ au vecteur vitesse,
\item $f◦φ$ à l'image de l'arc $φ$ par l'application $f$.
\end{itemize}

On obtient donc le vecteur vitesse de l'image d'un arc paramétré.

\subsection{Dérivées d'ordre supérieur}

\Para{Définition}

Soit $\Fn{f}{U}{ℝ^n}$ une fonction de classe $\CC1$ où $U$ est un ouvert de $ℝ^p$.
Si les applications $∂_k f \colon U \toℝ^n$ sont elles-mêmes
de classe $\CC1$ pour tout $k∈\Dcro{1,p}$,
alors on dit que $f$ est de classe $\CC2$ et on note $∀a∈U$,
\[∂_{k,l} f (a)
= \frac{∂^2 f}{∂x_k \,∂x_l}(a)
=∂_k (∂_l f) (a)
= \frac{∂}{∂x_k}
\left(\frac{∂f}{∂x_l}\right) (a).\]
On définit de même par récurrence les fonctions de classe $\CC k$, $k≥2$,
comme étant les fonctions admettant en tout point de $U$
des dérivées partielles de classe $\CC{k-1}$.

\Para{Théorème}[Schwarz]

Soit $f \colon U \toℝ^n$ de classe $\CC2$ où $U$ est un ouvert de $ℝ^p$.
Alors $∀a∈U$, $∀(i,j)∈\Dcro{1,p}^2$,
\[\frac{∂^2 f}{∂x_i \,∂x_j} (a)
= \frac{∂^2 f}{∂x_j \,∂x_i} (a).\]

% -----------------------------------------------------------------------------
\section{Cas des fonctions numériques}

\subsection{Équations aux dérivées partielles}

cf. exercices.

\subsection{Un soupçon de topologie}

\Para{Remarque}

Si $D⊂ℝ^p$, on peut définir le \emph{bord} de $D$, noté $∂D$.
Lorsque $D$ est une partie \og{}gentille\fg{} de $ℝ^p$,
$∂D$ est exactement ce que son nom laisse supposer.

\Para{Remarque}

En général, la définition est un peu technique et hors-programme.
\begin{multline*}
  ∂D = \Bigl\{ x∈ℝ^p \,\Bigm|\, ∀r>0, \\
  B(x,r) ∩ D≠∅ \text{ et }
  B(x,r) ∩ {}^c \! D≠∅
  \Bigr\}
\end{multline*}
Autrement dit, $x∈∂D$ si et seulement si $x$ est à la fois limite
d'une suite d'éléments de $D$ et limite d'une autre suite
d'éléments du complémentaire de $D$.

\Para{Proposition}

Soit $D⊂ℝ^p$.
\begin{itemize}
\item $D$ ouvert si et seulement si $∂D⊂^c D$.
\item $D$ fermé si et seulement si $∂D⊂D$.
\item L'intérieur de $D$ est égal à $\mathring{D} = D∖∂D$;
  c'est un ouvert.
\item L'adhérence de $D$ est égale à $\overline{D} = D∪∂D$;
  c'est un fermé.
\end{itemize}

\Para{Théorème}

Une fonction numérique \emph{continue} sur une partie \emph{fermée bornée} de $ℝ^p$ est bornée et atteint ses bornes.

\subsection{Extremum}

\Para{Définitions}

Soit $\Fn{f}{D}{ℝ}$ où $D$ est une partie de $ℝ^p$.
Soit $a∈D$.
\begin{itemize}
\item $f$ a un \emph{minimum global} en $a$ si et seulement si \[∀x∈D\+ f(x)≥f(a).\]
\item $f$ a un \emph{maximum global} en $a$ si et seulement si \[∀x∈D\+ f(x)≤f(a).\]
\item $f$ a un \emph{extremum global} en $a$ si et seulement si $f$ a un minimum ou un maximum global en $a$.
\item $f$ a un \emph{minimum local} en $a$ si et seulement si \[∃r > 0\+ ∀x∈B(a,r)∩D\+ f(x)≥f(a).\]
\item $f$ a un \emph{maximum local} en $a$ si et seulement si \[∃r > 0\+ ∀x∈B(a,r)∩D\+ f(x)≤f(a).\]
\item $f$ a un \emph{extremum local} en $a$ si et seulement si $f$ a un minimum ou un maximum local en $a$.
\end{itemize}

\Para{Définition}[gradient]

Soit $\Fn{f}{U}{ℝ}$ de classe $\CC1$ où $U$ est un ouvert de $ℝ^p$.
On appelle \emph{gradient de $f$} l'application
\[\Fonction{∇f}{U}{ℝ^p}{x}{
  \left(\Uplet{\DerPart{f}{x_1}(x)}{\DerPart{f}{x_p}(x)}\right).}\]

\Para{Proposition}

Avec les mêmes notations, on a, $∀a∈U$,
\[\DIF fah = \PS{∇f(a)}{h}\]

\Para{Définition}[point critique]

Soit $\Fn{f}{U}{ℝ}$ de classe $\CC1$ où $U$ est un ouvert de $ℝ^p$.
Soit $a∈U$.
On dit que $a$ est un \emph{point critique} de $f$ si et seulement si $∇f(a) = 0$.

\Para{Théorème}

Soit $\Fn{f}{U}{ℝ}$ de classe $\CC1$ où $U$ est un ouvert de $ℝ^p$.
\begin{itemize}
\item Tout extremum global de $f$ est un extremum local de $f$.
\item Tout extremum local de $f$ est un point critique de $f$.
\item Les deux réciproques sont fausses, même dans le cas $p=1$.
\item Si $U$ n'est pas un ouvert, le premier point reste vrai,
  mais généralement pas le second.
\end{itemize}

\Para{Remarque}

Ce théorème est très pratique pour déterminer les extrema
des fonctions numériques de plusieurs variables.

Si $f$ est défini sur un domaine $D$ quelconque,
on ne peut appliquer ce résultat que sur l'intérieur $U=\mathring{D}$.
Il faut étudier les valeurs de $f$ sur le bord $∂D$ par d'autres méthodes,
généralement par une étude directe.

% -----------------------------------------------------------------------------
\section{Exercices}

\subsection{Échauffement}

\Exercice

Étudier la continuité des fonctions suivantes:
\begin{enumerate}
\item $f(x,y) = \frac{\sin x + \cos y}{x^2 + y^2 + 1}$
\item $f(x,y) = x^y$
\item $f(x,y) = \begin{cases}
  \frac{x \sin^2 y}{x^2+y^2} & \text{si } (x,y)≠(0,0) \\
  \hfil 0                    & \text{si } x=y=0.
\end{cases}$
\end{enumerate}

\Exercice

Soit $f \colonℝ^p \toℝ^n$ linéaire.
Montrer que $f$ est de classe $\CC1$ et que $∀a∈ℝ^p$, $\DF fa = f$.

\Exercice

Soit $\Fonction{f}{ℝ^2}{ℝ}{(x,y)}{\max(\Abs x, \Abs y)}$

Déterminer le plus grand ouvert sur lequel $f$ est $\CC1$.

\Exercice

Soit $\Fn{f}{ℝ^2}{ℝ}$ une fonction de classe $\CC1$.
On pose $u(x) = f(x,x)$, $v(x,y)=f(y,x)$ et $w(x,y) = f(x+y,x^2 \sin y)$.
Calculer les dérivées partielles de ces fonctions de plusieurs façons:
\begin{itemize}
\item avec la règle de la chaîne;
\item avec les différentielles.
\end{itemize}

\Exercice

Soit \[\Fonction {Φ} {]0,∞[×]-π,π[} {ℝ^2∖\Ensemble{(x,0)}{x≤0}} {(ρ,θ)} {(x,y) = (ρ\cosθ,ρ\sinθ).}\]
\begin{enumerate}
\item Soit $f \colonℝ^2 \toℝ$ et $g = f◦Φ$.
  Informellement, on a $f(x,y) = g(ρ,θ)$.

  Exprimer les dérivées partielles de $g$ en fonction de celles de $f$.
\item Calculer $\Div g$ sachant que $\Div f = \frac{∂f}{∂x} + \frac{∂f}{∂y}$.
\item Calculer $Δg$ sachant que $Δf = \frac{∂^2 f}{∂x^2} + \frac{∂^2 f}{∂y^2}$.
\item On pose $f(x,y) = h(x,θ)$.
  Comparer les dérivées partielles $\DerPart fx$ et $\DerPart hx$.
  Expliquer la différence!
\end{enumerate}

\subsection{Classiques}

\Exercice[intégrales à paramètre, avec bornes variables]

Soit $φ$ et $ψ$ deux fonctions $ℝ\toℝ$ de classe $\CC1$.
Soit $f \colonℝ^2 \toℝ$ de classe $\CC1$.
On pose
\[g(x) = ∫_{φ(x)}^{ψ(x)} f(x,t) \D t \quad\text{et}\quad
h(a,b,x) = ∫_a^b f(x,t) \D t.\]
\begin{enumerate}
\item Montrer que $h$ est de classe $\CC1$.
\item En déduire que $g$ est également de classe $\CC1$ et calculer $g'(x)$.
\end{enumerate}

\Exercice

Soit $(p,q)∈ℕ^2$ et $f \colonℝ^2 \toℝ$ définie par
\[f(x,y) = \begin{cases}
  \frac{x^p y^q}{x^2+y^2} & \text{si } (x,y)≠(0,0) \\
  \hfil 0                 & \text{si } x=y=0.
\end{cases}\]
\begin{enumerate}
\item Montrer que $f$ est $\CC{∞}$ sur $ℝ^2∖\Acco{(0,0)}$.
\item Montrer que $f$ est continue ssi $p+q≥2$.
\item Montrer que $f$ est de classe $\CC1$ ssi $p+q≥3$.
\item Montrer que $f$ est de classe $\CC k$ ssi $p+q≥k+2$.
\end{enumerate}

\subsection{Équations aux Dérivées Partielles}

\Exercice

Déterminer les fonctions $\Fn{f}{(ℝ_+^*)^2}{ℝ}$ de classe $\CC1$ vérifiant:
\[x \DerPart fx = y \DerPart fy.\]
On pourra poser $u = xy$ et $v = \frac xy$.

\Exercice

Déterminer les fonctions $\Fn{f}{ℝ^2∖\{(0,0)\}}{ℝ}$ de classe $\CC1$ vérifiant:
\[x \DerPart fx = - y \DerPart fy\]
On pourra poser $x = ρ\cosθ$ et $y = ρ\sinθ$.

\Exercice

Résoudre l'équation
\[ \DerPart fu (u,v) + 2u \DerPart fv (u,v) = 0 \]
en effectuant le changement de variables
\[ ϕ(x,y) = (x, y+x^2). \]

\Exercice

Déterminer les fonctions $\Fn{f}{(ℝ_+^*)^2}{ℝ}$ de classe $\CC2$ vérifiant:
\[x^2 \frac{∂^2 f}{∂x^2} + 2xy \frac{∂^2 f}{∂x∂y} + y^2 \frac{∂^2 f}{∂y^2} = 0\]
On pourra poser $u = xy$ et $v = \frac xy$.

\Exercice

Soit $\Fn{f}{ℝ^2}{ℝ}$ de classe $\CC2$ et $g(u,v) = f(uv,u+v)$.
\begin{enumerate}
\item Calculer $\frac{∂^2 g}{∂u∂v}$.
\item Résoudre l'équation:
  \[x \frac{∂^2 f}{∂x^2} + y \frac{∂^2 f}{∂x∂y} + \frac{∂^2 f}{∂y^2} + \frac{∂f}{∂x} = y.\]
\end{enumerate}

\subsection{Tératologie}

\Exercice

Soit $f(x,y) = \begin{cases}
  \frac{x y}{x^2+y^2} & \text{si } (x,y)≠(0,0) \\
  \hfil 0             & \text{si } x=y=0.
\end{cases}$
\begin{enumerate}
\item Montrer que $∀x∈ℝ$, la fonction $y \mapsto f(x,y)$ est continue.
  De même, montrer que $∀y∈ℝ$, la fonction $x \mapsto f(x,y)$ est continue.
\item Montrer que $f$ admet des dérivées partielles en tout point $(x_0,y_0)∈ℝ^2$.
\item $f$ est-elle de classe $\CC1$?
\item Montrer que $f$ est continue sur $ℝ^2∖\Acco{(0,0)}$ mais n'est pas continue en $(0,0)$.
\end{enumerate}

\Exercice
\begin{enumerate}
\item Soit $f(x,y) = \frac{y^2}{x}$ si $x≠0$, et $f(0,y) = 0$.
  Montrer que $f$ est dérivable selon tout vecteur en $(0,0)$, mais qu'elle n'est pas continue en $(0,0)$.
\item Soit $f(x,y) = xy\frac{x^2-y^2}{x^2+y^2}$ si $(x,y)≠(0,0)$ et $f(0,0)=0$.
  Monter que $\frac{∂^2 f}{∂x∂y}(0,0)$ et $\frac{∂^2 f}{∂y∂x}(0,0)$ existent, mais qu'ils sont différents.
\end{enumerate}

\subsection{Recherche d'extremas}

\Exercice

Déterminer les extrema des fonctions suivantes:
\begin{enumerate}
\item $f(x,y) = x^2 y (x+y-4)$
  sur $D = \Ensemble{(x,y)∈ℝ^2}{x≥0, y≥0, x+y≤6}$
\item $f(x,y) = x^4 + y^4 - 2(x-y)^2$
\item $f(x,y) = x^2(x+1) + y^3$
\item $f(x,y) = x^3 + y^3 - 6(x^2 - y^2)$
\item $f(x,y) =√{x^2+y^2} + x^2 - 3$
  sur $D = \Ensemble{(x,y)∈ℝ^2}{x^2+y^2≤16}$
\item $f(x,y,z) = x^2 + y^2 + z^2 + 2xyz$
\item $f(x,y) = (y^2-x^2)(y^2-2x^2)$
\item $f(x,y) = x\ln y - y\ln x$
\item $f(x,y) = 4xy + \frac1x + \frac1y$
\item $f(x,y) = x^3+y^3-9xy+27$
\item $f \nUplet x1n = x_1 x_2 \cdots x_n$ sur $D = \Ensemble{\nUplet x1n∈ℝ_+^n}{x_1+x_2+\cdots+x_n≤1}$
\end{enumerate}

\subsection{Divers}

\Exercice

Soit une fonction $f$ de classe $\CC1$ de $ℝ^2$ dans $ℝ$
et soit $(p,q)∈ℝ^2$ tels que $(p,q)≠(0,0)$.
Montrer que les restrictions de $f$ aux droites d'équations
$px+qy=C^{\text{ste}}$
sont constantes £ssi.
\[ q \DerPart fx = p \DerPart fy. \]

\Exercice

Soit une fonction $\Fn{f}{ℝ^n∖\{0\}}{ℝ}$ de classe $\CC1$ et $k$ une constante réelle.
On dit que $f$ est homogène de degré $k$ si et seulement si
\[∀t > 0 \+∀x∈ℝ^n∖\{0\} \+
f(tx) = t^k f(x).\]
Montrer que $f$ est homogène de degré $k$
si et seulement si elle vérifie l'identité d'Euler:
\[∀x∈ℝ^n∖\{0\} \+ ∑_{i=1}^{n} \DerPart{f}{x_i}(x) = kf(x).\]

\Exercice

Soit $U$ un ouvert convexe de $ℝ^n$, $a$ et $b$ deux points distincts de $U$.
\begin{enumerate}
\item Si $f \colon U \toℝ$ est de classe $\CC1$ sur $U$,
  montrer qu'il existe $c∈\intO{a,b}$ tel que $f(b)-f(a) = \DIF{f}{c}{(b-a)}$.
      On rappelle que $\DIF{f}{c}{(b-a)} = \PS{∇f(c)}{b-a}$.
    \item De combien de décimales exactes des nombres $e$, $√2$ et $π$ a-t-on besoin
      pour pouvoir calculer à $10^{-20}$ près le nombre $\frac{√2}{e+π^3}$?
\end{enumerate}

\Exercice

Soit $\Fn{f}{ℝ^n}{ℝ}$ de classe $\CC1$ sur un ouvert $U$ contenant $\vec0$, telle que
\[∀x∈ℝ^n∖\{0\}\+ ∀t∈ℝ_+^*\+ f(tx) = tf(x)\]
Montrer que $f$ est linéaire.

\Exercice

Soit $f \colonℝ\toℝ$ de classe $\CC1$, et $F(x,y) = \frac{f(x)-f(y)}{x-y}$.
\begin{enumerate}
\item Montrer que $F$ se prolonge en une fonction continue sur $ℝ^2$, que l'on notera $\widetilde F$.
\item Si $f$ est de classe $\CC2$, montrer que $\widetilde F$ est de classe $\CC1$.
\end{enumerate}

\Exercice

Soit $f(x,y) = \frac{\sin x - \sin y}{\sh x - \sh y}$.
Montrer que $f$ se prolonge en une fonction $\CC∞$ sur $ℝ^2$.

\Exercice

Soit $\Fonction{f}{\MnR}{\MnR}{X}{X^2.}$
\begin{enumerate}
\item Montrer que $f$ est $\CC∞$.
\item Calculer $\DF fX$.
\item On suppose $X$ diagonalisable et que ses valeurs propres sont toutes strictement positives.
  Montrer alors que $\DF fX$ est un automorphisme de $\MnR$.
\end{enumerate}

\end{document}
