% autogenerated by ytex.rs

\documentclass{scrartcl}

\usepackage[francais]{babel}
\usepackage{geometry}
\usepackage{scrpage2}
\usepackage{lastpage}
\usepackage{multicol}
\usepackage{etoolbox}
\usepackage{xparse}
\usepackage{enumitem}
% \usepackage{csquotes}
\usepackage{amsmath}
\usepackage{amsfonts}
\usepackage{amssymb}
\usepackage{mathrsfs}
\usepackage{stmaryrd}
\usepackage{dsfont}
% \usepackage{eurosym}
% \usepackage{numprint}
\usepackage[most]{tcolorbox}
% \usepackage{tikz}
% \usepackage{tkz-tab}
\usepackage[unicode]{hyperref}
\usepackage[ocgcolorlinks]{ocgx2}

\let\ifTwoColumns\iftrue
\def\Classe{$\Psi$2019--2020}

% reproducible builds
% LuaTeX: \pdfvariable suppressoptionalinfo 1023 \relax
\pdfinfoomitdate=1
\pdftrailerid{}

\newif\ifDisplaystyle
\everymath\expandafter{\the\everymath\ifDisplaystyle\displaystyle\fi}
\newcommand\DS{\displaystyle}

\clearscrheadfoot
\pagestyle{scrheadings}
\thispagestyle{empty}
\ohead{\Classe}
\ihead{\thepage/\pageref*{LastPage}}

\setlist[itemize,1]{label=\textbullet}
\setlist[itemize,2]{label=\textbullet}

\ifTwoColumns
  \geometry{margin=1cm,top=2cm,bottom=3cm,foot=1cm}
  \setlist[enumerate]{leftmargin=*}
  \setlist[itemize]{leftmargin=*}
\else
  \geometry{margin=3cm}
\fi

\makeatletter
\let\@author=\relax
\let\@date=\relax
\renewcommand\maketitle{%
    \begin{center}%
        {\sffamily\huge\bfseries\@title}%
        \ifx\@author\relax\else\par\medskip{\itshape\Large\@author}\fi
        \ifx\@date\relax\else\par\bigskip{\large\@date}\fi
    \end{center}\bigskip
    \ifTwoColumns
        \par\begin{multicols*}{2}%
        \AtEndDocument{\end{multicols*}}%
        \setlength{\columnsep}{5mm}
    \fi
}
\makeatother

\newcounter{ParaNum}
\NewDocumentCommand\Para{smo}{%
  \IfBooleanF{#1}{\refstepcounter{ParaNum}}%
  \paragraph{\IfBooleanF{#1}{{\tiny\arabic{ParaNum}~}}#2\IfNoValueF{#3}{ (#3)}}}

\newcommand\I{i}
\newcommand\mi{i}
\def\me{e}

\def\do#1{\expandafter\undef\csname #1\endcsname}
\docsvlist{Ker,sec,csc,cot,sinh,cosh,tanh,coth,th}
\undef\do

\DeclareMathOperator\ch{ch}
\DeclareMathOperator\sh{sh}
\DeclareMathOperator\th{th}
\DeclareMathOperator\coth{coth}
\DeclareMathOperator\cotan{cotan}
\DeclareMathOperator\argch{argch}
\DeclareMathOperator\argsh{argsh}
\DeclareMathOperator\argth{argth}

\let\epsilon=\varepsilon
\let\phi=\varphi
\let\leq=\leqslant
\let\geq=\geqslant
\let\subsetneq=\varsubsetneq
\let\emptyset=\varnothing

\newcommand{\+}{,\;}

\undef\C
\newcommand\ninf{{n\infty}}
\newcommand\N{\mathbb{N}}
\newcommand\Z{\mathbb{Z}}
\newcommand\Q{\mathbb{Q}}
\newcommand\R{\mathbb{R}}
\newcommand\C{\mathbb{C}}
\newcommand\K{\mathbb{K}}
\newcommand\Ns{\N^*}
\newcommand\Zs{\Z^*}
\newcommand\Qs{\Q^*}
\newcommand\Rs{\R^*}
\newcommand\Cs{\C^*}
\newcommand\Ks{\K^*}
\newcommand\Rp{\R^+}
\newcommand\Rps{\R^+_*}
\newcommand\Rms{\R^-_*}
\newcommand{\Rpinf}{\Rp\cup\Acco{+\infty}}

\undef\B
\newcommand\B{\mathscr{B}}

\undef\P
\DeclareMathOperator\P{\mathbb{P}}
\DeclareMathOperator\E{\mathbb{E}}
\DeclareMathOperator\Var{\mathbb{V}}

\DeclareMathOperator*\PetitO{o}
\DeclareMathOperator*\GrandO{O}
\DeclareMathOperator*\Sim{\sim}
\DeclareMathOperator\Tr{tr}
\DeclareMathOperator\Ima{Im}
\DeclareMathOperator\Ker{Ker}
\DeclareMathOperator\Sp{Sp}
\DeclareMathOperator\Diag{diag}
\DeclareMathOperator\Rang{rang}
\DeclareMathOperator*\Coords{Coords}
\DeclareMathOperator*\Mat{Mat}
\DeclareMathOperator\Pass{Pass}
\DeclareMathOperator\Com{Com}
\DeclareMathOperator\Card{Card}
\DeclareMathOperator\Racines{Racines}
\DeclareMathOperator\Vect{Vect}
\DeclareMathOperator\Id{Id}

\newcommand\DerPart[2]{\frac{\partial #1}{\partial #2}}

\def\T#1{{#1}^T}

\def\pa#1{({#1})}
\def\Pa#1{\left({#1}\right)}
\def\bigPa#1{\bigl({#1}\bigr)}
\def\BigPa#1{\Bigl({#1}\Bigr)}
\def\biggPa#1{\biggl({#1}\biggr)}
\def\BiggPa#1{\Biggl({#1}\Biggr)}

\def\pafrac#1#2{\pa{\frac{#1}{#2}}}
\def\Pafrac#1#2{\Pa{\frac{#1}{#2}}}
\def\bigPafrac#1#2{\bigPa{\frac{#1}{#2}}}
\def\BigPafrac#1#2{\BigPa{\frac{#1}{#2}}}
\def\biggPafrac#1#2{\biggPa{\frac{#1}{#2}}}
\def\BiggPafrac#1#2{\BiggPa{\frac{#1}{#2}}}

\def\cro#1{[{#1}]}
\def\Cro#1{\left[{#1}\right]}
\def\bigCro#1{\bigl[{#1}\bigr]}
\def\BigCro#1{\Bigl[{#1}\Bigr]}
\def\biggCro#1{\biggl[{#1}\biggr]}
\def\BiggCro#1{\Biggl[{#1}\Biggr]}

\def\abs#1{\mathopen|{#1}\mathclose|}
\def\Abs#1{\left|{#1}\right|}
\def\bigAbs#1{\bigl|{#1}\bigr|}
\def\BigAbs#1{\Bigl|{#1}\Bigr|}
\def\biggAbs#1{\biggl|{#1}\biggr|}
\def\BiggAbs#1{\Biggl|{#1}\Biggr|}

\def\acco#1{\{{#1}\}}
\def\Acco#1{\left\{{#1}\right\}}
\def\bigAcco#1{\bigl\{{#1}\bigr\}}
\def\BigAcco#1{\Bigl\{{#1}\Bigr\}}
\def\biggAcco#1{\biggl\{{#1}\biggr\}}
\def\BiggAcco#1{\Biggl\{{#1}\Biggr\}}

\def\ccro#1{\llbracket{#1}\rrbracket}
\def\Dcro#1{\llbracket{#1}\rrbracket}

\def\floor#1{\lfloor#1\rfloor}
\def\Floor#1{\left\lfloor{#1}\right\rfloor}

\def\sEnsemble#1#2{\mathopen\{#1\mid#2\mathclose\}}
\def\bigEnsemble#1#2{\bigl\{#1\bigm|#2\bigr\}}
\def\BigEnsemble#1#2{\Bigl\{#1\Bigm|#2\Bigr\}}
\def\biggEnsemble#1#2{\biggl\{#1\biggm|#2\biggr\}}
\def\BiggEnsemble#1#2{\Biggl\{#1\Biggm|#2\Biggr\}}
\let\Ensemble=\bigEnsemble

\newcommand\IntO[1]{\left]#1\right[}
\newcommand\IntF[1]{\left[#1\right]}
\newcommand\IntOF[1]{\left]#1\right]}
\newcommand\IntFO[1]{\left[#1\right[}

\newcommand\intO[1]{\mathopen]#1\mathclose[}
\newcommand\intF[1]{\mathopen[#1\mathclose]}
\newcommand\intOF[1]{\mathopen]#1\mathclose]}
\newcommand\intFO[1]{\mathopen[#1\mathclose[}

\newcommand\Fn[3]{#1\colon#2\to#3}
\newcommand\CC[1]{\mathscr{C}^{#1}}
\newcommand\D{\mathop{}\!\mathrm{d}}

\newcommand\longto{\longrightarrow}

\undef\M
\newcommand\M[3]{\mathrm{#1}_{#2}\pa{#3}}
\newcommand\MnR{\M{M}{n}{\R}}
\newcommand\MnC{\M{M}{n}{\C}}
\newcommand\MnK{\M{M}{n}{\K}}
\newcommand\GLnR{\M{GL}{n}{\R}}
\newcommand\GLnC{\M{GL}{n}{\C}}
\newcommand\GLnK{\M{GL}{n}{\K}}
\newcommand\DnR{\M{D}{n}{\R}}
\newcommand\DnC{\M{D}{n}{\C}}
\newcommand\DnK{\M{D}{n}{\K}}
\newcommand\SnR{\M{S}{n}{\R}}
\newcommand\AnR{\M{A}{n}{\R}}
\newcommand\OnR{\M{O}{n}{\R}}
\newcommand\SnRp{\mathrm{S}_n^+(\R)}
\newcommand\SnRpp{\mathrm{S}_n^{++}(\R)}

\newcommand\LE{\mathscr{L}(E)}
\newcommand\GLE{\mathscr{GL}(E)}
\newcommand\SE{\mathscr{S}(E)}
\renewcommand\OE{\mathscr{O}(E)}

\newcommand\ImplD{$\Cro\Rightarrow$}
\newcommand\ImplR{$\Cro\Leftarrow$}
\newcommand\InclD{$\Cro\subset$}
\newcommand\InclR{$\Cro\supset$}
\newcommand\notInclD{$\Cro{\not\subset}$}
\newcommand\notInclR{$\Cro{\not\supset}$}

\newcommand\To[1]{\xrightarrow[#1]{}}
\newcommand\Toninf{\To{\ninf}}

\newcommand\Norm[1]{\|#1\|}
\newcommand\Norme{{\Norm{\cdot}}}

\newcommand\Int[1]{\mathring{#1}}
\newcommand\Adh[1]{\overline{#1}}

\newcommand\Uplet[2]{{#1},\dots,{#2}}
\newcommand\nUplet[3]{(\Uplet{{#1}_{#2}}{{#1}_{#3}})}

\newcommand\Fonction[5]{{#1}\left|\begin{aligned}{#2}&\;\longto\;{#3}\\{#4}&\;\longmapsto\;{#5}\end{aligned}\right.}

\DeclareMathOperator\orth{\bot}
\newcommand\Orth[1]{{#1}^\bot}
\newcommand\PS[2]{\langle#1,#2\rangle}

\newcommand{\Tribu}{\mathscr{T}}
\newcommand{\Part}{\mathcal{P}}
\newcommand{\Pro}{\bigPa{\Omega,\Tribu}}
\newcommand{\Prob}{\bigPa{\Omega,\Tribu,\P}}

\newcommand\DEMO{$\spadesuit$}
\newcommand\DUR{$\spadesuit$}

\newenvironment{psmallmatrix}{\left(\begin{smallmatrix}}{\end{smallmatrix}\right)}

% -----------------------------------------------------------------------------

\newcommand{\FIK}{\mathcal{F}(I,\K)}
\newcommand{\fn}{(f_n)_{n\in \N}}
\newcommand{\Sfn}{\sum_n f_n}

\begin{document}
\title{Suites et s\'eries de fonctions}
\maketitle

\Para{Notations}

\begin{itemize}
\item
  $I$ d\'esigne un intervalle de $\R$ non vide et non r\'eduit \`a un point;
\item
  $\K$ d\'esigne $\R$ ou $\C$;
\item
  $\K^I = \FIK$ d\'esigne le $\K$-espace vectoriel des fonctions de $I$ dans $\K$.
\end{itemize}

% -----------------------------------------------------------------------------
\section{G\'en\'eralit\'es}

\Para{D\'efinition}

On appelle \emph{suite de fonctions} toute suite $\fn$ \`a valeurs dans $\FIK$.
Autement dit, une suite de fonctions est une suite $\fn$ dont les \'el\'ements sont des fonctions $\Fn{f_n}{I}{\K}$.

\Para{D\'efinition}

Soit $\fn$ une suite de fonctions de $I$ dans $\K$.
On pose \[ \Fonction{S_n}{I}{\K}{x}{\sum_{k=0}^n f_k(x)} \]
On appelle \emph{s\'erie de fonctions} de $I$ dans $\K$
de terme g\'en\'eral $f_n$ la suite de fonctions $(S_n)_{n\in \N}$; on la note plut\^ot $\sum_n f_n$.

% -----------------------------------------------------------------------------
\section{Modes de convergence}

% -----------------------------------------------------------------------------
\subsection{Convergence simple}

\Para{D\'efinition}

Soit $\fn$ une suite de fonctions de $I$ dans $\K$, et soit $f \colon I \to\K$.
On dit que la suite de fonctions $\fn$ \emph{converge simplement} vers $f$ sur $I$ si et seulement si \[ \forall x\in I \+ f_n(x) \Toninf f(x). \]

\Para{D\'efinition}

On dit que la s\'erie $\Sfn$ \emph{converge simplement} sur $I$
si et seulement si la suite de fonction $(S_n)_{n\in \N}$ converge simplement sur $I$.

Autrement dit, la s\'erie $\Sfn$ converge simplement sur $I$ si et seulement si
\[ \tcboxmath{ \forall x\in I, \text{la s\'erie} \sum_n f_n(x) \text{ converge.} } \]

\Para{Proposition}[unicit\'e de la limite simple]

Soit $\fn$ une suite de fonctions de $I$ dans $\K$.
Soit $f$ et $g$ deux fonctions de $I$ dans $\K$.
Si $\fn$ converge simplement sur $I$ vers $f$ et vers $g$
alors $f = g$.

\Para{D\'efinition}

Soit $\fn$ une suite de fonctions de $I$ dans $\K$ convergeant simplement sur $I$ vers $f \colon I \to \K$.
$f$ s'appelle la \emph{limite simple} de la suite de fonctions $(f_n)_{n\in \N}$.

% -----------------------------------------------------------------------------
\subsection{Convergence uniforme}

\Para{D\'efinition}

Soit $\fn$ une suite de fonctions de $I$ dans $\K$, et soit $f \colon I \to\K$.
On dit que la suite de fonctions $\fn$ \emph{converge uniform\'ement} vers $f$ sur $I$ si et seulement si
\[ \forall \epsilon>0 \+ \exists N\in \N{} \+ \forall n\geq N \+ \forall x\in I \+ \abs{f_n(x) - f(x)} \leq{} \epsilon, \]
ou de fa\c con \'equivalente, si et seulement si
\[ \tcboxmath{ \sup_I \Abs{f_n - f} \Toninf 0. } \]

\Para{Proposition}

Si $\fn$ converge uniform\'ement vers $f$, alors $\fn$ converge simplement vers $f$.

\Para{D\'efinition}

On dit que la s\'erie $\Sfn$ \emph{converge uniform\'ement} sur $I$ si et seulement si la suite de fonction $(S_n)_{n\in \N}$ converge uniform\'ement sur $I$.

\Para{Proposition}

Soit $\Sfn$ une s\'erie de fonction qui converge simplement vers $f$ sur $I$.
Cette s\'erie converge uniform\'ement vers $f$ sur $I$ si et seulement si
la suite de fonctions $(R_n)$ converge uniform\'ement vers $0$ sur $I$
o\`u \[ R_n(x) = \sum_{k>n} f_k(x). \]

Ainsi, montrer que $\Sfn$ converge uniform\'ement sur $I$ revient \`a montrer
l'existence d'une suite r\'eelle $(\epsilon_n)_{n\in \N}$ telle que:
\begin{tcolorbox}
  \begin{itemize}
  \item
    $\forall n\in \N$, $\forall x\in I$, $\Abs{R_n(x)} \leq \epsilon_n$;
  \item
    la suite num\'erique $(\epsilon_n)$ tend vers $0$.
  \end{itemize}
\end{tcolorbox}

% -----------------------------------------------------------------------------
\subsection{Convergence normale}

Cette notion de convergence n'a de sens que pour une s\'erie de fonctions.

\Para{D\'efinition}

Soit $\Sfn$ une s\'erie de fonctions born\'ees de $I$ dans $\K$.
On dit que la s\'erie de fonctions $\Sfn$ \emph{converge normalement} sur $I$
si et seulement si la \emph{s\'erie num\'erique}
de terme g\'en\'eral $u_n = \sup_I \Abs{f_n}$ converge.

\Para{Proposition}

Soit $\Sfn$ une s\'erie de fonctions de $I$ dans $\K$.
Montrer que la s\'erie converge normalement sur $I$ revient \`a
montrer l'existence d'une suite r\'eelle $(\alpha_n)_{n\in \N}$ telle que:
\begin{tcolorbox}
  \begin{itemize}
  \item
    $\forall n\in \N$, $\forall x\in I$, $\Abs{f_n(x)}\leq \alpha_n$;
  \item
    la s\'erie num\'erique $\sum_n\alpha_n$ converge.
  \end{itemize}
\end{tcolorbox}

\Para{Proposition}

La convergence normale entra\^ine la convergence uniforme (qui entra\^ine la convergence simple).

% -----------------------------------------------------------------------------
\section{Th\'eor\`emes}

% -----------------------------------------------------------------------------
\subsection{Continuit\'e}\label{sec:cont}

\Para{Th\'eor\`eme de continuit\'e de la limite}

La limite uniforme d'une suite de fonctions continue est elle-m\^eme continue.
Plus pr\'ecis\'ement, si
\begin{itemize}
\item
  pour tout $n\in \N$, $f_n$ est continue sur $I$,
\item
  $\fn$ converge uniform\'ement vers $f$.
\end{itemize}
Alors $f$ est continue sur $I$.

\Para{Remarque}

Cela n'est pas vrai pour la limite simple;
consid\'erez par exemple la suite de fonction $f_n(x) = x^n$ sur $[0,1]$.

\Para{Th\'eor\`eme de continuit\'e de la somme}

Soit $\Sfn$ une s\'erie de fonctions de $I$ dans $\K$.

On suppose que:
\begin{itemize}
\item
  pour tout $n\in \N$, $f_n$ est continue sur $I$;
\item
  $\Sfn$ converge uniform\'ement sur $I$.
\end{itemize}

Alors la somme $f = \sum_{n=0}^{+\infty} f_n$ est une fonction d\'efinie et continue sur $I$.

% -----------------------------------------------------------------------------
\subsection{Permutation avec une limite}\label{sec:perm-lim}

\Para{Th\'eor\`eme de la double limite}

Soit $\fn$ une suite de fonctions de $I$ dans $\K$.
On suppose que:
\begin{itemize}
\item
  $a$ est une extr\'emit\'e de $I$ (\'eventuellement $\pm\infty$);
\item
  pour tout $n\in \N$, $\lim_a f_n = \ell_n$ existe et est finie;
\item
  $\fn$ converge uniform\'ement vers $f$ sur $I$.
\end{itemize}

Alors:
\begin{itemize}
\item
  la suite $(\ell_n)_{n\in \N}$ converge. Notons $\ell$ sa limite;
\item
  $f$ admet une limite en $a$;
\item
  $\lim_a f = \ell$, c.-\`a-d.
  \[ \tcboxmath{
      \lim_{x\to a} \; \lim_\ninf \; f_n(x) = \lim_\ninf \; \lim_{x\to a} \; f_n(x).
  } \]
\end{itemize}

\Para{Th\'eor\`eme de permutation limite-somme}

Soit $\Sfn$ une s\'erie de fonctions de $I$ dans $\K$.
On suppose que:
\begin{itemize}
\item
  $a$ est une extr\'emit\'e de $I$ (\'eventuellement $\pm\infty$);
\item
  pour tout $n\in \N$, $\lim_a f_n = \ell_n$ existe et est finie;
\item
  $\Sfn$ converge uniform\'ement vers $f$ sur $I$.
\end{itemize}

Alors:
\begin{itemize}
\item
  la s\'erie num\'erique $\sum_n \ell_n$ est convergente;
\item
  $f$ admet une limite en $a$;
\item
  $\lim_a f = \sum_{n=0}^{+\infty} \ell_n$, c.-\`a-d.
  \[ \tcboxmath{ \lim_{x \to a} \sum_{n=0}^{+\infty} f_n(x) = \sum_{n=0}^{+\infty} \lim_{x \to a} f_n(x). } \]
\end{itemize}

% -----------------------------------------------------------------------------
\subsection{Permutation avec une int\'egrale}\label{sec:perm-int}

\Para{Th\'eor\`eme de permutation limite/int\'egrale}

On suppose que:
\begin{itemize}
\item
  pour tout $n\in \N$, $f_n$ est continue sur le segment $[a,b]$;
\item
  $\fn$ converge uniform\'ement vers $f$ sur $[a,b]$.
\end{itemize}

Alors $f$ est continue sur $[a,b]$ et
\[ \tcboxmath{ \lim_\ninf \int_a^b f_n = \int_a^b \lim_\ninf f_n } = \int_a^b f. \]

\Para{Th\'eor\`eme de permutation somme/int\'egrale}

Soit $\Sfn$ une s\'erie de fonctions de $[a,b]$ dans $\K$.
On suppose que:
\begin{itemize}
\item
  pour tout $n\in \N$, $f_n$ est continue sur le segment $[a,b]$
\item
  la s\'erie de fonctions $\sum_n f_n$ converge uniform\'ement sur $[a,b]$ vers $f$
\end{itemize}

Alors $f$ est continue sur $[a,b]$ et
\[ \tcboxmath{ \sum_{n=0}^{+\infty} \int_a^b f_n = \int_a^b \sum_{n=0}^{+\infty} f_n } = \int_a^b f. \]

% -----------------------------------------------------------------------------
\subsection{D\'erivabilit\'e}\label{sec:deriv}

\Para{Th\'eor\`eme de d\'erivation de la limite}

On suppose que:
\begin{itemize}
\item
  pour tout $n\in \N$, $f_n$ est de classe $\CC1$ sur $I$;
\item
  $(f_n)$ converge simplement vers $f$ sur $I$;
\item
  $(f'_n)$ converge uniform\'ement vers $g$ sur $I$.
\end{itemize}

Alors:
\begin{itemize}
\item
  $f$ est de classe $\CC1$ sur $I$;
\item
  $f' = g$.
\end{itemize}

\Para{Th\'eor\`eme de d\'erivation de la somme}

Soit $\Sfn$ une s\'erie de fonctions de $I$ dans $\K$.

On suppose que:
\begin{itemize}
\item
  pour tout $n\in \N$, $f_n$ est de classe $\CC1$ sur $I$
\item
  $\sum_n f_n$ converge simplement sur $I$ vers $f$
\item
  $\sum_n f'_n$ converge uniform\'ement sur $I$
\end{itemize}

Alors $f$ est de classe $\CC1$ sur $I$ et
\[ \forall x\in I \+ f'(x) = \sum_{n=0}^{+\infty} f'_n(x). \]

% -----------------------------------------------------------------------------
\subsection{G\'en\'eralisation}\label{sec:deriv2}

\Para{Lemme}
Soit $(f_n)$ une suite de fonctions de $\intF{a,b}$ \`a valeurs dans $\K$.
On suppose que:
\begin{itemize}
\item
  pour tout $n\in \N$, $f_n$ est de classe $\CC1$ sur $\intF{a,b}$;
\item
  la suite de fonction $(f_n')$ converge uniform\'ement sur $\intF{a,b}$;
\item
  la suite num\'erique $(f_n(a))$ converge.
\end{itemize}
Alors la suite de fonctions $(f_n)$ converge uniform\'ement sur $\intF{a,b}$.

\Para{Th\'eor\`eme de d\'erivation de la limite}

On suppose que:
\begin{itemize}
\item
  pour tout $n\in \N$, $f_n$ est de classe $\CC{p}$ sur $I$, $p\geq1$;
\item
  pour $0\leq k\leq p-1$,
  la suite $(f^{(k)}_n)$ converge simplement vers $g_k$ sur $I$;
\item
  la suite $(f^{(p)}_n)$ converge uniform\'ement vers $g_p$ sur $I$.
\end{itemize}

Alors:
\begin{itemize}
\item
  $f=g_0$ est de classe $\CC{p}$;
\item
  pour tout $0\leq k\leq p$, $f^{(k)} = g_k$.
\end{itemize}

\Para{Th\'eor\`eme de d\'erivation de la somme}

Soit $\Sfn$ une s\'erie de fonctions de $I$ dans $\K$.
On suppose que:
\begin{itemize}
\item
  pour tout $n\in \N$, $f_n$ est de classe $\CC{p}$, $p\geq1$
\item
  pour tout $k\in\Dcro{0,p-1}$, $\sum_n f_n^{(k)}$ converge simplement sur $I$
\item
  $f$ est la somme de la s\'erie $\Sfn$, c.-\`a-d. $f =\sum_{n=0}^{+\infty} f_n$
\item
  $\sum_n f_n^{(p)}$ converge uniform\'ement sur $I$
\end{itemize}

Alors $f$ est de classe $\CC p$ sur $I$ et
\[ \forall k\in\ccro{0,p} \+ \forall x\in I \+ f^{(k)}(x) = \sum_{n=0}^{+\infty} f^{(k)}_n(x). \]

\Para{Remarque importante: caract\`ere local}

Les notions de continuit\'e et de d\'erivabilit\'e sont des notions \emph{locales}.
Cela signifie, par exemple, que pour montrer qu'une fonction $f$ est continue en~42, il suffit de s'int\'eresser \`a la continuit\'e de $f$ sur un voisinage de~42, par exemple le segment $[41,43]$.

Ainsi, dans les th\'eor\`emes sur la continuit\'e et sur la d\'erivabilit\'e
(sections \ref{sec:cont}, \ref{sec:deriv} et \ref{sec:deriv2}),
on peut remplacer l'hypoth\`ese de convergence uniforme sur $I$
par la convergence uniforme sur tout segment $K$ inclus dans $I$.

Cela n'est pas possible pour les th\'eor\`emes de la section~\ref{sec:perm-lim}.

% -----------------------------------------------------------------------------
\section{Exercices}

% -----------------------------------------------------------------------------
\par\pagebreak[1]\par
\paragraph{Exercice 1}%
\hfill{\tiny 5805}%
\begingroup~

Soit $\Fn{\varphi}{\Rp}{\R}$ continue, non identiquement nulle telle que $\varphi(0) = 0$ et $\lim_{+\infty} \varphi{} = 0$.

Que peut-on dire de la convergence des suites de fonctions suivantes?
En particulier, on se demandera s'il y a convergence simple?
uniforme?
uniforme sur tout segment?
sur quel(s) intervalle(s) de $\Rp$?
\begin{enumerate}
\item
  $a_n(x) = \varphi(x)/n$
\item
  $b_n(x) = \varphi(x/n)$
\item
  $c_n(x) = n\varphi(x)$
\item
  $d_n(x) = \varphi(nx)$
\item
  $e_n(x) = \varphi(nx)\varphi(x/n)$
\end{enumerate}
\endgroup

% -----------------------------------------------------------------------------
\par\pagebreak[1]\par
\paragraph{Exercice 2}%
\hfill{\tiny 5655}%
\begingroup~

Soit $\alpha\in \R$.
\'Etudier la convergence de la suite de fonctions
d\'efinie par $f_n(x) = n^\alpha{} x (1-x)^n$ sur $[0,1]$.
\endgroup

% -----------------------------------------------------------------------------
\par\pagebreak[1]\par
\paragraph{\href{https://psi.miomio.fr/exo/0562.pdf}{Exercice 3}}%
\hfill{\tiny 0562}%
\begingroup~

Soit $f_0(t) = 0$ et $f_{n+1}(t) = \sqrt{t+f_n(t)}$.
\begin{enumerate}
\item
  \'Etudier la convergence simple de la suite de fonctions $(f_n)$ sur $\Rp$.
\item
  Y a-t-il convergence uniforme?
\end{enumerate}
\endgroup

% -----------------------------------------------------------------------------
\par\pagebreak[1]\par
\paragraph{Exercice 4}%
\hfill{\tiny 3907}%
\begingroup~

Soit $f_n(x) = n \cos^n(x) \sin(x)$ d\'efinie sur $I = [0,\pi/2]$.
\begin{enumerate}
\item
  D\'eterminer la limite simple de la suite $(f_n)$.
\item
  Montrer que
  \[ \lim_\ninf \int_0^{\pi/2} f_n(x) \D x \neq{} \int_0^{\pi/2} \lim_\ninf f_n(x) \D x. \]
\item
  A-t-on convergence uniforme sur $I$ de la suite $(f_n)$?
\end{enumerate}
\endgroup

% -----------------------------------------------------------------------------
\par\pagebreak[1]\par
\paragraph{Exercice 5}%
\hfill{\tiny 4113}%
\begingroup~

\def\fn{(f_n)_{n\in \N}}
Soit $f_n \colon x \mapsto \frac{1}{n!} \, x^n \, e^{-x}$.
\begin{enumerate}
\item
  Montrer que $\fn$ converge uniform\'ement sur $\Rp$.
\item
  Calculer \[ \lim_\ninf \int_0^{+\infty} f_n(x) \D x, \]
  o\`u par d\'efinition \[ \int_0^{+\infty} f_n = \lim_{M \to +\infty} \int_0^M f_n. \]
  Le r\'esultat est-il surprenant?
\end{enumerate}
\endgroup

% -----------------------------------------------------------------------------
\par\pagebreak[1]\par
\paragraph{Exercice 6}%
\hfill{\tiny 2459}%
\begingroup~

Soit $\Fn{f}{[0,1]}{\R}$ de classe $\CC1$ telle que $f(1)\neq0$.
On pose \[ u_n = \int_0^1 x^n f(x) \D x. \]
\begin{enumerate}
\item
  D\'eterminer la limite $\ell$ de la suite $(u_n)_{n\in \N}$
\item
  D\'eterminer un \'equivalent de $u_n - \ell$.
  On pourra effectuer une int\'egration par parties.
\end{enumerate}
\endgroup

% -----------------------------------------------------------------------------
\par\pagebreak[1]\par
\paragraph{Exercice 7}%
\hfill{\tiny 0104}%
\begingroup~

Soit $\Fn{f}{[0,1]}{\R}$ continue.
D\'eterminer \[ \lim_\ninf \int_0^1 n x^n f(x^n) \D x. \]

On pourra commencer par le changement de variables $y=x^n$,
et majorer la diff\'erence entre l'int\'egrale et la limite conjectur\'ee.
\endgroup

% -----------------------------------------------------------------------------
\par\pagebreak[1]\par
\paragraph{\href{https://psi.miomio.fr/exo/7651.pdf}{Exercice 8}}%
\hfill{\tiny 7651}%
\begingroup~

D\'eterminer \[ \lim_{n\infty} \int_0^1 \frac{\D x}{1+x+x^2+\cdots+x^n}. \]
\endgroup

% -----------------------------------------------------------------------------
\par\pagebreak[1]\par
\paragraph{Exercice 9}%
\hfill{\tiny 0001}%
\begingroup~

Soit $\Fonction{f_n}{\intOF{0,1}}{\R}{x}{\frac{(-x\ln x)^n}{n!}}$
\begin{enumerate}
\item
  Montrer que $f_n$ se prolonge en une fonction continue sur $[0,1]$, que
  l'on notera \'egalement $f_n$.
\item
  Montrer que la s\'erie $\sum_n f_n$ converge normalement sur $[0,1]$.
\item
  \begin{enumerate}
  \item
    Pour $(p,q)\in \N^2$, on note $I_{p,q} =\int_0^1 x^p (-\ln x)^q \D x$.
    Montrer que pour $q\geq1$, on a $I_{p,q} = \frac{q}{p+1} I_{p,q-1}$.
  \item
    En d\'eduire la valeur de $\int_0^1 f_n$.
  \end{enumerate}
\item
  En d\'eduire que:
  \[ \int_0^1 \frac{\D x}{x^x} =\sum_{n=1}^{+\infty} \frac{1}{n^n} \]
\end{enumerate}
\endgroup

% -----------------------------------------------------------------------------
\par\pagebreak[1]\par
\paragraph{Exercice 10}%
\hfill{\tiny 8296}%
\begingroup~

Soit $\DS f(x) = \sum_{n=1}^{+\infty} \frac{\sin(nx)}{n2^n}$
\begin{enumerate}
\item
  V\'erifier que $f$ est d\'efinie, puis continue, puis de classe $\CC1$ sur $\R$.
\item
  Calculer la somme de la s\'erie $\sum_{n=1}^{+\infty} \pa{e^{ix}/2}^n$,
  et en d\'eduire une expression explicite de $f'$.
\item
  Expliciter $f$.
\end{enumerate}
\endgroup

% -----------------------------------------------------------------------------
\par\pagebreak[1]\par
\paragraph{Exercice 11}%
\hfill{\tiny 6953}%
\begingroup~

Soit $\DS f(x) = \sum_{n=0}^{+\infty} e^{-n^2x}$.
\begin{enumerate}
\item
  Montrer que $f$ est de classe $\CC\infty$ sur $\Rps$.
\item
  D\'eterminer $\lim_{0^+} f$ et $\lim_{+\infty} f$.
\end{enumerate}
\endgroup

% -----------------------------------------------------------------------------
\par\pagebreak[1]\par
\paragraph{Exercice 12}%
\hfill{\tiny 1010}%
\begingroup~

Soit $f(x) = \sum_{n=0}^{+\infty} \frac{x^n}{n!}$.
Montrer que $f$ est de classe $\CC1$ sur $\R$,
puis d\'eterminer $f$.
\endgroup

% -----------------------------------------------------------------------------
\par\pagebreak[1]\par
\paragraph{\href{https://psi.miomio.fr/exo/2542.pdf}{Exercice 13}}%
\hfill{\tiny 2542}%
\begingroup~

Soit $\DS f(x) = \sum_{n\geq1} \frac{nx}{2^n x^2 + n}$.
\begin{enumerate}
\item
  \'Etudier la convergence simple puis la convergence uniforme de la s\'erie.
\item
  Montrer que $f$ d\'efinit une fonction de classe $\CC1$ sur $\Rps$.
\item
  Montrer que $f$ n'est pas d\'erivable en z\'ero.
\item
  D\'eterminer un \'equivalent de $f$ au voisinage de $+\infty$.
\end{enumerate}
\endgroup

% -----------------------------------------------------------------------------
\par\pagebreak[1]\par
\paragraph{Exercice 14}%
\hfill{\tiny 8595}%
\begingroup~

Montrer que la s\'erie de fonctions $\sum f_n$ o\`u $f_n(x) = \frac{xe^{-nx}}{\ln n}$ pour $n\geq2$ converge normalement sur tout segment de $\Rps$ mais pas normalement sur $\Rp$.

% -----------------------------------------------------------------------------
\endgroup

% -----------------------------------------------------------------------------
\par\pagebreak[1]\par
\paragraph{Exercice 15}%
\hfill{\tiny 1037}%
\begingroup~

\def\fn{(f_n)_{n\in \N}}
Reprendre les exercices~6, 7 et 8
en admettant le th\'eor\`eme suivant, qui est une version plus forte du th\'eor\`eme de permutation limite/int\'egrale.
\emph{Ce th\'eor\`eme n'est pas au programme,}
mais il s'agit d'un cas particulier du th\'eor\`eme de convergence domin\'ee que nous verrons plus loin.

\Para{Th\'eor\`eme de convergence born\'ee}

On suppose que:
\begin{itemize}
\item
  pour tout $n\in \N$, $f_n$ est continue (par morceaux) sur le segment $[a,b]$;
\item
  $f_n$ converge simplement vers $f$ sur $[a,b]$;
\item
  $f$ est \'egalement continue (par morceaux);
\item
  la suite $\fn$ est born\'ee, c.-\`a-d.
  \[ \exists M \geq0 \+ \forall n \in \N\+ \forall x \in[a,b] \+ \abs{f_n(x)} \leq{} M \]
\end{itemize}
Alors \[ \tcboxmath{ \lim_\ninf \int_a^b f_n = \int_a^b \lim_\ninf f_n } = \int_a^b f. \]
\endgroup

\end{document}
