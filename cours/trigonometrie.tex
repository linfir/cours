% autogenerated by ytex.rs

\documentclass{scrartcl}

\usepackage[francais]{babel}
\usepackage{geometry}
\usepackage{scrpage2}
\usepackage{lastpage}
\usepackage{multicol}
\usepackage{etoolbox}
\usepackage{xparse}
\usepackage{enumitem}
% \usepackage{csquotes}
\usepackage{amsmath}
\usepackage{amsfonts}
\usepackage{amssymb}
\usepackage{mathrsfs}
\usepackage{stmaryrd}
\usepackage{dsfont}
% \usepackage{eurosym}
% \usepackage{numprint}
% \usepackage[most]{tcolorbox}
% \usepackage{tikz}
% \usepackage{tkz-tab}
\usepackage[unicode]{hyperref}
\usepackage[ocgcolorlinks]{ocgx2}

\let\ifTwoColumns\iftrue
\def\Classe{$\Psi$2019--2020}

% reproducible builds
% LuaTeX: \pdfvariable suppressoptionalinfo 1023 \relax
\pdfinfoomitdate=1
\pdftrailerid{}

\newif\ifDisplaystyle
\everymath\expandafter{\the\everymath\ifDisplaystyle\displaystyle\fi}
\newcommand\DS{\displaystyle}

\clearscrheadfoot
\pagestyle{scrheadings}
\thispagestyle{empty}
\ohead{\Classe}
\ihead{\thepage/\pageref*{LastPage}}

\setlist[itemize,1]{label=\textbullet}
\setlist[itemize,2]{label=\textbullet}

\ifTwoColumns
  \geometry{margin=1cm,top=2cm,bottom=3cm,foot=1cm}
  \setlist[enumerate]{leftmargin=*}
  \setlist[itemize]{leftmargin=*}
\else
  \geometry{margin=3cm}
\fi

\makeatletter
\let\@author=\relax
\let\@date=\relax
\renewcommand\maketitle{%
    \begin{center}%
        {\sffamily\huge\bfseries\@title}%
        \ifx\@author\relax\else\par\medskip{\itshape\Large\@author}\fi
        \ifx\@date\relax\else\par\bigskip{\large\@date}\fi
    \end{center}\bigskip
    \ifTwoColumns
        \par\begin{multicols*}{2}%
        \AtEndDocument{\end{multicols*}}%
        \setlength{\columnsep}{5mm}
    \fi
}
\makeatother

\newcounter{ParaNum}
\NewDocumentCommand\Para{smo}{%
  \IfBooleanF{#1}{\refstepcounter{ParaNum}}%
  \paragraph{\IfBooleanF{#1}{{\tiny\arabic{ParaNum}~}}#2\IfNoValueF{#3}{ (#3)}}}

\newcommand\I{i}
\newcommand\mi{i}
\def\me{e}

\def\do#1{\expandafter\undef\csname #1\endcsname}
\docsvlist{Ker,sec,csc,cot,sinh,cosh,tanh,coth,th}
\undef\do

\DeclareMathOperator\ch{ch}
\DeclareMathOperator\sh{sh}
\DeclareMathOperator\th{th}
\DeclareMathOperator\coth{coth}
\DeclareMathOperator\cotan{cotan}
\DeclareMathOperator\argch{argch}
\DeclareMathOperator\argsh{argsh}
\DeclareMathOperator\argth{argth}

\let\epsilon=\varepsilon
\let\phi=\varphi
\let\leq=\leqslant
\let\geq=\geqslant
\let\subsetneq=\varsubsetneq
\let\emptyset=\varnothing

\newcommand{\+}{,\;}

\undef\C
\newcommand\ninf{{n\infty}}
\newcommand\N{\mathbb{N}}
\newcommand\Z{\mathbb{Z}}
\newcommand\Q{\mathbb{Q}}
\newcommand\R{\mathbb{R}}
\newcommand\C{\mathbb{C}}
\newcommand\K{\mathbb{K}}
\newcommand\Ns{\N^*}
\newcommand\Zs{\Z^*}
\newcommand\Qs{\Q^*}
\newcommand\Rs{\R^*}
\newcommand\Cs{\C^*}
\newcommand\Ks{\K^*}
\newcommand\Rp{\R^+}
\newcommand\Rps{\R^+_*}
\newcommand\Rms{\R^-_*}
\newcommand{\Rpinf}{\Rp\cup\Acco{+\infty}}

\undef\B
\newcommand\B{\mathscr{B}}

\undef\P
\DeclareMathOperator\P{\mathbb{P}}
\DeclareMathOperator\E{\mathbb{E}}
\DeclareMathOperator\Var{\mathbb{V}}

\DeclareMathOperator*\PetitO{o}
\DeclareMathOperator*\GrandO{O}
\DeclareMathOperator*\Sim{\sim}
\DeclareMathOperator\Tr{tr}
\DeclareMathOperator\Ima{Im}
\DeclareMathOperator\Ker{Ker}
\DeclareMathOperator\Sp{Sp}
\DeclareMathOperator\Diag{diag}
\DeclareMathOperator\Rang{rang}
\DeclareMathOperator*\Coords{Coords}
\DeclareMathOperator*\Mat{Mat}
\DeclareMathOperator\Pass{Pass}
\DeclareMathOperator\Com{Com}
\DeclareMathOperator\Card{Card}
\DeclareMathOperator\Racines{Racines}
\DeclareMathOperator\Vect{Vect}
\DeclareMathOperator\Id{Id}

\newcommand\DerPart[2]{\frac{\partial #1}{\partial #2}}

\def\T#1{{#1}^T}

\def\pa#1{({#1})}
\def\Pa#1{\left({#1}\right)}
\def\bigPa#1{\bigl({#1}\bigr)}
\def\BigPa#1{\Bigl({#1}\Bigr)}
\def\biggPa#1{\biggl({#1}\biggr)}
\def\BiggPa#1{\Biggl({#1}\Biggr)}

\def\pafrac#1#2{\pa{\frac{#1}{#2}}}
\def\Pafrac#1#2{\Pa{\frac{#1}{#2}}}
\def\bigPafrac#1#2{\bigPa{\frac{#1}{#2}}}
\def\BigPafrac#1#2{\BigPa{\frac{#1}{#2}}}
\def\biggPafrac#1#2{\biggPa{\frac{#1}{#2}}}
\def\BiggPafrac#1#2{\BiggPa{\frac{#1}{#2}}}

\def\cro#1{[{#1}]}
\def\Cro#1{\left[{#1}\right]}
\def\bigCro#1{\bigl[{#1}\bigr]}
\def\BigCro#1{\Bigl[{#1}\Bigr]}
\def\biggCro#1{\biggl[{#1}\biggr]}
\def\BiggCro#1{\Biggl[{#1}\Biggr]}

\def\abs#1{\mathopen|{#1}\mathclose|}
\def\Abs#1{\left|{#1}\right|}
\def\bigAbs#1{\bigl|{#1}\bigr|}
\def\BigAbs#1{\Bigl|{#1}\Bigr|}
\def\biggAbs#1{\biggl|{#1}\biggr|}
\def\BiggAbs#1{\Biggl|{#1}\Biggr|}

\def\acco#1{\{{#1}\}}
\def\Acco#1{\left\{{#1}\right\}}
\def\bigAcco#1{\bigl\{{#1}\bigr\}}
\def\BigAcco#1{\Bigl\{{#1}\Bigr\}}
\def\biggAcco#1{\biggl\{{#1}\biggr\}}
\def\BiggAcco#1{\Biggl\{{#1}\Biggr\}}

\def\ccro#1{\llbracket{#1}\rrbracket}
\def\Dcro#1{\llbracket{#1}\rrbracket}

\def\floor#1{\lfloor#1\rfloor}
\def\Floor#1{\left\lfloor{#1}\right\rfloor}

\def\sEnsemble#1#2{\mathopen\{#1\mid#2\mathclose\}}
\def\bigEnsemble#1#2{\bigl\{#1\bigm|#2\bigr\}}
\def\BigEnsemble#1#2{\Bigl\{#1\Bigm|#2\Bigr\}}
\def\biggEnsemble#1#2{\biggl\{#1\biggm|#2\biggr\}}
\def\BiggEnsemble#1#2{\Biggl\{#1\Biggm|#2\Biggr\}}
\let\Ensemble=\bigEnsemble

\newcommand\IntO[1]{\left]#1\right[}
\newcommand\IntF[1]{\left[#1\right]}
\newcommand\IntOF[1]{\left]#1\right]}
\newcommand\IntFO[1]{\left[#1\right[}

\newcommand\intO[1]{\mathopen]#1\mathclose[}
\newcommand\intF[1]{\mathopen[#1\mathclose]}
\newcommand\intOF[1]{\mathopen]#1\mathclose]}
\newcommand\intFO[1]{\mathopen[#1\mathclose[}

\newcommand\Fn[3]{#1\colon#2\to#3}
\newcommand\CC[1]{\mathscr{C}^{#1}}
\newcommand\D{\mathop{}\!\mathrm{d}}

\newcommand\longto{\longrightarrow}

\undef\M
\newcommand\M[3]{\mathrm{#1}_{#2}\pa{#3}}
\newcommand\MnR{\M{M}{n}{\R}}
\newcommand\MnC{\M{M}{n}{\C}}
\newcommand\MnK{\M{M}{n}{\K}}
\newcommand\GLnR{\M{GL}{n}{\R}}
\newcommand\GLnC{\M{GL}{n}{\C}}
\newcommand\GLnK{\M{GL}{n}{\K}}
\newcommand\DnR{\M{D}{n}{\R}}
\newcommand\DnC{\M{D}{n}{\C}}
\newcommand\DnK{\M{D}{n}{\K}}
\newcommand\SnR{\M{S}{n}{\R}}
\newcommand\AnR{\M{A}{n}{\R}}
\newcommand\OnR{\M{O}{n}{\R}}
\newcommand\SnRp{\mathrm{S}_n^+(\R)}
\newcommand\SnRpp{\mathrm{S}_n^{++}(\R)}

\newcommand\LE{\mathscr{L}(E)}
\newcommand\GLE{\mathscr{GL}(E)}
\newcommand\SE{\mathscr{S}(E)}
\renewcommand\OE{\mathscr{O}(E)}

\newcommand\ImplD{$\Cro\Rightarrow$}
\newcommand\ImplR{$\Cro\Leftarrow$}
\newcommand\InclD{$\Cro\subset$}
\newcommand\InclR{$\Cro\supset$}
\newcommand\notInclD{$\Cro{\not\subset}$}
\newcommand\notInclR{$\Cro{\not\supset}$}

\newcommand\To[1]{\xrightarrow[#1]{}}
\newcommand\Toninf{\To{\ninf}}

\newcommand\Norm[1]{\|#1\|}
\newcommand\Norme{{\Norm{\cdot}}}

\newcommand\Int[1]{\mathring{#1}}
\newcommand\Adh[1]{\overline{#1}}

\newcommand\Uplet[2]{{#1},\dots,{#2}}
\newcommand\nUplet[3]{(\Uplet{{#1}_{#2}}{{#1}_{#3}})}

\newcommand\Fonction[5]{{#1}\left|\begin{aligned}{#2}&\;\longto\;{#3}\\{#4}&\;\longmapsto\;{#5}\end{aligned}\right.}

\DeclareMathOperator\orth{\bot}
\newcommand\Orth[1]{{#1}^\bot}
\newcommand\PS[2]{\langle#1,#2\rangle}

\newcommand{\Tribu}{\mathscr{T}}
\newcommand{\Part}{\mathcal{P}}
\newcommand{\Pro}{\bigPa{\Omega,\Tribu}}
\newcommand{\Prob}{\bigPa{\Omega,\Tribu,\P}}

\newcommand\DEMO{$\spadesuit$}
\newcommand\DUR{$\spadesuit$}

\newenvironment{psmallmatrix}{\left(\begin{smallmatrix}}{\end{smallmatrix}\right)}

% -----------------------------------------------------------------------------

\Displaystyletrue

\newcommand\Ei[1]{\me^{#1\I\pi}}
\DeclareMathOperator\Signe{signe}

\begin{document}
\title{Formulaire de trigonom\'etrie}
\maketitle

\section{Relations entre les fonctions trigonom\'etriques}

\begin{itemize}
\item
  $\cos^2 x + \sin^2 x = 1$
\item
  $\tan x   = \frac{\sin x}{\cos x}$
\item
  $\cotan x = \frac{\cos x}{\sin x}$
\item
  $1 + \tan^2 x   = \frac{1}{\cos^2 x}$
\item
  $1 + \cotan^2 x = \frac{1}{\sin^2 x}$
\end{itemize}

\section{Formules d'Euler}

\begin{itemize}
\item
  $\cos x     = \frac{\me^{\I x} + \me^{-\I x}}{2}$
\item
  $\sin x     = \frac{\me^{\I x} - \me^{-\I x}}{2\I}$
\item
  $\me^{\I x} = \cos x + \I \sin x$
\end{itemize}

Exemple d'utilisation: lin\'eariser $\cos^2 (x) \sin^4 (x)$.
\[ \begin{aligned}
    & \cos^2 (x) \sin^4 (x) \\
    & \quad = \Pafrac{\Ei{}+\Ei{-}}{2}^2\cdot\Pafrac{\Ei{}-\Ei{-}}{2\I}^4 \\
    & \begin{split}
      \quad ={} &\frac{1}{2^6} \Pa{\Ei{2} + 2 +\Ei{-2} } \\
    &\quad \cdot{} \Pa{\Ei{4} - 4\Ei{2} + 6 - 4\Ei{-2} +\Ei{-4} } \end{split} \\
    & \begin{split}
      \quad ={} & \frac{1}{2^6} \Bigl( \Ei{6} - 2\Ei{4} -\Ei{2} + 4 \\
    &\qquad -\Ei{-2} - 2\Ei{-4} +\Ei{-6} \Bigr) \end{split} \\
    & \quad = \frac{\cos 6x - 2 \cos 4x - \cos 2x + 2}{32}
\end{aligned} \]

\section{Sym\'etries}

\begin{itemize}
\item
  $\cos(-x)          = \cos x$
\item
  $\sin(-x)          = -\sin x$
\item
  $\tan(-x)          = -\tan x$
\item
  $\cos(\pi+x)         = -\cos x$
\item
  $\sin(\pi+x)         = -\sin x$
\item
  $\tan(\pi+x)         = \tan x$
\item
  $\cos(x+n\pi)        = (-1)^n \cos x$
\item
  $\cos(n\pi)          = (-1)^n$
\item
  $\sin(x+n\pi)        = (-1)^n \sin x$
\item
  $\sin(n\pi)          = 0$
\item
  $\cos(\pi-x)         = -\cos x$
\item
  $\sin(\pi-x)         = \sin x$
\item
  $\cos\BigPa{\frac\pi2 - x} = \sin x$
\item
  $\sin\BigPa{\frac\pi2 - x} = \cos x$
\item
  $\tan\BigPa{\frac\pi2 - x} = \frac{1}{\tan x} = \cotan x$
\item
  $\cos\BigPa{\frac\pi2 + x} = -\sin x = \cos'(x)$
\item
  $\sin\BigPa{\frac\pi2 + x} = \cos x = \sin'(x)$
\item
  $\tan\BigPa{\frac\pi2 + x} = -\frac{1}{\tan x} = -\cotan x$
\end{itemize}

\section{Formules de duplication}

\begin{itemize}
\item
  $\cos 2\theta= \cos^2\theta- \sin^2\theta= 2\cos^2\theta- 1 = 1 - 2\sin^2\theta$
\item
  $\sin 2\theta= 2 \cos\theta\sin\theta$
\item
  $\tan 2\theta= \frac{2\tan\theta}{1-\tan^2\theta}$
\item
  $\cos^2\theta= \frac{1+\cos 2\theta}{2}$
\item
  $\sin^2\theta= \frac{1-\cos 2\theta}{2}$
\item
  $1 + \cos\theta= 2 \cos^2 \BigPa{\frac\theta{2}}$
\item
  $1 - \cos\theta= 2 \sin^2 \BigPa{\frac\theta{2}}$
\end{itemize}

\section{Formules d'addition}

\begin{itemize}
\item
  $\cos(a+b)       = \cos a \cos b - \sin a \sin b$
\item
  $\cos(a-b)       = \cos a \cos b + \sin a \sin b$
\item
  $\sin(a+b)       = \sin a \cos b + \cos a \sin b$
\item
  $\sin(a-b)       = \sin a \cos b - \cos a \sin b$
\item
  $\tan(a+b)       = \frac{\tan a + \tan b}{1 - \tan(a)\tan(b)}$
\item
  $\tan(a-b)       = \frac{\tan a - \tan b}{1 + \tan(a)\tan(b)}$
\item
  $\sin p + \sin q =  2 \sin\BigPa{\frac{p+q}{2}} \cos\BigPa{\frac{p-q}{2}}$
\item
  $\sin p - \sin q =  2 \cos\BigPa{\frac{p+q}{2}} \sin\BigPa{\frac{p-q}{2}}$
\item
  $\cos p + \cos q =  2 \cos\BigPa{\frac{p+q}{2}} \cos\BigPa{\frac{p-q}{2}}$
\item
  $\cos p - \cos q = -2 \sin\BigPa{\frac{p+q}{2}} \sin\BigPa{\frac{p-q}{2}}$
\end{itemize}

\section{Tangente de l'arc moiti\'e}

\begin{itemize}
\item
  $t     = \tan\BigPa{\frac\theta{2}}$
\item
  $\cos\theta{} = \frac{1-t^2}{1+t^2}$
\item
  $\sin\theta{} = \frac{2t}{1+t^2}$
\item
  $\tan\theta{} = \frac{2t}{1-t^2}$
\end{itemize}

\section{Valeurs remarquables}

\begin{itemize}
\item
  $\sin 0 = 0$, $\cos 0 = 1$, $\tan 0 = 0$
\item
  $\sin\BigPa{\frac\pi6} = \frac12$, $\cos\BigPa{\frac\pi6} = \frac{\sqrt3}{2}$, $\tan\BigPa{\frac\pi6} = \frac{1}{\sqrt3}$
\item
  $\sin\BigPa{\frac\pi4} = \frac{\sqrt2}2$, $\cos\BigPa{\frac\pi4} = \frac{\sqrt2}{2}$, $\tan\BigPa{\frac\pi4} = 1$
\item
  $\sin\BigPa{\frac\pi3} = \frac{\sqrt3}{2}$, $\cos\BigPa{\frac\pi3} = \frac12$, $\tan\BigPa{\frac\pi3} = \sqrt3$
\item
  $\sin\BigPa{\frac\pi2} = 1$, $\cos\BigPa{\frac\pi2} = 0$, $\tan\BigPa{\frac\pi2} = \infty$
\end{itemize}

Bonus:

\begin{itemize}
\item
  $\cos\BigPa{\frac{2\pi}{5}} = \frac{\sqrt5-1}{4}$
\item
  La valeur de $\cos\BigPa{\frac{2\pi}{17}}$ que Gauss a trouv\'ee \`a 19 ans:
\end{itemize}
\[ \begin{aligned}
    & 16\cos\Pafrac{2\pi}{17} = -1 +\sqrt{17} +\sqrt{34-2\sqrt{17}} \\
    & \qquad + 2\sqrt{17 + 3\sqrt{17} -\sqrt{34-2\sqrt{17}} - 2\sqrt{34+2\sqrt{17}}}
\end{aligned} \]

\section{\'Equations trigonom\'etriques}

\begin{itemize}
\item
  $\cos x = \cos y$
  si et seulement si $(\exists k\in \Z\+ x = y + 2k\pi)$ ou $(\exists l\in \Z\+ x = -y + 2l\pi)$.
\item
  $\sin x = \sin y$
  si et seulement si $(\exists k\in \Z\+ x = y + 2k\pi)$ ou $(\exists l\in \Z\+ x =\pi-y + 2l\pi)$
\end{itemize}

\section{Fonctions r\'eciproques}

\subsection{Arc cosinus}

Soit $\Fonction{c}{\intF{0,\pi}}{\intF{-1,1}}{x}{\cos x.}$

La fonction $c$ est bijective, on note $\arccos$ sa r\'eciproque.

\begin{itemize}
\item
  $\arccos$ est une fonction continue sur $\intF{-1,1}$
  et de classe $\CC\infty$ sur $\intO{-1,1}$;
\item
  $\forall x\in\intF{-1,1}$, $\cos\bigl(\arccos(x)\bigr) = x$;
\item
  $\forall x\in\intF{0,\pi}$, $\arccos\bigl(\cos(x)\bigr) = x$,
  mais ce n'est pas vrai pour $x\in \R\setminus\intF{0,\pi}$.
\end{itemize}

\subsection{Arc sinus}

Soit $\Fonction{s}{\left[ -\frac\pi2,\frac\pi2 \right]}{\intF{-1,1}}{x}{\sin x.}$

La fonction $s$ est bijective, on note $\arcsin$ sa r\'eciproque.

\begin{itemize}
\item
  $\arcsin$ est une fonction continue sur $\intF{-1,1}$
  et de classe $\CC\infty$ sur $\intO{-1,1}$;
\item
  $\forall x\in\intF{-1,1}$, $\sin\bigl( \arcsin(x) \bigr) = x$;
\item
  $\forall x\in\intF{-\pi/2,\pi/2}$, $\arcsin\bigl( \sin(x) \bigr) = x$,
  mais ce n'est pas vrai pour $x\in \R\setminus\intF{-\pi/2,\pi/2}$.
\end{itemize}

\subsection{Arc tangente}

Soit $\Fonction{t}{\IntO{-\frac\pi2,\frac\pi2}}\R{x}{\tan x.}$

La fonction $t$ est bijective, on note $\arctan$ sa r\'eciproque.

\begin{itemize}
\item
  $\arctan$ est une fonction de classe $\CC\infty$ sur $\R$;
\item
  $\forall x\in \R$, $\tan\bigl( \arctan(x) \bigr) = x$;
\item
  $\forall x\in\intO{-\pi/2,\pi/2}$, $\arctan\bigl( \tan(x) \bigr) = x$,
  mais ce n'est pas vrai pour $x\in \R\setminus\intO{-\pi/2,\pi/2}$.
\end{itemize}

\subsection{Identit\'es}

\begin{itemize}
\item
  $\arccos(x) + \arcsin(x)       =  \frac\pi2$
\item
  $\arctan(x) + \arctan\BigPa{\frac1x} =  \Signe(x)\frac\pi2$
\end{itemize}

\section{D\'eriv\'ees}

\begin{itemize}
\item
  $\cos'(x)    = -\sin x$
\item
  $\sin'(x)    = \cos x$
\item
  $\tan'(x)    = 1 + \tan^2 x = \frac{1}{\cos^2 x}$
\item
  $\cotan'(x)  = -1 - \cotan^2 x = -\frac{1}{\sin^2 x}$
\item
  $\arccos'(x) = \frac{-1}{\sqrt{1-x^2}}$
\item
  $\arcsin'(x) = \frac{1}{\sqrt{1-x^2}}$
\item
  $\arctan'(x) = \frac{1}{1+x^2}$
\end{itemize}

\section{Trigonom\'etrie hyperbolique}

Il faut conna\^itre: $\ch^2 x - \sh^2 x = 1$.
Les autres identit\'es se d\'eduisent de
$\cos(\I z) = \ch z$ et $\sin(\I z) = \I \sh z$.

Par exemple,
\[ \begin{aligned} \ch 2\theta{} &= \cos(2\I\theta)
    = 1 - 2\sin^2(\I\theta)
    = 1 + 2\sh^2\theta{} \\
    \th 2\theta{} &= -\I\tan(2\I\theta)
    = -\I\cdot\frac{2\tan(\I\theta)}{1 - \tan^2(\I\theta)}
    = \frac{2\th\theta}{1 + \th^2\theta}
\end{aligned} \]

\section{Polyn\^omes de Tchebychev}

\subsection{Polyn\^omes de premi\`ere esp\`ece}

\begin{itemize}
\item
  $T_0(X)      = 1$
\item
  $T_1(X)      = X$
\item
  $T_{n+2}(X)  = 2X \, T_{n+1}(X) - T_n(X)$
\item
  $T_n(X)$ est un polyn\^ome de degr\'e $n$ et de coefficient dominant $2^{n-1}$.
\item
  $T_n(\cos\theta)  = \cos(n\theta)$
\item
  $T_n(\ch\theta) = \ch(n\theta)$
\end{itemize}

Par exemple, $T_3(X) = 4X^3 - 3X$ donc \[ \cos 3x = 4\cos^3 x - 3 \cos x. \]

\subsection{Polyn\^omes de seconde esp\`ece}

\begin{itemize}
\item
  $U_0(X)      = 1$
\item
  $U_1(X)      = 2X$
\item
  $U_{n+2}(X)  = 2X \, U_{n+1}(X) - U_n(X)$
\item
  $U_n(X)$ est un polyn\^ome de degr\'e $n$ et de coefficient dominant $2^n$.
\item
  $U_n(\cos\theta)  \sin\theta{}  = \sin\bigl((n+1)\theta\bigr)$
\item
  $U_n(\ch\theta) \sh\theta{} = \sh\bigl((n+1)\theta\bigr)$
\end{itemize}

\end{document}
