% autogenerated by ytex.rs

\documentclass{scrartcl}

\usepackage[francais]{babel}
\usepackage{geometry}
\usepackage{scrpage2}
\usepackage{lastpage}
\usepackage{ragged2e}
\usepackage{multicol}
\usepackage{etoolbox}
\usepackage{xparse}
\usepackage{enumitem}
\usepackage{csquotes}
\usepackage{amsmath}
\usepackage{amsfonts}
\usepackage{amssymb}
\usepackage{mathrsfs}
\usepackage{stmaryrd}
\usepackage{dsfont}
\usepackage{eurosym}
\usepackage{numprint}
\usepackage[most]{tcolorbox}
\usepackage{tikz}
\usepackage{tkz-tab}
\usepackage[unicode]{hyperref}
\usepackage[ocgcolorlinks]{ocgx2}

\let\ifTwoColumns\iftrue
\def\Classe{$\Psi$2019--2020}

% reproducible builds
% LuaTeX: \pdfvariable suppressoptionalinfo 1023 \relax
\pdfinfoomitdate=1
\pdftrailerid{}

\newif\ifDisplaystyle
\everymath\expandafter{\the\everymath\ifDisplaystyle\displaystyle\fi}
\newcommand\DS{\displaystyle}

\clearscrheadfoot
\pagestyle{scrheadings}
\thispagestyle{empty}
\ohead{\Classe}
\ihead{\thepage/\pageref*{LastPage}}

\setlist[itemize,1]{label=\textbullet}
\setlist[itemize,2]{label=\textbullet}

\ifTwoColumns
  \geometry{margin=1cm,top=2cm,bottom=3cm,foot=1cm}
  \setlist[enumerate]{leftmargin=*}
  \setlist[itemize]{leftmargin=*}
\else
  \geometry{margin=3cm}
\fi

\makeatletter
\let\@author=\relax
\let\@date=\relax
\renewcommand\maketitle{%
    \begin{center}%
        {\sffamily\huge\bfseries\@title}%
        \ifx\@author\relax\else\par\medskip{\itshape\Large\@author}\fi
        \ifx\@date\relax\else\par\bigskip{\large\@date}\fi
    \end{center}\bigskip
    \ifTwoColumns
        \par\begin{multicols*}{2}%
        \AtEndDocument{\end{multicols*}}%
        \setlength{\columnsep}{5mm}
    \fi
}
\makeatother

\newcounter{ParaNum}
\NewDocumentCommand\Para{smo}{%
  \IfBooleanF{#1}{\refstepcounter{ParaNum}}%
  \paragraph{\IfBooleanF{#1}{{\tiny\arabic{ParaNum}~}}#2\IfNoValueF{#3}{ (#3)}}}

\newcommand\I{i}
\newcommand\mi{i}
\def\me{e}

\def\do#1{\expandafter\undef\csname #1\endcsname}
\docsvlist{Ker,sec,csc,cot,sinh,cosh,tanh,coth,th}
\undef\do

\DeclareMathOperator\ch{ch}
\DeclareMathOperator\sh{sh}
\DeclareMathOperator\th{th}
\DeclareMathOperator\coth{coth}
\DeclareMathOperator\cotan{cotan}
\DeclareMathOperator\argch{argch}
\DeclareMathOperator\argsh{argsh}
\DeclareMathOperator\argth{argth}

\let\epsilon=\varepsilon
\let\phi=\varphi
\let\leq=\leqslant
\let\geq=\geqslant
\let\subsetneq=\varsubsetneq
\let\emptyset=\varnothing

\newcommand{\+}{,\;}

\undef\C
\newcommand\ninf{{n\infty}}
\newcommand\N{\mathbb{N}}
\newcommand\Z{\mathbb{Z}}
\newcommand\Q{\mathbb{Q}}
\newcommand\R{\mathbb{R}}
\newcommand\C{\mathbb{C}}
\newcommand\K{\mathbb{K}}
\newcommand\Ns{\N^*}
\newcommand\Zs{\Z^*}
\newcommand\Qs{\Q^*}
\newcommand\Rs{\R^*}
\newcommand\Cs{\C^*}
\newcommand\Ks{\K^*}
\newcommand\Rp{\R^+}
\newcommand\Rps{\R^+_*}
\newcommand\Rms{\R^-_*}
\newcommand{\Rpinf}{\Rp\cup\Acco{+\infty}}

\undef\B
\newcommand\B{\mathscr{B}}

\undef\P
\DeclareMathOperator\P{\mathbb{P}}
\DeclareMathOperator\E{\mathbb{E}}
\DeclareMathOperator\Var{\mathbb{V}}

\DeclareMathOperator*\PetitO{o}
\DeclareMathOperator*\GrandO{O}
\DeclareMathOperator*\Sim{\sim}
\DeclareMathOperator\Tr{tr}
\DeclareMathOperator\Ima{Im}
\DeclareMathOperator\Ker{Ker}
\DeclareMathOperator\Sp{Sp}
\DeclareMathOperator\Diag{diag}
\DeclareMathOperator\Rang{rang}
\DeclareMathOperator*\Coords{Coords}
\DeclareMathOperator*\Mat{Mat}
\DeclareMathOperator\Pass{Pass}
\DeclareMathOperator\Com{Com}
\DeclareMathOperator\Card{Card}
\DeclareMathOperator\Racines{Racines}
\DeclareMathOperator\Vect{Vect}
\DeclareMathOperator\Id{Id}

\newcommand\DerPart[2]{\frac{\partial #1}{\partial #2}}

\def\T#1{{#1}^T}

\def\pa#1{({#1})}
\def\Pa#1{\left({#1}\right)}
\def\bigPa#1{\bigl({#1}\bigr)}
\def\BigPa#1{\Bigl({#1}\Bigr)}
\def\biggPa#1{\biggl({#1}\biggr)}
\def\BiggPa#1{\Biggl({#1}\Biggr)}

\def\pafrac#1#2{\pa{\frac{#1}{#2}}}
\def\Pafrac#1#2{\Pa{\frac{#1}{#2}}}
\def\bigPafrac#1#2{\bigPa{\frac{#1}{#2}}}
\def\BigPafrac#1#2{\BigPa{\frac{#1}{#2}}}
\def\biggPafrac#1#2{\biggPa{\frac{#1}{#2}}}
\def\BiggPafrac#1#2{\BiggPa{\frac{#1}{#2}}}

\def\cro#1{[{#1}]}
\def\Cro#1{\left[{#1}\right]}
\def\bigCro#1{\bigl[{#1}\bigr]}
\def\BigCro#1{\Bigl[{#1}\Bigr]}
\def\biggCro#1{\biggl[{#1}\biggr]}
\def\BiggCro#1{\Biggl[{#1}\Biggr]}

\def\abs#1{\mathopen|{#1}\mathclose|}
\def\Abs#1{\left|{#1}\right|}
\def\bigAbs#1{\bigl|{#1}\bigr|}
\def\BigAbs#1{\Bigl|{#1}\Bigr|}
\def\biggAbs#1{\biggl|{#1}\biggr|}
\def\BiggAbs#1{\Biggl|{#1}\Biggr|}

\def\acco#1{\{{#1}\}}
\def\Acco#1{\left\{{#1}\right\}}
\def\bigAcco#1{\bigl\{{#1}\bigr\}}
\def\BigAcco#1{\Bigl\{{#1}\Bigr\}}
\def\biggAcco#1{\biggl\{{#1}\biggr\}}
\def\BiggAcco#1{\Biggl\{{#1}\Biggr\}}

\def\ccro#1{\llbracket{#1}\rrbracket}
\def\Dcro#1{\llbracket{#1}\rrbracket}

\def\floor#1{\lfloor#1\rfloor}
\def\Floor#1{\left\lfloor{#1}\right\rfloor}

\def\sEnsemble#1#2{\mathopen\{#1\mid#2\mathclose\}}
\def\bigEnsemble#1#2{\bigl\{#1\bigm|#2\bigr\}}
\def\BigEnsemble#1#2{\Bigl\{#1\Bigm|#2\Bigr\}}
\def\biggEnsemble#1#2{\biggl\{#1\biggm|#2\biggr\}}
\def\BiggEnsemble#1#2{\Biggl\{#1\Biggm|#2\Biggr\}}
\let\Ensemble=\bigEnsemble

\newcommand\IntO[1]{\left]#1\right[}
\newcommand\IntF[1]{\left[#1\right]}
\newcommand\IntOF[1]{\left]#1\right]}
\newcommand\IntFO[1]{\left[#1\right[}

\newcommand\intO[1]{\mathopen]#1\mathclose[}
\newcommand\intF[1]{\mathopen[#1\mathclose]}
\newcommand\intOF[1]{\mathopen]#1\mathclose]}
\newcommand\intFO[1]{\mathopen[#1\mathclose[}

\newcommand\Fn[3]{#1\colon#2\to#3}
\newcommand\CC[1]{\mathscr{C}^{#1}}
\newcommand\D{\mathop{}\!\mathrm{d}}

\newcommand\longto{\longrightarrow}

\undef\M
\newcommand\M[3]{\mathrm{#1}_{#2}\pa{#3}}
\newcommand\MnR{\M{M}{n}{\R}}
\newcommand\MnC{\M{M}{n}{\C}}
\newcommand\MnK{\M{M}{n}{\K}}
\newcommand\GLnR{\M{GL}{n}{\R}}
\newcommand\GLnC{\M{GL}{n}{\C}}
\newcommand\GLnK{\M{GL}{n}{\K}}
\newcommand\DnR{\M{D}{n}{\R}}
\newcommand\DnC{\M{D}{n}{\C}}
\newcommand\DnK{\M{D}{n}{\K}}
\newcommand\SnR{\M{S}{n}{\R}}
\newcommand\AnR{\M{A}{n}{\R}}
\newcommand\OnR{\M{O}{n}{\R}}
\newcommand\SnRp{\mathrm{S}_n^+(\R)}
\newcommand\SnRpp{\mathrm{S}_n^{++}(\R)}

\newcommand\LE{\mathscr{L}(E)}
\newcommand\GLE{\mathscr{GL}(E)}
\newcommand\SE{\mathscr{S}(E)}
\renewcommand\OE{\mathscr{O}(E)}

\newcommand\ImplD{$\Cro\Rightarrow$}
\newcommand\ImplR{$\Cro\Leftarrow$}
\newcommand\InclD{$\Cro\subset$}
\newcommand\InclR{$\Cro\supset$}
\newcommand\notInclD{$\Cro{\not\subset}$}
\newcommand\notInclR{$\Cro{\not\supset}$}

\newcommand\To[1]{\xrightarrow[#1]{}}
\newcommand\Toninf{\To{\ninf}}

\newcommand\Norm[1]{\|#1\|}
\newcommand\Norme{{\Norm{\cdot}}}

\newcommand\Int[1]{\mathring{#1}}
\newcommand\Adh[1]{\overline{#1}}

\newcommand\Uplet[2]{{#1},\dots,{#2}}
\newcommand\nUplet[3]{(\Uplet{{#1}_{#2}}{{#1}_{#3}})}

\newcommand\Fonction[5]{{#1}\left|\begin{aligned}{#2}&\;\longto\;{#3}\\{#4}&\;\longmapsto\;{#5}\end{aligned}\right.}

\DeclareMathOperator\orth{\bot}
\newcommand\Orth[1]{{#1}^\bot}
\newcommand\PS[2]{\langle#1,#2\rangle}

\newcommand{\Tribu}{\mathscr{T}}
\newcommand{\Part}{\mathcal{P}}
\newcommand{\Pro}{\bigPa{\Omega,\Tribu}}
\newcommand{\Prob}{\bigPa{\Omega,\Tribu,\P}}

\newcommand\DEMO{$\spadesuit$}
\newcommand\DUR{$\spadesuit$}

\newenvironment{psmallmatrix}{\left(\begin{smallmatrix}}{\end{smallmatrix}\right)}


% -----------------------------------------------------------------------------

\Displaystyletrue

\begin{document}
\title{Int\'egration}
\maketitle

Ce chapitre a pour objectif d'\'etendre la notion d'int\'egrale
vue en premi\`ere ann\'ee
pour pouvoir traiter des int\'egrales comme
\[
  \int_{-\infty}^{+\infty} \frac{\D t}{1 + t^2} = \pi{}
  \quad \text{ou encore} \quad
  \int_0^1 \frac{\D x}{\sqrt{x}} = 2.
\]

\Para{Notations}

Comme d'habitude,
\begin{itemize}
\item $\K$ d\'esigne le corps des r\'eels $\R$ ou bien le corps des complexes $\C$;
\item $I$ d\'esigne un intervalle de $\R$ non vide et non r\'eduit \`a un point.
\end{itemize}

\section{Rappels}

Si $\Fn{f}{\intF{a,b}}{\K}$ est \emph{continue par morceaux sur un segment},
on sait d\'efinir l'int\'egrale $\int_a^b f$, dite \enquote{int\'egrale propre}.

Les principales propri\'et\'es de l'int\'egrale sont:
\begin{itemize}
\item la lin\'earit\'e,
\item la relation de Chasles;
\item la positivit\'e (et la croissance),
\item la stricte positivit\'e pour les fonctions continues,
\item l'in\'egalit\'e de la moyenne.
\end{itemize}
Les d\'etails sont dans le chapitre \enquote{R\'evisions d'analyse}.

\Para{Attention}

La fonction \[ f(x) = \begin{cases} 1/x & \text{si } x \in{} \intOF{0,1} \\ \hfil0 & \text{si } x \in{} \intF{-1,0} \end{cases} \]
n'est pas continue par morceaux sur $\intF{-1,1}$.

\section{Int\'egrales impropres}

\Para{D\'efinition}

Soit $I$ un intervalle quelconque de $\R$.
On dit que $f$ est \emph{continue par morceaux sur $I$}
si $f$ est continue par morceaux sur tout segment $K$ inclus dans $I$.
On peut noter que cela correspond \`a la d\'efinition usuelle si $I$ est un segment.

\Para{Propositions}

\begin{itemize}
\item Toute fonction continue de $I$ dans $\K$ est continue par morceaux.
\item Si $\Fn{f}{I}{\K}$ est continue par morceaux, alors $f$ est born\'ee sur tout segment $\intF{a,b} \subset{} I$.
\item L'ensemble $\mathcal{CM}(I,\K)$ des fonctions continues par morceaux de $I$ dans $\K$ est un sous-espace vectoriel
  de l'espace vectoriel $\K^I$ des fonctions de $\K$ dans $I$.
\end{itemize}

\Para{Exemples}

cf. exercice~9.

\Para{D\'efinitions}[int\'egrale impropre]

Soit $\Fn{f}{I}{\K}$ continue par morceaux.
Soit $-\infty{} \leq{} a < b \leq{} +\infty$.
\begin{enumerate}
\item
  Si $I = \intFO{a,b}$, on dit que l'int\'egrale $\int_a^b f$ \emph{converge} si et seulement si $\ell= \lim_{\beta\to b^-}\int_a^\beta f$ existe et est finie. On note alors $\int_a^b f =\ell$.
\item
  Si $I = \intOF{a,b}$, on dit que l'int\'egrale $\int_a^b f$ \emph{converge} si et seulement si $\ell= \lim_{\alpha\to a^+}\int_\alpha^b f$ existe et est finie. On note alors $\int_a^b f =\ell$.
\item
  Si $I = \intO{a,b}$, on dit que l'int\'egrale $\int_a^b f$ \emph{converge} si et seulement s'il existe $c \in{} I$
  tel que les int\'egrales $\int_a^c f$ et $\int_c^b f$ convergent.
  On note alors $\int_a^b f = \int_a^c f + \int_c^b f$.

  \emph{Remarque: } la convergence de l'int\'egrale ainsi que sa valeur ne d\'ependent pas du choix de $c \in{} I$.
\end{enumerate}

\Para{Attention}

Soit $\Fonction{f}{\R}{\R}{x}{x.}$
On a:
\begin{enumerate}
\item
  $f$ est continue par morceaux sur $\R$;
\item
  $\lim_{M \to +\infty}\int_{-M}^M f$ existe et est nulle;
\item
  pourtant, l'int\'egrale $\int_{-\infty}^{+\infty} f$ diverge.
\end{enumerate}

\Para{Exemples}[int\'egrales de r\'ef\'erence]

\begin{enumerate}
\item
  Int\'egrales de Riemann
  \begin{enumerate}
  \item
    $\int_1^{+\infty} \frac{\D x}{x^\alpha}$ converge si et seulement si $\alpha>1$
  \item
    $\int_0^1    \frac{\D x}{x^\alpha}$ converge si et seulement si $\alpha<1$
  \end{enumerate}
\item
  $\int_0^{+\infty} e^{-\alpha x} \D x$ converge si et seulement si $\alpha>0$.
\item
  $\int_0^1 \ln x \D x$ converge
\end{enumerate}

\Para{D\'efinition}

Soit $\Fn{f}{\intO{a,b}}{\K}$ continue par morceaux o\`u $-\infty{} \leq{} a \leq{} b \leq{} +\infty$.
Si l'int\'egrale $\int_a^b f$ converge, on note par d\'efinition
$\int_b^a f = -\int_a^b f$.

\Para{Proposition (propri\'et\'es de l'int\'egrale)}

Soit $I$ un intervalle de $\R$.
\begin{enumerate}
\item
  \emph{Lin\'earit\'e}

  Soit $f$ et $g$ deux fonctions continues par morceaux de $I$ dans $\K$.
  Soit $(\lambda,\mu)\in \K^2$.
  Si les int\'egrales $\int_a^b f$ et $\int_a^b g$ convergent, alors l'int\'egrale $\int_a^b (\lambda f +\mu g)$ converge et est \'egale \`a $\lambda\int_a^b f +\mu\int_a^b g$.
\item
  \emph{Relation de Chasles}

  Soit $\Fn{f}{I}{\K}$ continue par morceaux.
  Si les int\'egrales $\int_a^b f$ et $\int_b^c f$ convergent, alors l'int\'egrale $\int_a^c f$ converge et est \'egale \`a $\int_a^b f +\int_b^c f$.
\item
  \emph{Positivit\'e}

  Soit $\Fn{f}{I}{\Rp}$ continue par morceaux et \emph{$a\leq b$}.
  Si l'int\'egrale $\int_a^b f$ converge, alors $\int_a^b f \geq0$.
\item
  \emph{Stricte positivit\'e}

  Soit $\Fn{f}{I}{\Rp}$ une fonction \emph{continue}.
  Si l'int\'egrale $\int_a^b f$ converge et si $\int_a^b f = 0$, alors $f = \tilde0$.
\item
  \emph{Croissance}

  Soit $f, g \colon I \to\R$ continues par morceaux et \emph{$a\leq b$}.
  On suppose $\forall x\in\intO{a,b} \+ f(x)\leq g(x)$.
  Si les int\'egrales $\int_a^b f$ et $\int_a^b g$ convergent, alors $\int_a^b f\leq\int_a^b g$.
\item
  \emph{In\'egalit\'e de la moyenne}

  Soit $\Fn fI\K$ continue par morceaux et \emph{$a\leq b$}.
  Si les int\'egrales $\int_a^b f$ et $\int_a^b \Abs{f}$ convergent, alors $\left| \int_a^b f \right| \leq\int_a^b \Abs{f}$.
  Sans l'hypoth\`ese $a\leq b$, il faudrait \'ecrire:
  $\left| \int_a^b f \right| \leq\left| \int_a^b \Abs{f} \right|$.
\end{enumerate}

\Para{Th\'eor\`eme}[changement de variables]

Soit $\Fn{f}{\intO{a,b}}{\K}$ une fonction continue par morceaux, $\Fn{\varphi}{\intO{\alpha,\beta}}{\intO{a,b}}$ une bijection strictement croissante de classe $\CC1$. Alors l'int\'egrale
$\int_\alpha^\beta(f\circ \varphi)\varphi'$ est convergente si et seulement si
$\int_a^b f$ est convergente et, si tel est le cas, elles sont \'egales.

\Para{Proposition}[int\'egration par parties]

Soit $f$ et $g$ deux fonctions $\intO{a,b} \to\K$ d\'erivables telles que $f'$ et $g'$ sont continues par morceaux. On suppose que $fg$ a une limite en $a$ et en $b$.
Alors les int\'egrales $\int_a^b fg'$ et $\int_a^b f'g$ sont de m\^eme nature.
De plus, si elles convergent, on a
\[ \int_a^b fg' = \biggl[fg\biggr]_a^b -\int_a^b f'g \]
o\`u $\Bigl[fg\Bigr]_a^b = \lim\limits_{x \to b^-} \bigl( f(x)g(x) \bigr) - \lim\limits_{x \to a^+} \bigl( f(x)g(x) \bigr)$.

\section{Cas des fonctions positives}

\Para{Th\'eor\`eme}

Soit $f$, $\Fn{g}{\intFO{a,b}}{\Rp}$ continues par morceaux.
On suppose que:
\[ \tag{$H$} \forall x\in\intFO{a,b} \+ f(x)\leq g(x). \]
Alors:

\begin{itemize}
\item
  si l'int\'egrale $\int_a^b g$ converge, alors l'int\'egrale $\int_a^b f$ converge \'egalement;
\item
  si l'int\'egrale $\int_a^b f$ diverge, alors l'int\'egrale $\int_a^b g$ diverge \'egalement.
\end{itemize}

Le r\'esultat est encore vrai si on remplace $(H)$ par
\[ \tag{$H'$} f(x) = \GrandO \bigPa{g(x)} \text{ au voisinage de $b$}. \]

\Para{Th\'eor\`eme}

Soit $f$, $\Fn{g}{\intFO{a,b}}{\Rp}$ continues par morceaux.
On suppose que $f(x) \sim g(x)$ au voisinage de $b$.
Alors les int\'egrales $\int_a^b f$ et $\int_a^b g$ sont de m\^eme nature.

\Para{Corollaire}

Soit $f$, $\Fn{g}{\intFO{a,b}}{\R}$ continues par morceaux.
On suppose que:

\begin{itemize}
\item
  $f(x) \sim g(x)$ au voisinage de $b$;
\item
  $g$ est de signe constant au voisinage de $b$.
  Autrement dit, l'une des deux propositions suivantes est vraie:
  \begin{itemize}
  \item
    il existe $c\in\intFO{a,b}$ tel que pour tout $x\in\intFO{c,b}$, on ait $g(x) \geq{} 0$,
  \item
    il existe $c\in\intFO{a,b}$ tel que pour tout $x\in\intFO{c,b}$, on ait $g(x) \leq{} 0$.
  \end{itemize}
\end{itemize}

Alors les int\'egrales $\int_a^b f$ et $\int_a^b g$ sont de m\^eme nature.

\section{Int\'egrales absolument convergentes}

\Para{D\'efinition}

Soit $I$ un intervalle de $\R$ et $\Fn fI\K$ une fonction continue par morceaux.

On dit que l'int\'egrale $\int_a^b f$ est \emph{absolument convergente} si et seulement si l'int\'egrale $\int_a^b \Abs{f(x)} \D x$ converge.

\Para{Th\'eor\`eme}

Une int\'egrale absolument convergente est convergente.

\Para{D\'efinition}

Une int\'egrale convergente mais non absolument convergente est dite \emph{semi-convergente}.

\Para{Exemple}

L'int\'egrale $\int_0^{+\infty} \frac{\sin x}{x} \D x$ est semi-convergente.

\Para{Proposition}

Soit $f$ et $g$ deux fonctions continues par morceaux sur $\intFO{a,b}$.
Si $f(x) = \GrandO \bigPa{g(x)}$ au voisinage de~$b^-$ et si $\int_a^b g$ converge absolument, alors $\int_a^b f$ converge.

On peut d\'eduire de ceci les r\`egles suivantes:

\Para{Corollaire}

Soit $\Fn{f}{\intFO{a,+\infty}}{\K}$ continue par morceaux.
S'il existe $\alpha> 1$ tel que $x^\alpha f(x) \to 0$ quand $x \to +\infty$,
alors l'int\'egrale $\int_a^{+\infty} f$ converge.

\Para{Corollaire}

Soit $\Fn{f}{\intOF{0,a}}{\K}$ continue par morceaux.
S'il existe $\alpha< 1$ tel que $x^\alpha f(x) \to 0$ quand $x \to 0^+$,
alors l'int\'egrale $\int_0^a f$ converge.

\section{Lien avec les s\'eries num\'eriques}

\Para{Remarque}

La th\'eorie des int\'egrales impropres ressemble beaucoup \`a celle des s\'eries num\'eriques.
Il y a une diff\'erence notable:

\Para{Attention}

Il existe des fonctions $\Fn f\Rp\Rp$ continues
telles que $\int_0^{+\infty} f$ converge mais $f$ ne tend pas vers 0 en $+\infty$.

\Para{Th\'eor\`eme}[comparaison s\'erie-int\'egrale]

Soit $f$ une fonction $\intFO{a,+\infty} \to \Rp$ continue par morceaux d\'ecroissante.
Alors l'int\'egrale $\int_a^{+\infty} f(x) \D x$ et la s\'erie $\sum_{n\geq a} f(n)$ sont de m\^eme nature.

\section{Int\'egrabilit\'e}

\Para{Proposition-D\'efinition}

Soit $\Fn fI\K$ continue par morceaux.
Notons $a$ et $b$ les extr\'emit\'es de $I$, \'eventuellement $\pm\infty$.
Les conditions suivantes sont \'equivalentes:
\begin{enumerate}
\item
  l'int\'egrale $\int_a^b f$ est absolument convergente;
\item
  il existe $M\in\Rp$ tel que pour tout segment $[\alpha,\beta]\subset I$, on a $\int_\alpha^\beta{} \Abs{f} \leq{} M$.
\end{enumerate}

Si ces conditions sont v\'erifi\'ees,
on dit que $f$ est \emph{int\'egrable sur $I$}.

\Para{Proposition}

Soit $f$ et $g$ deux fonctions continues par morceaux sur $I$ telles que $\Abs{f}\leq\Abs{g}$ sur $I$.
Si $g$ est int\'egrable sur $I$, alors $f$ est \'egalement int\'egrable sur $I$.

\Para{Proposition}[propri\'et\'es de l'int\'egrale]

Soit $I$ et $J$ deux intervalles de $\R$.

\begin{enumerate}
\item
  \emph{Lin\'earit\'e}

  Soit $f$, $\Fn gI\K$ deux fonctions continues par morceaux int\'egrables sur $I$.
  Soit $(\lambda,\mu)\in \K^2$.
  Alors $\lambda f +\mu g$ est int\'egrable sur $I$ et
  $\int_I (\lambda f +\mu g) =\lambda\int_I f +\mu\int_I g$.
\item
  \emph{Relation de Chasles}

  Soit $\Fn{f}{I\cup J}{\K}$ continue par morceaux, int\'egrable sur $I$ et sur $J$.
  Alors $f$ est int\'egrable sur $I\cup J$.
  Si de plus, $I\cap J$ est vide ou r\'eduit \`a un point, on a:
  $\int_{I\cup J} f =\int_I f +\int_J f$.
\item
  \emph{Positivit\'e}

  Soit $\Fn fI\Rp$ continue par morceaux int\'egrable sur $I$.
  Alors $\int_I f\geq0$.
\item
  \emph{Stricte positivit\'e}

  Soit $\Fn{f}{I}{\R}$ une fonction \emph{continue} et int\'egrable sur $I$.
  Si $\int_I \Abs{f} = 0$, alors $f = \tilde0$.
\item
  \emph{Croissance}

  Soit $f$, $\Fn{g}{I}{\R}$ continues par morceaux int\'egrables sur $I$.
  Si $\forall x\in I\+ f(x)\leq g(x)$, alors $\int_I f\leq\int_I g$.
\item
  \emph{In\'egalit\'e de la moyenne}

  Soit $\Fn fI\K$ continue par morceaux int\'egrable sur $I$.
  Alors $\left| \int_I f \right| \leq\int_I \Abs{f}$.
\end{enumerate}

\Para{Th\'eor\`eme}[changement de variables]

Soit $\Fn fI\K$ continue par morceaux int\'egrable sur $I$ et $\Fn\varphi JI$ une \emph{bijection} de classe~$\CC1$.
Alors la fonction $(f\circ \varphi)\cdot\Abs{\varphi'}$ est int\'egrable sur $J$ et:
\[ \int_I f =\int_J (f\circ \varphi)\cdot\Abs{\varphi'} \]

\Para{Remarque}

Cette formule n'est autre que la formule usuelle
de changement de variables $y = \varphi(x)$:
\[ \int_I f(y) \D y =\int_J f(\varphi(x))\cdot\Abs{\varphi'(x)} \D x \]
La pr\'esence de la valeur absolue vient du fait que $\varphi$ est soit croissante, soit d\'ecroissante (et change alors l'ordre des bornes).

\section{Exercices}

% -----------------------------------------------------------------------------
\par\pagebreak[1]\par
\paragraph{Exercice 1 (int\'egrales de Bertrand)}%
\hfill{\tiny 4776}%
\begingroup~

On se propose de montrer que l'int\'egrale
\[ \int_2^{+\infty} \frac{\D x}{x^\alpha\ln^\beta x}, \]
o\`u $(\alpha,\beta)\in \R^2$ converge si et seulement si
\[ \alpha>1 \quad \text{ou} \quad (\alpha=1 \text{ et } \beta>1). \]

Soit $\Fonction{f}{\intFO{2,+\infty}}{\R}{x}{\frac{1}{x^\alpha\ln^\beta x}}$

\begin{enumerate}
\item
  V\'erifier que $f$ est continue et positive sur $\intFO{2,+\infty}$.
\item
  On suppose $\alpha{} > 1$.
  On choisit alors un $\gamma$ tel que $1 < \gamma{} < \alpha$.

  \begin{enumerate}
  \item
    Montrer que $f(x) = \PetitO\BigPa{\frac{1}{x^\gamma}}$ quand $x \to +\infty$.
  \item
    En d\'eduire que l'int\'egrale $\int_2^{+\infty} f$ converge.
  \end{enumerate}
\item
  On suppose $\alpha{} < 1$.
  On choisit alors un $\gamma$ tel que $\alpha{} < \gamma{} < 1$.

  \begin{enumerate}
  \item
    Montrer que $\frac{1}{x^\gamma} = \PetitO\bigPa{f(x)}$ quand $x \to +\infty$.
  \item
    En d\'eduire que l'int\'egrale $\int_2^{+\infty} f$ diverge.
  \end{enumerate}
\item
  On suppose $\alpha{} = 1$.
  Faire le changement de variables $y = \ln x$.
  Conclure.
\end{enumerate}
\endgroup

% -----------------------------------------------------------------------------
\par\pagebreak[1]\par
\paragraph{Exercice 2 (int\'egrales de Bertrand, suite)}%
\hfill{\tiny 5755}%
\begingroup~

Montrer que l'int\'egrale
\[ \int_0^{\frac12} \frac{\D x}{x^\alpha(-\ln x)^\beta}, \]
o\`u $(\alpha,\beta)\in \R^2$ converge si et seulement si
\[ \alpha<1 \quad \text{ou} \quad (\alpha=1 \text{ et } \beta>1). \]
\endgroup

% -----------------------------------------------------------------------------
\par\pagebreak[1]\par
\paragraph{Exercice 3}%
\hfill{\tiny 4182}%
\begingroup~

Pour $\alpha>0$, on pose \[ f_\alpha(x) = \frac{\sin x}{x^\alpha}. \]

\begin{enumerate}
\item
  Montrer que l'int\'egrale $\int_1^{+\infty} f_\alpha$ converge pour tout $\alpha>0$.
\item
  Montrer que $f_\alpha$ est int\'egrable sur $\intFO{1,+\infty}$ si et seulement si $\alpha>1$.
\end{enumerate}
\endgroup

% -----------------------------------------------------------------------------
\par\pagebreak[1]\par
\paragraph{Exercice 4}%
\hfill{\tiny 7263}%
\begingroup~

\'Etudier l'existence des int\'egrales suivantes:
\begin{enumerate}
\item
  $\int_0^{+\infty} x^\alpha{} e^{-x} \ln x \D x$
\item
  $\int_0^{+\infty} \frac{\sin x}{1 + e^x + \cos x} \D x$
\item
  $\int_0^1 \frac{\ln x}{(x^\alpha-1)(x+1)^3} \D x$
\item
  $\int_0^1 t^{\alpha-1} (1-t)^{\beta-1} \D t$
\item
  $\int_0^{+\infty} \frac{e^{-x}}{\sqrt{x\ln(1+x)}} \D x$
\item
  $\int_0^1 (-\ln x)^\alpha\D x$
\item
  $\int_0^{+\infty} \frac{e^x}{\sqrt{\sh(\alpha x)}} \D x$
\item
  $\int_0^{+\infty} \cos(e^x) \D x$
\item
  $\int_0^{+\infty} \frac{\sin t}{\sqrt t + \cos t} \D t$
\item
  $\int_0^{+\infty} \ln\left(1 + \frac{\sin x}{x^\alpha}\right) \D x$ o\`u $\alpha>0$
\item
  $\int_0^{+\infty} \frac{\sqrt{x}\sin(1/x^2)}{\ln(1+x)} \D x$
\end{enumerate}
\endgroup

% -----------------------------------------------------------------------------
\par\pagebreak[1]\par
\paragraph{Exercice 5}%
\hfill{\tiny 1112}%
\begingroup~

Justifier l'existence et calculer les int\'egrales suivantes:

\begin{enumerate}
\item
  $\int_a^b \frac{\D x}{\sqrt{(x-a)(b-x)}}$
\item
  $\int_0^{+\infty} xe^{-x}\sin x \D x$
\item
  $\int_0^1 \frac{\D x}{(1+x)\sqrt{1-x^2}}$
\item
  $\int_2^{+\infty} \frac{\D x}{x^2\sqrt{x^2-4}}$
\item
  $\int_0^{\frac\pi2}\sqrt{\tan x} \D x$
\item
  $\int_0^1 \frac{\ln(1-t^2)}{t^2} \D t$
\item
  $\int_{-\infty}^{+\infty} \frac{\D x}{(x^2+1)\sqrt{x^2+4}}$
\item
  $\int_0^{+\infty} \frac{\D x}{(1+x^2)(1+x^\alpha)}$; poser $t=\frac1x$
\item
  $\int_0^1 \frac{t^n-1}{\ln t} \D t$
\item
  $\int_0^{+\infty} \frac{\sin^3 x}{x^2} \D x$
\item
  $\int_0^{+\infty} \left( \int_x^{+\infty} e^{-t^2} \D t \right) \D x$
\item
  $\int_0^{+\infty} \frac{t^3}{e^t-1} \D t$
\item
  $\int_1^{+\infty} \frac{x-E(x)}{x^2} \D x$
\end{enumerate}
\endgroup

% -----------------------------------------------------------------------------
\par\pagebreak[1]\par
\paragraph{Exercice 6}%
\hfill{\tiny 4937}%
\begingroup~

On cherche \`a calculer l'int\'egrale $I =\int_0^{\frac\pi2} \ln(\sin x) \D x$.

\begin{enumerate}
\item
  Montrer que $I =\int_0^{\frac\pi2} \ln(\cos x) \D x$.
\item
  En d\'eduire que $2I =\int_0^{\frac\pi2} \ln\Pa{\frac{\sin 2x}{2}} \D x$.
\item
  Conclure.
\end{enumerate}
\endgroup

% -----------------------------------------------------------------------------
\par\pagebreak[1]\par
\paragraph{Exercice 7}%
\hfill{\tiny 3827}%
\begingroup~

Soit $\Fn{f}{\Rps}{\R}$ continue.
On suppose que $\lim\limits_{0^+} f =\lambda\in \R$
et $\lim\limits_{+\infty} f = \mu\in \R$.

Soit $(a,b)\in(\Rps)^2$.
\begin{enumerate}
\item
  Montrer que l'int\'egrale $\int_0^{+\infty} \frac{f(bx)-f(ax)}{x} \D x$ converge, et vaut $(\mu-\lambda)\ln\bigl(\frac ba\bigr)$.
\item
  En d\'eduire la valeur des int\'egrales suivantes:
  \begin{enumerate}
  \item
    $\int_0^{+\infty} \frac{e^{-ax}-e^{-bx}}{x} \D x$
  \item
    $\int_0^{+\infty} \frac{\arctan 2x - \arctan x}{x} \D x$
  \item
    $\int_0^1 \frac{x-1}{\ln x} \D x$
  \end{enumerate}
\end{enumerate}
\endgroup

% -----------------------------------------------------------------------------
\par\pagebreak[1]\par
\paragraph{\href{https://psi.miomio.fr/exo/7294.pdf}{Exercice 8}}%
\hfill{\tiny 7294}%
\begingroup~

\begin{enumerate}
\item
  Montrer que $\int_1^x e^t \ln t \D t \sim e^x \ln x$ quand $x\to+\infty$
\item
  Montrer que $\int_x^{+\infty} e^{-t^2} \D t \sim \frac{e^{-x^2}}{2x}$ quand $x\to+\infty$.
\item
  En d\'eduire, pour $0 < a < b$, la valeur de \[ \lim_\ninf \left(\int_a^b e^{-nt^2} \D t\right)^{\frac1n}. \]
\end{enumerate}
\endgroup

% -----------------------------------------------------------------------------
\par\pagebreak[1]\par
\paragraph{Exercice 9}%
\hfill{\tiny 6493}%
\begingroup~

\begin{enumerate}
\item
  Montrer que la fonction partie enti\`ere est continue par morceaux sur $\R$.
\item
  Soit $f(x) = x \floor{1/x}$ si $x>0$ et $f(0) = 0$.

  \begin{enumerate}
  \item
    Montrer que la fonction $f$ est continue en 0.
  \item
    Montrer que $f$ est continue par morceaux sur $\Rps$.
  \item
    Montrer que $f$ n'est pas continue par morceaux sur $\Rp$.
  \end{enumerate}
\end{enumerate}
\endgroup

% -----------------------------------------------------------------------------
\par\pagebreak[1]\par
\paragraph{Exercice 10}%
\hfill{\tiny 0721}%
\begingroup~

Montrer que $t \mapsto \frac{\sin^2 t}{t}$
n'est pas int\'egrable sur $\intFO{0,+\infty}$.
\endgroup

% -----------------------------------------------------------------------------
\par\pagebreak[1]\par
\paragraph{Exercice 11}%
\hfill{\tiny 9519}%
\begingroup~

Convergence et calcul de \[ \int_0^1 x\lfloor 1/x \rfloor \D x. \]
\endgroup

% -----------------------------------------------------------------------------
\par\pagebreak[1]\par
\paragraph{Exercice 12}%
\hfill{\tiny 4271}%
\begingroup~

Convergence et calcul de \[ \int_1^{+\infty} \frac{\arcsin(1/\sqrt t)}{t^2} \D t. \]
\endgroup

% -----------------------------------------------------------------------------
\par\pagebreak[1]\par
\paragraph{Exercice 13}%
\hfill{\tiny 7676}%
\begingroup~

\'Etudier l'int\'egrabilit\'e
de $x \mapsto \frac{\ln x}{x^2+1}$ sur $\intO{0,+\infty}$.

\'Etudier l'int\'egrabilit\'e
de $x \mapsto \frac{e^{-x}}{\sqrt{x-1}}$ sur $\intO{1,+\infty}$.
\endgroup

% -----------------------------------------------------------------------------
\par\pagebreak[1]\par
\paragraph{Exercice 14}%
\hfill{\tiny 0374}%
\begingroup~

\begin{enumerate}
\item Convergence et calcul de $\DS I_n = \int_0^{+\infty} t \me^{-nt} \D t$.
\item Convergence et calcul de
  $\DS I = \int_0^{+\infty} \frac{\me^{-\sqrt{t}}}{1-\me^{-\sqrt{t}}} \D t$.
\end{enumerate}
\endgroup

% -----------------------------------------------------------------------------
\par\pagebreak[1]\par
\paragraph{Exercice 15}%
\hfill{\tiny 1344}%
\begingroup~

D\'eterminer la limite de $\DS \int_x^{2x} \frac{\tan t}{t^2} \D t$ quand $x \to 0^+$.
\endgroup

% -----------------------------------------------------------------------------
\par\pagebreak[1]\par
\paragraph{Exercice 16}%
\hfill{\tiny 5697}%
\begingroup~

Trouver un \'equivalent de
\[ \int_0^{1/n^2} \frac{\me^{-nx^2}}{1+n^2x^2} \D x. \]
\endgroup

% -----------------------------------------------------------------------------
\par\pagebreak[1]\par
\paragraph{Exercice 17 (lemme de Riemann-Lebesgue)}%
\hfill{\tiny 3351}%
\begingroup~

Soit $a < b$ deux r\'eels et $\Fn{f}{[a,b]}{\R}$ continue par morceaux.
On pose pour $n\in \N$: $I_n =\int_a^b f(t) \sin(nt) \D t$.
\begin{enumerate}
\item
  Montrer que la suite $(I_n)$ est born\'ee.
\item
  Si $f$ est de classe $\CC 1$, montrer que $I_n \Toninf 0$;
  on pourra effectuer une int\'egration par parties.
\item
  Si $f$ est constante, calculer $I_n$ et en d\'eduire que $I_n \Toninf 0$.
\item
  Si $f$ est en escaliers, montrer que $I_n \Toninf 0$.
\item
  Dans le cas g\'en\'eral, montrer qu'on a $I_n \Toninf 0$.
\end{enumerate}
\endgroup

% -----------------------------------------------------------------------------
\par\pagebreak[1]\par
\paragraph{\href{https://psi.miomio.fr/exo/6977.pdf}{Exercice 18} (premi\`ere formule de la moyenne)}%
\hfill{\tiny 6977}%
\begingroup~

Soit $\Fn{f}{[a,b]}{\R}$ continue et $\Fn{g}{[a,b]}{\Rp}$ continue par morceaux. Montrer que:
\[ \exists c\in[a,b]\+\int_a^b fg = f(c)\int_a^b g \]

On pourra commencer par encadrer $f$ par ses bornes.
\endgroup

% -----------------------------------------------------------------------------
\par\pagebreak[1]\par
\paragraph{\href{https://psi.miomio.fr/exo/7412.pdf}{Exercice 19}}%
\hfill{\tiny 7412}%
\begingroup~

Soit $\Fn{f}{[0,1]}{\R}$ continue.
En utilisant l'exercice~18, d\'eterminer
\[ \lim_{x\to0^+}\int_x^{2x} \frac{f(t)}{t} \D t. \]
Red\'emontrer ensuite le r\'esultat \emph{\`a la main}, avec des $\epsilon$.
\endgroup

% -----------------------------------------------------------------------------
\par\pagebreak[1]\par
\paragraph{\href{https://psi.miomio.fr/exo/2044.pdf}{Exercice 20} (seconde formule de la moyenne)}%
\hfill{\tiny 2044}%
\begingroup~

Soit $\Fn{f}{[a,b]}{\R}$ positive, d\'ecroissante, de classe $\CC1$ et $\Fn{g}{[a,b]}{\R}$ continue.
Montrer qu'il existe $c\in[a,b]$ tel que
\[ \int_a^b f(t)g(t) \D t = f(a)\int_a^c g(t) \D t. \]
\emph{Remarque:} le r\'esultat reste vrai en supposant seulement $f$ continue, mais la preuve est plus complexe.
\endgroup

% -----------------------------------------------------------------------------
\par\pagebreak[1]\par
\paragraph{Exercice 21 (m\'ethode des r\'esidus)}%
\hfill{\tiny 9523}%
\begingroup~

\def\Res{\mathop{\mathrm{Res}}}
Soit $P$ et $Q$ deux polyn\^omes r\'eels non nuls, premiers entre eux
et $F$ la fraction rationnelle $F = \frac{P}{Q}$.
On rappelle qu'un \emph{p\^ole} de $F$ est par d\'efinition une racine de $Q$.
On suppose que:
\begin{itemize}
\item
  la fraction rationnelle $F$ n'a pas de p\^ole r\'eel
\item
  $\deg Q \geq\deg P + 2$
\end{itemize}

Pour tout p\^ole $\alpha\in \C$, on appelle \emph{r\'esidu} de $F(X)$ en $\alpha$ le coefficient de $\frac{1}{X-\alpha}$ dans la d\'ecomposition en \'el\'ements simples de $F(X)$, et on le note $\Res(F,\alpha)$.
On pose:
\begin{itemize}
\item
  $\mathcal{P}$ l'ensemble des p\^oles de $F$. On a $\mathcal{P}\subset \C\setminus \R$;
\item
  $\mathcal{P}^+$ l'ensemble des p\^oles de $F$ ayant une partie imaginaire positive.
\end{itemize}

On cherche \`a montrer la formule suivante, qui est un cas simple de la \emph{formule int\'egrale de Cauchy}
\[ \int_{-\infty}^{+\infty} F(t) \D t = 2i\pi\sum_{\alpha\in\mathcal{P}^+} \Res(F,\alpha) \]

\begin{enumerate}
\item
  Montrer que la fonction $t \mapsto F(t)$ est int\'egrable sur $\R$.
\item
  Notons $\Uplet{\alpha_1}{\alpha_n}$ les racines distinctes de $Q$ et $m_k$ la multiplicit\'e de $\alpha_k$.
  Montrer qu'il existe des complexes $\beta_{k,l}$ tels que
  \[ F(X) = \sum_{k=1}^n \sum_{l=1}^{m_k} \frac{\beta_{k,l}}{(X-\alpha_k)^l}. \]
\item
  En consid\'erant $\lim_{t\to+\infty} tF(t)$,
  montrer que \[ \sum_{\alpha\in\mathcal{P}} \Res(F,\alpha) = 0. \]
\item
  Si $\alpha\in \C\setminus \R$ et $l\geq2$, montrer que
  \[ \int_{-M}^M \frac{\D t}{(t-\alpha)^l} \longto 0 \]
  quand $M\to+\infty$.
\item
  Si $\alpha=a+ib$ avec $(a,b)\in \R^2$ et $b\neq0$, montrer que
  \[ \int_{-M}^M \frac{\D t}{t-\alpha} \longto
    \begin{cases}
      \phantom{-}i\pi{} & \text{si $b>0$} \\
      -i\pi{} & \text{si $b<0$}
  \end{cases} \]
  quand $M\to+\infty$.
\item
  En d\'eduire que
  \[ \int_{-M}^M F \longto i\pi\sum_{\alpha\in\mathcal{P}^+} \Res(F,\alpha) - i\pi\sum_{\alpha\in\mathcal{P}^-} \Res(F,\alpha) \]
  quand $M\to+\infty$
  o\`u $\mathcal{P}^- = \mathcal{P} \setminus{} \mathcal{P}^+$.

\item
  Conclure: montrer la formule de Cauchy.
\item
  \emph{Application:} Calculer les int\'egrales suivantes:
  \begin{enumerate}
  \item
    $\DS \int_{-\infty}^{+\infty} \frac{\D t}{(t^2+1)^2}$
  \item
    $\DS \int_{-\infty}^{+\infty} \frac{t^2 \D t}{t^4 + 1}$
  \item
    $\DS \int_{-\infty}^{+\infty} \frac{\D t}{1+t^{2n}}$
  \end{enumerate}
\end{enumerate}
\endgroup

% -----------------------------------------------------------------------------
\par\pagebreak[1]\par
\paragraph{Exercice 22 (irrationalit\'e de $\pi$)}%
\hfill{\tiny 7552}%
\begingroup~

On veut montrer que $\pi$ est irrationnel.
On suppose par l'absurde que $\pi{} = a/b$ avec $a, b \in{} \Ns$.
\begin{enumerate}
\item Montrer que pour tout $q\in \N$, $q^n = o(n!)$.
\item Soient, pour $n\in \N$,
  $Q_n(X) = \frac{1}{n!} X^n (bX-a)^n$
  et $I_n = \int_0^\pi{} Q_n(x) \sin(x) \D x$.
  Montrer que la suite $(I_n)$ tend vers~0.
\item
  \'Etablir la relation $Q_n' = (2bX-a) Q_{n-1}$.
  Puis, \`a l'aide de cette relation et de la formule de Leibniz,
  exprimer les d\'eriv\'ees successives de $Q_n$ en fonction de celles de $Q_{n-1}$.
\item
  Montrer que $Q_n^{(k)}(0)\in \Z$ et $Q_n^{(k)}(\pi)\in \Z$ pour tout $k\in \N$.
\item
  Montrer que $I_n\in \Z$ et $I_n>0$ puis conclure.
\end{enumerate}
\endgroup

\end{document}
