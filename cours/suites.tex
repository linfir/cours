% autogenerated by ytex.rs

\documentclass{scrartcl}

\usepackage[francais]{babel}
\usepackage{geometry}
\usepackage{scrpage2}
\usepackage{lastpage}
\usepackage{multicol}
\usepackage{etoolbox}
\usepackage{xparse}
\usepackage{enumitem}
% \usepackage{csquotes}
\usepackage{amsmath}
\usepackage{amsfonts}
\usepackage{amssymb}
\usepackage{mathrsfs}
\usepackage{stmaryrd}
\usepackage{dsfont}
\usepackage{eurosym}
% \usepackage{numprint}
% \usepackage[most]{tcolorbox}
% \usepackage{tikz}
% \usepackage{tkz-tab}
\usepackage[unicode]{hyperref}
\usepackage[ocgcolorlinks]{ocgx2}

\let\ifTwoColumns\iftrue
\def\Classe{$\Psi$2019--2020}

% reproducible builds
% LuaTeX: \pdfvariable suppressoptionalinfo 1023 \relax
\pdfinfoomitdate=1
\pdftrailerid{}

\newif\ifDisplaystyle
\everymath\expandafter{\the\everymath\ifDisplaystyle\displaystyle\fi}
\newcommand\DS{\displaystyle}

\clearscrheadfoot
\pagestyle{scrheadings}
\thispagestyle{empty}
\ohead{\Classe}
\ihead{\thepage/\pageref*{LastPage}}

\setlist[itemize,1]{label=\textbullet}
\setlist[itemize,2]{label=\textbullet}

\ifTwoColumns
  \geometry{margin=1cm,top=2cm,bottom=3cm,foot=1cm}
  \setlist[enumerate]{leftmargin=*}
  \setlist[itemize]{leftmargin=*}
\else
  \geometry{margin=3cm}
\fi

\makeatletter
\let\@author=\relax
\let\@date=\relax
\renewcommand\maketitle{%
    \begin{center}%
        {\sffamily\huge\bfseries\@title}%
        \ifx\@author\relax\else\par\medskip{\itshape\Large\@author}\fi
        \ifx\@date\relax\else\par\bigskip{\large\@date}\fi
    \end{center}\bigskip
    \ifTwoColumns
        \par\begin{multicols*}{2}%
        \AtEndDocument{\end{multicols*}}%
        \setlength{\columnsep}{5mm}
    \fi
}
\makeatother

\newcounter{ParaNum}
\NewDocumentCommand\Para{smo}{%
  \IfBooleanF{#1}{\refstepcounter{ParaNum}}%
  \paragraph{\IfBooleanF{#1}{{\tiny\arabic{ParaNum}~}}#2\IfNoValueF{#3}{ (#3)}}}

\newcommand\I{i}
\newcommand\mi{i}
\def\me{e}

\def\do#1{\expandafter\undef\csname #1\endcsname}
\docsvlist{Ker,sec,csc,cot,sinh,cosh,tanh,coth,th}
\undef\do

\DeclareMathOperator\ch{ch}
\DeclareMathOperator\sh{sh}
\DeclareMathOperator\th{th}
\DeclareMathOperator\coth{coth}
\DeclareMathOperator\cotan{cotan}
\DeclareMathOperator\argch{argch}
\DeclareMathOperator\argsh{argsh}
\DeclareMathOperator\argth{argth}

\let\epsilon=\varepsilon
\let\phi=\varphi
\let\leq=\leqslant
\let\geq=\geqslant
\let\subsetneq=\varsubsetneq
\let\emptyset=\varnothing

\newcommand{\+}{,\;}

\undef\C
\newcommand\ninf{{n\infty}}
\newcommand\N{\mathbb{N}}
\newcommand\Z{\mathbb{Z}}
\newcommand\Q{\mathbb{Q}}
\newcommand\R{\mathbb{R}}
\newcommand\C{\mathbb{C}}
\newcommand\K{\mathbb{K}}
\newcommand\Ns{\N^*}
\newcommand\Zs{\Z^*}
\newcommand\Qs{\Q^*}
\newcommand\Rs{\R^*}
\newcommand\Cs{\C^*}
\newcommand\Ks{\K^*}
\newcommand\Rp{\R^+}
\newcommand\Rps{\R^+_*}
\newcommand\Rms{\R^-_*}
\newcommand{\Rpinf}{\Rp\cup\Acco{+\infty}}

\undef\B
\newcommand\B{\mathscr{B}}

\undef\P
\DeclareMathOperator\P{\mathbb{P}}
\DeclareMathOperator\E{\mathbb{E}}
\DeclareMathOperator\Var{\mathbb{V}}

\DeclareMathOperator*\PetitO{o}
\DeclareMathOperator*\GrandO{O}
\DeclareMathOperator*\Sim{\sim}
\DeclareMathOperator\Tr{tr}
\DeclareMathOperator\Ima{Im}
\DeclareMathOperator\Ker{Ker}
\DeclareMathOperator\Sp{Sp}
\DeclareMathOperator\Diag{diag}
\DeclareMathOperator\Rang{rang}
\DeclareMathOperator*\Coords{Coords}
\DeclareMathOperator*\Mat{Mat}
\DeclareMathOperator\Pass{Pass}
\DeclareMathOperator\Com{Com}
\DeclareMathOperator\Card{Card}
\DeclareMathOperator\Racines{Racines}
\DeclareMathOperator\Vect{Vect}
\DeclareMathOperator\Id{Id}

\newcommand\DerPart[2]{\frac{\partial #1}{\partial #2}}

\def\T#1{{#1}^T}

\def\pa#1{({#1})}
\def\Pa#1{\left({#1}\right)}
\def\bigPa#1{\bigl({#1}\bigr)}
\def\BigPa#1{\Bigl({#1}\Bigr)}
\def\biggPa#1{\biggl({#1}\biggr)}
\def\BiggPa#1{\Biggl({#1}\Biggr)}

\def\pafrac#1#2{\pa{\frac{#1}{#2}}}
\def\Pafrac#1#2{\Pa{\frac{#1}{#2}}}
\def\bigPafrac#1#2{\bigPa{\frac{#1}{#2}}}
\def\BigPafrac#1#2{\BigPa{\frac{#1}{#2}}}
\def\biggPafrac#1#2{\biggPa{\frac{#1}{#2}}}
\def\BiggPafrac#1#2{\BiggPa{\frac{#1}{#2}}}

\def\cro#1{[{#1}]}
\def\Cro#1{\left[{#1}\right]}
\def\bigCro#1{\bigl[{#1}\bigr]}
\def\BigCro#1{\Bigl[{#1}\Bigr]}
\def\biggCro#1{\biggl[{#1}\biggr]}
\def\BiggCro#1{\Biggl[{#1}\Biggr]}

\def\abs#1{\mathopen|{#1}\mathclose|}
\def\Abs#1{\left|{#1}\right|}
\def\bigAbs#1{\bigl|{#1}\bigr|}
\def\BigAbs#1{\Bigl|{#1}\Bigr|}
\def\biggAbs#1{\biggl|{#1}\biggr|}
\def\BiggAbs#1{\Biggl|{#1}\Biggr|}

\def\acco#1{\{{#1}\}}
\def\Acco#1{\left\{{#1}\right\}}
\def\bigAcco#1{\bigl\{{#1}\bigr\}}
\def\BigAcco#1{\Bigl\{{#1}\Bigr\}}
\def\biggAcco#1{\biggl\{{#1}\biggr\}}
\def\BiggAcco#1{\Biggl\{{#1}\Biggr\}}

\def\ccro#1{\llbracket{#1}\rrbracket}
\def\Dcro#1{\llbracket{#1}\rrbracket}

\def\floor#1{\lfloor#1\rfloor}
\def\Floor#1{\left\lfloor{#1}\right\rfloor}

\def\sEnsemble#1#2{\mathopen\{#1\mid#2\mathclose\}}
\def\bigEnsemble#1#2{\bigl\{#1\bigm|#2\bigr\}}
\def\BigEnsemble#1#2{\Bigl\{#1\Bigm|#2\Bigr\}}
\def\biggEnsemble#1#2{\biggl\{#1\biggm|#2\biggr\}}
\def\BiggEnsemble#1#2{\Biggl\{#1\Biggm|#2\Biggr\}}
\let\Ensemble=\bigEnsemble

\newcommand\IntO[1]{\left]#1\right[}
\newcommand\IntF[1]{\left[#1\right]}
\newcommand\IntOF[1]{\left]#1\right]}
\newcommand\IntFO[1]{\left[#1\right[}

\newcommand\intO[1]{\mathopen]#1\mathclose[}
\newcommand\intF[1]{\mathopen[#1\mathclose]}
\newcommand\intOF[1]{\mathopen]#1\mathclose]}
\newcommand\intFO[1]{\mathopen[#1\mathclose[}

\newcommand\Fn[3]{#1\colon#2\to#3}
\newcommand\CC[1]{\mathscr{C}^{#1}}
\newcommand\D{\mathop{}\!\mathrm{d}}

\newcommand\longto{\longrightarrow}

\undef\M
\newcommand\M[3]{\mathrm{#1}_{#2}\pa{#3}}
\newcommand\MnR{\M{M}{n}{\R}}
\newcommand\MnC{\M{M}{n}{\C}}
\newcommand\MnK{\M{M}{n}{\K}}
\newcommand\GLnR{\M{GL}{n}{\R}}
\newcommand\GLnC{\M{GL}{n}{\C}}
\newcommand\GLnK{\M{GL}{n}{\K}}
\newcommand\DnR{\M{D}{n}{\R}}
\newcommand\DnC{\M{D}{n}{\C}}
\newcommand\DnK{\M{D}{n}{\K}}
\newcommand\SnR{\M{S}{n}{\R}}
\newcommand\AnR{\M{A}{n}{\R}}
\newcommand\OnR{\M{O}{n}{\R}}
\newcommand\SnRp{\mathrm{S}_n^+(\R)}
\newcommand\SnRpp{\mathrm{S}_n^{++}(\R)}

\newcommand\LE{\mathscr{L}(E)}
\newcommand\GLE{\mathscr{GL}(E)}
\newcommand\SE{\mathscr{S}(E)}
\renewcommand\OE{\mathscr{O}(E)}

\newcommand\ImplD{$\Cro\Rightarrow$}
\newcommand\ImplR{$\Cro\Leftarrow$}
\newcommand\InclD{$\Cro\subset$}
\newcommand\InclR{$\Cro\supset$}
\newcommand\notInclD{$\Cro{\not\subset}$}
\newcommand\notInclR{$\Cro{\not\supset}$}

\newcommand\To[1]{\xrightarrow[#1]{}}
\newcommand\Toninf{\To{\ninf}}

\newcommand\Norm[1]{\|#1\|}
\newcommand\Norme{{\Norm{\cdot}}}

\newcommand\Int[1]{\mathring{#1}}
\newcommand\Adh[1]{\overline{#1}}

\newcommand\Uplet[2]{{#1},\dots,{#2}}
\newcommand\nUplet[3]{(\Uplet{{#1}_{#2}}{{#1}_{#3}})}

\newcommand\Fonction[5]{{#1}\left|\begin{aligned}{#2}&\;\longto\;{#3}\\{#4}&\;\longmapsto\;{#5}\end{aligned}\right.}

\DeclareMathOperator\orth{\bot}
\newcommand\Orth[1]{{#1}^\bot}
\newcommand\PS[2]{\langle#1,#2\rangle}

\newcommand{\Tribu}{\mathscr{T}}
\newcommand{\Part}{\mathcal{P}}
\newcommand{\Pro}{\bigPa{\Omega,\Tribu}}
\newcommand{\Prob}{\bigPa{\Omega,\Tribu,\P}}

\newcommand\DEMO{$\spadesuit$}
\newcommand\DUR{$\spadesuit$}

\newenvironment{psmallmatrix}{\left(\begin{smallmatrix}}{\end{smallmatrix}\right)}

% -----------------------------------------------------------------------------

\def\U{(u_n)_{n\in \N}}
\def\V{(v_n)_{n\in \N}}

\begin{document}
\title{Suites num\'eriques}
\maketitle

% -----------------------------------------------------------------------------
\section{G\'en\'eralit\'es}

\Para{D\'efinitions}
\begin{itemize}
\item
  Une \emph{suite} \`a valeurs dans un ensemble $X$ est une application $\Fn{u}{\N}{X}$.
  On note fr\'equemment $u_n$ au lieu de $u(n)$ et $\U$ au lieu de $u$.
\item
  On dit que $u_n$ est le \emph{terme g\'en\'eral} de la suite $\U$.
\item
  Une \emph{suite num\'erique} est une suite \`a valeurs dans $\R$ ou $\C$.
\item
  Une \emph{suite r\'eelle} est une suite \`a valeurs dans $\R$.
\item
  Une suite r\'eelle $\U$ est dite \emph{croissante} si et seulement si
  $\forall n\in \N$, $u_n\leq u_{n+1}$.
\item
  Une suite r\'eelle $\U$ est dite \emph{d\'ecroissante} si et seulement si
  $\forall n\in \N$, $u_n\geq u_{n+1}$.
\item
  Une suite num\'erique $\U$ est dite \emph{stationnaire} si et seulement s'il existe
  $c \in{} \K$ et $n_0 \in{} \N$ tels que $\forall n \geq{} n_0$, $u_n = c$.
\end{itemize}

\Para{D\'efinitions}
\begin{itemize}
\item
  Soit $\U$ une suite num\'erique et $\ell\in \K$.
  On dit que la suite $\U$ \emph{converge} vers $\ell$ si et seulement si
  \[ \forall \epsilon> 0 \+ \exists n_0\in \N{} \+ \forall n\geq n_0 \+ \Abs{u_n -\ell} \leq{} \epsilon. \]
  $\ell$ est n\'ecessairement unique et s'appelle \emph{la limite} de la suite $\U$.
  On \'ecrit $\lim\limits_{\ninf} u_n = \ell$ ou encore $u_n \Toninf \ell$.
\item
  Une suite num\'erique $\U$ est dite \emph{convergente} si et seulement s'il
  existe $\ell\in \K$ tel que $\U$ converge vers $\ell$.
  Elle est dite \emph{divergente} dans le cas contraire.
\end{itemize}

\Para{D\'efinitions}

Soit $\U$ une suite r\'eelle.
\begin{itemize}
\item
  On dit que \emph{$u_n \Toninf +\infty$} si et seulement si
  \[ \forall M\in \R{} \+ \exists n_0\in \N{} \+ \forall n\geq n_0 \+ u_n \geq{} M. \]
\item
  On dit que \emph{$u_n \Toninf -\infty$} si et seulement si
  \[ \forall M\in \R{} \+ \exists n_0\in \N{} \+ \forall n\geq n_0 \+ u_n \leq{} M. \]
\end{itemize}

\Para{Remarque}

Soit $\U$ une suite r\'eelle telle que $u_n \Toninf \pm\infty$.
Alors $\U$ est une suite divergente.

\emph{La r\'eciproque est fausse.}

\Para{D\'efinitions}

\begin{itemize}
\item
  Une suite r\'eelle $\U$ est dite \emph{major\'ee} si et seulement si
  \[ \exists A\in \R{} \+ \forall n\in \N{} \+ u_n\leq A. \]
\item
  Une suite r\'eelle $\U$ est dite \emph{minor\'ee} si et seulement si
  \[ \exists B\in \R{} \+ \forall n\in \N{} \+ u_n\geq B. \]
\item
  Une suite num\'erique $\U$ est dite \emph{born\'ee} si et seulement si
  \[ \exists M\in \R{} \+ \forall n\in \N{} \+ \Abs{u_n}\leq M. \]
\end{itemize}

\Para{D\'efinitions}

\begin{itemize}
\item
  Une \emph{extractrice} est une fonction $\N\to\N$ strictement croissante.
\item
  Une \emph{sous-suite} ou \emph{suite extraite} de la suite $\U$
  est une suite de la forme $(u_{\sigma(n)})_{n\in \N}$ o\`u $\sigma$
  est une extractrice.
\end{itemize}

\Para{Proposition}

Une sous-suite d'une suite convergente est \'egalement convergente.

Plus pr\'ecis\'ement:
si $\U$ est une suite num\'erique telle que $u_n \Toninf \ell{} \in{} \K$,
alors pour toute extractrice $\sigma$, on a $u_{\sigma(n)} \Toninf \ell$.

\Para{Proposition}

Soit $\U$ une suite num\'erique et $\ell\in \K$.
On suppose $u_{2n} \Toninf \ell$ et $u_{2n+1} \Toninf \ell$.
Alors $u_n \Toninf \ell$.

L'\'enonc\'e reste vrai si $\K=\R$ et $\ell{} = \pm\infty$.

\subsection{Th\'eor\`emes}

\Para{Th\'eor\`eme de convergence des suites monotones}

Soit $\U$ une suite r\'eelle croissante.
\begin{itemize}
\item si $\U$ est major\'ee, alors elle converge.
\item sinon, $u_n \Toninf +\infty$.
\end{itemize}

On a un \'enonc\'e analogue pour les suites d\'ecroissantes.

\Para{Th\'eor\`eme des gendarmes}

Soit $(u_n)_{n\in \N}$, $(v_n)_{n\in \N}$ et $(w_n)_{n\in \N}$ trois suites r\'eelles et $\ell\in \R$.
On suppose que:
\begin{itemize}
\item
  $u_n \Toninf \ell$;
\item
  $w_n \Toninf \ell$;
\item
  $\exists n_0\in \N{} \+ \forall n\geq n_0 \+ u_n \leq{} v_n \leq{} w_n$.
\end{itemize}
Alors $(v_n)_{n\in \N}$ est une suite convergente et $v_n \Toninf \ell$.

\Para{Th\'eor\`eme}

Soit $\U$ une suite r\'eelle.
Soit $I$ un intervalle de $\R$ et $\Fn{f}{I}{\R}$.
On suppose que:
\begin{itemize}
\item
  $u_n \Toninf \ell$;
\item
  $\forall{} n \in{} \N$, $u_n \in{} I$;
\item
  $\ell\in I$;
\item
  $f$ continue en $\ell$.
\end{itemize}

Alors $f(u_n) \Toninf f(\ell)$.

\Para{D\'efinition}

Soit $\U$ et $\V$ deux suites r\'eelles.
On dit qu'elles sont \emph{adjacentes} si et seulement si:
\begin{itemize}
\item
  $\U$ est croissante;
\item
  $\V$ est d\'ecroissante;
\item
  $v_n - u_n \Toninf 0$.
\end{itemize}

\Para{Th\'eor\`eme des suites adjacentes}

Soit $\U$ et $\V$ deux suites adjacentes.
Alors elles convergent toutes les deux vers la m\^eme limite $\ell\in \R$.
De plus:
\begin{multline*}
  u_0\leq\cdots\leq u_n\leq u_{n+1}\leq\cdots\leq \ell{} \\
  \ell\leq\cdots\leq v_{n+1}\leq v_n\leq\cdots\leq v_0
\end{multline*}

\Para{Th\'eor\`eme de Ces\`aro}

Soit $\U$ une suite num\'erique telle que $u_n \Toninf \ell{} \in{} \K$.
On pose \[ \mu_n = \frac{1}{n} \sum_{k=0}^{n-1} u_k, \]
il s'agit de la moyenne arithm\'etique des $n$ premiers termes de la suite $\U$.

Alors $\mu_n \Toninf \ell$.

De plus, si $\U$ est une suite r\'eelle qui tend vers $\pm\infty$, le r\'esultat est encore vrai.

\Para{Th\'eor\`eme de Bolzano-Weierstrass}[hors-programme]

Toute suite r\'eelle born\'ee admet une sous-suite convergente.

\subsection{Comparaison}

\Para{D\'efinitions}

Soit $\U$ et $\V$ deux suites num\'eriques.
\begin{itemize}
\item
  On dit que \emph{$u_n = O(v_n)$} si et seulement s'il existe
  une suite num\'erique $(e_n)_{n\in \N}$ \emph{born\'ee} telle que
  $\exists n_0\in \N$, $\forall n\geq n_0$, $u_n = e_n v_n$.
\item
  On dit que \emph{$u_n = o(v_n)$} si et seulement s'il existe
  une suite num\'erique $(e_n)_{n\in \N}$ \emph{qui tend vers 0} telle que
  $\exists n_0\in \N$, $\forall n\geq n_0$, $u_n = e_n v_n$.
\item
  On dit que \emph{$u_n \sim v_n$} si et seulement s'il existe
  une suite num\'erique $(e_n)_{n\in \N}$ \emph{qui tend vers 1} telle que
  $\exists n_0\in \N$, $\forall n\geq n_0$, $u_n = e_n v_n$.
  Il s'agit d'une relation d'\'equivalence.
\end{itemize}

\Para{Proposition}

Soit $\U$ et $\V$ deux suites num\'eriques.
Alors \[ u_n \sim v_n \iff u_n = v_n + o(v_n). \]

\Para{Proposition}

Soit $\U$ et $\V$ deux suites num\'eriques.
On suppose de plus que:
\[ \exists n_0\in \N{} \+ \forall n\geq n_0 \+ v_n\neq0. \]
Alors:
\begin{itemize}
\item
  $u_n = O(v_n)$ si et seulement si la suite $\bigPa{\frac{u_n}{v_n}}_{n\geq n_0}$ est born\'ee.
\item
  $u_n = o(v_n)$ si et seulement si $\frac{u_n}{v_n} \Toninf 0$.
\item
  $u_n \sim v_n$ si et seulement si $\frac{u_n}{v_n} \Toninf 1$.
\end{itemize}

% -----------------------------------------------------------------------------
\section{Suites r\'ecurrentes du premier ordre}

\Para{D\'efinition}

Soit $\U$ une suite \`a valeurs dans un ensemble $X$
et $\Fn fXX$.

On dit que $\U$ est une \emph{suite r\'ecurrente (du premier ordre)}
associ\'ee \`a la fonction $f$
si et seulement si \[ \forall n\in \N, u_{n+1} = f(u_n). \]

\Para{Proposition}

Soit $\U$ une suite r\'ecurrente associ\'ee \`a $f$.
Si:
\begin{itemize}
\item
  $u_n \Toninf \ell$;
\item
  $\ell\in X$;
\item
  $f$ continue en $\ell$.
\end{itemize}

Alors $f(\ell) =\ell$.

\Para{Remarque}

Cela sert \`a trouver les limites \emph{potentielles} de $\U$.

\Para{D\'efinition}

Soit $\Fn fXX$ et $Y\subset X$.
On dit que $Y$ est \emph{stable} par $f$ si et seulement si $f(Y)\subset Y$.

\Para{Proposition}

Soit $\U$ une suite r\'ecurrente r\'eelle associ\'ee \`a $f$
et $I$ un intervalle stable par $f$ tel que $u_{n_0}\in I$.
Alors $\forall n\geq n_0$, $u_n\in I$.

\Para{Proposition}

Soit $\Fn{f}{I}{I}$ o\`u $I$ est un intervalle de $\R$
et $\U$ une suite r\'ecurrente associ\'ee \`a $f$.
\begin{enumerate}
\item
  Si $f$ est croissante sur $I$, alors $\U$ est monotone;
\item
  Si $f$ est d\'ecroissante sur $I$, alors les suites extraites
  $(u_{2n})_{n\in \N}$ et $(u_{2n+1})_{n\in \N}$ sont monotones;
\item
  Dans le cas g\'en\'eral, on ne peut rien dire: $\U$ peut avoir un comportement tr\`es complexe.
\end{enumerate}

\Para{Exemples}
\begin{itemize}
\item
  \emph{Suites arithm\'etiques:}
  $\forall n\in \N{} \+ u_{n+1} = u_n + b$.
  On a alors $\forall n\in \N{} \+ u_n = u_0 + nb$.
\item
  \emph{Suites g\'eom\'etriques:}
  $\forall n\in \N{} \+ u_{n+1} = a u_n$.
  On a alors $\forall n\in \N{} \+ u_n = a^n u_0$.
\item
  \emph{Suites arithm\'etico-g\'eom\'etriques:}
  $\forall n\in \N{} \+ u_{n+1} = au_n + b$ o\`u $a\neq1$.
  On pose $\ell$ tel que $\ell= a\ell+ b$, puis $v_n = u_n -\ell$.
  On v\'erifie alors que $\V$ est une suite g\'eom\'etrique.
\item
  \emph{R\'ecurrence homographique:}
  cf. exercice~29.
\end{itemize}

\section{Suites r\'ecurrentes du second ordre}

\Para{D\'efinition}

Soit $\U$ une suite \`a valeurs dans un ensemble $X$,
et $\Fn{f}{X^2}{X}$.
On dit que $\U$ est une \emph{suite r\'ecurrente du second ordre} associ\'ee \`a la fonction $f$
si et seulement si \[ \forall n\in \N, u_{n+2} = f(u_n, u_{n+1}). \]

\subsection*{Cas des suites r\'ecurrentes lin\'eaires du second ordre \`a coefficients constants sans second membre}

\Para{Th\'eor\`eme}

Soit $\U$ une suite num\'erique et $(a,b,c)\in \K^3$ avec $a\neq0$ et $(b,c)\neq(0,0)$
tels que \[ \forall n\in \N{} \+ a u_{n+2} + b u_{n+1} + c u_n = 0. \]

On forme l'\emph{\'equation caract\'eristique}:
\[ ar^2 + br + c = 0. \tag{EC} \]

Si \emph{$\K=\C$}, il y a deux cas:
\begin{enumerate}
\item
  Si $\Delta\neq0$,
  (EC) admet deux racines distinctes $r_1$ et $r_2$, et
  \[ \exists(A,B)\in \C^2 \+ \forall n\in \N{} \+ u_n = A r_1^n + B r_2^n \]
\item
  Si $\Delta=0$,
  (EC) admet une racine double $r_0\neq0$, et
  \[ \exists(A,B)\in \C^2 \+ \forall n\in \N{} \+ u_n = (An + B) r_0^n \]
\end{enumerate}

Si \emph{$\K=\R$}, il y a trois cas:
\begin{enumerate}
\item
  Si $\Delta>0$,
  (EC) admet deux racines distinctes $r_1$ et $r_2$, et
  \[ \exists(A,B)\in \R^2 \+ \forall n\in \N{} \+ u_n = A r_1^n + B r_2^n \]
\item
  Si $\Delta=0$,
  (EC) admet une racine double $r_0\neq0$, et
  \[ \exists(A,B)\in \R^2 \+ \forall n\in \N{} \+ u_n = (An + B) r_0^n \]
\item
  Si $\Delta<0$,
  (EC) admet deux racines complexes conjugu\'ees $\rho e^{\pm\I\theta}$, et
  \[ \exists(A,B)\in \R^2 \+ \forall n\in \N{} \+ u_n = \rho^n \BigPa{ A \cos(n\theta) + B \sin(n\theta) } \]
\end{enumerate}

% -----------------------------------------------------------------------------
\section{Exercices}

% -----------------------------------------------------------------------------
\par\pagebreak[1]\par
\paragraph{Exercice 1}%
\hfill{\tiny 5758}%
\begingroup~

Trouver un exemple de suite r\'eelle $(u_n)$ qui soit
\begin{enumerate}
\item ni minor\'ee, ni major\'ee;
\item minor\'ee, non major\'ee et qui ne tend pas vers $+\infty$;
\item positive et qui tend vers 0 sans \^etre d\'ecroissante.
\end{enumerate}
\endgroup

% -----------------------------------------------------------------------------
\par\pagebreak[1]\par
\paragraph{Exercice 2}%
\hfill{\tiny 2883}%
\begingroup~

Soit $(u_n)$ une suite r\'eelle convergeant vers $\ell\in \R$.
Si $\floor\cdot$ d\'esigne la fonction partie enti\`ere,
la suite $\bigPa{\floor{u_n}}_{n\in \N}$ est-elle convergente?
\endgroup

% -----------------------------------------------------------------------------
\par\pagebreak[1]\par
\paragraph{Exercice 3}%
\hfill{\tiny 2047}%
\begingroup~

Soit $(u_n)$ une suite r\'eelle croissante de r\'eels strictements positifs.
On suppose que la suite $(u_n)$ converge.
Montrer que sa limite est strictement positive.
\endgroup

% -----------------------------------------------------------------------------
\par\pagebreak[1]\par
\paragraph{Exercice 4}%
\hfill{\tiny 3604}%
\begingroup~

Trouver un exemple de suite qui diverge mais dont la moyenne de Ces\`aro converge.
\endgroup

% -----------------------------------------------------------------------------
\par\pagebreak[1]\par
\paragraph{Exercice 5}%
\hfill{\tiny 8193}%
\begingroup~

Montrer que l'ensemble des entiers naturels $n$ tels que $2^{n^2} < (4n)!$ est fini.
\endgroup

% -----------------------------------------------------------------------------
\par\pagebreak[1]\par
\paragraph{Exercice 6}%
\hfill{\tiny 3930}%
\begingroup~

Soit $(u_n)$ une suite telle que les sous-suites $(u_{2n})$, $(u_{2n+1})$ et $(u_{n^2})$ convergent.
Montrer que $(u_n)$ converge.
\endgroup

% -----------------------------------------------------------------------------
\par\pagebreak[1]\par
\paragraph{\href{https://psi.miomio.fr/exo/2133.pdf}{Exercice 7}}%
\hfill{\tiny 2133}%
\begingroup~

\Displaystyletrue

D\'eterminer la limite, ou montrer la divergence des suites $(u_n)$ d\'efinies par
\begin{enumerate}
\item
  $u_n = \frac{3^n-(-2)^n}{3^n+(-2)^n}$
\item
  $u_n = \sqrt{n^2+n+1} - \sqrt{n^2-n+1}$
\item
  $u_n = \frac{n-\sqrt{n^2+1}}{n+\sqrt{n^2+1}}$
\item
  $u_n = \frac{1}{n^2} \sum_{k=1}^n k$
\item
  $u_n = \sqrt[n]{n^2}$
\item
  $u_n = \Pa{\sin\frac1n}^{1/n}$
\item
  $u_n = \frac{\sin n}{n+(-1)^{n+1}}$
\item
  $u_n = \frac{n!}{n^n}$
\item
  $u_n = \frac{n-(-1)^n}{n+(-1)^n}$
\item
  $u_n = \frac{e^n}{n^n}$
\item
  $u_n = \sqrt[n]{2+(-1)^n}$
\item
  $u_n = \Pa{1+\frac xn}^n$
\item
  $u_n = \Pa{\frac{n-1}{n+1}}^{n+2}$
\item
  $u_n = n^2 \Pa{\cos\frac{1}{n} - \cos\frac{1}{n+1}}$
\item
  $u_n = \Pa{\tan\BigPa{\frac\pi4 + \frac\alpha{n}}}^n$
\item
  $u_n = \Pa{\frac{\ln(n+1)}{\ln(n)}}^{n\ln n}$
\item
  $u_n = \Pa{\frac{\arctan(n+1)}{\arctan(n)}}^{n^2}$
\item
  $u_n = \cos\Pa{\pi n^2 \ln(1-1/n)}$
\end{enumerate}
\endgroup

% -----------------------------------------------------------------------------
\par\pagebreak[1]\par
\paragraph{Exercice 8}%
\hfill{\tiny 0156}%
\begingroup~

\Displaystyletrue

\'Etudier la convergence des suites r\'eelles $(u_n)$ d\'efinies par:
\begin{enumerate}
\item
  $u_0 = a > 0, u_{n+1} = \frac12\left(u_n + \frac{a}{u_n}\right)$
\item
  $u_0 = 1$, $u_{n+1} = \frac{u_n}{u_n^2+1}$
\item
  $u_0\geq0$, $u_{n+1} = \frac16(u_n^2+8)$
\item
  $u_0 = 1$, $u_{n+1} = \frac{1}{2+u_n}$
\item
  $u_{n+1} = \cos u_n$
\item
  $0 < u_0 < \frac{\sqrt5-1}{2}, u_{n+1} = 1 - u_n^2$
\item
  $u_{n+1} = u_n - u_n^2$
\item
  $u_{n+1} = u_n + \frac{1+u_n}{1+2u_n}$
\item
  $0\leq u_0\leq1$, $u_{n+1} = \frac{\sqrt{u_n}}{\sqrt{u_n}+\sqrt{1-u_n}}$
\item
  $u_{n+1} =\sqrt{2-u_n}$
\end{enumerate}
\endgroup

% -----------------------------------------------------------------------------
\par\pagebreak[1]\par
\paragraph{Exercice 9 (lemme des petits pas)}%
\hfill{\tiny 2586}%
\begingroup~

Soit $(u_n)$ une suite num\'erique telle que $u_{n+1} - u_n \to \ell$.
Montrer que $u_n = n \ell{} + o(n)$ quand $n \to +\infty$.
\endgroup

% -----------------------------------------------------------------------------
\par\pagebreak[1]\par
\paragraph{\href{https://psi.miomio.fr/exo/2397.pdf}{Exercice 10}}%
\hfill{\tiny 2397}%
\begingroup~

Soit $(u_n)$ et $(v_n)$ deux suites r\'eelles telles que $u_n^2 + u_n v_n + v_n^2 \to 0$.
Que peut-on dire de $(u_n)$ et de $(v_n)$?
\endgroup

% -----------------------------------------------------------------------------
\par\pagebreak[1]\par
\paragraph{\href{https://psi.miomio.fr/exo/4810.pdf}{Exercice 11}}%
\hfill{\tiny 4810}%
\begingroup~

Montrer que pour tout $n \in \N^*$, l'\'equation $x-e^{-x}=n$ admet une unique solution $x_n$.

Montrer que $x_n \in[n,n+1]$. Donner un \'equivalent et un d\'eveloppement limit\'e \`a l'ordre 2 de $x_n - n$ en $+\infty$.
\endgroup

% -----------------------------------------------------------------------------
\par\pagebreak[1]\par
\paragraph{Exercice 12}%
\hfill{\tiny 0194}%
\begingroup~

\'Etudier la convergence des suites $(u_n)$ et $(v_n)$ d\'efinies par
$0 < u_0 < v_0$,
$u_{n+1} = \frac{2u_n+v_n}{3}$ et
$v_{n+1} = \frac{2v_n+u_n}{3}$.
\endgroup

% -----------------------------------------------------------------------------
\par\pagebreak[1]\par
\paragraph{Exercice 13}%
\hfill{\tiny 7949}%
\begingroup~

Soit $(u_n)$ et $(v_n)$ deux suites r\'eelles v\'erifiant
\[ \left\{ \begin{array}{l}
      0\leq u_n\leq1, \\
      0\leq u_n\leq1, \\
      u_n v_n \Toninf 1.
\end{array} \right. \]
Que peut-on dire de ces deux suites?
\endgroup

% -----------------------------------------------------------------------------
\par\pagebreak[1]\par
\paragraph{\href{https://psi.miomio.fr/exo/9102.pdf}{Exercice 14}}%
\hfill{\tiny 9102}%
\begingroup~

Soit $(u_n)$ une suite \`a valeurs positives et $\lambda{} \in{} \intO{0,1}$.
\begin{enumerate}
\item
  On suppose que $u_{n+1} \leq{} \lambda u_n$ pour tout entier~$n$.
  Montrer que $u_n \to 0$.
\item
  On suppose qu'il existe une suite r\'eelle $(v_n)$ tendant vers~0 telle que $u_{n+1} \leq{} \lambda{} (u_n + v_n)$ pour tout entier~$n$.
  A-t-on $u_n \to 0$?
\end{enumerate}
\endgroup

% -----------------------------------------------------------------------------
\par\pagebreak[1]\par
\paragraph{Exercice 15 (D'Alembert)}%
\hfill{\tiny 6566}%
\begingroup~

Soit $(u_n)$ une suite \`a valeurs strictement positives
telle que $\frac{u_{n+1}}{u_n} \to \ell$.
\begin{enumerate}
\item
  Si $\ell{} < 1$, montrer que $u_n \to 0$.
\item
  Si $\ell{} > 1$, montrer que $u_n \to +\infty$.
\item
  Montrer qu'on ne peut pas conclure dans le cas $\ell{} = 1$.
\end{enumerate}
\endgroup

% -----------------------------------------------------------------------------
\par\pagebreak[1]\par
\paragraph{Exercice 16}%
\hfill{\tiny 8311}%
\begingroup~

Soit $(u_n)$ une suite injective \`a valeurs dans $\N$.
Montrer que $u_n \to +\infty$.
\endgroup

% -----------------------------------------------------------------------------
\par\pagebreak[1]\par
\paragraph{Exercice 17}%
\hfill{\tiny 1515}%
\begingroup~

\'Etudier la suite $(u_n)_{n \in \N}$ d\'efinie par son premier terme $u_0 > 0$ et v\'erifiant
$u_{n+1} = 1+\frac{1}{u_n}$.
\endgroup

% -----------------------------------------------------------------------------
\par\pagebreak[1]\par
\paragraph{Exercice 18}%
\hfill{\tiny 9376}%
\begingroup~

Soit $(u_n)$ une suite r\'eelle d\'ecroissante telle que
$u_n + u_{n+1} \Sim \limits_\ninf \frac1n$.
Montrer que $u_n$ tend vers~$0$.
Trouver un \'equivalent de $u_n$.
\endgroup

% -----------------------------------------------------------------------------
\par\pagebreak[1]\par
\paragraph{Exercice 19}%
\hfill{\tiny 8644}%
\begingroup~

\'Etudier la suite $(S_n)$ d\'efinie par $\DS S_n = \sum_{k=n}^{2n-1} \frac{1}{2k+1}$.
\endgroup

% -----------------------------------------------------------------------------
\par\pagebreak[1]\par
\paragraph{\href{https://psi.miomio.fr/exo/8043.pdf}{Exercice 20}}%
\hfill{\tiny 8043}%
\begingroup~

Pour $n\in \N$, on pose
\[ u_n = \sum_{k=0}^n \frac{1}{k!} \qquad \text{et} \qquad
v_n = u_n + \frac{1}{n\cdot n!} \]
\begin{enumerate}
\item
  Montrer que $(u_n)$ et $(v_n)$ sont deux suites adjacentes;
  on note $\ell$ leur limite commune.
\item
  D\'eterminer un $n$ tel que $v_n - u_n \leq{} 10^{-7}$.
  En d\'eduire une valeur approch\'ee de $\ell$ \`a $10^{-7}$ pr\`es.
\item
  Montrer par l'absurde que $\ell$ est irrationnel.
\item
  En utilisant l'in\'egalit\'e de Taylor-Lagrange appliqu\'ee \`a l'exponentielle,
  montrer que $\ell{} = e$.
\end{enumerate}
\endgroup

% -----------------------------------------------------------------------------
\par\pagebreak[1]\par
\paragraph{Exercice 21}%
\hfill{\tiny 8952}%
\begingroup~

Soit $(z_n)$ d\'efinie par $z_0 \in \C$ et $z_{n+1} = \frac12 \Pa{z_n + \Abs{z_n}}$.
Donner le module et l'argument de $z_n$ en fonction de $z_0$.
Construire g\'eom\'etriquement $z_{n+1}$ en fonction de $z_n$.
\'Etudier la convergence de la suite $(z_n)$.
\endgroup

% -----------------------------------------------------------------------------
\par\pagebreak[1]\par
\paragraph{Exercice 22}%
\hfill{\tiny 4186}%
\begingroup~

On d\'efinit les suites $(a_n)$ et $(b_n)$ de la mani\`ere suivante:
$b_0 > a_0 > 0$,
$a_{n+1} = \frac{a_n^2}{a_n+b_n}$ et
$b_{n+1} = \frac{b_n^2}{a_n+b_n}$.

\'Etudier la convergence de ces deux suites.
\endgroup

% -----------------------------------------------------------------------------
\par\pagebreak[1]\par
\paragraph{Exercice 23}%
\hfill{\tiny 6888}%
\begingroup~

Soit $I = \intF{a,b}$ un intervalle de $\R$ et $\Fn{f}{I}{I}$ de classe $\CC1$.
On suppose que
\[ \exists{} k < 1 \+ \forall{} x \in{} I \+ \abs{f'(x)} \leq{} k. \]

Soit $(u_n)$ une suite telle que $u_0 \in{} I$ et pour tout $n\in \N$, $u_{n+1} = f(u_n)$.

\begin{enumerate}
\item
  On pose $g(x) = f(x) - x$.
  Montrer que $g$ est strictement d\'ecroissante et en d\'eduire
  qu'il existe un unique $\alpha{} \in{} I$ tel que $f(\alpha) = \alpha$.
\item
  On pose $\delta_n = \Abs{u_n - \alpha}$.
  En utilisant l'in\'egalit\'e des accroissements finis, montrer que $\delta_{n+1} \leq{} k \delta_n$.
\item
  En d\'eduire que $u_n \to \alpha$.
\end{enumerate}
\endgroup

% -----------------------------------------------------------------------------
\par\pagebreak[1]\par
\paragraph{\href{https://psi.miomio.fr/exo/7018.pdf}{Exercice 24}}%
\hfill{\tiny 7018}%
\begingroup~

D\'eveloppement asymptotique \`a deux termes de la suite d\'efinie par $x_0 > 0$
et $x_{n+1} = x_n + \frac{1}{\sqrt{x_n}}$.
\endgroup

% -----------------------------------------------------------------------------
\par\pagebreak[1]\par
\paragraph{Exercice 25}%
\hfill{\tiny 7379}%
\begingroup~

Soit $(u_n)$ et $(v_n)$ les suites r\'eelles d\'efinies par
$u_0 = a > 0$,
$v_0 = b > 0$,
$u_{n+1} = \sqrt{u_n v_n}$ et
$v_{n+1} = \frac{u_n+v_n}{2}$.

Montrer que $(u_n)$ et $(v_n)$ sont des suites adjacentes.

\emph{Remarque:}
leur limite commune $M(a,b)$ s'appelle la \emph{moyenne arithm\'etico-g\'eom\'etrique} de $a$ et $b$.
On peut prouver que:
\[ \frac\pi{M(a,b)} =\int_{-\infty}^{+\infty} \frac{\D x}{\sqrt{(x^2+a^2)(x^2+b^2)}} \]
\endgroup

% -----------------------------------------------------------------------------
\par\pagebreak[1]\par
\paragraph{Exercice 26}%
\hfill{\tiny 5838}%
\begingroup~

\'Etudier la suite d\'efinie par son premier terme $u_0$
et v\'erifiant $u_{n+1} = \frac{3}{\pi} \sin(u_n) + 1$.
\endgroup

% -----------------------------------------------------------------------------
\par\pagebreak[1]\par
\paragraph{Exercice 27}%
\hfill{\tiny 9757}%
\begingroup~

\begin{enumerate}
\item Montrer que l'ensemble $E$ des suites complexes de terme g\'en\'eral $a_n$ v\'erifiant
  $a_{n+3} = a_{n+2} + a_{n+1} + a_n$
  est un $\C$-espace vectoriel de dimension 3.
\item Montrer que $F = \Ensemble{(a_n) \in E}{\lim_\ninf a_n = 0}$
  est un sous-espace vectoriel de $E$.

  Quelle est sa dimension? En donner une base.
\end{enumerate}
\endgroup

% -----------------------------------------------------------------------------
\par\pagebreak[1]\par
\paragraph{Exercice 28}%
\hfill{\tiny 0855}%
\begingroup~

\begin{enumerate}
\item
  \begin{enumerate}
  \item
    R\'esoudre $u_0 = a$, $u_1 = b$ et $u_{n+2} = \frac{u_{n+1} + u_n}{2}$.
  \item
    D\'eterminer $\lim_\ninf u_n$.
  \end{enumerate}
\item
  \begin{enumerate}
  \item
    R\'esoudre $v_0 =\alpha> 0$, $v_1 =\beta> 0$ et $v_{n+2} =\sqrt{ v_{n+1} v_n }$
  \item
    D\'eterminer $\lim_\ninf v_n$.
  \end{enumerate}
\end{enumerate}
\endgroup

% -----------------------------------------------------------------------------
\par\pagebreak[1]\par
\paragraph{Exercice 29 (r\'ecurrence homographique)}%
\hfill{\tiny 3156}%
\begingroup~

Pour toute matrice $A \in{} \M{M}{2}{\C}$, on pose
\[ f_A(z) = \frac{az+b}{cz+d} \quad \text{si } A = \begin{pmatrix} a & b \\ c & d \end{pmatrix}. \]
\begin{enumerate}
\item
  V\'erifier que $f_{I_2} = \Id_\C$.
\item
  Montrer que $f_A \circ{} f_B = f_{AB}$ sur un ensemble \`a pr\'eciser.
\item
  On pose $u_0 \in{} \C$ et
  \[ \forall{} n \in{} \N{} \+ u_{n+1} = \frac{a u_n + b}{c u_n + d}. \]
  Exprimer $u_n$ en fonction des coefficients de la matrice $A^n$.
\item \emph{pour les $5/2$:}
  Si $A$ est diagonalisable, comment calcule-t-on $A^n$?

  Si $A$ ne l'est pas, montrer que $A$ est semblable \`a une matrice de la forme
  \[ B = \begin{pmatrix} \lambda & \mu \\ 0 & \lambda \end{pmatrix} \quad \text{o\`u } \mu{} \neq{} 0. \]
  Comment peut-on calculer $A^n$?
\end{enumerate}
\endgroup

% -----------------------------------------------------------------------------
\par\pagebreak[1]\par
\paragraph{Exercice 30 (rigolo)}%
\hfill{\tiny 0923}%
\begingroup~

\'Etablir
\[ \sqrt{1+\sqrt{1+\sqrt{1+\cdots}}}
  = 1 + \cfrac{1}{1 + \cfrac{1}{1 + \cdots}}
\]
\endgroup

\end{document}
