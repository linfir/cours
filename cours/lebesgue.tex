% autogenerated by ytex.rs

\documentclass{scrartcl}

\usepackage[francais]{babel}
\usepackage{geometry}
\usepackage{scrpage2}
\usepackage{lastpage}
\usepackage{multicol}
\usepackage{etoolbox}
\usepackage{xparse}
\usepackage{enumitem}
% \usepackage{csquotes}
\usepackage{amsmath}
\usepackage{amsfonts}
\usepackage{amssymb}
\usepackage{mathrsfs}
\usepackage{stmaryrd}
\usepackage{dsfont}
\usepackage{eurosym}
% \usepackage{numprint}
% \usepackage[most]{tcolorbox}
% \usepackage{tikz}
% \usepackage{tkz-tab}
\usepackage[unicode]{hyperref}
\usepackage[ocgcolorlinks]{ocgx2}

\let\ifTwoColumns\iftrue
\def\Classe{$\Psi$2019--2020}

% reproducible builds
% LuaTeX: \pdfvariable suppressoptionalinfo 1023 \relax
\pdfinfoomitdate=1
\pdftrailerid{}

\newif\ifDisplaystyle
\everymath\expandafter{\the\everymath\ifDisplaystyle\displaystyle\fi}
\newcommand\DS{\displaystyle}

\clearscrheadfoot
\pagestyle{scrheadings}
\thispagestyle{empty}
\ohead{\Classe}
\ihead{\thepage/\pageref*{LastPage}}

\setlist[itemize,1]{label=\textbullet}
\setlist[itemize,2]{label=\textbullet}

\ifTwoColumns
  \geometry{margin=1cm,top=2cm,bottom=3cm,foot=1cm}
  \setlist[enumerate]{leftmargin=*}
  \setlist[itemize]{leftmargin=*}
\else
  \geometry{margin=3cm}
\fi

\makeatletter
\let\@author=\relax
\let\@date=\relax
\renewcommand\maketitle{%
    \begin{center}%
        {\sffamily\huge\bfseries\@title}%
        \ifx\@author\relax\else\par\medskip{\itshape\Large\@author}\fi
        \ifx\@date\relax\else\par\bigskip{\large\@date}\fi
    \end{center}\bigskip
    \ifTwoColumns
        \par\begin{multicols*}{2}%
        \AtEndDocument{\end{multicols*}}%
        \setlength{\columnsep}{5mm}
    \fi
}
\makeatother

\newcounter{ParaNum}
\NewDocumentCommand\Para{smo}{%
  \IfBooleanF{#1}{\refstepcounter{ParaNum}}%
  \paragraph{\IfBooleanF{#1}{{\tiny\arabic{ParaNum}~}}#2\IfNoValueF{#3}{ (#3)}}}

\newcommand\I{i}
\newcommand\mi{i}
\def\me{e}

\def\do#1{\expandafter\undef\csname #1\endcsname}
\docsvlist{Ker,sec,csc,cot,sinh,cosh,tanh,coth,th}
\undef\do

\DeclareMathOperator\ch{ch}
\DeclareMathOperator\sh{sh}
\DeclareMathOperator\th{th}
\DeclareMathOperator\coth{coth}
\DeclareMathOperator\cotan{cotan}
\DeclareMathOperator\argch{argch}
\DeclareMathOperator\argsh{argsh}
\DeclareMathOperator\argth{argth}

\let\epsilon=\varepsilon
\let\phi=\varphi
\let\leq=\leqslant
\let\geq=\geqslant
\let\subsetneq=\varsubsetneq
\let\emptyset=\varnothing

\newcommand{\+}{,\;}

\undef\C
\newcommand\ninf{{n\infty}}
\newcommand\N{\mathbb{N}}
\newcommand\Z{\mathbb{Z}}
\newcommand\Q{\mathbb{Q}}
\newcommand\R{\mathbb{R}}
\newcommand\C{\mathbb{C}}
\newcommand\K{\mathbb{K}}
\newcommand\Ns{\N^*}
\newcommand\Zs{\Z^*}
\newcommand\Qs{\Q^*}
\newcommand\Rs{\R^*}
\newcommand\Cs{\C^*}
\newcommand\Ks{\K^*}
\newcommand\Rp{\R^+}
\newcommand\Rps{\R^+_*}
\newcommand\Rms{\R^-_*}
\newcommand{\Rpinf}{\Rp\cup\Acco{+\infty}}

\undef\B
\newcommand\B{\mathscr{B}}

\undef\P
\DeclareMathOperator\P{\mathbb{P}}
\DeclareMathOperator\E{\mathbb{E}}
\DeclareMathOperator\Var{\mathbb{V}}

\DeclareMathOperator*\PetitO{o}
\DeclareMathOperator*\GrandO{O}
\DeclareMathOperator*\Sim{\sim}
\DeclareMathOperator\Tr{tr}
\DeclareMathOperator\Ima{Im}
\DeclareMathOperator\Ker{Ker}
\DeclareMathOperator\Sp{Sp}
\DeclareMathOperator\Diag{diag}
\DeclareMathOperator\Rang{rang}
\DeclareMathOperator*\Coords{Coords}
\DeclareMathOperator*\Mat{Mat}
\DeclareMathOperator\Pass{Pass}
\DeclareMathOperator\Com{Com}
\DeclareMathOperator\Card{Card}
\DeclareMathOperator\Racines{Racines}
\DeclareMathOperator\Vect{Vect}
\DeclareMathOperator\Id{Id}

\newcommand\DerPart[2]{\frac{\partial #1}{\partial #2}}

\def\T#1{{#1}^T}

\def\pa#1{({#1})}
\def\Pa#1{\left({#1}\right)}
\def\bigPa#1{\bigl({#1}\bigr)}
\def\BigPa#1{\Bigl({#1}\Bigr)}
\def\biggPa#1{\biggl({#1}\biggr)}
\def\BiggPa#1{\Biggl({#1}\Biggr)}

\def\pafrac#1#2{\pa{\frac{#1}{#2}}}
\def\Pafrac#1#2{\Pa{\frac{#1}{#2}}}
\def\bigPafrac#1#2{\bigPa{\frac{#1}{#2}}}
\def\BigPafrac#1#2{\BigPa{\frac{#1}{#2}}}
\def\biggPafrac#1#2{\biggPa{\frac{#1}{#2}}}
\def\BiggPafrac#1#2{\BiggPa{\frac{#1}{#2}}}

\def\cro#1{[{#1}]}
\def\Cro#1{\left[{#1}\right]}
\def\bigCro#1{\bigl[{#1}\bigr]}
\def\BigCro#1{\Bigl[{#1}\Bigr]}
\def\biggCro#1{\biggl[{#1}\biggr]}
\def\BiggCro#1{\Biggl[{#1}\Biggr]}

\def\abs#1{\mathopen|{#1}\mathclose|}
\def\Abs#1{\left|{#1}\right|}
\def\bigAbs#1{\bigl|{#1}\bigr|}
\def\BigAbs#1{\Bigl|{#1}\Bigr|}
\def\biggAbs#1{\biggl|{#1}\biggr|}
\def\BiggAbs#1{\Biggl|{#1}\Biggr|}

\def\acco#1{\{{#1}\}}
\def\Acco#1{\left\{{#1}\right\}}
\def\bigAcco#1{\bigl\{{#1}\bigr\}}
\def\BigAcco#1{\Bigl\{{#1}\Bigr\}}
\def\biggAcco#1{\biggl\{{#1}\biggr\}}
\def\BiggAcco#1{\Biggl\{{#1}\Biggr\}}

\def\ccro#1{\llbracket{#1}\rrbracket}
\def\Dcro#1{\llbracket{#1}\rrbracket}

\def\floor#1{\lfloor#1\rfloor}
\def\Floor#1{\left\lfloor{#1}\right\rfloor}

\def\sEnsemble#1#2{\mathopen\{#1\mid#2\mathclose\}}
\def\bigEnsemble#1#2{\bigl\{#1\bigm|#2\bigr\}}
\def\BigEnsemble#1#2{\Bigl\{#1\Bigm|#2\Bigr\}}
\def\biggEnsemble#1#2{\biggl\{#1\biggm|#2\biggr\}}
\def\BiggEnsemble#1#2{\Biggl\{#1\Biggm|#2\Biggr\}}
\let\Ensemble=\bigEnsemble

\newcommand\IntO[1]{\left]#1\right[}
\newcommand\IntF[1]{\left[#1\right]}
\newcommand\IntOF[1]{\left]#1\right]}
\newcommand\IntFO[1]{\left[#1\right[}

\newcommand\intO[1]{\mathopen]#1\mathclose[}
\newcommand\intF[1]{\mathopen[#1\mathclose]}
\newcommand\intOF[1]{\mathopen]#1\mathclose]}
\newcommand\intFO[1]{\mathopen[#1\mathclose[}

\newcommand\Fn[3]{#1\colon#2\to#3}
\newcommand\CC[1]{\mathscr{C}^{#1}}
\newcommand\D{\mathop{}\!\mathrm{d}}

\newcommand\longto{\longrightarrow}

\undef\M
\newcommand\M[3]{\mathrm{#1}_{#2}\pa{#3}}
\newcommand\MnR{\M{M}{n}{\R}}
\newcommand\MnC{\M{M}{n}{\C}}
\newcommand\MnK{\M{M}{n}{\K}}
\newcommand\GLnR{\M{GL}{n}{\R}}
\newcommand\GLnC{\M{GL}{n}{\C}}
\newcommand\GLnK{\M{GL}{n}{\K}}
\newcommand\DnR{\M{D}{n}{\R}}
\newcommand\DnC{\M{D}{n}{\C}}
\newcommand\DnK{\M{D}{n}{\K}}
\newcommand\SnR{\M{S}{n}{\R}}
\newcommand\AnR{\M{A}{n}{\R}}
\newcommand\OnR{\M{O}{n}{\R}}
\newcommand\SnRp{\mathrm{S}_n^+(\R)}
\newcommand\SnRpp{\mathrm{S}_n^{++}(\R)}

\newcommand\LE{\mathscr{L}(E)}
\newcommand\GLE{\mathscr{GL}(E)}
\newcommand\SE{\mathscr{S}(E)}
\renewcommand\OE{\mathscr{O}(E)}

\newcommand\ImplD{$\Cro\Rightarrow$}
\newcommand\ImplR{$\Cro\Leftarrow$}
\newcommand\InclD{$\Cro\subset$}
\newcommand\InclR{$\Cro\supset$}
\newcommand\notInclD{$\Cro{\not\subset}$}
\newcommand\notInclR{$\Cro{\not\supset}$}

\newcommand\To[1]{\xrightarrow[#1]{}}
\newcommand\Toninf{\To{\ninf}}

\newcommand\Norm[1]{\|#1\|}
\newcommand\Norme{{\Norm{\cdot}}}

\newcommand\Int[1]{\mathring{#1}}
\newcommand\Adh[1]{\overline{#1}}

\newcommand\Uplet[2]{{#1},\dots,{#2}}
\newcommand\nUplet[3]{(\Uplet{{#1}_{#2}}{{#1}_{#3}})}

\newcommand\Fonction[5]{{#1}\left|\begin{aligned}{#2}&\;\longto\;{#3}\\{#4}&\;\longmapsto\;{#5}\end{aligned}\right.}

\DeclareMathOperator\orth{\bot}
\newcommand\Orth[1]{{#1}^\bot}
\newcommand\PS[2]{\langle#1,#2\rangle}

\newcommand{\Tribu}{\mathscr{T}}
\newcommand{\Part}{\mathcal{P}}
\newcommand{\Pro}{\bigPa{\Omega,\Tribu}}
\newcommand{\Prob}{\bigPa{\Omega,\Tribu,\P}}

\newcommand\DEMO{$\spadesuit$}
\newcommand\DUR{$\spadesuit$}

\newenvironment{psmallmatrix}{\left(\begin{smallmatrix}}{\end{smallmatrix}\right)}

% -----------------------------------------------------------------------------

\newcommand{\FIK}{\mathcal{F}(I,\K)}
\newcommand{\fn}{(f_n)_{n\in \N}}
\newcommand{\Sfn}{\sum_n f_n}

\begin{document}
\title{Th\'eor\`emes de Lebesgue}
\maketitle

% -----------------------------------------------------------------------------

Ces deux th\'eor\`emes du programme proviennent de la th\'eorie de l'int\'egrale de Lebesgue,
qui n'est pas au programme des classes pr\'eparatoires.
Ils fournissent des versions beaucoup plus puissantes des th\'eor\`emes de permutations limite/int\'egrale et somme/int\'egrale.
En particulier, ils permettent d'effectuer des permutations avec des int\'egrales g\'en\'eralis\'ees, et non plus seulement avec des int\'egrales propres sur un segment $[a,b]$.

% -----------------------------------------------------------------------------
\section{Th\'eor\`eme de convergence domin\'ee}

Soit $\fn$ une suite de fonctions de $I$ dans $\K$.
On suppose que:
\begin{enumerate}[label={\emph{\roman*)}}]
\item
  pour tout $n\in \N$, $f_n \colon I \to\K$ est continue par morceaux;
\item
  $f \colon I \to\K$ est continue par morceaux;
\item
  la suite de fonctions $\fn$ converge simplement vers $f$ sur $I$;
\item
  \emph{Hypoth\`ese de domination.}
  Il existe $\varphi\colon I \to \Rp$ continue par morceaux telle que:
  \begin{enumerate}[label={\emph{\alph*)}}]
  \item
    $\forall n\in \N\+\forall x\in I\+ \Abs{f_n(x)}\leq \varphi(x)$;
  \item
    $\varphi$ est \emph{int\'egrable sur $I$}.
  \end{enumerate}
\end{enumerate}

Alors:
\begin{itemize}
\item
  pour tout $n\in \N$, $f_n$ est int\'egrable sur $I$
\item
  $f$ est int\'egrable sur $I$
\item
  $\DS \lim_{n\infty}\int_I f_n =\int_I f$.
\end{itemize}

% -----------------------------------------------------------------------------
\section{Th\'eor\`eme d'int\'egration terme \`a terme}

Soit $\Sfn$ une s\'erie de fonctions de $I$ dans $\K$.
On suppose que:
\begin{enumerate}[label={\emph{\roman*)}}]
\item
  pour tout $n\in \N$, $f_n$ est continue par morceaux et int\'egrable sur $I$;
\item
  la s\'erie de fonctions $\sum_n f_n$ converge simplement sur $I$
  vers une fonction $f \colon I \to\K$, elle-m\^eme continue par morceaux;
\item
  \emph{la s\'erie num\'erique $\DS \sum_n \int_I \Abs{f_n}$ converge.}
\end{enumerate}

Alors:
\begin{itemize}
\item
  $f$ est int\'egrable sur $I$
\item
  la s\'erie num\'erique $\DS \sum_n \int_I f_n$ converge
\item
  $\DS \int_I f =\sum_{n=0}^{+\infty}\int_I f_n$, autrement dit:
  \[ \int_I \sum_{n=0}^{+\infty} f_n =\sum_{n=0}^{+\infty} \int_I f_n. \]
\end{itemize}

% -----------------------------------------------------------------------------
\section{Un peu de topologie}

\Para{Crit\`ere s\'equentiel}

Soit $\Fn{f}{I}{\R}$, $a$ un point de $I$ ou une extr\'emit\'e de $I$ (\'eventuellement $\pm\infty$)
et $\ell$ un r\'eel (ou $\pm\infty$).
Les conditions suivantes sont \'equivalentes:
\begin{enumerate}[label={\emph{\roman*)}}]
\item
  $\DS \lim_{x \to a} f(x) = \ell$;
\item
  pour toute suite $(u_n)$ \`a valeurs dans $I$
  telle que $u_n \Toninf a$,
  on a $f(u_n) \Toninf \ell$.
\end{enumerate}

% -----------------------------------------------------------------------------
\section{Exercices}

% -----------------------------------------------------------------------------
\par\pagebreak[1]\par
\paragraph{Exercice 1}%
\hfill{\tiny 8730}%
\begingroup~

Soit $\DS f_n(x) = \frac{n\sin(x/n)}{x(1+x^2)}$ et $\DS a_n = \int_0^{+\infty} f_n$.
D\'eterminer la limite de la suite $(a_n)$.
\endgroup

% -----------------------------------------------------------------------------
\par\pagebreak[1]\par
\paragraph{Exercice 2}%
\hfill{\tiny 0117}%
\begingroup~

Soit $\Fn{f}{\Rp}{\R}$ continue born\'ee telle que $f(0)\neq0$.
D\'eterminer un \'equivalent de \[ u_n = \int_0^{+\infty} f(x) e^{-nx} \D x. \]
\endgroup

% -----------------------------------------------------------------------------
\par\pagebreak[1]\par
\paragraph{Exercice 3}%
\hfill{\tiny 5975}%
\begingroup~

Soit $\Fonction{f_n}{\R}{\R}{x}{\frac{1}{1+(x-n)^2}}$
\begin{enumerate}
\item
  Montrer que la suite $(f_n)$ converge simplement et d\'eterminer sa limite $f$.
\item
  Calculer $\DS \int_\R{} f_n$ et $\DS \int_\R f$.
\item
  Expliquer ce r\'esultat.
\end{enumerate}
\endgroup

% -----------------------------------------------------------------------------
\par\pagebreak[1]\par
\paragraph{Exercice 4}%
\hfill{\tiny 8108}%
\begingroup~

Montrer que:
\[ \int_0^{+\infty} \frac{\sin x}{e^x - 1} \D x = \sum_{n=1}^{+\infty} \frac{1}{n^2+1} \]

\emph{Astuces:} on pourra remarquer que, pour $x > 0$:
\begin{itemize}
\item
  $\DS \frac{1}{e^x-1} = \frac{e^{-x}}{1-e^{-x}} = \sum_{n=1}^{+\infty} e^{-nx}$.
\item
  $\Abs{\sin(x)} \leq{} x$.
\end{itemize}
\endgroup

% -----------------------------------------------------------------------------
\par\pagebreak[1]\par
\paragraph{Exercice 5}%
\hfill{\tiny 5978}%
\begingroup~

\begin{enumerate}
\item
  Montrer que $\DS \int_0^{+\infty} \frac{\sqrt t}{e^t - 1} \D t
  = \frac{\sqrt \pi}{2} \sum_{n=1}^{+\infty} \frac{1}{n^{3/2}}$.

  On admettra que $\DS \int_0^{+\infty} e^{-x^2} \D x = \frac{\sqrt \pi}{2}$;
  il s'agit de l'\emph{int\'egrale de Gauss}.
\item
  Plus g\'en\'eralement, montrer que pour tout $x > 1$, on a:
  \[ \int_0^{+\infty} \frac{t^{x-1}}{e^t-1} \D t = \Gamma(x)\zeta(x) \]
  o\`u $\DS \zeta(x) = \sum_{n=1}^{+\infty} \frac{1}{n^x}$
  et $\DS \Gamma(x) = \int_0^{+\infty} t^{x-1} e^{-t} \D t$.
\end{enumerate}
\endgroup

% -----------------------------------------------------------------------------
\par\pagebreak[1]\par
\paragraph{Exercice 6}%
\hfill{\tiny 0223}%
\begingroup~

Pour $n\in\Ns$, on pose $\DS f_n(x) = \BigPa{1+\frac{x^2}{n}}^{-n}$.
Soit $\DS u_n = \int_0^{+\infty} f_n$.
\begin{enumerate}
\item
  Montrer que $f_n$ est continue et int\'egrable sur $\Rp$.
\item
  D\'eterminer la limite simple $f$ de la suite de fonctions $(f_n)$.
\item
  \begin{enumerate}
  \item
    Justifier l'in\'egalit\'e suivante, valable $\forall(a,b)\in(\Rps)^2$
    et $\forall \lambda\in[0,1]$:
    \[ \ln\Big(\lambda a + (1-\lambda)b\Big)\geq \lambda\ln a + (1-\lambda) \ln b \]
  \item
    Appliquer l'in\'egalit\'e pr\'ec\'edente avec $a=1+x^2$, $b=1$ et $\lambda=\frac1n$.
  \item
    En d\'eduire que $\forall n\geq1$, $\forall x\in\Rp$, $f_n(x)\leq f_1(x)$.
  \end{enumerate}
\item
  En d\'eduire que: $\DS u_n \Toninf\int_0^{+\infty} e^{-x^2} \D x$.
\item
  On note $\DS W_n = \int_0^{\frac\pi2} \cos^n(\theta) \D\theta$ la
  $n$-i\`eme int\'egrale de Wallis.
  Exprimer $u_n$ en fonction des int\'egrales de Wallis; on pourra
  effectuer le changement de variables $x=\sqrt{n}\tan\theta$.
\item
  On rappelle (cf. TD sur les s\'eries num\'eriques) que
  $W_n \sim \sqrt{\pi/2n}$.
  En d\'eduire la valeur de l'int\'egrale de Gauss $\DS G = \int_0^{+\infty} e^{-x^2} \D x$.
\end{enumerate}
\endgroup

% -----------------------------------------------------------------------------
\par\pagebreak[1]\par
\paragraph{Exercice 7}%
\hfill{\tiny 9748}%
\begingroup~

Soit $\Fn{f}{[0,1]}{\R}$ continue.
Pour $x > 0$, on pose $\DS g(x) = \int_0^1 \frac{xf(t)}{x^2+t^2} \D t$.
\begin{enumerate}
\item
  Soit $(x_n)_{n\in \N}$ est une suite de r\'eels strictement positifs qui tend vers 0.
  Montrer que $g(x_n) \Toninf \frac{\pi f(0)}{2}$.
  On pourra commencer par le changement de variables $t = x_n y$.
\item
  En d\'eduire que $\DS \lim_{0^+} g = \frac{\pi f(0)}{2}$.
\end{enumerate}
\endgroup

% -----------------------------------------------------------------------------
\par\pagebreak[1]\par
\paragraph{Exercice 8}%
\hfill{\tiny 5434}%
\begingroup~

On pose $\DS u_n = \int_0^1 \frac{\D x}{1+x^n}$.
\begin{enumerate}
\item
  D\'eterminer la limite $\ell$ de la suite $(u_n)_{n\in \N}$.
\item
  Exprimer $n(1-u_n)$ sous forme d'une int\'egrale et effectuer le changement
  de variables $y=x^n$.
\item
  En d\'eduire un d\'eveloppement asymptotique de $u_n$ \`a deux termes.
\item
  Bonus:
  montrer que \[ u_n = 1 - \frac{\ln 2}{n} + \frac{\pi^2}{12n^2} + o\BigPa{\frac{1}{n^2}}. \]
\end{enumerate}
\endgroup

% -----------------------------------------------------------------------------
\par\pagebreak[1]\par
\paragraph{Exercice 9}%
\hfill{\tiny 7025}%
\begingroup~

Pour $n\in \N$, on pose \[ f_n(x) = \Pa{1-\frac xn}^{n-1} \ln(x) \text{ si } x\in\intOF{0,n} \]
et $f_n(x) = 0$ sinon.
\begin{enumerate}
\item
  Montrer que la suite de fonctions $(f_n)$ converge simplement sur $\Rps$ vers une fonction $f$ \`a pr\'eciser.
\item
  Montrer que $\DS \int_0^n f_n \Toninf \int_0^{+\infty} f$.
\item
  Calculer $\DS \int_0^n f_n$.
\item
  Sachant que
  $\DS \sum_{k=1}^n \frac1k = \ln(n) + \gamma{} + o(1)$,
  montrer que
  $\DS \int_0^{+\infty} e^{-x} \ln(x) \D x = -\gamma$.
\end{enumerate}
\endgroup

% -----------------------------------------------------------------------------
\par\pagebreak[1]\par
\paragraph{\href{https://psi.miomio.fr/exo/8085.pdf}{Exercice 10}}%
\hfill{\tiny 8085}%
\begingroup~

Pour $n\in \N$, on pose \[ a_n = \int_0^{\pi/4} \tan^n(t) \D t. \]
\'Etudier la s\'erie enti\`ere $\sum{} a_n x^n$:
rayon de convergence,
\'etude aux bornes du domaine de d\'efinition,
calcul de la somme.
\endgroup

% -----------------------------------------------------------------------------
\par\pagebreak[1]\par
\paragraph{Exercice 11}%
\hfill{\tiny 5538}%
\begingroup~

Soit $\DS u_n = \int_0^1 \frac{\D x}{1+x+x^2+\dots+x^n}$.
\begin{enumerate}
\item
  D\'eterminer la limite $\ell$ de la suite $(u_n)$.
\item
  D\'eterminer un \'equivalent de $u_n - \ell$.
\item
  Bonus: d\'eterminer un terme de plus du d\'eveloppment asymptotique de $u_n$.
\end{enumerate}
\endgroup

\end{document}
