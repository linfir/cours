% autogenerated by ytex.rs

\documentclass{scrartcl}

\usepackage[francais]{babel}
\usepackage{geometry}
\usepackage{scrpage2}
\usepackage{lastpage}
\usepackage{ragged2e}
\usepackage{multicol}
\usepackage{etoolbox}
\usepackage{xparse}
\usepackage{enumitem}
\usepackage{csquotes}
\usepackage{amsmath}
\usepackage{amsfonts}
\usepackage{amssymb}
\usepackage{mathrsfs}
\usepackage{stmaryrd}
\usepackage{dsfont}
\usepackage{eurosym}
\usepackage{numprint}
\usepackage[most]{tcolorbox}
\usepackage{tikz}
\usepackage{tkz-tab}
\usepackage[unicode]{hyperref}
\usepackage[ocgcolorlinks]{ocgx2}

\let\ifTwoColumns\iftrue
\def\Classe{$\Psi$2019--2020}

% reproducible builds
% LuaTeX: \pdfvariable suppressoptionalinfo 1023 \relax
\pdfinfoomitdate=1
\pdftrailerid{}

\newif\ifDisplaystyle
\everymath\expandafter{\the\everymath\ifDisplaystyle\displaystyle\fi}
\newcommand\DS{\displaystyle}

\clearscrheadfoot
\pagestyle{scrheadings}
\thispagestyle{empty}
\ohead{\Classe}
\ihead{\thepage/\pageref*{LastPage}}

\setlist[itemize,1]{label=\textbullet}
\setlist[itemize,2]{label=\textbullet}

\ifTwoColumns
  \geometry{margin=1cm,top=2cm,bottom=3cm,foot=1cm}
  \setlist[enumerate]{leftmargin=*}
  \setlist[itemize]{leftmargin=*}
\else
  \geometry{margin=3cm}
\fi

\makeatletter
\let\@author=\relax
\let\@date=\relax
\renewcommand\maketitle{%
    \begin{center}%
        {\sffamily\huge\bfseries\@title}%
        \ifx\@author\relax\else\par\medskip{\itshape\Large\@author}\fi
        \ifx\@date\relax\else\par\bigskip{\large\@date}\fi
    \end{center}\bigskip
    \ifTwoColumns
        \par\begin{multicols*}{2}%
        \AtEndDocument{\end{multicols*}}%
        \setlength{\columnsep}{5mm}
    \fi
}
\makeatother

\newcounter{ParaNum}
\NewDocumentCommand\Para{smo}{%
  \IfBooleanF{#1}{\refstepcounter{ParaNum}}%
  \paragraph{\IfBooleanF{#1}{{\tiny\arabic{ParaNum}~}}#2\IfNoValueF{#3}{ (#3)}}}

\newcommand\I{i}
\newcommand\mi{i}
\def\me{e}

\def\do#1{\expandafter\undef\csname #1\endcsname}
\docsvlist{Ker,sec,csc,cot,sinh,cosh,tanh,coth,th}
\undef\do

\DeclareMathOperator\ch{ch}
\DeclareMathOperator\sh{sh}
\DeclareMathOperator\th{th}
\DeclareMathOperator\coth{coth}
\DeclareMathOperator\cotan{cotan}
\DeclareMathOperator\argch{argch}
\DeclareMathOperator\argsh{argsh}
\DeclareMathOperator\argth{argth}

\let\epsilon=\varepsilon
\let\phi=\varphi
\let\leq=\leqslant
\let\geq=\geqslant
\let\subsetneq=\varsubsetneq
\let\emptyset=\varnothing

\newcommand{\+}{,\;}

\undef\C
\newcommand\ninf{{n\infty}}
\newcommand\N{\mathbb{N}}
\newcommand\Z{\mathbb{Z}}
\newcommand\Q{\mathbb{Q}}
\newcommand\R{\mathbb{R}}
\newcommand\C{\mathbb{C}}
\newcommand\K{\mathbb{K}}
\newcommand\Ns{\N^*}
\newcommand\Zs{\Z^*}
\newcommand\Qs{\Q^*}
\newcommand\Rs{\R^*}
\newcommand\Cs{\C^*}
\newcommand\Ks{\K^*}
\newcommand\Rp{\R^+}
\newcommand\Rps{\R^+_*}
\newcommand\Rms{\R^-_*}
\newcommand{\Rpinf}{\Rp\cup\Acco{+\infty}}

\undef\B
\newcommand\B{\mathscr{B}}

\undef\P
\DeclareMathOperator\P{\mathbb{P}}
\DeclareMathOperator\E{\mathbb{E}}
\DeclareMathOperator\Var{\mathbb{V}}

\DeclareMathOperator*\PetitO{o}
\DeclareMathOperator*\GrandO{O}
\DeclareMathOperator*\Sim{\sim}
\DeclareMathOperator\Tr{tr}
\DeclareMathOperator\Ima{Im}
\DeclareMathOperator\Ker{Ker}
\DeclareMathOperator\Sp{Sp}
\DeclareMathOperator\Diag{diag}
\DeclareMathOperator\Rang{rang}
\DeclareMathOperator*\Coords{Coords}
\DeclareMathOperator*\Mat{Mat}
\DeclareMathOperator\Pass{Pass}
\DeclareMathOperator\Com{Com}
\DeclareMathOperator\Card{Card}
\DeclareMathOperator\Racines{Racines}
\DeclareMathOperator\Vect{Vect}
\DeclareMathOperator\Id{Id}

\newcommand\DerPart[2]{\frac{\partial #1}{\partial #2}}

\def\T#1{{#1}^T}

\def\pa#1{({#1})}
\def\Pa#1{\left({#1}\right)}
\def\bigPa#1{\bigl({#1}\bigr)}
\def\BigPa#1{\Bigl({#1}\Bigr)}
\def\biggPa#1{\biggl({#1}\biggr)}
\def\BiggPa#1{\Biggl({#1}\Biggr)}

\def\pafrac#1#2{\pa{\frac{#1}{#2}}}
\def\Pafrac#1#2{\Pa{\frac{#1}{#2}}}
\def\bigPafrac#1#2{\bigPa{\frac{#1}{#2}}}
\def\BigPafrac#1#2{\BigPa{\frac{#1}{#2}}}
\def\biggPafrac#1#2{\biggPa{\frac{#1}{#2}}}
\def\BiggPafrac#1#2{\BiggPa{\frac{#1}{#2}}}

\def\cro#1{[{#1}]}
\def\Cro#1{\left[{#1}\right]}
\def\bigCro#1{\bigl[{#1}\bigr]}
\def\BigCro#1{\Bigl[{#1}\Bigr]}
\def\biggCro#1{\biggl[{#1}\biggr]}
\def\BiggCro#1{\Biggl[{#1}\Biggr]}

\def\abs#1{\mathopen|{#1}\mathclose|}
\def\Abs#1{\left|{#1}\right|}
\def\bigAbs#1{\bigl|{#1}\bigr|}
\def\BigAbs#1{\Bigl|{#1}\Bigr|}
\def\biggAbs#1{\biggl|{#1}\biggr|}
\def\BiggAbs#1{\Biggl|{#1}\Biggr|}

\def\acco#1{\{{#1}\}}
\def\Acco#1{\left\{{#1}\right\}}
\def\bigAcco#1{\bigl\{{#1}\bigr\}}
\def\BigAcco#1{\Bigl\{{#1}\Bigr\}}
\def\biggAcco#1{\biggl\{{#1}\biggr\}}
\def\BiggAcco#1{\Biggl\{{#1}\Biggr\}}

\def\ccro#1{\llbracket{#1}\rrbracket}
\def\Dcro#1{\llbracket{#1}\rrbracket}

\def\floor#1{\lfloor#1\rfloor}
\def\Floor#1{\left\lfloor{#1}\right\rfloor}

\def\sEnsemble#1#2{\mathopen\{#1\mid#2\mathclose\}}
\def\bigEnsemble#1#2{\bigl\{#1\bigm|#2\bigr\}}
\def\BigEnsemble#1#2{\Bigl\{#1\Bigm|#2\Bigr\}}
\def\biggEnsemble#1#2{\biggl\{#1\biggm|#2\biggr\}}
\def\BiggEnsemble#1#2{\Biggl\{#1\Biggm|#2\Biggr\}}
\let\Ensemble=\bigEnsemble

\newcommand\IntO[1]{\left]#1\right[}
\newcommand\IntF[1]{\left[#1\right]}
\newcommand\IntOF[1]{\left]#1\right]}
\newcommand\IntFO[1]{\left[#1\right[}

\newcommand\intO[1]{\mathopen]#1\mathclose[}
\newcommand\intF[1]{\mathopen[#1\mathclose]}
\newcommand\intOF[1]{\mathopen]#1\mathclose]}
\newcommand\intFO[1]{\mathopen[#1\mathclose[}

\newcommand\Fn[3]{#1\colon#2\to#3}
\newcommand\CC[1]{\mathscr{C}^{#1}}
\newcommand\D{\mathop{}\!\mathrm{d}}

\newcommand\longto{\longrightarrow}

\undef\M
\newcommand\M[3]{\mathrm{#1}_{#2}\pa{#3}}
\newcommand\MnR{\M{M}{n}{\R}}
\newcommand\MnC{\M{M}{n}{\C}}
\newcommand\MnK{\M{M}{n}{\K}}
\newcommand\GLnR{\M{GL}{n}{\R}}
\newcommand\GLnC{\M{GL}{n}{\C}}
\newcommand\GLnK{\M{GL}{n}{\K}}
\newcommand\DnR{\M{D}{n}{\R}}
\newcommand\DnC{\M{D}{n}{\C}}
\newcommand\DnK{\M{D}{n}{\K}}
\newcommand\SnR{\M{S}{n}{\R}}
\newcommand\AnR{\M{A}{n}{\R}}
\newcommand\OnR{\M{O}{n}{\R}}
\newcommand\SnRp{\mathrm{S}_n^+(\R)}
\newcommand\SnRpp{\mathrm{S}_n^{++}(\R)}

\newcommand\LE{\mathscr{L}(E)}
\newcommand\GLE{\mathscr{GL}(E)}
\newcommand\SE{\mathscr{S}(E)}
\renewcommand\OE{\mathscr{O}(E)}

\newcommand\ImplD{$\Cro\Rightarrow$}
\newcommand\ImplR{$\Cro\Leftarrow$}
\newcommand\InclD{$\Cro\subset$}
\newcommand\InclR{$\Cro\supset$}
\newcommand\notInclD{$\Cro{\not\subset}$}
\newcommand\notInclR{$\Cro{\not\supset}$}

\newcommand\To[1]{\xrightarrow[#1]{}}
\newcommand\Toninf{\To{\ninf}}

\newcommand\Norm[1]{\|#1\|}
\newcommand\Norme{{\Norm{\cdot}}}

\newcommand\Int[1]{\mathring{#1}}
\newcommand\Adh[1]{\overline{#1}}

\newcommand\Uplet[2]{{#1},\dots,{#2}}
\newcommand\nUplet[3]{(\Uplet{{#1}_{#2}}{{#1}_{#3}})}

\newcommand\Fonction[5]{{#1}\left|\begin{aligned}{#2}&\;\longto\;{#3}\\{#4}&\;\longmapsto\;{#5}\end{aligned}\right.}

\DeclareMathOperator\orth{\bot}
\newcommand\Orth[1]{{#1}^\bot}
\newcommand\PS[2]{\langle#1,#2\rangle}

\newcommand{\Tribu}{\mathscr{T}}
\newcommand{\Part}{\mathcal{P}}
\newcommand{\Pro}{\bigPa{\Omega,\Tribu}}
\newcommand{\Prob}{\bigPa{\Omega,\Tribu,\P}}

\newcommand\DEMO{$\spadesuit$}
\newcommand\DUR{$\spadesuit$}

\newenvironment{psmallmatrix}{\left(\begin{smallmatrix}}{\end{smallmatrix}\right)}


% -----------------------------------------------------------------------------


\renewcommand{\Tribu}{\Part(\Omega)}

\begin{document}
\title{Probabilit\'es finies}
\maketitle

% -----------------------------------------------------------------------------
\section{G\'en\'eralit\'es}

\Para{D\'efinitions}

\begin{itemize}
\item
  Un \emph{espace probabilisable fini} est un couple $\Pro$
  o\`u $\Omega$ est un ensemble fini
  et $\Part(\Omega)$ est l'ensemble des parties de $\Omega$.
\item
  $\Omega$ s'appelle l'\emph{univers}.
\item
  Un \'el\'ement de $\Omega$ s'appelle un \emph{r\'esultat} ou une \emph{r\'ealisation}.
\item
  Un \'el\'ement de $\Part(\Omega)$ s'appelle un \emph{\'ev\'enement}.
\end{itemize}

\Para{D\'efinitions}

Soit $\Pro$ un espace probabilisable fini.
Soit $A$ et $B$ deux \'ev\'enements.
\begin{itemize}
\item
  L'\'ev\'enement $\emptyset$ est l'\'ev\'enement impossible.
\item
  L'\'ev\'enement $\Omega$ est l'\'ev\'enement certain.
\item
  Les \'ev\'enements $\emptyset$ et $\Omega$ sont les \'ev\'enements triviaux.
\item
  Un \'ev\'enement de la forme $\{x\}$ o\`u $x\in \Omega$ s'appelle un \emph{\'ev\'enement \'el\'ementaire}.
\item
  L'\'ev\'enement \og non $A$\fg{} est $\bar A = \Omega\setminus A$.
\item
  L'\'ev\'enement \og$A$ ou $B$\fg{} est $A\cup B$.
\item
  L'\'ev\'enement \og$A$ et $B$\fg{} est $A\cap B$.
\item
  Les \'ev\'enements $A$ et $B$ sont \emph{incompatibles} s'ils sont disjoints, c.-\`a-d. si $A\cap B=\emptyset$.
\end{itemize}

\Para{D\'efinitions}

Soit $\Pro$ un espace probabilisable fini.
Une \emph{probabilit\'e} sur $\Pro$ est une application
$\Fn\P{\Part(\Omega)}{[0,1]}$ telle que
\begin{itemize}
\item
  $\P(\Omega)=1$.
\item
  Si $A$ et $B$ sont des \'ev\'enements incompatibles,
  alors $\P(A\cup B)=\P(A)+\P(B)$.
\end{itemize}

Le triplet $\Prob$ s'appelle un \emph{espace probabilis\'e fini}.

\Para{Proposition}

Soit $\Prob$ un espace probabilis\'e.
Si $A_1, \dots, A_n$ est une famille d'\'ev\'enements deux \`a deux incompatibles,
alors \[ \P\left( \bigcup_{i=1}^n A_i \right) = \sum_{i=1}^n \P(A_i) \]

\Para{Proposition}

Soit $\Prob$ un espace probabilis\'e.
Pour tout \'ev\'enement $A$ on a
\[ \P(A) = \sum_{x\in A} \P\bigl(\{x\}\bigr) \]

\Para{Proposition}

Soit $\Prob$ un espace probabilis\'e.
\begin{itemize}
\item
  Si $A$ est un \'ev\'enement, alors $\P(\bar A) = 1 - \P(A)$.
\item
  Si $A$ et $B$ sont deux \'ev\'enements tels que $A\subset B$, alors $\P(A)\leq \P(B)$.
\item
  Si $A$ et $B$ sont deux \'ev\'enements, alors $\P(A\cup B)=\P(A)+\P(B)-\P(A\cap B)$.
\end{itemize}

\Para{D\'efinition}

Soit $\Pro$ un espace probabilisable fini.
Un \emph{syst\`eme complet d'\'ev\'enements} est une famille $\nUplet A1n$
d'\'ev\'enements deux \`a deux incompatibles tels que
\[ \bigcup_{i=1}^n A_i = \Omega. \]

\Para{Proposition}

Soit $\Prob$ un espace probabilis\'e.
Soit $\nUplet A1n$ un syst\`eme complet d'\'ev\'enements.
Pour tout \'ev\'enement $B$, on a
\[ \P(B) = \sum_{i=1}^n \P(B\cap A_i). \]

\Para{Proposition-D\'efinition}[probabilit\'e uniforme]

Soit $\Pro$ un espace probabilisable fini.
Alors l'application
\[ \Fonction\P{\Part(\Omega)}{[0,1]}{A}{\frac{\Card{A}}{\Card\Omega}} \]
est une probabilit\'e appel\'ee \emph{probabilit\'e uniforme sur $\Omega$}.

% -----------------------------------------------------------------------------
\section{Ind\'ependance}

\Para{D\'efinition}

Soit $\Prob$ un espace probabilis\'e.
\begin{itemize}
\item
  Deux \'ev\'enements $A$ et $B$ sont \emph{ind\'ependants} si $\P(A\cap B)=\P(A)\P(B)$.
\item
  Une famille $\nUplet A1n$ d'\'evenements sont \emph{(mutuellement) ind\'ependants}
  si pour toute partie $I\subset\Dcro{1,n}$, on a
  \[ \P\left( \bigcap_{i\in I} A_i \right) = \prod_{i\in I} \P(A_i) \]
\end{itemize}

\Para{Attention}

Trois \'ev\'enements peuvent \^etre ind\'ependants deux \`a deux, sans pour autant \^etre mutuellement ind\'ependants.

\Para{Proposition}

Soit $\Prob$ un espace probabilis\'e.
Soit $\nUplet A1n$ des \'ev\'enements mutuellement ind\'ependants.
\begin{itemize}
\item
  Si $I\subset\Dcro{1,n}$,
  alors $(A_i)_{i\in I}$ sont des \'ev\'enements mutuellement ind\'ependants.
\item
  Si pour tout $i\in\Dcro{1,n}$, $B_i \in{} \bigl\{ \emptyset, A_i, \bar A_i,\Omega{} \bigr\}$,
  alors $\nUplet B1n$ sont des \'ev\'enements mutuellement ind\'ependants.
\end{itemize}

% -----------------------------------------------------------------------------
\section{Probabilit\'e conditionnelle}

\Para{Proposition-D\'efinition}

Soit $\Prob$ un espace probabilis\'e.
Soit $A$ un \'ev\'enement tel que $\P(A)>0$.
L'application
\[ \Fonction{\P_A}{\Part(\Omega)}{[0,1]}{B}{\frac{\P(A\cap B)}{\P(A)}} \]
est une probabilit\'e sur $\Pro$
appel\'ee \emph{probabilit\'e conditionnellement \`a $A$},
ou \emph{probabilit\'e sachant $A$}.

On note $\P(B \mid A) = \P_A(B)$.

\Para{Proposition}[lien avec l'ind\'ependance]

Soit $\Prob$ un espace probabilis\'e.
Soit $A$ un \'ev\'enement de probabilit\'e non nulle
et $B$ un \'ev\'enement quelconque.
Alors les \'ev\'enements $A$ et $B$ sont ind\'ependants si et seulement si $\P(B \mid A)=\P(B)$.

\Para{Formule des probabilit\'es totales}

Soit $\Prob$ un espace probabilis\'e.
Soit $\nUplet A1n$ un syst\`eme complet d'\'ev\'enements de probabilit\'es non nulles.
Pour tout \'ev\'enement $B$, on a
\[ \P(B) = \sum_{i=1}^n \P(B \mid A_i) \P(A_i). \]

\Para{Cas particulier}

Soit $\Prob$ un espace probabilis\'e.
Soit $A$ un \'ev\'enement de probabilit\'e $\P(A) \in{} \intO{0,1}$.
Pour tout \'ev\'enement $B$, on a
\[ \P(B) = \P(B \mid A)\P(A) + \P(B \mid \bar A)\P(\bar A) \]

\Para{Formule de Bayes}

Soit $\Prob$ un espace probabilis\'e.
Soit $A$ et $B$ deux \'ev\'enements de probabilit\'es non nulles.
Alors
\[ \P(A \mid B) = \frac{ \P(B \mid A)\P(A) }{ \P(B) } \]

\Para{Corollaire}[formule de Bayes usuelle]

Soit $\Prob$ un espace probabilis\'e.
Soit $\nUplet A1n$ un syst\`eme complet d'\'ev\'enements de probabilit\'es non nulles.
Pour tout \'ev\'enement $B$ de probabilit\'e non nulle,
et pour tout $k\in\Dcro{1,n}$, on a
\[ \P(A_k \mid B) = \frac{ \P(B \mid A_k)\P(A_k) }{ \DS \sum_{i=1}^n \P(B \mid A_i)\P(A_i) }. \]

% -----------------------------------------------------------------------------
\section{Exercices}

% -----------------------------------------------------------------------------
\par\pagebreak[1]\par
\paragraph{Exercice 1}%
\hfill{\tiny 2598}%
\begingroup~

Soit $A$ et $B$ deux \'ev\'enements tels que $\P(A)=\P(B)=\frac34$.
Donner un encadrement, le meilleur possible, de $\P(A\cap B)$?
\endgroup

% -----------------------------------------------------------------------------
\par\pagebreak[1]\par
\paragraph{Exercice 2}%
\hfill{\tiny 7630}%
\begingroup~

Soit $\Omega{} = \{ 1,2,3,4 \}$ muni de la probabilit\'e uniforme.
Soit $A = \{ 1,2 \}$, $B = \{ 1,3 \}$ et $C = \{ 1,4 \}$
Montrer que $A$, $B$ et $C$ sont deux \`a deux ind\'ependants,
mais qu'ils ne sont pas mutuellement ind\'ependants.
\endgroup

% -----------------------------------------------------------------------------
\par\pagebreak[1]\par
\paragraph{Exercice 3}%
\hfill{\tiny 6875}%
\begingroup~

On jette deux d\'es (\`a 6 faces). Expliciter l'univers $\Omega$.

Soit $A_0$ l'\'ev\'enement \og la somme des points est paire\fg,
$A_1$ l'\'ev\'enement \og la somme des points est impaire\fg{}
et $B$ l'\'ev\'enement \og la valeur absolue de la diff\'erence des points est \'egale \`a 4\fg.
Combien comptez-vous d'\'ev\'enements \'el\'ementaires dans $A_0\setminus B$,
dans $A_1\setminus B$?
\endgroup

% -----------------------------------------------------------------------------
\par\pagebreak[1]\par
\paragraph{Exercice 4}%
\hfill{\tiny 1323}%
\begingroup~

L'irradiation par les rayons $X$ de vers \`a soie induit certaines anomalies.
La probabilit\'e d'une anomalie particuli\`ere est $p=\frac1{10}$.
\begin{itemize}
\item
  Quelle est la probabilit\'e de trouver au moins un embryon pr\'esentant
  cette anomalie, sur dix diss\'equ\'es?
\item
  Combien faut-il en diss\'equer pour trouver au moins une anomalie
  avec une probabilit\'e sup\'erieure \`a $50\%$? \`a $95\%$?
\end{itemize}
\endgroup

% -----------------------------------------------------------------------------
\par\pagebreak[1]\par
\paragraph{Exercice 5}%
\hfill{\tiny 1145}%
\begingroup~

Un professeur d\'ecide de faire passer rapidement l'oral de \og probabilit\'es\fg.
L'\'etudiant est autoris\'e \`a r\'epartir quatre boules, deux blanches et deux noires,
entre deux urnes. Le professeur choisit au hasard une des urnes et en extrait
une boule. Si la boule est noire, l'\'etudiant est re\c cu.
Comment r\'epartiriez-vous les boules?
\endgroup

% -----------------------------------------------------------------------------
\par\pagebreak[1]\par
\paragraph{Exercice 6}%
\hfill{\tiny 9539}%
\begingroup~

On plombe un d\'e \`a 6 faces de sorte que la probabilit\'e d'apparition d'une face
donn\'ee est proportionnelle au nombre de points de cette face.
On lance le d\'e deux fois.
Quelle est la probabilit\'e d'obtenir une somme des points \'egale \`a 4?
\endgroup

% -----------------------------------------------------------------------------
\par\pagebreak[1]\par
\paragraph{Exercice 7}%
\hfill{\tiny 5456}%
\begingroup~

Un jeu consiste \`a lancer une pi\`ece (diam\`etre $3$ cm) sur une table
quadrill\'ee par des carr\'es de $4$ cm de c\^ot\'es.
On gagne si la pi\`ece tombe enti\`erement \`a l'int\'erieur d'un carr\'e.
Quelle est la probabilit\'e de gagner \`a ce jeu?
\endgroup

% -----------------------------------------------------------------------------
\par\pagebreak[1]\par
\paragraph{Exercice 8}%
\hfill{\tiny 2970}%
\begingroup~

Un appareil contient $6$ transistors dont $2$ exactement sont d\'efectueux.
On les identifie en testant les transistors l'un apr\`es l'autre.
Le test s'arr\^ete quand les $2$ transistors d\'efectueux sont trouv\'es.
Calculer la probabilit\'e pour que le test:
\begin{itemize}
\item
  soit termin\'e au bout de $2$ op\'erations?
\item
  n\'ecessite strictement plus de $3$ op\'erations?
\end{itemize}
\endgroup

% -----------------------------------------------------------------------------
\par\pagebreak[1]\par
\paragraph{Exercice 9}%
\hfill{\tiny 9743}%
\begingroup~

Dans une course de $20$ chevaux, quelle est la probabilit\'e, en jouant
$3$ chevaux, de gagner le tierc\'e dans l'ordre?
dans l'ordre ou le d\'esordre?
dans le d\'esordre?
\endgroup

% -----------------------------------------------------------------------------
\par\pagebreak[1]\par
\paragraph{Exercice 10}%
\hfill{\tiny 2732}%
\begingroup~

Au Loto, on doit cocher 6 cases dans une grille comportant 49 num\'eros.
\begin{enumerate}
\item
  Quelle est la probabilit\'e de gagner le gros lot (c'est-\`a-dire d'avoir les 6 bons num\'eros)?
\item
  On gagne quelque chose \`a partir du moment o\`u l'on a au moins 3 bons num\'eros.
  Avec quelle probabilit\'e cela arrive-t-il?
\item
  En fait, on peut aussi (en payant plus cher) cocher 7, 8, 9 ou m\^eme 10 num\'eros
  sur la grille. Dans chacun des cas, quelle est la probabilit\'e de gagner le
  gros lot?
\end{enumerate}
\endgroup

% -----------------------------------------------------------------------------
\par\pagebreak[1]\par
\paragraph{Exercice 11}%
\hfill{\tiny 1115}%
\begingroup~

On choisit au hasard un comit\'e de quatre personnes parmi huit am\'ericains, cinq anglais et trois fran\c cais. Quelle est la probabilit\'e:
\begin{itemize}
\item
  qu'il ne se compose que d'am\'ericains?
\item
  qu'aucun am\'ericain ne figure dans ce comit\'e?
\item
  qu'au moins un membre de chaque nation figure dans le comit\'e?
\end{itemize}
\endgroup

% -----------------------------------------------------------------------------
\par\pagebreak[1]\par
\paragraph{Exercice 12 (\og{}paradoxe\fg{} des anniversaires)}%
\hfill{\tiny 9940}%
\begingroup~

Dans une classe de $n$ \'el\`eves, quelle est la probabilit\'e pour que
deux \'etudiants au moins aient m\^eme anniversaire?

Quel est le nombre minimum de personnes dans le groupe pour que
cette probabilit\'e soit d'au moins $50\%$? de $90\%$?
\endgroup

% -----------------------------------------------------------------------------
\par\pagebreak[1]\par
\paragraph{Exercice 13}%
\hfill{\tiny 9201}%
\begingroup~

Alice, Bob, Charly et Denis jouent au bridge, et
re\c coivent chacuns $13$ cartes d'un (m\^eme) jeu de $52$ cartes.

Sachant qu'Alice et Charly ont \`a eux deux $8$ Piques,
on en d\'eduit que Bob et Denis ont $5$ Piques \`a eux deux.
Quelle est la probabilit\'e pour que les Piques soient \og bien r\'epartis\fg,
c.-\`a-d. pour que la r\'epartition des $5$ Piques soit $3-2$ ou $2-3$ entre Bob et Denis?
\endgroup

% -----------------------------------------------------------------------------
\par\pagebreak[1]\par
\paragraph{Exercice 14 (inclusion-exclusion)}%
\hfill{\tiny 4441}%
\begingroup~

Une \'ecole d'ing\'enieur propose \`a ses $320$ \'etudiants deux cours
de math\'ematiques, un en analyse, et un en probabilit\'es.
On sait qu'il y a $140$ \'etudiants qui choisissent l'analyse,
$170$ qui n'assistent pas au cours de probabilit\'es,
et $190$ qui suivent exactement un cours de math\'ematiques.

On choisit un \'etudiant au hasard. Quelle est la probabilit\'e pour
qu'il ne suive pas les deux cours de math\'ematiques?
Qu'il suive au moins un cours de math\'ematiques?
Qu'il suive l'analyse mais pas les probabilit\'es?
\endgroup

% -----------------------------------------------------------------------------
\par\pagebreak[1]\par
\paragraph{Exercice 15}%
\hfill{\tiny 9860}%
\begingroup~

Un \'etudiant, Ulysse, sort habituellement le vendredi soir, avec une probabilit\'e $2/3$.
Or ce jour-l\`a, il y a justement un devoir de math\'ematiques le lendemain matin.
On suppose qu'Ulysse r\'eussit \`a avoir la moyenne avec une probabilit\'e
de $3/4$ s'il n'est pas sorti la veille, et de seulement $1/2$ dans le cas contraire.

Sachant qu'Ulysse n'a pas obtenu la moyenne, quelle est la probabilit\'e pour qu'il soit sorti la veille?
\endgroup

% -----------------------------------------------------------------------------
\par\pagebreak[1]\par
\paragraph{Exercice 16}%
\hfill{\tiny 9752}%
\begingroup~

Un nouveau test de d\'epistage d'une maladie rare et incurable, touchant
environ une personne sur $\numprint{100000}$, vient d'\^etre mis au point.
Pour tester sa validit\'e, on a effectu\'e une \'etude statistique:
sur $534$ sujets sains, le test a \'et\'e positif $1$ seule fois,
et, sur $17$ sujets malades, il a \'et\'e positif $16$ fois.

Une personne effectue ce test, et le r\'esultat est positif.
Quelle est la probabilit\'e pour qu'elle soit atteinte de cette maladie?
Faut-il commercialiser ce test?
\endgroup

% -----------------------------------------------------------------------------
\par\pagebreak[1]\par
\paragraph{Exercice 17}%
\hfill{\tiny 5261}%
\begingroup~

Lo\"ic joue au Loto; s'il gagne, il part aux Seychelles pour un mois complet \`a
coup s\^ur. S'il perd, il ne partira probablement pas (seulement avec une
probabilit\'e de $1/\numprint{10000}$).
\begin{enumerate}
\item
  Quelle est la probabilit\'e pour que Lo\"ic parte aux Seychelles demain?
\item
  Sachant qu'il est parti aux Seychelles, quelle est la probabilit\'e qu'il ait gagn\'e au Loto?
\end{enumerate}
\endgroup

% -----------------------------------------------------------------------------
\par\pagebreak[1]\par
\paragraph{Exercice 18}%
\hfill{\tiny 1387}%
\begingroup~

Trois personnes (Alduire, Basilis et Cl\'eophie)
jouent \`a la roulette russe
de la fa\c con suivante: on fait tourner une fois le barillet au d\'ebut,
puis chacun appuie sur la d\'etente \`a son tour
(Alduire, puis Basilis, puis Cl\'eophie).
Pr\'ef\'ereriez-vous \^etre \`a la place d'Alduire, de Basilis ou de Cl\'eophie?
\endgroup

% -----------------------------------------------------------------------------
\par\pagebreak[1]\par
\paragraph{Exercice 19}%
\hfill{\tiny 5853}%
\begingroup~

Trois machines $A$, $B$ et $C$ fournissent respectivement $50\%$, $30\%$
et $20\%$ de la production d'une usine. Les pourcentages de pi\`eces
d\'efectueuses sont respectivement de $3\%$, $4\%$ et $5\%$.
\begin{enumerate}
\item
  Quelle est la probabilit\'e qu'une pi\`ece, prise au hasard dans la production, soit d\'efecteuse?
\item
  Quelle est la probabilit\'e pour qu'une pi\`ece d\'efectueuse prise au hasard
  provienne de $A$? de $B$? de $C$?
\end{enumerate}
\endgroup

% -----------------------------------------------------------------------------
\par\pagebreak[1]\par
\paragraph{Exercice 20}%
\hfill{\tiny 3772}%
\begingroup~

Alice et Bob jouent aux fl\'echettes.
\`A chaque manche, Alice gagne avec une probabilit\'e $p$.
La partie se d\'eroule en deux manches gagnantes.
Quelle est la probabilit\'e $p'$ pour que Alice gagne?

Quand a-t-on $p' < p$, $p' = p$, $p' > p$? Commenter le r\'esultat.
\endgroup

% -----------------------------------------------------------------------------
\par\pagebreak[1]\par
\paragraph{Exercice 21 (Monty Hall)}%
\hfill{\tiny 8017}%
\begingroup~

Hildebert joue \`a un jeu t\'el\'evis\'e.
Il a face \`a lui, trois portes ($A$, $B$ et $C$) identiques;
derri\`ere l'une d'entre elle se trouve le gros lot, mais derri\`ere
les deux autres, rien du tout.

Hildebert choisit une des portes (disons, la $A$), et alors le
pr\'esentateur (qui conna\^it la porte gagnante) ouvre une autre porte
(disons, la $C$) et montre \`a tous qu'il n'y a rien derri\`ere
celle-ci (la porte $C$).
Il demande ensuite \`a Hildebert s'il pr\'ef\`ere rester sur son choix
ou s'il veut changer.

Hildebert, se disant que cela ne fait aucune diff\'erence,
reste sur son choix (la porte $A$).
A-t-il raison d'agir ainsi?
\endgroup

% -----------------------------------------------------------------------------
\par\pagebreak[1]\par
\paragraph{Exercice 22}%
\hfill{\tiny 5283}%
\begingroup~

\def\Tribu{\Part(\Omega)}
Soit $\Prob$ un espace probabilis\'e fini tel que $\Card\Omega$ est un nombre premier $p$ et $\P$ est la probabilit\'e uniforme.
Soit $A$ et $B$ deux \'ev\'enements non triviaux.
Montrer que $A$ et $B$ ne sont pas ind\'ependants.
\endgroup

% -----------------------------------------------------------------------------
\par\pagebreak[1]\par
\paragraph{Exercice 23}%
\hfill{\tiny 0133}%
\begingroup~

Un couple a deux enfants dont une fille.
Quelle est la probabilit\'e que l'autre enfant soit un gar\c con?
\endgroup

% -----------------------------------------------------------------------------
\par\pagebreak[1]\par
\paragraph{Exercice 24}%
\hfill{\tiny 7845}%
\begingroup~

On consid\`ere un jeu de $52$ cartes.
Apr\`es avoir m\'elang\'e les cartes, quelle est la probabilit\'e que
\begin{enumerate}
\item
  l'as de c\oe ur soit plac\'e avant l'as de pique?
\item
  les as de c\oe ur et de pique soient voisins?
\end{enumerate}
\endgroup

% -----------------------------------------------------------------------------
\par\pagebreak[1]\par
\paragraph{Exercice 25 (la cha\^ine des menteurs)}%
\hfill{\tiny 9242}%
\begingroup~

On suppose qu'un message binaire ($0$ ou $1$) est transmis depuis un emetteur $M_0$ \`a travers une cha\^ine $M_1, M_2, \dots, M_n$ de messagers menteurs, qui transmettent correctement le message avec une probabilit\'e $p$, mais qui changent sa valeur avec une probabilit\'e $1-p$.

Si l'on note $a_n$ la probabilit\'e que l'information transmise par $M_n$ soit identique \`a celle envoy\'ee par $M_0$ (avec comme convention que $a_0=1$), d\'eterminer $a_{n+1}$ en fonction de $a_n$, puis une expression explicite de $a_n$ en fonction de $n$, ainsi que la valeur limite de la suite $(a_n)_{n\in \N}$. Le r\'esultat est-il conforme \`a ce \`a quoi l'on pouvait s'attendre?
\endgroup

% -----------------------------------------------------------------------------
\par\pagebreak[1]\par
\paragraph{\href{https://psi.miomio.fr/exo/4429.pdf}{Exercice 26} (urne de Polya)}%
\hfill{\tiny 4429}%
\begingroup~

Soit $(a,b,c) \in{} \N^3$ avec $a+b>0$.
On consid\`ere une urne contenant $a$ boules blanches et $b$ boules noires.
Apr\`es chaque tirage, on remet la boule choisie dans l'urne avec $c$ nouvelles boules de la m\^eme couleur.
D\'eterminer la probabilit\'e que la $n$-i\`eme boule tir\'ee soit blanche.
\endgroup

% -----------------------------------------------------------------------------
\par\pagebreak[1]\par
\paragraph{Exercice 27}%
\hfill{\tiny 8726}%
\begingroup~

On dispose 8 tours sur un \'echiquier.
Quelle est la probabilit\'e pour qu'aucune tour ne puisse en capturer une autre?
\endgroup

\end{document}
