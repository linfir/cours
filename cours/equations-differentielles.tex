% autogenerated by ytex.rs

\documentclass{scrartcl}

\usepackage[francais]{babel}
\usepackage{geometry}
\usepackage{scrpage2}
\usepackage{lastpage}
\usepackage{ragged2e}
\usepackage{multicol}
\usepackage{etoolbox}
\usepackage{xparse}
\usepackage{enumitem}
\usepackage{csquotes}
\usepackage{amsmath}
\usepackage{amsfonts}
\usepackage{amssymb}
\usepackage{mathrsfs}
\usepackage{stmaryrd}
\usepackage{dsfont}
\usepackage{eurosym}
\usepackage{numprint}
\usepackage[most]{tcolorbox}
\usepackage{tikz}
\usepackage{tkz-tab}
\usepackage[unicode]{hyperref}
\usepackage[ocgcolorlinks]{ocgx2}

\let\ifTwoColumns\iftrue
\def\Classe{$\Psi$2019--2020}

% reproducible builds
% LuaTeX: \pdfvariable suppressoptionalinfo 1023 \relax
\pdfinfoomitdate=1
\pdftrailerid{}

\newif\ifDisplaystyle
\everymath\expandafter{\the\everymath\ifDisplaystyle\displaystyle\fi}
\newcommand\DS{\displaystyle}

\clearscrheadfoot
\pagestyle{scrheadings}
\thispagestyle{empty}
\ohead{\Classe}
\ihead{\thepage/\pageref*{LastPage}}

\setlist[itemize,1]{label=\textbullet}
\setlist[itemize,2]{label=\textbullet}

\ifTwoColumns
  \geometry{margin=1cm,top=2cm,bottom=3cm,foot=1cm}
  \setlist[enumerate]{leftmargin=*}
  \setlist[itemize]{leftmargin=*}
\else
  \geometry{margin=3cm}
\fi

\makeatletter
\let\@author=\relax
\let\@date=\relax
\renewcommand\maketitle{%
    \begin{center}%
        {\sffamily\huge\bfseries\@title}%
        \ifx\@author\relax\else\par\medskip{\itshape\Large\@author}\fi
        \ifx\@date\relax\else\par\bigskip{\large\@date}\fi
    \end{center}\bigskip
    \ifTwoColumns
        \par\begin{multicols*}{2}%
        \AtEndDocument{\end{multicols*}}%
        \setlength{\columnsep}{5mm}
    \fi
}
\makeatother

\newcounter{ParaNum}
\NewDocumentCommand\Para{smo}{%
  \IfBooleanF{#1}{\refstepcounter{ParaNum}}%
  \paragraph{\IfBooleanF{#1}{{\tiny\arabic{ParaNum}~}}#2\IfNoValueF{#3}{ (#3)}}}

\newcommand\I{i}
\newcommand\mi{i}
\def\me{e}

\def\do#1{\expandafter\undef\csname #1\endcsname}
\docsvlist{Ker,sec,csc,cot,sinh,cosh,tanh,coth,th}
\undef\do

\DeclareMathOperator\ch{ch}
\DeclareMathOperator\sh{sh}
\DeclareMathOperator\th{th}
\DeclareMathOperator\coth{coth}
\DeclareMathOperator\cotan{cotan}
\DeclareMathOperator\argch{argch}
\DeclareMathOperator\argsh{argsh}
\DeclareMathOperator\argth{argth}

\let\epsilon=\varepsilon
\let\phi=\varphi
\let\leq=\leqslant
\let\geq=\geqslant
\let\subsetneq=\varsubsetneq
\let\emptyset=\varnothing

\newcommand{\+}{,\;}

\undef\C
\newcommand\ninf{{n\infty}}
\newcommand\N{\mathbb{N}}
\newcommand\Z{\mathbb{Z}}
\newcommand\Q{\mathbb{Q}}
\newcommand\R{\mathbb{R}}
\newcommand\C{\mathbb{C}}
\newcommand\K{\mathbb{K}}
\newcommand\Ns{\N^*}
\newcommand\Zs{\Z^*}
\newcommand\Qs{\Q^*}
\newcommand\Rs{\R^*}
\newcommand\Cs{\C^*}
\newcommand\Ks{\K^*}
\newcommand\Rp{\R^+}
\newcommand\Rps{\R^+_*}
\newcommand\Rms{\R^-_*}
\newcommand{\Rpinf}{\Rp\cup\Acco{+\infty}}

\undef\B
\newcommand\B{\mathscr{B}}

\undef\P
\DeclareMathOperator\P{\mathbb{P}}
\DeclareMathOperator\E{\mathbb{E}}
\DeclareMathOperator\Var{\mathbb{V}}

\DeclareMathOperator*\PetitO{o}
\DeclareMathOperator*\GrandO{O}
\DeclareMathOperator*\Sim{\sim}
\DeclareMathOperator\Tr{tr}
\DeclareMathOperator\Ima{Im}
\DeclareMathOperator\Ker{Ker}
\DeclareMathOperator\Sp{Sp}
\DeclareMathOperator\Diag{diag}
\DeclareMathOperator\Rang{rang}
\DeclareMathOperator*\Coords{Coords}
\DeclareMathOperator*\Mat{Mat}
\DeclareMathOperator\Pass{Pass}
\DeclareMathOperator\Com{Com}
\DeclareMathOperator\Card{Card}
\DeclareMathOperator\Racines{Racines}
\DeclareMathOperator\Vect{Vect}
\DeclareMathOperator\Id{Id}

\newcommand\DerPart[2]{\frac{\partial #1}{\partial #2}}

\def\T#1{{#1}^T}

\def\pa#1{({#1})}
\def\Pa#1{\left({#1}\right)}
\def\bigPa#1{\bigl({#1}\bigr)}
\def\BigPa#1{\Bigl({#1}\Bigr)}
\def\biggPa#1{\biggl({#1}\biggr)}
\def\BiggPa#1{\Biggl({#1}\Biggr)}

\def\pafrac#1#2{\pa{\frac{#1}{#2}}}
\def\Pafrac#1#2{\Pa{\frac{#1}{#2}}}
\def\bigPafrac#1#2{\bigPa{\frac{#1}{#2}}}
\def\BigPafrac#1#2{\BigPa{\frac{#1}{#2}}}
\def\biggPafrac#1#2{\biggPa{\frac{#1}{#2}}}
\def\BiggPafrac#1#2{\BiggPa{\frac{#1}{#2}}}

\def\cro#1{[{#1}]}
\def\Cro#1{\left[{#1}\right]}
\def\bigCro#1{\bigl[{#1}\bigr]}
\def\BigCro#1{\Bigl[{#1}\Bigr]}
\def\biggCro#1{\biggl[{#1}\biggr]}
\def\BiggCro#1{\Biggl[{#1}\Biggr]}

\def\abs#1{\mathopen|{#1}\mathclose|}
\def\Abs#1{\left|{#1}\right|}
\def\bigAbs#1{\bigl|{#1}\bigr|}
\def\BigAbs#1{\Bigl|{#1}\Bigr|}
\def\biggAbs#1{\biggl|{#1}\biggr|}
\def\BiggAbs#1{\Biggl|{#1}\Biggr|}

\def\acco#1{\{{#1}\}}
\def\Acco#1{\left\{{#1}\right\}}
\def\bigAcco#1{\bigl\{{#1}\bigr\}}
\def\BigAcco#1{\Bigl\{{#1}\Bigr\}}
\def\biggAcco#1{\biggl\{{#1}\biggr\}}
\def\BiggAcco#1{\Biggl\{{#1}\Biggr\}}

\def\ccro#1{\llbracket{#1}\rrbracket}
\def\Dcro#1{\llbracket{#1}\rrbracket}

\def\floor#1{\lfloor#1\rfloor}
\def\Floor#1{\left\lfloor{#1}\right\rfloor}

\def\sEnsemble#1#2{\mathopen\{#1\mid#2\mathclose\}}
\def\bigEnsemble#1#2{\bigl\{#1\bigm|#2\bigr\}}
\def\BigEnsemble#1#2{\Bigl\{#1\Bigm|#2\Bigr\}}
\def\biggEnsemble#1#2{\biggl\{#1\biggm|#2\biggr\}}
\def\BiggEnsemble#1#2{\Biggl\{#1\Biggm|#2\Biggr\}}
\let\Ensemble=\bigEnsemble

\newcommand\IntO[1]{\left]#1\right[}
\newcommand\IntF[1]{\left[#1\right]}
\newcommand\IntOF[1]{\left]#1\right]}
\newcommand\IntFO[1]{\left[#1\right[}

\newcommand\intO[1]{\mathopen]#1\mathclose[}
\newcommand\intF[1]{\mathopen[#1\mathclose]}
\newcommand\intOF[1]{\mathopen]#1\mathclose]}
\newcommand\intFO[1]{\mathopen[#1\mathclose[}

\newcommand\Fn[3]{#1\colon#2\to#3}
\newcommand\CC[1]{\mathscr{C}^{#1}}
\newcommand\D{\mathop{}\!\mathrm{d}}

\newcommand\longto{\longrightarrow}

\undef\M
\newcommand\M[3]{\mathrm{#1}_{#2}\pa{#3}}
\newcommand\MnR{\M{M}{n}{\R}}
\newcommand\MnC{\M{M}{n}{\C}}
\newcommand\MnK{\M{M}{n}{\K}}
\newcommand\GLnR{\M{GL}{n}{\R}}
\newcommand\GLnC{\M{GL}{n}{\C}}
\newcommand\GLnK{\M{GL}{n}{\K}}
\newcommand\DnR{\M{D}{n}{\R}}
\newcommand\DnC{\M{D}{n}{\C}}
\newcommand\DnK{\M{D}{n}{\K}}
\newcommand\SnR{\M{S}{n}{\R}}
\newcommand\AnR{\M{A}{n}{\R}}
\newcommand\OnR{\M{O}{n}{\R}}
\newcommand\SnRp{\mathrm{S}_n^+(\R)}
\newcommand\SnRpp{\mathrm{S}_n^{++}(\R)}

\newcommand\LE{\mathscr{L}(E)}
\newcommand\GLE{\mathscr{GL}(E)}
\newcommand\SE{\mathscr{S}(E)}
\renewcommand\OE{\mathscr{O}(E)}

\newcommand\ImplD{$\Cro\Rightarrow$}
\newcommand\ImplR{$\Cro\Leftarrow$}
\newcommand\InclD{$\Cro\subset$}
\newcommand\InclR{$\Cro\supset$}
\newcommand\notInclD{$\Cro{\not\subset}$}
\newcommand\notInclR{$\Cro{\not\supset}$}

\newcommand\To[1]{\xrightarrow[#1]{}}
\newcommand\Toninf{\To{\ninf}}

\newcommand\Norm[1]{\|#1\|}
\newcommand\Norme{{\Norm{\cdot}}}

\newcommand\Int[1]{\mathring{#1}}
\newcommand\Adh[1]{\overline{#1}}

\newcommand\Uplet[2]{{#1},\dots,{#2}}
\newcommand\nUplet[3]{(\Uplet{{#1}_{#2}}{{#1}_{#3}})}

\newcommand\Fonction[5]{{#1}\left|\begin{aligned}{#2}&\;\longto\;{#3}\\{#4}&\;\longmapsto\;{#5}\end{aligned}\right.}

\DeclareMathOperator\orth{\bot}
\newcommand\Orth[1]{{#1}^\bot}
\newcommand\PS[2]{\langle#1,#2\rangle}

\newcommand{\Tribu}{\mathscr{T}}
\newcommand{\Part}{\mathcal{P}}
\newcommand{\Pro}{\bigPa{\Omega,\Tribu}}
\newcommand{\Prob}{\bigPa{\Omega,\Tribu,\P}}

\newcommand\DEMO{$\spadesuit$}
\newcommand\DUR{$\spadesuit$}

\newenvironment{psmallmatrix}{\left(\begin{smallmatrix}}{\end{smallmatrix}\right)}


% -----------------------------------------------------------------------------


\newcommand{\eq}[1]{\mathrm{(#1)}}
\newcommand{\mtag}[1]{\tag{$\mathrm{#1}$}}
\newcommand{\solI}[1]{\mathcal{S}_I(#1)}
\newcommand{\solJ}[1]{\mathcal{S}_J(#1)}
\newcommand{\VnK}{\mathrm{M}_{n,1}(\K)}

\begin{document}
\title{\'Equations diff\'erentielles}
\maketitle

Ce chapitre reprend rapidement et \'etend l'\'etude des \'equations diff\'erentielles lin\'eaires vues en premi\`ere ann\'ee.

\Para{Notations}

Dans ce chapitre,
\begin{itemize}
\item $\K$ d\'esigne le corps $\R$ ou $\C$,
\item $I$ d\'esigne un intervalle (non vide et non r\'eduit \`a un point) de $\R$,
\item $J$ d\'esigne un intervalle (non vide et non r\'eduit \`a un point) \emph{inclus dans $I$},
\item toutes les fonctions qui interviennent sont (au moins) continues,
\item $t\in I$ est la variable par rapport \`a laquelle on d\'erive,
\item dans le cas scalaire, la fonction inconnue est $x \colon t \mapsto x(t)\in \K$,
\item dans le cas vectoriel,
  \begin{itemize}
  \item on identifie $\VnK$ et $\K^n$,
  \item la fonction inconnue est $X \colon t \mapsto X(t)\in \K^n$.
  \end{itemize}
\end{itemize}

% -----------------------------------------------------------------------------
\section{\'Equations diff\'erentielles lin\'eaires scalaires du premier ordre}

\subsection{G\'en\'eralit\'es}

\Para{D\'efinitions}

Une \emph{\'equation diff\'erentielle lin\'eaire scalaire du premier ordre}
est une \'equation de la forme
\[\mtag{E} a(t) x'(t) + b(t) x(t) = c(t),\]
o\`u $a$, $b$ et $c$ sont des fonctions \emph{continues} de $I$ dans $\K$,
et $x$ une fonction inconnue de $I$ dans $\K$.

Une \emph{solution sur $J\subset I$} de cette \'equation est une fonction $f$ de $J$ dans $\K$,
d\'erivable en tout point de $J$, telle que pour tout $t\in J$, on ait
\[a(t) f'(t) + b(t) f(t) = c(t).\]
On note $\solJ{E}$ l'ensemble des solutions de $\eq{E}$ sur $J$.

\emph{R\'esoudre} $\eq{E}$, c'est d\'eterminer $\solJ{E}$
pour tout intervalle $J\subset I$.

\Para{D\'efinition}

Avec les m\^emes notations, dans le cas $\K=\R$, on consid\`ere l'\'equation diff\'erentielle
\[\mtag{E} a(t) x'(t) + b(t) x(t) = c(t).\]
Les courbes repr\'esentatrices des solutions de $\eq{E}$ s'appellent \emph{courbes int\'egrales} de $\eq{E}$.

\Para{D\'efinitions}

On consid\`ere l'\'equation diff\'erentielle

\[\mtag{E} a(t) x'(t) + b(t) x(t) = c(t).\]
\begin{itemize}
\item $\eq{E}$ est dite \emph{\`a coefficients constants}
  si et seulement si les fonctions $a$ et $b$ sont constantes.
\item $\eq{E}$ est dite \emph{normalis\'ee} ou \emph{r\'esolue en $x'$}
  si et seulement si \[\forall t\in I\+ a(t) = 1.\]
\item $\eq{E}$ est dite \emph{homog\`ene} ou \emph{sans second membre}
  si et seulement si \[\forall t\in I\+ c(t) = 0.\]
\item On appelle \emph{\'equation diff\'erentielle homog\`ene associ\'ee} \`a $\eq{E}$
  ou \emph{\'equation diff\'erentielle sans second membre associ\'ee} \`a $\eq{E}$
  l'\'equation diff\'erentielle
  \[\mtag{E_0} a(t) x'(t) + b(t) x(t) = 0.\]
\end{itemize}

\Para{D\'efinition}

Un \emph{probl\`eme de Cauchy} du premier ordre
est la donn\'ee d'une \'equation diff\'erentielle du premier ordre
normalis\'ee et d'une condition initiale.

Un probl\`eme de Cauchy lin\'eaire du premier ordre est donc de la forme
\[\mtag{C} \begin{cases}
    \forall t\in I\+ x'(t) + b(t)x(t) = c(t), \\
    x(t_0) = x_0 \text{ o\`u } t_0\in I \text{ et } x_0\in \K.
\end{cases}\]

\subsection{\'Etude th\'eorique}

\Para{Remarque}

\'Etant donn\'e l'\'equation diff\'erentielle lin\'eaire scalaire du premier ordre
\[\mtag{E} a(t) x'(t) + b(t) x(t) = c(t),\]
on peut se ramener \`a une \'equation normalis\'ee:
\begin{enumerate}
\item si $a$ ne s'annule pas sur $I$, il suffit de diviser par $a(t)$;
\item par contre, si $a$ s'annule, on d\'ecoupe $I$ en intervalles
  o\`u $a$ ne s'annule pas, puis on fait une \'etude sur chaque intervalle.
  \`A la fin, on cherche \`a \og raccorder\fg{} les solutions.
\end{enumerate}

\Para{Proposition}

Soit $f$ une solution sur $J$ de l'\'equation diff\'erentielle normalis\'ee
\[\mtag{E} x'(t) + b(t) x(t) = c(t).\]
Alors $f$ est de classe $\CC1$ sur $J$.

De plus, si $b$ et $c$ sont de classe $\CC p$ sur $J$,
alors $f$ est de classe $\CC{p+1}$ sur $J$;
si $b$ et $c$ sont de classe $\CC\infty$ sur $J$, alors $f$ l'est \'egalement.

\Para{Th\'eor\`eme}[Cauchy-Lipschitz lin\'eaire]

Soit $\Fn bI\K$ et $\Fn cI\K$ deux fonctions \emph{continues}.
Soit $t_0\in I$ et $x_0\in \K$.
Alors le probl\`eme de Cauchy
\[\mtag{C}
  \begin{cases}
    \forall t\in I\+ x'(t) + b(t) x(t) = c(t), \\
    x(t_0) = x_0
\end{cases}\]
admet une unique solution sur tout intervalle $J$
tel que $t_0\in J$ et $J\subset I$.

De plus, la solution sur $J$ n'est autre que la restriction \`a $J$
de la solution sur $I$.

\Para{Corollaire}

Soit $\eq{E}$ une \'equation diff\'erentielle lin\'eaire scalaire normalis\'ee
du premier ordre sur $I$
\[\mtag{E} x'(t) + b(t)x(t) = c(t).\]
Alors les courbes int\'egrales de $\eq{E}$ sont disjointes.
Plus pr\'ecis\'ement, elles forment une partition de $I\times \R$.

\Para{Proposition}

Soit $f$ une solution sur $J$ de l'\'equation diff\'erentielle lin\'eaire scalaire
normalis\'ee homog\`ene du premier ordre
\[\mtag{E_0} x'(t) + b(t) x(t) = 0.\]
Alors
\begin{itemize}
\item soit $f$ est identiquement nulle sur $J$,
\item soit $f$ ne s'annule pas sur $J$.
\end{itemize}

\Para{Th\'eor\`eme}[de structure]

On consid\`ere les \'equations diff\'erentielles lin\'eaires scalaires
normalis\'ees du premier ordre
\[\mtag{E}   x'(t) + b(t)x(t) = c(t)\]
\[\mtag{E_0} x'(t) + b(t)x(t) = 0\]
\begin{itemize}
\item $\solJ{E_0}$ est un sous-espace vectoriel de $\CC1(I,\K)$ de dimension $1$.
\item $\solJ{E}$ est un sous-espace \emph{affine} de $\CC1(I,\K)$ de dimension $1$ et de direction $\solJ{E_0}$.
\end{itemize}

\subsection{R\'esolution pratique}

\subsubsection{Principe de superposition}

\Para{Corollaire}

\'Etant donn\'e une \'equation lin\'eaire $\eq{E}$,
la solution g\'en\'erale de $\eq{E}$ est donn\'ee
par la somme de la solution g\'en\'erale de l'\'equation homog\`ene $\eq{E_0}$
et d'une solution particuli\`ere de $\eq{E}$.

Plus pr\'ecis\'ement, soit $x_p$ une solution particuli\`ere de $\eq{E}$ sur $J$.
Alors toute solution $x$ de $\eq{E}$ sur $J$ est de la forme
\[x = x_0 + x_p,\]
o\`u $x_0$ est une solution de $\eq{E_0}$ sur $J$.

\emph{Pour r\'esoudre $\eq{E}$, il suffit donc de r\'esoudre $\eq{E_0}$ et de trouver une solution particuli\`ere de $\eq{E}$.}

\Para{Remarque}

Si $\eq{E}$ est de la forme
\[\mtag{E} a(t) x'(t) + b(t) x(t) = c_1(t) + c_2(t),\]
pour trouver une solution particuli\`ere de $\eq{E}$, il suffit de
\begin{itemize}
\item trouver une solution particuli\`ere $x_{p,1}$ de
  \[\mtag{E_1} a(t) x'(t) + b(t) x(t) = c_1(t),\]
\item trouver une solution particuli\`ere $x_{p,2}$ de
  \[\mtag{E_2} a(t) x'(t) + b(t) x(t) = c_2(t),\]
\item poser $x_p = x_{p,1} + x_{p,2}$.
\end{itemize}

Dans ces conditions, $x_p$ est une solution particuli\`ere de $\eq{E}$.

\subsubsection{R\'esolution de l'\'equation homog\`ene}

\Para{Proposition}

On consid\`ere l'\'equation diff\'erentielle
\[\mtag{E_0} x'(t) + a(t) x(t) = 0,\]
o\`u $\Fn aI\K$ est une fonction continue.

On note $A$ une primitive de $a$ sur $I$.
Alors, pour $J\subset I$,
\[\solJ{E_0} = \Bigl\{ \Fn fJ\K\,\Bigm|\, \exists K\in \K\+ \forall t\in J \+ f(t) = K \me^{-A(t)} \Bigr\}.\]
Moins formellement, on dit que la \emph{solution g\'en\'erale} de $(E_0)$
sur $J$ est donn\'ee par
\[x(t) = K \me^{-A(t)} \quad \text{o\`u} \quad K\in \K.\]

\Para{Remarque}

La \og r\'esolution\fg{} suivante peut permettre de retrouver la formule
\begin{enumerate}
\item $x'(t) + a(t) x(t) = 0$
\item $\frac{x'(t)}{x(t)} = -a(t)$
\item $\ln\Abs{x(t)} = -A(t) + C_0$ o\`u $C_0\in \R$
\item $\Abs{x(t)} = \me^{C_0} \me^{-A(t)} = C_1 \me^{-A(t)}$ o\`u $C_1\in\Rp$
\item $x(t) = K \me^{-A(t)}$ o\`u $K\in \R$
\end{enumerate}

N\'eanmoins, \emph{il ne s'agit pas d'une preuve} \`a cause
\begin{itemize}
\item de la division par $x(t)$ qui pourrait s'annuler, et
\item du passage de $\Abs{x(t)}$ \`a $x(t)$ n\'ecessite des justifications
  (on pourrait imaginer que $x$ change de signe et donc que $K$ soit une fonction de $t$...)
\end{itemize}

Bref, \`a \'eviter sur une copie, sauf si l'on sait d\'ej\`a que $\forall x\in I$, $x(t)>0$.

\subsubsection{Recherche d'une solution particuli\`ere}

\Para{Proposition}[cas d'une \'equation \`a coefficients constants]

On consid\`ere l'\'equation diff\'erentielle
\[\mtag{E} x'(t) + ax(t) = P(t) \me^{\alpha t},\]
o\`u $P\in \K_d[X]$.

Il existe une unique solution particuli\`ere de la forme
\begin{enumerate}
\item Si $\alpha\neq-a$, on prend $x_p(t) = Q(t) \me^{\alpha t}$
  o\`u $Q\in \K_d[X]$ \`a d\'eterminer.
\item Si $\alpha= -a$, on prend $x_p(t) = tQ(t) \me^{\alpha t}$
  o\`u $Q\in \K_d[X]$ \`a d\'eterminer.
\end{enumerate}

Bien s\^ur, si le second membre est une somme de termes de ce type,
il suffit de superposer les solutions particuli\`eres correspondantes.

\Para{Exemples}
\begin{enumerate}
\item Pour $x' + x = t\me^t$, on prend $x_p = (at+b)\me^t$.
\item Pour $x' - x = (t+1)\me^t$, on prend $x_p = t(at+b)\me^t$.
\item Pour $x' + x = \sin t$, on prend $x_p = a\cos t + b\sin t$.
\item Pour $x' + 3x = t\cos 2t$, on prend $x_p = (at+b)\cos 2t + (ct+d)\sin 2t$.
\item Pour $x' + x = t\me^t - \me^{2t}$, on prend $x_p = (at+b)\me^t + c\me^{2t}$.
\end{enumerate}

\Para{Proposition}[m\'ethode de variation de la constante]

On consid\`ere les \'equations diff\'erentielles
\[\mtag{E}   x'(t) + a(t) x(t) = b(t),\]
\[\mtag{E_0} x'(t) + a(t) x(t) = 0.\]
Soit $\varphi$ une solution non nulle de $\eq{E_0}$ sur $J$.
On rappelle que $\varphi$ ne s'annule pas et que
la solution g\'en\'erale de $\eq{E_0}$ est donn\'ee par
\[x_0(t) = K\varphi(t) \quad \text{o\`u } K\in \K.\]

On recherche alors les solutions de $\eq{E}$ sous la forme
$x(t) =\lambda(t)\varphi(t)$, et l'on obtient
\[\mtag{E'}\lambda'(t)\varphi(t) = b(t),\]
qui est facile \`a r\'esoudre.

\Para{Corollaire}

La solution du probl\`eme de Cauchy
\[\begin{cases}
    \forall t\in I\+ x'(t) + a(t)x(t) = b(t), \\
    x(t_0) = x_0
\end{cases}\]
est donn\'ee par
\[x(t) = \me^{-A(t)} \left( \me^{A(t_0)} x_0 + \int_{t_0}^{t} \me^{A(u)}b(u) \D u \right),\]
o\`u $A$ est une primitive de $a$ sur $I$.

Cette formule \emph{n'est pas \`a conna\^itre}, mais \`a savoir retrouver au besoin.

\subsubsection{Probl\`emes de raccords}

\Para{Remarque}

On consid\`ere l'\'equation diff\'erentielle sur $I$
\[\mtag{E} a(t)x'(t) + b(t)x(t) = c(t).\]
Si $a$ s'annule sur $I$, on est face \`a un probl\`eme de raccord.

Supposons par exemple que $I=\R$ et que $a$ s'annule uniquement en $1$.
\begin{itemize}
\item On pose alors $I_1 = \intO{-\infty,1}$ et $I_2 = \intO{1,+\infty}$.
  Sur ces deux intervalles, $\eq{E}$ est une \'equation normalis\'ee, donc
  on peut utiliser les r\'esultats pr\'ec\'edents.
\item On r\'esoud $\eq{E}$ sur $I_1$ et sur $I_2$.
\item On d\'etermine, g\'en\'eralement par la m\'ethode d'analyse-synth\`ese,
  quelles sont les fonctions $f$
  solutions de $\eq{E}$ sur $\R$, sachant que la restriction de $f$ \`a $I_k$
  est n\'ecessairement une solution de $\eq{E}$ sur $I_k$.
\end{itemize}

\Para{Exemples}
\begin{enumerate}
\item R\'esoudre sur $\R$ l'\'equation $tx'(t) - 2x(t) = 0$.
\item R\'esoudre sur $\R$ l'\'equation $2tx'(t) - x(t) = 0$.
\end{enumerate}

% -----------------------------------------------------------------------------
\section{Syst\`emes diff\'erentiels lin\'eaires}

\Para{Exemple}

On cherche \`a r\'esoudre le syst\`eme diff\'erentiel
\[\mtag{S} \left\{ \begin{aligned}
      x'(t) &= a(t) x(t) + b(t) y(t) + c(t) z(t) + j(t) \\
      y'(t) &= d(t) x(t) + e(t) y(t) + f(t) z(t) + k(t) \\
      z'(t) &= g(t) x(t) + h(t) y(t) + i(t) z(t) + l(t)
\end{aligned} \right.\]
Ce syst\`eme peut s'\'ecrire
\[\mtag{S} X'(t) = A(t) X(t) + B(t)\]
o\`u $X(t) = \begin{pmatrix} x(t) \\ y(t) \\ z(t) \end{pmatrix}$,
$A(t) = \begin{pmatrix} a(t) & b(t) & c(t) \\ d(t) & e(t) & f(t) \\ g(t) & h(t) & i(t) \end{pmatrix}$, \\
et $B(t) = \begin{pmatrix} j(t) \\ k(t) \\ l(t) \end{pmatrix}$.

\subsection{G\'en\'eralit\'es}

\Para{D\'efinitions}

Un \emph{syst\`eme diff\'erentiel lin\'eaire du premier ordre}
est une \'equation de la forme
\[\mtag{S} X'(t) = A(t) X(t) + B(t)\]
o\`u $\Fn{A}{I}{\MnK}$ et $\Fn{B}{I}{\K^n}$ sont des fonctions continues,
et $\Fn{X}{I}{\K^n}$ est une fonction inconnue.

Une \emph{solution sur $J\subset I$} de ce syst\`eme est une fonction $f$ de $J$ dans $\K^n$,
d\'erivable en tout point de $J$, telle que pour tout $t\in J$, on ait
\[f'(t) = A(t) f(t) + B(t).\]
On note $\solJ{S}$ l'ensemble des solutions de $\eq{S}$ sur $J$.

\emph{R\'esoudre} $\eq{S}$, c'est d\'eterminer $\solJ{S}$
pour tout intervalle $J\subset I$.

\Para{D\'efinitions}

On consid\`ere le syst\`eme diff\'erentiel lin\'eaire du premier ordre
\[\mtag{S} X'(t) = A(t) X(t) + B(t)\]
\begin{itemize}
\item $\eq{S}$ est dit \emph{\`a coefficients constants}
  si et seulement si la fonction $A$ est constante.
\item $\eq{S}$ est dit \emph{homog\`ene} ou \emph{sans second membre}
  si et seulement si \[\forall t\in I \+ B(t) = 0.\]
\item On appelle \emph{syst\`eme diff\'erentiel homog\`ene associe} \`a $\eq{S}$
  le syst\`eme diff\'erentiel
  \[\mtag{S_0} X'(t) = A(t) X(t).\]
\end{itemize}

\Para{D\'efinition}

Un \emph{probl\`eme de Cauchy} du premier ordre
est la donn\'ee d'un syst\`eme diff\'erentiel lin\'eaire du premier ordre
et d'une condition initiale.

Un probl\`eme de Cauchy lin\'eaire du premier ordre est donc de la forme
\[\mtag{C} \begin{cases}
    \forall t\in I\+ X'(t) = A(t)X(t) + B(t), \\
    X(t_0) = X_0 \text{ o\`u $t_0 \in I$ et $X_0\in \K$.}
\end{cases}\]

\subsection{\'Etude th\'eorique}

\Para{Proposition}

Soit $f$ une solution sur $J$ du syst\`eme diff\'erentiel
\[\mtag{S} X'(t) = A(t) X(t) + B(t).\]
Alors $f$ est de classe $\CC1$ sur $J$.

De plus, si $B$ et $C$ sont de classe $\CC p$ sur $J$,
alors $f$ est de classe $\CC{p+1}$ sur $J$;
si $b$ et $c$ sont de classe $\CC\infty$ sur $J$, alors $f$ l'est \'egalement.

\Para{Th\'eor\`eme}[Cauchy-Lipschitz lin\'eaire]

Soit $\Fn{A}{I}{\MnK}$ et $\Fn{B}{I}{\K^n}$ deux fonctions \emph{continues}.
Soit $t_0\in I$ et $X_0\in \K^n$.
Alors le probl\`eme de Cauchy
\[\mtag{C} \begin{cases}
    \forall t\in I \+ X'(t) = A(t) X(t) + B(t), \\
    X(t_0) = X_0
\end{cases}\]
admet une unique solution sur tout intervalle $J$ tel que $t_0\in J$ et $J\subset I$.

De plus, la solution sur $J$ n'est autre que la restriction \`a $J$
de la solution sur $I$.

\Para{Proposition}

Soit $t_0\in J$.
On consid\`ere le syst\`eme diff\'erentiel homog\`ene
\[\mtag{S_0} X'(t) = A(t)X(t)\]
Alors l'application
\[\Fonction{\Phi}{\solJ{S_0}}{\K^n}{X}{X(t_0)}\]
est un isomorphisme (c.-\`a-d. une application lin\'eaire bijective).

\Para{Th\'eor\`eme}[de structure]

On consid\`ere les syst\`emes diff\'erentiels lin\'eaires
\[\mtag{S}   X'(t) = A(t)X(t) + B(t)\]
\[\mtag{S_0} X'(t) = A(t)X(t)\]
Alors:
\begin{itemize}
\item $\solJ{S_0}$ est un sous-espace vectoriel de $\CC1(I,\K^n)$ de dimension $n$.
\item $\solJ{S}$ est un sous-espace \emph{affine} de $\CC1(I,\K^n)$ de dimension $n$ et de direction $\solJ{S_0}$.
\end{itemize}

\subsection{R\'esolution pratique}

\subsubsection{Principe de superposition}

\Para{Corollaire}

\'Etant donn\'e un syst\`eme diff\'erentiel lin\'eaire $\eq{S}$,
la solution g\'en\'erale de $\eq{S}$ est donn\'ee
par la somme de la solution g\'en\'erale du syst\`eme homog\`ene associ\'e $\eq{S_0}$
et d'une solution particuli\`ere de $\eq{S}$.

Plus pr\'ecis\'ement, soit $X_p$ une solution particuli\`ere de $\eq{S}$ sur $J$.
Alors toute solution $X$ de $\eq{S}$ sur $J$ est de la forme
\[X = X_0 + X_p,\]
o\`u $X_0$ est une solution de $\eq{S_0}$ sur $J$.

\emph{Pour r\'esoudre $\eq{S}$, il suffit donc de r\'esoudre $\eq{S_0}$ et de trouver une solution particuli\`ere de $\eq{S}$.}

\Para{Remarque}

Si $\eq{S}$ est de la forme
\[\mtag{S} X'(t) = A(t) X(t) + B_1(t) + B_2(t),\]
pour trouver une solution particuli\`ere de $\eq{S}$, il suffit de
\begin{itemize}
\item trouver une solution particuli\`ere $X_{p,1}$ de
  \[\mtag{S_1} X'(t) = A(t) X(t) + B_1(t),\]
\item trouver une solution particuli\`ere $X_{p,2}$ de
  \[\mtag{S_2} X'(t) = A(t) X(t) + B_2(t),\]
\item poser $X_p = X_{p,1} + X_{p,2}$.
\end{itemize}

Dans ces conditions, $X_p$ est une solution particuli\`ere de $\eq{S}$.

\subsubsection{R\'esolution d'un syst\`eme diff\'erentiel homog\`ene \`a coefficients constants}

\Para{Th\'eor\`eme}

Soit $A\in\MnK$ et $\Fn{X}{\R}{\K^n}$ une solution de
\[\mtag{S} X'(t) = A X(t).\]
On suppose que $A$ est diagonalisable:
soit $\nUplet X1n$ une base de vecteurs propres
associ\'es aux valeurs propres $\nUplet\lambda1n$.
Alors il existe $\nUplet a1n\in \K^n$ tels que
\[\forall t\in \R\+ X(t) = \sum_{k=1}^n a_k \me^{\lambda_k t} X_k.\]

\Para{Remarque}

Si on ne peut pas diagonaliser $A$, mais seulement la trigonaliser,
$A = P T P^{-1}$, on pose $X(t) = P Y(t)$, et le syst\`eme devient
\[\mtag{S'} Y'(t) = T Y(t),\]
ce qui donne un syst\`eme triangulaire que l'on peut r\'esoudre,
en r\'esolvant successivement les \'equations diff\'erentielles scalaires.

% -----------------------------------------------------------------------------
\section{\'Equations diff\'erentielles lin\'eaires scalaires du second ordre}

\subsection{G\'en\'eralit\'es}

\Para{D\'efinitions}

Une \emph{\'equation diff\'erentielle lin\'eaire scalaire du second ordre}
est une \'equation de la forme
\[\mtag{E} a(t) x''(t) + b(t) x'(t) + c(t) x(t) = d(t),\]
o\`u $a$, $b$, $c$ et $d$ sont des fonctions \emph{continues} de $I$ dans $\K$,
et $x$ une fonction inconnue de $I$ dans $\K$.

Une \emph{solution sur $J\subset I$} de cette \'equation est une fonction $f$ de $J$ dans $\K$,
deux fois d\'erivable en tout point de $J$, telle que pour tout $t\in J$, on ait
\[a(t) f''(t) + b(t) f'(t) + c(t) f(t) = d(t).\]
On note $\solJ{E}$ l'ensemble des solutions de $\eq{E}$ sur $J$.

\emph{R\'esoudre} $\eq{E}$, c'est d\'eterminer $\solJ{E}$ pour tout intervalle $J\subset I$.

\Para{D\'efinition}

Avec les m\^emes notations, dans le cas $\K=\R$, on consid\`ere l'\'equation diff\'erentielle
\[\mtag{E} a(t) x''(t) + b(t) x'(t) + c(t) x(t) = d(t).\]
Les courbes repr\'esentatrices des solutions de $\eq{E}$ s'appellent
\emph{courbes int\'egrales} de $\eq{E}$.

\Para{D\'efinitions}

On consid\`ere l'\'equation diff\'erentielle
\[\mtag{E} a(t) x''(t) + b(t) x'(t) + c(t) x(t) = d(t).\]
\begin{itemize}
\item $\eq{E}$ est dite \emph{\`a coefficients constants}
  si et seulement si les fonctions $a$, $b$ et $c$ sont constantes.
\item $\eq{E}$ est dite \emph{normalis\'ee} ou \emph{r\'esolue en $x''$}
  si et seulement si \[\forall t\in I\+ a(t) = 1.\]
\item $\eq{E}$ est dite \emph{homog\`ene} ou \emph{sans second membre}
  si et seulement si \[\forall t\in I\+ d(t) = 0.\]
\item On appelle \emph{\'equation diff\'erentielle homog\`ene associ\'ee} \`a $\eq{E}$
  ou \emph{\'equation diff\'erentielle sans second membre associ\'ee} \`a $\eq{E}$
  l'\'equation diff\'erentielle
  \[\mtag{E_0} a(t) x''(t) + b(t) x'(t) + c(t) x(t) = 0.\]
\end{itemize}

\Para{D\'efinition}

Un \emph{probl\`eme de Cauchy} du second ordre
est la donn\'ee d'une \'equation diff\'erentielle du second ordre
normalis\'ee et de deux conditions initiales.

Un probl\`eme de Cauchy lin\'eaire du second ordre est donc de la forme
\[\mtag{C}
  \begin{cases}
    \forall t\in I\+ x''(t) + b(t)x'(t) + c(t) x(t) = d(t), \\
    x(t_0) =\alpha, \\
    x'(t_0) =\beta\text{ o\`u } (t_0,\alpha,\beta)\in I\times \K^2.
\end{cases}\]

\subsection{\'Etude th\'eorique}

\Para{Remarque}

\'Etant donn\'e l'\'equation diff\'erentielle lin\'eaire scalaire du premier ordre
\[\mtag{E} a(t) x''(t) + b(t) x'(t) + c(t) x(t) = d(t),\]
on peut se ramener \`a une \'equation normalis\'ee:
\begin{enumerate}
\item si $a$ ne s'annule pas sur $I$, il suffit de diviser par $a(t)$;
\item par contre, si $a$ s'annule, on d\'ecoupe $I$ en intervalles
  o\`u $a$ ne s'annule pas, puis on fait une \'etude sur chaque intervalle.
  \`A la fin, on peut chercher \`a \og raccorder\fg{} les solutions.
\end{enumerate}

\Para{Remarque}

On consid\`ere l'\'equation diff\'erentielle suivante:
\[\mtag{E} x''(t) + b(t) x'(t) + c(t) x(t) = d(t).\]
On peut se ramener \`a un syst\`eme diff\'erentiel
\[\mtag{S} X'(t) = A(t) X(t) + B(t)\]
en posant
$X(t) = \begin{pmatrix} x(t) \\ x'(t) \end{pmatrix}$,
$A(t) = \begin{pmatrix} 0 & 1 \\ -c(t) & -b(t) \end{pmatrix}$,
$B(t) = \begin{pmatrix} 0 \\ d(t) \end{pmatrix}$

\Para{Proposition}

Soit $f$ une solution sur $J$ de l'\'equation diff\'erentielle normalis\'ee
\[\mtag{E} x''(t) + b(t) x'(t) + c(t) x(t) = d(t).\]
Alors $f$ est de classe $\CC2$ sur $J$.

De plus, si $b$, $c$ et $d$ sont de classe $\CC p$ sur $J$,
alors $f$ est de classe $\CC{p+2}$ sur $J$;
si $b$, $c$ et $d$ sont de classe $\CC\infty$ sur $J$, alors $f$ l'est \'egalement.

\Para{Remarque}

Toute \'equation diff\'erentielle lin\'eaire scalaire du second ordre sous forme r\'esolue
\[\mtag{E} x''(t) + b(t) x'(t) + c(t) x(t) = d(t)\]
peut s'\'ecrire sous la forme d'un syst\`eme diff\'erentiel
\[\mtag{S} X'(t) = A(t) X(t) + B(t)\] o\`u
$X(t) = \begin{pmatrix} x(t) \\ x'(t) \end{pmatrix}$,
$A(t) = \begin{pmatrix} 0 & 1 \\ -c(t) & -b(t) \end{pmatrix}$, \\
$B(t) = \begin{pmatrix} 0 \\ d(t) \end{pmatrix}$.

\Para{Th\'eor\`eme}[Cauchy-Lipschitz lin\'eaire]

Soit $\Fn bI\K$, $\Fn cI\K$ et $\Fn dI\K$
trois fonctions \emph{continues}.
Soit $t_0\in I$ et $(\alpha,\beta)\in \K^2$.

Alors le probl\`eme de Cauchy
\[\mtag{C}
  \begin{cases}
    \forall t\in I\+ x''(t) + b(t) x'(t) + c(t) x(t) = d(t), \\
    x(t_0) =\alpha, \\
    x'(t_0) =\beta{}
\end{cases}\]
admet une unique solution sur tout intervalle $J$
tel que $t_0\in J$ et $J\subset I$.

De plus, la solution sur $J$ n'est autre que la restriction \`a $J$
de la solution sur $I$.

\Para{Remarque}

Les courbes int\'egrales de $\eq{E}$ ne sont pas disjointes.

\Para{Th\'eor\`eme}[de structure]

On consid\`ere les \'equations diff\'erentielles lin\'eaires scalaires normalis\'ees du second ordre
\[\mtag{E} x''(t) + b(t)x'(t) + c(t) x(t) = d(t)\]
\[\mtag{E_0} x''(t) + b(t)x'(t) + c(t) x(t) = 0\]
\begin{itemize}
\item $\solJ{E_0}$ est un sous-espace vectoriel de $\CC2(I,\K)$ de dimension 2.
\item $\solJ{E}$ est un sous-espace \emph{affine} de $\CC2(I,\K)$ de dimension 2 et de direction $\solJ{E_0}$.
\end{itemize}

\Para{D\'efinition}

On appelle \emph{syst\`eme fondamental} des solutions de l'\'equation homog\`ene
\[\mtag{E_0} x''(t) + b(t)x'(t) + c(t) x(t) = 0\]
toute base $(\varphi,\psi)$ de $\solI{E_0}$.

\Para{Proposition}

Soit $(\varphi,\psi)$ un syst\`eme fondamental de $\eq{E_0}$.
Alors
\[\begin{split} \solJ{E_0} &= \Bigl\{ \Fn fJ\K\;\Bigm|\; \exists(A,B)\in \K\, \\
& \forall t\in J\+, f(t) = A\varphi(t) + B\psi(t) \Bigr\}. \end{split}\]

Autrement dit, la solution g\'en\'erale de $\eq{E_0}$ est
\[x_0(t) = A\varphi(t) + B\psi(t)\]
o\`u $(A,B)\in \K^2$.

\Para{D\'efinition}

Avec les m\^emes notations,
soit $\varphi$ et $\psi$ deux solutions de $\eq{E_0}$.
On appelle \emph{Wronskien} de $\varphi$ et $\psi$ la quantit\'e
\[W(t) = \begin{vmatrix} \varphi(t) & \psi(t) \\ \varphi'(t) & \psi'(t) \end{vmatrix} =\varphi(t)\psi'(t) -\varphi'(t)\psi(t).\]

\Para{Remarque}

$W$ est solution de l'\'equation diff\'erentielle
\[W'(t) + b(t)W(t) = 0.\]

\Para{Proposition}

Les conditions suivantes sont \'equivalentes:
\begin{enumerate}
\item $(\varphi,\psi)$ est un syst\`eme fondamental pour $\eq{E_0}$,
\item $(\varphi,\psi)$ est une famille libre,
\item $W$ ne s'annule pas sur $I$,
\item $W$ n'est pas identiquement nul sur $I$.
\end{enumerate}

\subsection{R\'esolution pratique}

\subsubsection{Principe de superposition}

\Para{Proposition}

\'Etant donn\'e une \'equation lin\'eaire $\eq{E}$,
la solution g\'en\'erale de $\eq{E}$ est donn\'ee
par la somme de la solution g\'en\'erale de l'\'equation homog\`ene $\eq{E_0}$
et d'une solution particuli\`ere de l'\'equation compl\`ete $\eq{E}$.

Plus pr\'ecis\'ement, soit $x_p$ une solution particuli\`ere de $\eq{E}$ sur $J$.
Alors toute solution $x$ de $\eq{E}$ sur $J$ est de la forme
\[x = x_0 + x_p,\]
o\`u $x_0$ est une solution de $\eq{E}$ sur $J$.

Pour r\'esoudre $\eq{E}$,
il suffit donc de r\'esoudre $\eq{E_0}$
et de trouver une solution particuli\`ere de $\eq{E}$.

\Para{Remarque}

Si $\eq{E}$ est de la forme
\[\mtag{E} a(t) x''(t) + b(t)x'(t) + c(t)x(t) = d_1(t) + d_2(t),\]
pour trouver une solution particuli\`ere de $\eq{E}$,
il suffit de
\begin{itemize}
\item trouver une solution particuli\`ere $x_{p,1}$ de
  \[\mtag{E_1} a(t) x''(t) + b(t) x'(t) + c(t) x(t) = d_1(t),\]
\item trouver une solution particuli\`ere $x_{p,2}$ de
  \[\mtag{E_2} a(t) x''(t) + b(t) x'(t) + c(t) x(t) = d_2(t),\]
\item poser $x_p = x_{p,1} + x_{p,2}$.
\end{itemize}

Dans ces conditions, $x_p$ est une solution particuli\`ere de $\eq{E}$.

\subsubsection{R\'esolution de l'\'equation homog\`ene}

\Para{Proposition}[cas des coefficients constants]

Soit l'\'equation diff\'erentielle
\[\mtag{E} ax''(t) + bx'(t) + cx(t) = 0,\]
o\`u $(a,b,c)\in \K^3$, $a\neq0$.

On forme l'\emph{\'equation caract\'eristique} de discriminant $\Delta= b^2-4ac$
\[\mtag{E_c} ar^2 + br + c = 0.\]

Alors, dans le cas $\K=\R$,
\begin{itemize}
\item Si $\Delta> 0$, la solution g\'en\'erale est donn\'ee par
  \[x(t) = A \me^{r_1t} + B \me^{r_2t},\]
  o\`u $(A,B)\in \R^2$ et $r_1$ et $r_2$ sont les racines distinctes de $\eq{E_c}$.
\item Si $\Delta= 0$, la solution g\'en\'erale est donn\'ee par
  \[x(t) = (A t+B) \me^{r_0t},\]
  o\`u $(A,B)\in \R^2$ et $r_0$ est la racine double de $\eq{E_c}$.
\item Si $\Delta< 0$, la solution g\'en\'erale est donn\'ee par
  \[x(t) = \me^{\alpha t} \BigPa{ A \cos(\omega t) + B \sin(\omega t) },\]
  o\`u $(A,B)\in \R^2$ et $\alpha\pm i\omega$ sont les
  racines complexes conjugu\'ees de $\eq{E_c}$.
\end{itemize}

Alors, dans le cas $\K=\C$,
\begin{itemize}
\item Si $\Delta\neq0$, la solution g\'en\'erale est donn\'ee par
  \[x(t) = A \me^{r_1 t} + B \me^{r_2 t},\]
  o\`u $(A,B)\in \C^2$ et $r_1$ et $r_2$ sont les racines distinctes de $\eq{E_c}$.
\item Si $\Delta= 0$, la solution g\'en\'erale est donn\'ee par
  \[x(t) = (A t+B) \me^{r_0 t},\]
  o\`u $(A,B)\in \C^2$ et $r_0$ est la racine double de $\eq{E_c}$.
\end{itemize}

\Para{Remarque}[cas g\'en\'eral]

\emph{Il n'existe pas de m\'ethode g\'en\'erale!}
\begin{itemize}
\item Dans le cas d'une \'equation \`a coefficients constants, on a des formules.
\item On peut tenter de chercher une solution sous la forme
  d'un mon\^ome, d'un polyn\^ome, d'une s\'erie enti\`ere, etc...
\item L'\'enonc\'e peut sugg\'erer un changement de variable.
\item Si on a trouv\'e une solution $f$ qui ne s'annule pas,
  on peut chercher les autres solutions sous la forme $x(t) = f(t) y(t)$
  o\`u $y$ est la nouvelle fonction inconnue.
\end{itemize}

et c'est \`a peu pr\`es tout.

\subsubsection{Recherche d'une solution particuli\`ere}

\Para{Proposition}[cas d'une \'equation \`a coefficients constants]

On consid\`ere l'\'equation diff\'erentielle
\[\mtag{E} ax''(t) + bx'(t) + cx(t) = P(t) \me^{\alpha t},\]
o\`u $a\neq0$ et $P\in \K_d[X]$.
On note $\eq{E_c}$ l'\'equation caract\'eristique.

Il existe une unique solution particuli\`ere de la forme
\begin{enumerate}
\item Si $\alpha$ n'est pas racine de $\eq{E_c}$,
  on prend $x_p(t) = Q(t) \me^{\alpha t}$ o\`u $Q\in \K_d[X]$ \`a d\'eterminer.
\item Si $\alpha$ est racine simple de $\eq{E_c}$,
  on prend $x_p(t) = tQ(t) \me^{\alpha t}$ o\`u $Q\in \K_d[X]$ \`a d\'eterminer.
\item Si $\alpha$ est racine double de $\eq{E_c}$,
  on prend $x_p(t) = t^2Q(t) \me^{\alpha t}$ o\`u $Q\in \K_d[X]$ \`a d\'eterminer.
\end{enumerate}

Bien s\^ur, si le second membre est une somme de termes de ce type,
il suffit de superposer les solutions particuli\`eres correspondantes.

\Para{Th\'eor\`eme}[Hors-programme, m\'ethode de variation des constantes]

On consid\`ere l'\'equation diff\'erentielle
\[\mtag{E} x''(t) + b(t)x'(t) + c(t)x(t) = d(t)\]
et l'on note $\eq{E_0}$ l'\'equation homog\`ene associ\'ee.
On suppose que $(\varphi,\psi)$ est un syst\`eme fondamental de solutions de $\eq{E_0}$.

Alors la solution g\'en\'erale de $\eq{E}$ est donn\'ee par
\[x(t) =\lambda(t)\varphi(t) +\mu(t)\psi(t),\]
o\`u $\lambda$ et $\mu$ sont solutions du syst\`eme de Cramer
\[\mtag{S} \left\{ \begin{aligned}
      \lambda'(t)\varphi(t)  + \mu'(t)\psi(t)  &= 0 \\
      \lambda'(t)\varphi'(t) + \mu'(t)\psi'(t) &= d(t).
\end{aligned} \right.\]

% % -----------------------------------------------------------------------------
% \section{Notions sur les \'equations diff\'erentielles non lin\'eaires (hors-programme)}
%
% \Para{Contexte}
%
% Les \'equations diff\'erentielles non lin\'eaires sont bien plus g\'en\'erales
% et bien plus complexes que les \'equations diff\'erentielles lin\'eaires.
%
% Par exemple, les \'equations diff\'erentielles scalaires du premier ordre
% sous forme r\'esolue sont les \'equations de la forme
% \[x'(t) = f(t, x(t)),\]
% o\`u $f$ est une fonction de deux variables, d\'efinie sur un ouvert $\Omega$
% de $\R^2$.
%
% La th\'eorie g\'en\'erale n'est pas au programme, nous traiterons quelques cas particuliers.
%
% \subsection{\'Equations \`a variables s\'eparables}
%
% \Para{Exemple}
%
% On consid\`ere l'\'equation diff\'erentielle
% \[\mtag{E} (1+t^2)x'(t) - 1 - x^2(t) = 0.\]
% \`A cause du terme en $x^2$, il ne s'agit pas d'une \'equation lin\'eaire.
% Celle-ci n'est pas trop m\'echante, car on peut \og s\'eparer les variables\fg:
% \[\frac{x'}{1+x^2} = \frac{1}{1+t^2},\]
% ce qui s'int\`egre en
% \[\mtag{E'} \arctan x(t) = \arctan t + C.\]
% Avec un peu de travail, on obtient les solutions suivantes:
% \begin{itemize}
% \item $t \mapsto \frac{kt+1}{k-t}$
%   sur $\intO{-\infty,k}$ et sur $\intO{k,+\infty}$ o\`u $k\in \R$
% \item $t \mapsto t$ sur $\R$
% \end{itemize}
%
% \Para{Remarque}
%
% Au lieu de l'\'equation
% \[\frac{x'}{1+x^2} = \frac{1}{1+t^2},\]
% on \'ecrit parfois, de fa\c con un peu abusive,
% \[\frac{\D x}{1+x^2} = \frac{\D t}{1+t^2},\]
% ce qui rend encore plus visible encore l'aspect
% \og \`a variables s\'epar\'ees\fg.
% On r\'esout en
% \[\int\frac{\D x}{1+x^2} =\int\frac{\D t}{1+t^2}.\]
%
% \Para{D\'efinition}
%
% Une \'equation diff\'erentielle scalaire est dite
% \og \`a variables s\'eparables\fg{} si elle est de la forme
% \[a(x(t))x'(t) = b(t),\]
% o\`u $a$ et $b$ sont deux fonctions num\'eriques continues.
%
% \Para{Proposition}
%
% On consid\`ere l'\'equation diff\'erentielle
% \[\mtag{E} a(x(t))x'(t) = b(t),\]
% o\`u $a$ et $b$ sont continues.
% Notons $A$ (resp. $B$) une primtive de $a$ (resp. $b$).
%
% L'\'equation $\eq{E}$ s'int\`egre en
% \[A(x(t)) = B(t) + C,\]
% o\`u $C$ est une constante.
%
% On ne peut pas toujours exprimer $x(t)$ en fonction de $t$,
% mais on peut g\'en\'eralement tracer les courbes int\'egrales.
%
% \Para{Exemple}
%
% D\'eterminer les courbes int\'egrales de l'\'equation
% \[xx' + t = 0.\]
%
% \subsection{Syst\`emes autonomes}
%
% \Para{D\'efinition}
%
% Un \emph{syst\`eme autonome} de deux \'equations diff\'erentielles du premier ordre
% est un syst\`eme diff\'erentiel de la forme
% \[\mtag{S} \left\{ \begin{aligned}
%       \frac{\D x}{\D t} &=\varphi(x,y) \\
%       \frac{\D y}{\D t} &=\psi(x,y)
% \end{aligned} \right.\]
% Ce syst\`eme est qualifi\'e d'\emph{autonome}
% car $\varphi$ et $\psi$ ne d\'ependent pas de $t$.
%
% \Para{Proposition}
%
% Avec les m\^emes notations, on suppose de plus que $\varphi$ et $\psi$ sont
% de classe $\CC1$ sur un ouvert $\Omega$ de $\R^2$ et que $\varphi$ ne s'annule pas sur $\Omega$.
% Alors on peut exprimer $y$ en fonction de $x$ en remarquant que
% \[\frac{\D y}{\D x} = \frac{\psi(x,y)}{\varphi(x,y)}.\]
% La courbe reliant $x$ et $y$ s'appelle \emph{courbe int\'egrale}
% du syst\`eme diff\'erentiel~$\eq{S}$.
%
% On dit parfois aussi qu'il s'agit de la courbe int\'egrale
% du champ de vecteur
% \[(x,y) \mapsto \bigPa{\varphi(x,y),\psi(x,y)}.\]
%
% \Para{Exemple}[pendule simple]
%
% L'\'equation du pendule est donn\'ee par (apr\`es normalisation)
% \[\frac{\mathrm{d}^2 x}{\mathrm{d}t^2} + \sin x = 0.\]
% En notant $v = \frac{\D x}{\D t}$, on obtient le syst\`eme autonome en $(x,v)$
% \[\left\{ \begin{aligned}
%       \frac{\D x}{\D t} &= v \\
%       \frac{\D v}{\D t} &= -\sin x
% \end{aligned} \right.\]
% On en d\'eduit
% \[\frac{\D v}{\D x} = -\frac{\sin x}{v},\]
% ce qui est une \'equation diff\'erentielle non lin\'eaire.
%
% Pour la r\'esoudre, il faut une petite astuce
% \[v\frac{\D v}{\D x} = -\sin x\]
% \[\frac12 v^2 = \cos x + C^\mathrm{ste}\]
%
% Au passage, nous retrouvons l'\'energie du syst\`eme
% \[E = \frac12 v^2 - \cos x = C.\]
%
% Avec cela, nous pouvons tracer les courbes int\'egrales.
% \[v = \pm\sqrt{2\cos x + C}.\]

% -----------------------------------------------------------------------------
\section{Exercices}

% -----------------------------------------------------------------------------
\par\pagebreak[1]\par
\paragraph{Exercice 1}%
\hfill{\tiny 9170}%
\begingroup~

R\'esoudre les \'equations diff\'erentielles suivantes:
\begin{enumerate}
\item $(1+x^2)y'+4xy=0$
\item $x^n y' -\alpha y = 0$ o\`u $(n,\alpha)\in\Ns\times\Rps$.
\item $y'+2xy-e^{x-x^2} = 0$
\item $x(x^2+1)y' + y + x = 0$
\item $xy'-y + \ln x = 0$
\item $x^3\ln\Abs{x}y' - x^2y - (2\ln\Abs{x}+1)=0$
\item $(\sin x)y'-y+1=0$
\item $(\sh x \ch^3 x)y' + 3(\ch^4 x)y - 1 = 0$
\end{enumerate}
\endgroup

% -----------------------------------------------------------------------------
\par\pagebreak[1]\par
\paragraph{Exercice 2}%
\hfill{\tiny 1732}%
\begingroup~

R\'esoudre les \'equations diff\'erentielles suivantes:
\begin{enumerate}
\item $y''-2y'+y = 2\sh x$
\item $2y''+2y'+y = xe^{-x}$
\item $y''-4y = 4e^{-2x}$
\item $y''-3y'+2y = (x^2+1)e^x$
\item $y''+2y'+y = \sin^2 x$
\item $y''+2y'+2y = \ch x \cos x$
\item $y''+y = \Abs{x}+1$
\item $y''-2y'+y = e^{\Abs x}$
\end{enumerate}
\endgroup

% -----------------------------------------------------------------------------
\par\pagebreak[1]\par
\paragraph{Exercice 3}%
\hfill{\tiny 4353}%
\begingroup~

R\'esoudre les \'equations diff\'erentielles suivantes:
\begin{enumerate}
\item $(2t+1)x'' + (4t-2)x' - 8x = 0$
\item $x'' + (4e^t-1)x' + 4e^{2t}x = 0$
\item $(t^2+1)^2x'' + 2t(t^2+1)x' + x = (t^2+1)^2$
\item $\begin{vmatrix} x'' & x' & x  \\  -\sin t & \cos t & \sin t  \\  -4\sin 2t &  2 \cos 2t &  \sin 2t \end{vmatrix} = 0$
\item $y''+y = \Abs{x^2-\pi^2}$ avec $y(0)=y'(0) = 0$
\end{enumerate}
\endgroup

% -----------------------------------------------------------------------------
\par\pagebreak[1]\par
\paragraph{Exercice 4}%
\hfill{\tiny 1953}%
\begingroup~

\begin{enumerate}
\item Trouver toutes les applications $\Fn f\R \R$ d\'erivables telles que:
  \[\forall x\in \R\+ f'(x) f(-x) = 1\]
\item Trouver toutes les applications $\Fn f\R \R$ d\'erivables telles que:
  \[\forall x\in \R\+ f'(x) + f(-x) = e^x\]
\item Trouver toutes les applications $\Fn f\R \R$ d\'erivables en $0$ telles que:
  \[\forall x\in \R\+\forall y\in \R\+ f(x+y) = e^x f(y) + e^y f(x)\]
\end{enumerate}
\endgroup

% -----------------------------------------------------------------------------
\par\pagebreak[1]\par
\paragraph{Exercice 5}%
\hfill{\tiny 7355}%
\begingroup~

Soit $f\in\mathcal{C}(\R,\R)$ ayant une limite finie en $+\infty$.
Montrer que toute solution de l'\'equation diff\'erentielle $y'+y = f$
admet une limite finie en $+\infty$.
\endgroup

% -----------------------------------------------------------------------------
\par\pagebreak[1]\par
\paragraph{Exercice 6}%
\hfill{\tiny 2770}%
\begingroup~

Soit l'\'equation diff\'erentielle \[x^2 y'' + 4xy' + (2-x^2)y = 1.\]
R\'esoudre cette \'equation diff\'erentielle sur $\R_+^*$ et sur $\R_-^*$
en posant $u = x^2 y$, puis \'etudier le raccordement en~$0$.
\endgroup

% -----------------------------------------------------------------------------
\par\pagebreak[1]\par
\paragraph{Exercice 7}%
\hfill{\tiny 0701}%
\begingroup~

Calculer explicitement les fonctions suivantes:
\[\begin{aligned}
    u(x) &= \int_0^{+\infty} \frac{e^{-t} \cos(xt)}{\sqrt t} \D t \\
    v(x) &= \int_0^{+\infty} \frac{e^{-t} \sin(xt)}{\sqrt t} \D t
\end{aligned}\]
\endgroup

% -----------------------------------------------------------------------------
\par\pagebreak[1]\par
\paragraph{Exercice 8}%
\hfill{\tiny 0227}%
\begingroup~

R\'esoudre le syst\`eme diff\'erentiel:
\[\left\{
    \begin{aligned}
      x' &= 5x  -   y  -  2z  +  e^t    \\
      y' &=  x  +  3y  -  2z  +  e^{2t} \\
      z' &= -x  -   y  +  4z  +  t e^t
    \end{aligned}
\right.\]
\endgroup

% -----------------------------------------------------------------------------
\par\pagebreak[1]\par
\paragraph{Exercice 9}%
\hfill{\tiny 3582}%
\begingroup~

Soit $\Fn f\R \R$ de classe $\CC2$ telle que
\[\forall x\in \R\+ f''(x) + f(x)\geq0.\]
Montrer que $\forall x\in \R\+ f(x) + f(x+\pi)\geq0$.
On pourra introduire
\[W(x) = \begin{vmatrix} f(x) & \sin(x) \\ f'(x) & \cos(x) \end{vmatrix}.\]
\endgroup

% -----------------------------------------------------------------------------
\par\pagebreak[1]\par
\paragraph{Exercice 10}%
\hfill{\tiny 9094}%
\begingroup~

Soit $\Fn A\R \R$ continue et born\'ee et $k > 0$.
Montrer que l'\'equation diff\'erentielle
\[y' - ky = A\]
admet une unique solution born\'ee.
\endgroup

% -----------------------------------------------------------------------------
\par\pagebreak[1]\par
\paragraph{\href{https://psi.miomio.fr/exo/1626.pdf}{Exercice 11}}%
\hfill{\tiny 1626}%
\begingroup~

Soit $\Fn f\Rp\R$ de classe $\CC1$ telle que
\[ \lim_{x \to +\infty} \bigPa{ f(x) + f'(x) } = 0.\]

Que dire de $\DS \lim_{x \to +\infty} f(x)$?
\endgroup

% -----------------------------------------------------------------------------
\par\pagebreak[1]\par
\paragraph{Exercice 12}%
\hfill{\tiny 3512}%
\begingroup~

R\'esoudre l'\'equation \[(-x^2+4x)f'(x) + (x+2)f(x) = x.\]
On pourra commencer par rechercher les solutions
d\'eveloppables en s\'erie enti\`ere au voisinage de 0.
\endgroup

% -----------------------------------------------------------------------------
\par\pagebreak[1]\par
\paragraph{Exercice 13}%
\hfill{\tiny 9713}%
\begingroup~

Soit $\Fn{f}{[a,b]}{\R}$ non nulle de classe $\CC2$ et solution de l'\'equation diff\'erentielle
$y'' + py' + qy = 0$, o\`u $p$ et $q$ sont continues de $[a,b]$ dans $\R$.
Montrer que $f$ admet un nombre fini de z\'eros.
\endgroup

% -----------------------------------------------------------------------------
\par\pagebreak[1]\par
\paragraph{Exercice 14}%
\hfill{\tiny 2414}%
\begingroup~

Soit $I$ un intervalle de $\R$, $r$ et $s$ deux fonctions continues $I \to\R$ telles que $r\leq s$.
Soit $x$ une solution non nulle de l'\'equation diff\'erentielle $x'' + rx = 0$,
et $y$ une solution non nulle de l'\'equation diff\'erentielle $y'' + sy = 0$.
Soit $t_1$ et $t_2$ deux z\'eros cons\'ecutifs de $x$.
Montrer que $y$ s'annule sur $\intO{t_1,t_2}$ sauf si $x$ et $y$ sont proportionnelles.
\endgroup

% -----------------------------------------------------------------------------
\par\pagebreak[1]\par
\paragraph{Exercice 15 (applications de l'exercice pr\'ec\'edent)}%
\hfill{\tiny 1228}%
\begingroup~

Soit $\Fn rI\R$ une fonction continue,
et $x$ une solution non nulle de l'\'equation diff\'erentielle
\[x'' + rx = 0.\]
\begin{enumerate}
\item On suppose $\forall t\in I, r(t)\leq0$.
  Montrer que $x$ s'annule au plus une fois dans $I$.
\item Soit $\mu> 0$. On suppose $\forall t\in I, r(t)\leq \mu^2$.
  Montrer que $t_2\geq t_1 + \frac{\pi}{\mu}$.
\item Soit $\lambda> 0$. On suppose $\forall t\in I, r(t)\geq \lambda^2$.
  Soit $t_1\in I$ telle que $t_1 + \frac{\pi}{\lambda}\in I$.

  Montrer que toute solution de l'\'equation diff\'erentielle $x'' + rx = 0$
  s'annule au moins une fois sur $\intO{t_1,t_1+\frac{\pi}{\lambda}}$.
\end{enumerate}
\endgroup

% -----------------------------------------------------------------------------
\par\pagebreak[1]\par
\paragraph{Exercice 16}%
\hfill{\tiny 8973}%
\begingroup~

Que peut-on dire des z\'eros des solutions sur $\R_+^*$ de l'\'equation diff\'erentielle suivante?
\[x'' + \frac{x'}{t} + \left( 1 - \frac{\alpha^2}{t^2} \right) x = 0\]
\endgroup

% -----------------------------------------------------------------------------
\par\pagebreak[1]\par
\paragraph{\href{https://psi.miomio.fr/exo/8698.pdf}{Exercice 17} (lemme de Gronwall)}%
\hfill{\tiny 8698}%
\begingroup~

Soit $\Fn{a}{\Rp}{\R}$ une fonction continue et
$\Fn{f}{\Rp}{\R}$ de classe $\CC1$ telle que
\[ \forall x\geq0 \+ f'(x)\leq a(x) f(x). \]

Montrer que
\[ \forall x\geq0 \+ f(x)\leq f(0) \exp\Pa{ \int_0^x a(t) \D t }. \]
\endgroup

\end{document}
