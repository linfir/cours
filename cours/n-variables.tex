% autogenerated by ytex.rs

\documentclass{scrartcl}

\usepackage[francais]{babel}
\usepackage{geometry}
\usepackage{scrpage2}
\usepackage{lastpage}
\usepackage{multicol}
\usepackage{etoolbox}
\usepackage{xparse}
\usepackage{enumitem}
% \usepackage{csquotes}
\usepackage{amsmath}
\usepackage{amsfonts}
\usepackage{amssymb}
\usepackage{mathrsfs}
\usepackage{stmaryrd}
\usepackage{dsfont}
\usepackage{eurosym}
% \usepackage{numprint}
% \usepackage[most]{tcolorbox}
% \usepackage{tikz}
% \usepackage{tkz-tab}
\usepackage[unicode]{hyperref}
\usepackage[ocgcolorlinks]{ocgx2}

\let\ifTwoColumns\iftrue
\def\Classe{$\Psi$2019--2020}

% reproducible builds
% LuaTeX: \pdfvariable suppressoptionalinfo 1023 \relax
\pdfinfoomitdate=1
\pdftrailerid{}

\newif\ifDisplaystyle
\everymath\expandafter{\the\everymath\ifDisplaystyle\displaystyle\fi}
\newcommand\DS{\displaystyle}

\clearscrheadfoot
\pagestyle{scrheadings}
\thispagestyle{empty}
\ohead{\Classe}
\ihead{\thepage/\pageref*{LastPage}}

\setlist[itemize,1]{label=\textbullet}
\setlist[itemize,2]{label=\textbullet}

\ifTwoColumns
  \geometry{margin=1cm,top=2cm,bottom=3cm,foot=1cm}
  \setlist[enumerate]{leftmargin=*}
  \setlist[itemize]{leftmargin=*}
\else
  \geometry{margin=3cm}
\fi

\makeatletter
\let\@author=\relax
\let\@date=\relax
\renewcommand\maketitle{%
    \begin{center}%
        {\sffamily\huge\bfseries\@title}%
        \ifx\@author\relax\else\par\medskip{\itshape\Large\@author}\fi
        \ifx\@date\relax\else\par\bigskip{\large\@date}\fi
    \end{center}\bigskip
    \ifTwoColumns
        \par\begin{multicols*}{2}%
        \AtEndDocument{\end{multicols*}}%
        \setlength{\columnsep}{5mm}
    \fi
}
\makeatother

\newcounter{ParaNum}
\NewDocumentCommand\Para{smo}{%
  \IfBooleanF{#1}{\refstepcounter{ParaNum}}%
  \paragraph{\IfBooleanF{#1}{{\tiny\arabic{ParaNum}~}}#2\IfNoValueF{#3}{ (#3)}}}

\newcommand\I{i}
\newcommand\mi{i}
\def\me{e}

\def\do#1{\expandafter\undef\csname #1\endcsname}
\docsvlist{Ker,sec,csc,cot,sinh,cosh,tanh,coth,th}
\undef\do

\DeclareMathOperator\ch{ch}
\DeclareMathOperator\sh{sh}
\DeclareMathOperator\th{th}
\DeclareMathOperator\coth{coth}
\DeclareMathOperator\cotan{cotan}
\DeclareMathOperator\argch{argch}
\DeclareMathOperator\argsh{argsh}
\DeclareMathOperator\argth{argth}

\let\epsilon=\varepsilon
\let\phi=\varphi
\let\leq=\leqslant
\let\geq=\geqslant
\let\subsetneq=\varsubsetneq
\let\emptyset=\varnothing

\newcommand{\+}{,\;}

\undef\C
\newcommand\ninf{{n\infty}}
\newcommand\N{\mathbb{N}}
\newcommand\Z{\mathbb{Z}}
\newcommand\Q{\mathbb{Q}}
\newcommand\R{\mathbb{R}}
\newcommand\C{\mathbb{C}}
\newcommand\K{\mathbb{K}}
\newcommand\Ns{\N^*}
\newcommand\Zs{\Z^*}
\newcommand\Qs{\Q^*}
\newcommand\Rs{\R^*}
\newcommand\Cs{\C^*}
\newcommand\Ks{\K^*}
\newcommand\Rp{\R^+}
\newcommand\Rps{\R^+_*}
\newcommand\Rms{\R^-_*}
\newcommand{\Rpinf}{\Rp\cup\Acco{+\infty}}

\undef\B
\newcommand\B{\mathscr{B}}

\undef\P
\DeclareMathOperator\P{\mathbb{P}}
\DeclareMathOperator\E{\mathbb{E}}
\DeclareMathOperator\Var{\mathbb{V}}

\DeclareMathOperator*\PetitO{o}
\DeclareMathOperator*\GrandO{O}
\DeclareMathOperator*\Sim{\sim}
\DeclareMathOperator\Tr{tr}
\DeclareMathOperator\Ima{Im}
\DeclareMathOperator\Ker{Ker}
\DeclareMathOperator\Sp{Sp}
\DeclareMathOperator\Diag{diag}
\DeclareMathOperator\Rang{rang}
\DeclareMathOperator*\Coords{Coords}
\DeclareMathOperator*\Mat{Mat}
\DeclareMathOperator\Pass{Pass}
\DeclareMathOperator\Com{Com}
\DeclareMathOperator\Card{Card}
\DeclareMathOperator\Racines{Racines}
\DeclareMathOperator\Vect{Vect}
\DeclareMathOperator\Id{Id}

\newcommand\DerPart[2]{\frac{\partial #1}{\partial #2}}

\def\T#1{{#1}^T}

\def\pa#1{({#1})}
\def\Pa#1{\left({#1}\right)}
\def\bigPa#1{\bigl({#1}\bigr)}
\def\BigPa#1{\Bigl({#1}\Bigr)}
\def\biggPa#1{\biggl({#1}\biggr)}
\def\BiggPa#1{\Biggl({#1}\Biggr)}

\def\pafrac#1#2{\pa{\frac{#1}{#2}}}
\def\Pafrac#1#2{\Pa{\frac{#1}{#2}}}
\def\bigPafrac#1#2{\bigPa{\frac{#1}{#2}}}
\def\BigPafrac#1#2{\BigPa{\frac{#1}{#2}}}
\def\biggPafrac#1#2{\biggPa{\frac{#1}{#2}}}
\def\BiggPafrac#1#2{\BiggPa{\frac{#1}{#2}}}

\def\cro#1{[{#1}]}
\def\Cro#1{\left[{#1}\right]}
\def\bigCro#1{\bigl[{#1}\bigr]}
\def\BigCro#1{\Bigl[{#1}\Bigr]}
\def\biggCro#1{\biggl[{#1}\biggr]}
\def\BiggCro#1{\Biggl[{#1}\Biggr]}

\def\abs#1{\mathopen|{#1}\mathclose|}
\def\Abs#1{\left|{#1}\right|}
\def\bigAbs#1{\bigl|{#1}\bigr|}
\def\BigAbs#1{\Bigl|{#1}\Bigr|}
\def\biggAbs#1{\biggl|{#1}\biggr|}
\def\BiggAbs#1{\Biggl|{#1}\Biggr|}

\def\acco#1{\{{#1}\}}
\def\Acco#1{\left\{{#1}\right\}}
\def\bigAcco#1{\bigl\{{#1}\bigr\}}
\def\BigAcco#1{\Bigl\{{#1}\Bigr\}}
\def\biggAcco#1{\biggl\{{#1}\biggr\}}
\def\BiggAcco#1{\Biggl\{{#1}\Biggr\}}

\def\ccro#1{\llbracket{#1}\rrbracket}
\def\Dcro#1{\llbracket{#1}\rrbracket}

\def\floor#1{\lfloor#1\rfloor}
\def\Floor#1{\left\lfloor{#1}\right\rfloor}

\def\sEnsemble#1#2{\mathopen\{#1\mid#2\mathclose\}}
\def\bigEnsemble#1#2{\bigl\{#1\bigm|#2\bigr\}}
\def\BigEnsemble#1#2{\Bigl\{#1\Bigm|#2\Bigr\}}
\def\biggEnsemble#1#2{\biggl\{#1\biggm|#2\biggr\}}
\def\BiggEnsemble#1#2{\Biggl\{#1\Biggm|#2\Biggr\}}
\let\Ensemble=\bigEnsemble

\newcommand\IntO[1]{\left]#1\right[}
\newcommand\IntF[1]{\left[#1\right]}
\newcommand\IntOF[1]{\left]#1\right]}
\newcommand\IntFO[1]{\left[#1\right[}

\newcommand\intO[1]{\mathopen]#1\mathclose[}
\newcommand\intF[1]{\mathopen[#1\mathclose]}
\newcommand\intOF[1]{\mathopen]#1\mathclose]}
\newcommand\intFO[1]{\mathopen[#1\mathclose[}

\newcommand\Fn[3]{#1\colon#2\to#3}
\newcommand\CC[1]{\mathscr{C}^{#1}}
\newcommand\D{\mathop{}\!\mathrm{d}}

\newcommand\longto{\longrightarrow}

\undef\M
\newcommand\M[3]{\mathrm{#1}_{#2}\pa{#3}}
\newcommand\MnR{\M{M}{n}{\R}}
\newcommand\MnC{\M{M}{n}{\C}}
\newcommand\MnK{\M{M}{n}{\K}}
\newcommand\GLnR{\M{GL}{n}{\R}}
\newcommand\GLnC{\M{GL}{n}{\C}}
\newcommand\GLnK{\M{GL}{n}{\K}}
\newcommand\DnR{\M{D}{n}{\R}}
\newcommand\DnC{\M{D}{n}{\C}}
\newcommand\DnK{\M{D}{n}{\K}}
\newcommand\SnR{\M{S}{n}{\R}}
\newcommand\AnR{\M{A}{n}{\R}}
\newcommand\OnR{\M{O}{n}{\R}}
\newcommand\SnRp{\mathrm{S}_n^+(\R)}
\newcommand\SnRpp{\mathrm{S}_n^{++}(\R)}

\newcommand\LE{\mathscr{L}(E)}
\newcommand\GLE{\mathscr{GL}(E)}
\newcommand\SE{\mathscr{S}(E)}
\renewcommand\OE{\mathscr{O}(E)}

\newcommand\ImplD{$\Cro\Rightarrow$}
\newcommand\ImplR{$\Cro\Leftarrow$}
\newcommand\InclD{$\Cro\subset$}
\newcommand\InclR{$\Cro\supset$}
\newcommand\notInclD{$\Cro{\not\subset}$}
\newcommand\notInclR{$\Cro{\not\supset}$}

\newcommand\To[1]{\xrightarrow[#1]{}}
\newcommand\Toninf{\To{\ninf}}

\newcommand\Norm[1]{\|#1\|}
\newcommand\Norme{{\Norm{\cdot}}}

\newcommand\Int[1]{\mathring{#1}}
\newcommand\Adh[1]{\overline{#1}}

\newcommand\Uplet[2]{{#1},\dots,{#2}}
\newcommand\nUplet[3]{(\Uplet{{#1}_{#2}}{{#1}_{#3}})}

\newcommand\Fonction[5]{{#1}\left|\begin{aligned}{#2}&\;\longto\;{#3}\\{#4}&\;\longmapsto\;{#5}\end{aligned}\right.}

\DeclareMathOperator\orth{\bot}
\newcommand\Orth[1]{{#1}^\bot}
\newcommand\PS[2]{\langle#1,#2\rangle}

\newcommand{\Tribu}{\mathscr{T}}
\newcommand{\Part}{\mathcal{P}}
\newcommand{\Pro}{\bigPa{\Omega,\Tribu}}
\newcommand{\Prob}{\bigPa{\Omega,\Tribu,\P}}

\newcommand\DEMO{$\spadesuit$}
\newcommand\DUR{$\spadesuit$}

\newenvironment{psmallmatrix}{\left(\begin{smallmatrix}}{\end{smallmatrix}\right)}

% -----------------------------------------------------------------------------

\newcommand{\DF}[2]{\D{#1} ({#2})}
\newcommand{\DIF}[3]{\DF{#1}{#2}\cdot{#3}}

\begin{document}
\title{Fonctions de plusieurs variables}
\maketitle

\Para{Contexte}

On s'int\'eresse \`a l'\'etude des fonctions qui d\'ependent de $p$ variables, c.-\`a-d. aux fonctions $\Fn{f}{U}{\R^n}$ o\`u $U$ est un ouvert de $\R^p$.

\Para{D\'efinition}

On dit que $f$ est une \emph{fonction num\'erique} si et seulement si $n=1$.

\Para{Notation}

En g\'en\'eral, on notera
\[f(x) = \bigl(\Uplet{f_1(x)}{f_n(x)}\bigr) \quad \text{o\`u} \quad x = \nUplet x1p,\]
de sorte que $\Fn{f_k}{U}{\R}$ pour $k\in\Dcro{1,n}$ est une fonction num\'erique.
Les $f_k$ s'appellent les \emph{composantes} de $f$.

% -----------------------------------------------------------------------------
\section{Applications continues (rappels)}

\Para{D\'efinition}

Soit $\Fn{f}{U}{\R^n}$ o\`u $U$ est un ouvert de $\R^p$.
Soit $a\in U$.
Les conditions suivantes sont \'equivalentes:
\begin{enumerate}
\item la fonction $f$ est continue en $a$,
\item $\lim\limits_{x \to a} f(x) = f(a)$,
\item $\lim\limits_{\Norm{x-a} \to 0} \Norm{f(x)-f(a)} = 0$,
\item $\forall \epsilon>0$, $\exists \eta>0$, $\Norm{x-a}<\eta\implies \Norm{f(x)-f(a)}<\epsilon$.
\end{enumerate}

\Para{Remarque}

Puisque les normes sont \'equivalentes en dimension finie,
la continuit\'e de $f$ ne d\'epend pas du choix des normes sur $\R^p$ et sur $\R^n$.

% -----------------------------------------------------------------------------
\section{Applications de classe $\CC1$}

\subsection{D\'eriv\'ees}

\Para{D\'efinition}[d\'eriv\'ee selon un vecteur]

Soit $\Fn{f}{U}{\R^n}$ o\`u $U$ est un ouvert de $\R^p$.
Soit $a\in U$ et $h\in \R^p$.
On pose $\varphi(t) = f(a+th)$.
Alors $\varphi$ est d\'efinie au voisinage de 0.

Si $\varphi$ est d\'erivable en $0$,
on dit que $f$ est \emph{d\'erivable en $a$ selon le vecteur $h$},
et l'on note
\[\DIF fah = \varphi'(0)
= \lim_{\substack{t \to 0 \\ t\in \R^*}} \left(\frac{f(a+th)-f(a)}{t}\right).\]
$\DIF fah$ s'appelle la \emph{d\'eriv\'ee de $f$ en $a$ selon le vecteur $h$}.

\Para{Notation}

On note $\nUplet e1p$ la base canonique de $\R^p$.

\Para{D\'efinition}[d\'eriv\'ees partielles]

Soit $\Fn{f}{U}{\R^n}$ o\`u $U$ est un ouvert de $\R^p$.
Soit $a\in U$.
On appelle \emph{$k$-i\`eme d\'eriv\'ee partielle de $f$}
la d\'eriv\'ee de $f$ selon le vecteur $e_k$, si elle existe.
On la note alors $\partial_k f(a)$ ou encore $\DerPart{f}{x_k}(a)$.
Ainsi,
\[\begin{aligned}
    \DerPart{f}{x_k}(a) = \lim_{\substack{h\to 0 \\ h\neq0}} \frac{1}{h} \Bigl[
    & f(a_1, \ldots, a_{k-1}, a_k + h, a_{k+1}, \ldots, a_p) \\
    & - f(a_1, \ldots, a_p) \Bigr].
\end{aligned}\]

\Para{Attention}

Une fonction peut admettre des d\'eriv\'ees partielles en tout point sans pour autant \^etre continue.
cf. exercice 12.

% -----------------------------------------------------------------------------
\subsection{Applications de classe $\CC1$}

\Para{D\'efinition}

Soit $\Fn{f}{U}{\R^n}$ o\`u $U$ est un ouvert de $\R^p$.
On dit que \emph{$f$ est de classe $\CC1$ sur $U$}
si
\begin{itemize}
\item $f$ admet des d\'eriv\'ees partielles en tout point de $U$;
\item les d\'eriv\'ees partielles $\partial_k f$ sont \emph{continues} sur $U$ pour tout $k\in\Dcro{1,p}$.
\end{itemize}

\Para{Th\'eor\`eme}

Soit $\Fn{f}{U}{\R^n}$ une fonction de classe $\CC1$ o\`u $U$ est un ouvert de $\R^p$.
Alors $f$ est continue sur $U$,
et admet en tout point de $U$ des d\'eriv\'ees selon tout vecteur.
De plus, on a $\forall a\in U$, $\forall h = \nUplet h1p\in \R^p$,
\[\DIF fah = \sum_{k=1}^p h_k \DerPart{f}{x_k}(a).\]

\Para{Proposition}

Soit $\Fn{f}{U}{\R^n}$ une fonction de classe $\CC1$ o\`u $U$ est un ouvert de $\R^p$.
Soit $a\in U$.
On a
\[f(a+h) = f(a) + \DIF fah + \PetitO(h) \quad \text{quand } h\to0.\]

\Para{Remarque}

Soit $U$ un ouvert de $\R^p$ contenant $0$ et $\Fn{f}{U}{\R^n}$.
Les conditions suivantes sont \'equivalentes:
\begin{enumerate}
\item $f(x) = \PetitO(x)$ quand $x \to 0$;
\item $\Norm{f(x)} = \PetitO\left(\Norm x\right)$ quand $x \to 0$;
\item $\lim\limits_{x\to\vec0} \left(\frac{\Norm{f(x)}}{\Norm x}\right) = 0$.
\end{enumerate}

% -----------------------------------------------------------------------------
\subsection{Diff\'erentielle}

\Para{D\'efinition}

Soit $\Fn{f}{U}{\R^n}$ une fonction de classe $\CC1$ o\`u $U$ est un ouvert de $\R^p$.
Soit $a\in U$.
L'application \[\Fonction{\DF fa}{\R^p}{\R^n}{h}{\DIF fah}\]
est une application \emph{lin\'eaire},
appel\'ee \emph{diff\'erentielle de $f$ en $a$}
et not\'ee $\DF fa$.

\Para{Remarque}

Notons $\D{x_k}$ l'application coordonn\'ee
\[\Fonction{\D{x_k}}{\R^p}{\R}{\nUplet x1p}{x_k.}\]
On a alors
\[\DF fa = \sum_{k=1}^p \DerPart{f}{x_k}(a) \D{x_k},\]
ou encore, en omettant le point $a$,
\[\D f = \DerPart{f}{x_1} \D{x_1} + \cdots + \DerPart{f}{x_p} \D{x_p}.\]

\Para{Proposition}[r\'eduction \`a des fonctions num\'eriques]

Soit $\Fn{f}{U}{\R^n}$ o\`u $U$ est un ouvert de $\R^p$.
On note $f(x) = \left( \Uplet{f_1(x)}{f_n(x)} \right)$ o\`u $f_k \colon U \to\R$.
Alors
\begin{enumerate}
\item $f$ est continue sur $U$ si et seulement si les fonctions $\Uplet{f_1}{f_n}$ le sont.
\item $f$ est de classe $\CC1$ si et seulement si les fonctions $\Uplet{f_1}{f_n}$ le sont.
\item Plus g\'en\'eralement, avec les d\'efinitions que l'on verra plus loin, on a pour tout $k\in \N\cup\Acco{\infty}$:
  $f$ est de classe $\CC k$ si et seulement si les fonctions $\Uplet{f_1}{f_n}$ le sont.
\end{enumerate}

\subsection{Op\'erations sur les fonctions de classe $\CC1$}

\Para{Proposition}[somme et produit]

$U$ d\'esigne un ouvert de $\R^p$.
\begin{enumerate}
\item Si $f \colon U \to\R^n$ et $g \colon U \to\R^n$ sont de classe $\CC1$, alors $f+g$ l'est.
  De plus, $\DerPart{(f+g)}{x_k} = \DerPart{f}{x_k} + \DerPart{g}{x_k}$.
\item Si $f \colon U \to\R$ et $g \colon U \to\R^n$ sont de classe $\CC1$, alors $fg$ l'est.
  De plus, $\DerPart{(fg)}{x_k} = \DerPart{f}{x_k} g + f \DerPart{g}{x_k}$.
\end{enumerate}

\Para{Th\'eor\`eme}[composition]

Soit $\Fn{f}{U}{\R^n}$ et $\Fn{g}{V}{\R^m}$ o\`u $U$ est un ouvert de $\R^p$ et $V$ un ouvert de $\R^n$.
On suppose que $f$ et $g$ sont de classe $\CC1$ et que $f(U)\subset V$.
Alors $\Fn{g\circ f}{U}{\R^m}$ est de classe $\CC1$.
De plus, pour tout $a\in U$,
\[\DF{(g\circ f)}{a} = \Big[ \D{g}\bigl(f(a)\bigr) \Big] \circ\Big[ \DF fa \Big].\]
Autrement dit,
\[\DerPart{(g\circ f)_i}{x_j}(a) = \sum_{k=1}^{n} \DerPart{g_i}{y_k}\bigl(f(a)\bigr) \DerPart{f_k}{x_j}(a).\]

\Para{Corollaire}

Soit $I$ un intervalle de $\R$, $U$ un ouvert de $\R^p$.
Soit $\varphi\colon I \to U$ et $\Fn{f}{U}{\R^n}$ deux applications de classe $\CC1$.
Alors $f\circ \varphi\colon I \to\R^n$ est de classe $\CC1$ et
\[\begin{aligned}
    (f\circ \varphi)'(t)
    &= \DF{f}{\varphi(t)} (\varphi'(t)) \\
    &= \sum_{k=1}^p \DerPart{f}{x_k} \bigl(\varphi_k(t)\bigr) \, \varphi_k'(t).
\end{aligned}\]

Cela peut s'interpr\'eter g\'eom\'etriquement:
\begin{itemize}
\item $\varphi$ correspond \`a un arc param\'etr\'e,
\item $\varphi'$ au vecteur vitesse,
\item $f\circ \varphi$ \`a l'image de l'arc $\varphi$ par l'application $f$.
\end{itemize}

On obtient donc le vecteur vitesse de l'image d'un arc param\'etr\'e.

\subsection{D\'eriv\'ees d'ordre sup\'erieur}

\Para{D\'efinition}

Soit $\Fn{f}{U}{\R^n}$ une fonction de classe $\CC1$ o\`u $U$ est un ouvert de $\R^p$.
Si les applications $\partial_k f \colon U \to\R^n$ sont elles-m\^emes
de classe $\CC1$ pour tout $k\in\Dcro{1,p}$,
alors on dit que $f$ est de classe $\CC2$ et on note $\forall a\in U$,
\[\partial_{k,l} f (a)
  = \frac{\partial^2 f}{\partial x_k \,\partial x_l}(a)
  =\partial_k (\partial_l f) (a)
  = \frac{\partial}{\partial x_k}
\left(\frac{\partial f}{\partial x_l}\right) (a).\]
On d\'efinit de m\^eme par r\'ecurrence les fonctions de classe $\CC k$, $k\geq2$,
comme \'etant les fonctions admettant en tout point de $U$
des d\'eriv\'ees partielles de classe $\CC{k-1}$.

\Para{Th\'eor\`eme}[Schwarz]

Soit $f \colon U \to\R^n$ de classe $\CC2$ o\`u $U$ est un ouvert de $\R^p$.
Alors $\forall a\in U$, $\forall(i,j)\in\Dcro{1,p}^2$,
\[\frac{\partial^2 f}{\partial x_i \,\partial x_j} (a)
= \frac{\partial^2 f}{\partial x_j \,\partial x_i} (a).\]

% -----------------------------------------------------------------------------
\section{Cas des fonctions num\'eriques}

\subsection{\'Equations aux d\'eriv\'ees partielles}

cf. exercices.

\subsection{Un soup\c con de topologie}

\Para{Remarque}

Si $D\subset \R^p$, on peut d\'efinir le \emph{bord} de $D$, not\'e $\partial D$.
Lorsque $D$ est une partie \og gentille\fg{} de $\R^p$,
$\partial D$ est exactement ce que son nom laisse supposer.

\Para{Remarque}

En g\'en\'eral, la d\'efinition est un peu technique et hors-programme.
\begin{multline*}
  \partial D = \Bigl\{ x\in \R^p \,\Bigm|\, \forall r>0, \\
  B(x,r) \cap{} D\neq\emptyset{} \text{ et }
  B(x,r) \cap{} {}^c \! D\neq\emptyset{}
  \Bigr\}
\end{multline*}
Autrement dit, $x\in\partial D$ si et seulement si $x$ est \`a la fois limite
d'une suite d'\'el\'ements de $D$ et limite d'une autre suite
d'\'el\'ements du compl\'ementaire de $D$.

\Para{Proposition}

Soit $D\subset \R^p$.
\begin{itemize}
\item $D$ ouvert si et seulement si $\partial D\subset^c D$.
\item $D$ ferm\'e si et seulement si $\partial D\subset D$.
\item L'int\'erieur de $D$ est \'egal \`a $\mathring{D} = D\setminus\partial D$;
  c'est un ouvert.
\item L'adh\'erence de $D$ est \'egale \`a $\overline{D} = D\cup\partial D$;
  c'est un ferm\'e.
\end{itemize}

\Para{Th\'eor\`eme}

Une fonction num\'erique \emph{continue} sur une partie \emph{ferm\'ee born\'ee} de $\R^p$ est born\'ee et atteint ses bornes.

\subsection{Extremum}

\Para{D\'efinitions}

Soit $\Fn{f}{D}{\R}$ o\`u $D$ est une partie de $\R^p$.
Soit $a\in D$.
\begin{itemize}
\item $f$ a un \emph{minimum global} en $a$ si et seulement si \[\forall x\in D\+ f(x)\geq f(a).\]
\item $f$ a un \emph{maximum global} en $a$ si et seulement si \[\forall x\in D\+ f(x)\leq f(a).\]
\item $f$ a un \emph{extremum global} en $a$ si et seulement si $f$ a un minimum ou un maximum global en $a$.
\item $f$ a un \emph{minimum local} en $a$ si et seulement si \[\exists r > 0\+ \forall x\in B(a,r)\cap D\+ f(x)\geq f(a).\]
\item $f$ a un \emph{maximum local} en $a$ si et seulement si \[\exists r > 0\+ \forall x\in B(a,r)\cap D\+ f(x)\leq f(a).\]
\item $f$ a un \emph{extremum local} en $a$ si et seulement si $f$ a un minimum ou un maximum local en $a$.
\end{itemize}

\Para{D\'efinition}[gradient]

Soit $\Fn{f}{U}{\R}$ de classe $\CC1$ o\`u $U$ est un ouvert de $\R^p$.
On appelle \emph{gradient de $f$} l'application
\[\Fonction{\nabla f}{U}{\R^p}{x}{
\left(\Uplet{\DerPart{f}{x_1}(x)}{\DerPart{f}{x_p}(x)}\right).}\]

\Para{Proposition}

Avec les m\^emes notations, on a, $\forall a\in U$,
\[\DIF fah = \PS{\nabla f(a)}{h}\]

\Para{D\'efinition}[point critique]

Soit $\Fn{f}{U}{\R}$ de classe $\CC1$ o\`u $U$ est un ouvert de $\R^p$.
Soit $a\in U$.
On dit que $a$ est un \emph{point critique} de $f$ si et seulement si $\nabla f(a) = 0$.

\Para{Th\'eor\`eme}

Soit $\Fn{f}{U}{\R}$ de classe $\CC1$ o\`u $U$ est un ouvert de $\R^p$.
\begin{itemize}
\item Tout extremum global de $f$ est un extremum local de $f$.
\item Tout extremum local de $f$ est un point critique de $f$.
\item Les deux r\'eciproques sont fausses, m\^eme dans le cas $p=1$.
\item Si $U$ n'est pas un ouvert, le premier point reste vrai,
  mais g\'en\'eralement pas le second.
\end{itemize}

\Para{Remarque}

Ce th\'eor\`eme est tr\`es pratique pour d\'eterminer les extrema
des fonctions num\'eriques de plusieurs variables.

Si $f$ est d\'efini sur un domaine $D$ quelconque,
on ne peut appliquer ce r\'esultat que sur l'int\'erieur $U=\mathring{D}$.
Il faut \'etudier les valeurs de $f$ sur le bord $\partial D$ par d'autres m\'ethodes,
g\'en\'eralement par une \'etude directe.

% -----------------------------------------------------------------------------
\section{Exercices}

\subsection{\'Echauffement}

% -----------------------------------------------------------------------------
\par\pagebreak[1]\par
\paragraph{Exercice 1}%
\hfill{\tiny 9654}%
\begingroup~

\'Etudier la continuit\'e des fonctions suivantes:
\begin{enumerate}
\item $f(x,y) = \frac{\sin x + \cos y}{x^2 + y^2 + 1}$
\item $f(x,y) = x^y$
\item $f(x,y) = \begin{cases}
      \frac{x \sin^2 y}{x^2+y^2} & \text{si } (x,y)\neq(0,0) \\
      \hfil 0                    & \text{si } x=y=0.
  \end{cases}$
\end{enumerate}
\endgroup

% -----------------------------------------------------------------------------
\par\pagebreak[1]\par
\paragraph{Exercice 2}%
\hfill{\tiny 9794}%
\begingroup~

Soit $\Fn{f}{\R^p}{\R^n}$ lin\'eaire.
Montrer que $f$ est de classe $\CC1$ et que $\forall a\in \R^p$, $\D f(a) = f$.
\endgroup

% -----------------------------------------------------------------------------
\par\pagebreak[1]\par
\paragraph{Exercice 3}%
\hfill{\tiny 6726}%
\begingroup~

Soit $\Fonction{f}{\R^2}{\R}{(x,y)}{\max(\Abs x, \Abs y)}$

D\'eterminer le plus grand ouvert sur lequel $f$ est $\CC1$.
\endgroup

% -----------------------------------------------------------------------------
\par\pagebreak[1]\par
\paragraph{Exercice 4}%
\hfill{\tiny 5850}%
\begingroup~

Soit $\Fn{f}{\R^2}{\R}$ une fonction de classe $\CC1$.
On pose $u(x) = f(x,x)$, $v(x,y)=f(y,x)$ et $w(x,y) = f(x+y,x^2 \sin y)$.
Calculer les d\'eriv\'ees partielles de ces fonctions de plusieurs fa\c cons:
\begin{itemize}
\item avec la r\`egle de la cha\^ine;
\item avec les diff\'erentielles.
\end{itemize}
\endgroup

% -----------------------------------------------------------------------------
\par\pagebreak[1]\par
\paragraph{Exercice 5}%
\hfill{\tiny 7032}%
\begingroup~

\def\Div{\mathop{\mathrm{div}}}
Soit \[\Fonction {\Phi} {\intO{0,\infty}\times\intO{-\pi,\pi}} {\R^2\setminus\Ensemble{(x,0)}{x\leq0}} {(\rho,\theta)} {(x,y) = (\rho\cos\theta,\rho\sin\theta).}\]
\begin{enumerate}
\item Soit $f \colon\R^2 \to\R$ et $g = f\circ \Phi$.
  Informellement, on a $f(x,y) = g(\rho,\theta)$.

  Exprimer les d\'eriv\'ees partielles de $g$ en fonction de celles de $f$.
\item Calculer $\Div g$ sachant que $\Div f = \frac{\partial f}{\partial x} + \frac{\partial f}{\partial y}$.
\item Calculer $\Delta g$ sachant que $\Delta f = \frac{\partial^2 f}{\partial x^2} + \frac{\partial^2 f}{\partial y^2}$.
\item On pose $f(x,y) = h(x,\theta)$.
  Comparer les d\'eriv\'ees partielles $\DerPart fx$ et $\DerPart hx$.
  Expliquer la diff\'erence!
\end{enumerate}
\endgroup

\subsection{Classiques}

% -----------------------------------------------------------------------------
\par\pagebreak[1]\par
\paragraph{Exercice 6 (int\'egrales \`a param\`etre, avec bornes variables)}%
\hfill{\tiny 1421}%
\begingroup~

Soit $\varphi$ et $\psi$ deux fonctions $\R\to\R$ de classe $\CC1$.
Soit $f \colon\R^2 \to\R$ de classe $\CC1$.
On pose
\[g(x) = \int_{\varphi(x)}^{\psi(x)} f(x,t) \D t \quad\text{et}\quad
h(a,b,x) = \int_a^b f(x,t) \D t.\]
\begin{enumerate}
\item Montrer que $h$ est de classe $\CC1$.
\item En d\'eduire que $g$ est \'egalement de classe $\CC1$ et calculer $g'(x)$.
\end{enumerate}
\endgroup

% -----------------------------------------------------------------------------
\par\pagebreak[1]\par
\paragraph{Exercice 7}%
\hfill{\tiny 4156}%
\begingroup~

Soit $(p,q)\in \N^2$ et $f \colon\R^2 \to\R$ d\'efinie par
\[f(x,y) = \begin{cases}
    \frac{x^p y^q}{x^2+y^2} & \text{si } (x,y)\neq(0,0) \\
    \hfil 0                 & \text{si } x=y=0.
\end{cases}\]
\begin{enumerate}
\item Montrer que $f$ est $\CC{\infty}$ sur $\R^2\setminus\Acco{(0,0)}$.
\item Montrer que $f$ est continue ssi $p+q\geq2$.
\item Montrer que $f$ est de classe $\CC1$ ssi $p+q\geq3$.
\item Montrer que $f$ est de classe $\CC k$ ssi $p+q\geq k+2$.
\end{enumerate}
\endgroup

\subsection{\'Equations aux D\'eriv\'ees Partielles}

% -----------------------------------------------------------------------------
\par\pagebreak[1]\par
\paragraph{Exercice 8}%
\hfill{\tiny 6898}%
\begingroup~

D\'eterminer les fonctions $\Fn{f}{(\R_+^*)^2}{\R}$ de classe $\CC1$ v\'erifiant:
\[x \DerPart fx = y \DerPart fy.\]
On pourra poser $u = xy$ et $v = \frac xy$.
\endgroup

% -----------------------------------------------------------------------------
\par\pagebreak[1]\par
\paragraph{Exercice 9}%
\hfill{\tiny 3393}%
\begingroup~

D\'eterminer les fonctions $\Fn{f}{\R^2\setminus\{(0,0)\}}{\R}$ de classe $\CC1$ v\'erifiant:
\[x \DerPart fx = - y \DerPart fy\]
On pourra poser $x = \rho\cos\theta$ et $y = \rho\sin\theta$.
\endgroup

% -----------------------------------------------------------------------------
\par\pagebreak[1]\par
\paragraph{Exercice 10}%
\hfill{\tiny 7278}%
\begingroup~

R\'esoudre l'\'equation
\[ \DerPart fu (u,v) + 2u \DerPart fv (u,v) = 0 \]
en effectuant le changement de variables
\[ \phi(x,y) = (x, y+x^2). \]
\endgroup

% -----------------------------------------------------------------------------
\par\pagebreak[1]\par
\paragraph{Exercice 11}%
\hfill{\tiny 7261}%
\begingroup~

D\'eterminer les fonctions $\Fn{f}{(\R_+^*)^2}{\R}$ de classe $\CC2$ v\'erifiant:
\[x^2 \frac{\partial^2 f}{\partial x^2} + 2xy \frac{\partial^2 f}{\partial x\partial y} + y^2 \frac{\partial^2 f}{\partial y^2} = 0\]
On pourra poser $u = xy$ et $v = \frac xy$.
\endgroup

% -----------------------------------------------------------------------------
\par\pagebreak[1]\par
\paragraph{Exercice 12}%
\hfill{\tiny 2824}%
\begingroup~

Soit $\Fn{f}{\R^2}{\R}$ de classe $\CC2$ et $g(u,v) = f(uv,u+v)$.
\begin{enumerate}
\item Calculer $\frac{\partial^2 g}{\partial u\partial v}$.
\item R\'esoudre l'\'equation:
  \[x \frac{\partial^2 f}{\partial x^2} + y \frac{\partial^2 f}{\partial x\partial y} + \frac{\partial^2 f}{\partial y^2} + \frac{\partial f}{\partial x} = y.\]
\end{enumerate}
\endgroup

\subsection{T\'eratologie}

% -----------------------------------------------------------------------------
\par\pagebreak[1]\par
\paragraph{Exercice 13}%
\hfill{\tiny 2632}%
\begingroup~

Soit $f(x,y) = \begin{cases}
    \frac{x y}{x^2+y^2} & \text{si } (x,y)\neq(0,0) \\
    \hfil 0             & \text{si } x=y=0.
\end{cases}$
\begin{enumerate}
\item Montrer que $\forall x\in \R$, la fonction $y \mapsto f(x,y)$ est continue.
  De m\^eme, montrer que $\forall y\in \R$, la fonction $x \mapsto f(x,y)$ est continue.
\item Montrer que $f$ admet des d\'eriv\'ees partielles en tout point $(x_0,y_0)\in \R^2$.
\item $f$ est-elle de classe $\CC1$?
\item Montrer que $f$ est continue sur $\R^2\setminus\Acco{(0,0)}$ mais n'est pas continue en $(0,0)$.
\end{enumerate}
\endgroup

% -----------------------------------------------------------------------------
\par\pagebreak[1]\par
\paragraph{Exercice 14}%
\hfill{\tiny 4611}%
\begingroup~

\begin{enumerate}
\item Soit $f(x,y) = \frac{y^2}{x}$ si $x\neq0$, et $f(0,y) = 0$.
  Montrer que $f$ est d\'erivable selon tout vecteur en $(0,0)$, mais qu'elle n'est pas continue en $(0,0)$.
\item Soit $f(x,y) = xy\frac{x^2-y^2}{x^2+y^2}$ si $(x,y)\neq(0,0)$ et $f(0,0)=0$.
  Monter que $\frac{\partial^2 f}{\partial x\partial y}(0,0)$ et $\frac{\partial^2 f}{\partial y\partial x}(0,0)$ existent, mais qu'ils sont diff\'erents.
\end{enumerate}
\endgroup

\subsection{Recherche d'extremas}

% -----------------------------------------------------------------------------
\par\pagebreak[1]\par
\paragraph{\href{https://psi.miomio.fr/exo/2158.pdf}{Exercice 15}}%
\hfill{\tiny 2158}%
\begingroup~

\newcommand\Dom{\mathcal{D}}
Extrema de
$f(x,y) = x^2 y (x+y-4)$
sur $\Dom = \Ensemble{(x,y)\in \R^2}{x\geq0, y\geq0, x+y\leq6}$.
\endgroup

% -----------------------------------------------------------------------------
\par\pagebreak[1]\par
\paragraph{\href{https://psi.miomio.fr/exo/7939.pdf}{Exercice 16}}%
\hfill{\tiny 7939}%
\begingroup~

Extrema de
$f(x,y) = x^4 + y^4 - 2(x-y)^2$.
\endgroup

% -----------------------------------------------------------------------------
\par\pagebreak[1]\par
\paragraph{\href{https://psi.miomio.fr/exo/4117.pdf}{Exercice 17}}%
\hfill{\tiny 4117}%
\begingroup~

Extrema de
$f(x,y) = x^2(x+1) + y^3$.
\endgroup

% -----------------------------------------------------------------------------
\par\pagebreak[1]\par
\paragraph{\href{https://psi.miomio.fr/exo/5095.pdf}{Exercice 18}}%
\hfill{\tiny 5095}%
\begingroup~

Extrema de
$f(x,y) = x^3 + y^3 - 6(x^2 - y^2)$.
\endgroup

% -----------------------------------------------------------------------------
\par\pagebreak[1]\par
\paragraph{\href{https://psi.miomio.fr/exo/7023.pdf}{Exercice 19}}%
\hfill{\tiny 7023}%
\begingroup~

\newcommand\Dom{\mathcal{D}}
Extrema de
$f(x,y) = \sqrt{x^2+y^2}+x^2-3$
sur $\Dom = \Ensemble{(x,y)\in \R^2}{x^2+y^2\leq16}$.
\endgroup

% -----------------------------------------------------------------------------
\par\pagebreak[1]\par
\paragraph{\href{https://psi.miomio.fr/exo/9605.pdf}{Exercice 20}}%
\hfill{\tiny 9605}%
\begingroup~

Extrema de
$f(x,y,z) = x^2 + y^2 + z^2 + 2xyz$.
\endgroup

% -----------------------------------------------------------------------------
\par\pagebreak[1]\par
\paragraph{\href{https://psi.miomio.fr/exo/6934.pdf}{Exercice 21}}%
\hfill{\tiny 6934}%
\begingroup~

Extrema de
$f(x,y) = (y^2-x^2)(y^2-2x^2)$.
\endgroup

% -----------------------------------------------------------------------------
\par\pagebreak[1]\par
\paragraph{\href{https://psi.miomio.fr/exo/6549.pdf}{Exercice 22}}%
\hfill{\tiny 6549}%
\begingroup~

Extrema de
$f(x,y) = x\ln y - y\ln x$.
\endgroup

% -----------------------------------------------------------------------------
\par\pagebreak[1]\par
\paragraph{\href{https://psi.miomio.fr/exo/4801.pdf}{Exercice 23}}%
\hfill{\tiny 4801}%
\begingroup~

Extrema de
$f(x,y) = 4xy + \frac1x + \frac1y$.
\endgroup

% -----------------------------------------------------------------------------
\par\pagebreak[1]\par
\paragraph{\href{https://psi.miomio.fr/exo/2351.pdf}{Exercice 24}}%
\hfill{\tiny 2351}%
\begingroup~

Extrema de
$f(x,y) = x^3+y^3-9xy+27$.
\endgroup

% -----------------------------------------------------------------------------
\par\pagebreak[1]\par
\paragraph{Exercice 25}%
\hfill{\tiny 4002}%
\begingroup~

\newcommand\Dom{\mathcal{D}}
Extrema de
$f \nUplet x1n = x_1 x_2 \cdots x_n$ sur $\Dom = \Ensemble{\nUplet x1n\in \R_+^n}{x_1+x_2+\cdots+x_n\leq1}$.
\endgroup

\subsection{Divers}

% -----------------------------------------------------------------------------
\par\pagebreak[1]\par
\paragraph{Exercice 26}%
\hfill{\tiny 4717}%
\begingroup~

Soit une fonction $f$ de classe $\CC1$ de $\R^2$ dans $\R$
et soit $(p,q)\in \R^2$ tels que $(p,q)\neq(0,0)$.
Montrer que les restrictions de $f$ aux droites d'\'equations
$px+qy=C^{\text{ste}}$
sont constantes si et seulement si
\[ q \DerPart fx = p \DerPart fy. \]
\endgroup

% -----------------------------------------------------------------------------
\par\pagebreak[1]\par
\paragraph{Exercice 27}%
\hfill{\tiny 9697}%
\begingroup~

Soit une fonction $\Fn{f}{\R^n\setminus\{0\}}{\R}$ de classe $\CC1$ et $k$ une constante r\'eelle.
On dit que $f$ est homog\`ene de degr\'e $k$ si et seulement si
\[\forall t > 0 \+\forall x\in \R^n\setminus\{0\} \+
f(tx) = t^k f(x).\]
Montrer que $f$ est homog\`ene de degr\'e $k$
si et seulement si elle v\'erifie l'identit\'e d'Euler:
\[\forall x\in \R^n\setminus\{0\} \+ \sum_{i=1}^{n} \DerPart{f}{x_i}(x) = kf(x).\]
\endgroup

% -----------------------------------------------------------------------------
\par\pagebreak[1]\par
\paragraph{Exercice 28}%
\hfill{\tiny 3040}%
\begingroup~

Soit $U$ un ouvert convexe de $\R^n$, $a$ et $b$ deux points distincts de $U$.
\begin{enumerate}
\item Si $f \colon U \to\R$ est de classe $\CC1$ sur $U$,
  montrer qu'il existe $c\in\intO{a,b}$ tel que $f(b)-f(a) = \D f(c)\cdot(b-a)$.
  On rappelle que $\D f(c)\cdot(b-a) = \PS{\nabla f(c)}{b-a}$.
\item De combien de d\'ecimales exactes des nombres $e$, $\sqrt2$ et $\pi$ a-t-on besoin
  pour pouvoir calculer \`a $10^{-20}$ pr\`es le nombre $\frac{\sqrt2}{e+\pi^3}$?
\end{enumerate}
\endgroup

% -----------------------------------------------------------------------------
\par\pagebreak[1]\par
\paragraph{Exercice 29}%
\hfill{\tiny 2618}%
\begingroup~

Soit $\Fn{f}{\R^n}{\R}$ de classe $\CC1$ sur un ouvert $U$ contenant $\vec0$, telle que
\[\forall x\in \R^n\setminus\{0\}\+ \forall t\in \R_+^*\+ f(tx) = tf(x)\]
Montrer que $f$ est lin\'eaire.
\endgroup

% -----------------------------------------------------------------------------
\par\pagebreak[1]\par
\paragraph{Exercice 30}%
\hfill{\tiny 9534}%
\begingroup~

Soit $f \colon\R\to\R$ de classe $\CC1$, et $F(x,y) = \frac{f(x)-f(y)}{x-y}$.
\begin{enumerate}
\item Montrer que $F$ se prolonge en une fonction continue sur $\R^2$, que l'on notera $\widetilde F$.
\item Si $f$ est de classe $\CC2$, montrer que $\widetilde F$ est de classe $\CC1$.
\end{enumerate}
\endgroup

% -----------------------------------------------------------------------------
\par\pagebreak[1]\par
\paragraph{Exercice 31}%
\hfill{\tiny 1006}%
\begingroup~

Soit $f(x,y) = \frac{\sin x - \sin y}{\sh x - \sh y}$.
Montrer que $f$ se prolonge en une fonction $\CC\infty$ sur $\R^2$.
\endgroup

% -----------------------------------------------------------------------------
\par\pagebreak[1]\par
\paragraph{Exercice 32}%
\hfill{\tiny 3170}%
\begingroup~

Soit $\Fonction{f}{\MnR}{\MnR}{X}{X^2.}$
\begin{enumerate}
\item Montrer que $f$ est $\CC\infty$.
\item Calculer $\D f(X)$.
\item On suppose $X$ diagonalisable et que ses valeurs propres sont toutes strictement positives.
  Montrer alors que $\D f(X)$ est un automorphisme de $\MnR$.
\end{enumerate}
\endgroup

\end{document}
