% autogenerated by ytex.rs

\documentclass{scrartcl}

\usepackage[francais]{babel}
\usepackage{geometry}
\usepackage{scrpage2}
\usepackage{lastpage}
\usepackage{ragged2e}
\usepackage{multicol}
\usepackage{etoolbox}
\usepackage{xparse}
\usepackage{enumitem}
\usepackage{csquotes}
\usepackage{amsmath}
\usepackage{amsfonts}
\usepackage{amssymb}
\usepackage{mathrsfs}
\usepackage{stmaryrd}
\usepackage{dsfont}
\usepackage{eurosym}
\usepackage{numprint}
\usepackage[most]{tcolorbox}
\usepackage{tikz}
\usepackage{tkz-tab}
\usepackage[unicode]{hyperref}
\usepackage[ocgcolorlinks]{ocgx2}

\let\ifTwoColumns\iftrue
\def\Classe{$\Psi$2019--2020}

% reproducible builds
% LuaTeX: \pdfvariable suppressoptionalinfo 1023 \relax
\pdfinfoomitdate=1
\pdftrailerid{}

\newif\ifDisplaystyle
\everymath\expandafter{\the\everymath\ifDisplaystyle\displaystyle\fi}
\newcommand\DS{\displaystyle}

\clearscrheadfoot
\pagestyle{scrheadings}
\thispagestyle{empty}
\ohead{\Classe}
\ihead{\thepage/\pageref*{LastPage}}

\setlist[itemize,1]{label=\textbullet}
\setlist[itemize,2]{label=\textbullet}

\ifTwoColumns
  \geometry{margin=1cm,top=2cm,bottom=3cm,foot=1cm}
  \setlist[enumerate]{leftmargin=*}
  \setlist[itemize]{leftmargin=*}
\else
  \geometry{margin=3cm}
\fi

\makeatletter
\let\@author=\relax
\let\@date=\relax
\renewcommand\maketitle{%
    \begin{center}%
        {\sffamily\huge\bfseries\@title}%
        \ifx\@author\relax\else\par\medskip{\itshape\Large\@author}\fi
        \ifx\@date\relax\else\par\bigskip{\large\@date}\fi
    \end{center}\bigskip
    \ifTwoColumns
        \par\begin{multicols*}{2}%
        \AtEndDocument{\end{multicols*}}%
        \setlength{\columnsep}{5mm}
    \fi
}
\makeatother

\newcounter{ParaNum}
\NewDocumentCommand\Para{smo}{%
  \IfBooleanF{#1}{\refstepcounter{ParaNum}}%
  \paragraph{\IfBooleanF{#1}{{\tiny\arabic{ParaNum}~}}#2\IfNoValueF{#3}{ (#3)}}}

\newcommand\I{i}
\newcommand\mi{i}
\def\me{e}

\def\do#1{\expandafter\undef\csname #1\endcsname}
\docsvlist{Ker,sec,csc,cot,sinh,cosh,tanh,coth,th}
\undef\do

\DeclareMathOperator\ch{ch}
\DeclareMathOperator\sh{sh}
\DeclareMathOperator\th{th}
\DeclareMathOperator\coth{coth}
\DeclareMathOperator\cotan{cotan}
\DeclareMathOperator\argch{argch}
\DeclareMathOperator\argsh{argsh}
\DeclareMathOperator\argth{argth}

\let\epsilon=\varepsilon
\let\phi=\varphi
\let\leq=\leqslant
\let\geq=\geqslant
\let\subsetneq=\varsubsetneq
\let\emptyset=\varnothing

\newcommand{\+}{,\;}

\undef\C
\newcommand\ninf{{n\infty}}
\newcommand\N{\mathbb{N}}
\newcommand\Z{\mathbb{Z}}
\newcommand\Q{\mathbb{Q}}
\newcommand\R{\mathbb{R}}
\newcommand\C{\mathbb{C}}
\newcommand\K{\mathbb{K}}
\newcommand\Ns{\N^*}
\newcommand\Zs{\Z^*}
\newcommand\Qs{\Q^*}
\newcommand\Rs{\R^*}
\newcommand\Cs{\C^*}
\newcommand\Ks{\K^*}
\newcommand\Rp{\R^+}
\newcommand\Rps{\R^+_*}
\newcommand\Rms{\R^-_*}
\newcommand{\Rpinf}{\Rp\cup\Acco{+\infty}}

\undef\B
\newcommand\B{\mathscr{B}}

\undef\P
\DeclareMathOperator\P{\mathbb{P}}
\DeclareMathOperator\E{\mathbb{E}}
\DeclareMathOperator\Var{\mathbb{V}}

\DeclareMathOperator*\PetitO{o}
\DeclareMathOperator*\GrandO{O}
\DeclareMathOperator*\Sim{\sim}
\DeclareMathOperator\Tr{tr}
\DeclareMathOperator\Ima{Im}
\DeclareMathOperator\Ker{Ker}
\DeclareMathOperator\Sp{Sp}
\DeclareMathOperator\Diag{diag}
\DeclareMathOperator\Rang{rang}
\DeclareMathOperator*\Coords{Coords}
\DeclareMathOperator*\Mat{Mat}
\DeclareMathOperator\Pass{Pass}
\DeclareMathOperator\Com{Com}
\DeclareMathOperator\Card{Card}
\DeclareMathOperator\Racines{Racines}
\DeclareMathOperator\Vect{Vect}
\DeclareMathOperator\Id{Id}

\newcommand\DerPart[2]{\frac{\partial #1}{\partial #2}}

\def\T#1{{#1}^T}

\def\pa#1{({#1})}
\def\Pa#1{\left({#1}\right)}
\def\bigPa#1{\bigl({#1}\bigr)}
\def\BigPa#1{\Bigl({#1}\Bigr)}
\def\biggPa#1{\biggl({#1}\biggr)}
\def\BiggPa#1{\Biggl({#1}\Biggr)}

\def\pafrac#1#2{\pa{\frac{#1}{#2}}}
\def\Pafrac#1#2{\Pa{\frac{#1}{#2}}}
\def\bigPafrac#1#2{\bigPa{\frac{#1}{#2}}}
\def\BigPafrac#1#2{\BigPa{\frac{#1}{#2}}}
\def\biggPafrac#1#2{\biggPa{\frac{#1}{#2}}}
\def\BiggPafrac#1#2{\BiggPa{\frac{#1}{#2}}}

\def\cro#1{[{#1}]}
\def\Cro#1{\left[{#1}\right]}
\def\bigCro#1{\bigl[{#1}\bigr]}
\def\BigCro#1{\Bigl[{#1}\Bigr]}
\def\biggCro#1{\biggl[{#1}\biggr]}
\def\BiggCro#1{\Biggl[{#1}\Biggr]}

\def\abs#1{\mathopen|{#1}\mathclose|}
\def\Abs#1{\left|{#1}\right|}
\def\bigAbs#1{\bigl|{#1}\bigr|}
\def\BigAbs#1{\Bigl|{#1}\Bigr|}
\def\biggAbs#1{\biggl|{#1}\biggr|}
\def\BiggAbs#1{\Biggl|{#1}\Biggr|}

\def\acco#1{\{{#1}\}}
\def\Acco#1{\left\{{#1}\right\}}
\def\bigAcco#1{\bigl\{{#1}\bigr\}}
\def\BigAcco#1{\Bigl\{{#1}\Bigr\}}
\def\biggAcco#1{\biggl\{{#1}\biggr\}}
\def\BiggAcco#1{\Biggl\{{#1}\Biggr\}}

\def\ccro#1{\llbracket{#1}\rrbracket}
\def\Dcro#1{\llbracket{#1}\rrbracket}

\def\floor#1{\lfloor#1\rfloor}
\def\Floor#1{\left\lfloor{#1}\right\rfloor}

\def\sEnsemble#1#2{\mathopen\{#1\mid#2\mathclose\}}
\def\bigEnsemble#1#2{\bigl\{#1\bigm|#2\bigr\}}
\def\BigEnsemble#1#2{\Bigl\{#1\Bigm|#2\Bigr\}}
\def\biggEnsemble#1#2{\biggl\{#1\biggm|#2\biggr\}}
\def\BiggEnsemble#1#2{\Biggl\{#1\Biggm|#2\Biggr\}}
\let\Ensemble=\bigEnsemble

\newcommand\IntO[1]{\left]#1\right[}
\newcommand\IntF[1]{\left[#1\right]}
\newcommand\IntOF[1]{\left]#1\right]}
\newcommand\IntFO[1]{\left[#1\right[}

\newcommand\intO[1]{\mathopen]#1\mathclose[}
\newcommand\intF[1]{\mathopen[#1\mathclose]}
\newcommand\intOF[1]{\mathopen]#1\mathclose]}
\newcommand\intFO[1]{\mathopen[#1\mathclose[}

\newcommand\Fn[3]{#1\colon#2\to#3}
\newcommand\CC[1]{\mathscr{C}^{#1}}
\newcommand\D{\mathop{}\!\mathrm{d}}

\newcommand\longto{\longrightarrow}

\undef\M
\newcommand\M[3]{\mathrm{#1}_{#2}\pa{#3}}
\newcommand\MnR{\M{M}{n}{\R}}
\newcommand\MnC{\M{M}{n}{\C}}
\newcommand\MnK{\M{M}{n}{\K}}
\newcommand\GLnR{\M{GL}{n}{\R}}
\newcommand\GLnC{\M{GL}{n}{\C}}
\newcommand\GLnK{\M{GL}{n}{\K}}
\newcommand\DnR{\M{D}{n}{\R}}
\newcommand\DnC{\M{D}{n}{\C}}
\newcommand\DnK{\M{D}{n}{\K}}
\newcommand\SnR{\M{S}{n}{\R}}
\newcommand\AnR{\M{A}{n}{\R}}
\newcommand\OnR{\M{O}{n}{\R}}
\newcommand\SnRp{\mathrm{S}_n^+(\R)}
\newcommand\SnRpp{\mathrm{S}_n^{++}(\R)}

\newcommand\LE{\mathscr{L}(E)}
\newcommand\GLE{\mathscr{GL}(E)}
\newcommand\SE{\mathscr{S}(E)}
\renewcommand\OE{\mathscr{O}(E)}

\newcommand\ImplD{$\Cro\Rightarrow$}
\newcommand\ImplR{$\Cro\Leftarrow$}
\newcommand\InclD{$\Cro\subset$}
\newcommand\InclR{$\Cro\supset$}
\newcommand\notInclD{$\Cro{\not\subset}$}
\newcommand\notInclR{$\Cro{\not\supset}$}

\newcommand\To[1]{\xrightarrow[#1]{}}
\newcommand\Toninf{\To{\ninf}}

\newcommand\Norm[1]{\|#1\|}
\newcommand\Norme{{\Norm{\cdot}}}

\newcommand\Int[1]{\mathring{#1}}
\newcommand\Adh[1]{\overline{#1}}

\newcommand\Uplet[2]{{#1},\dots,{#2}}
\newcommand\nUplet[3]{(\Uplet{{#1}_{#2}}{{#1}_{#3}})}

\newcommand\Fonction[5]{{#1}\left|\begin{aligned}{#2}&\;\longto\;{#3}\\{#4}&\;\longmapsto\;{#5}\end{aligned}\right.}

\DeclareMathOperator\orth{\bot}
\newcommand\Orth[1]{{#1}^\bot}
\newcommand\PS[2]{\langle#1,#2\rangle}

\newcommand{\Tribu}{\mathscr{T}}
\newcommand{\Part}{\mathcal{P}}
\newcommand{\Pro}{\bigPa{\Omega,\Tribu}}
\newcommand{\Prob}{\bigPa{\Omega,\Tribu,\P}}

\newcommand\DEMO{$\spadesuit$}
\newcommand\DUR{$\spadesuit$}

\newenvironment{psmallmatrix}{\left(\begin{smallmatrix}}{\end{smallmatrix}\right)}


% -----------------------------------------------------------------------------


\begin{document}
\title{Les in\'egalit\'es}
\maketitle

Il s'agit d'un chapitre de r\'evisions.

\section{Relation d'ordre}

\Para{Propri\'et\'es}
Pour tous r\'eels $a,b,c,d$, on a
\begin{itemize}
\item
  $a \leq{} a$
\item
  si $a \leq{} b$ et $b \leq{} c$,
  alors $a \leq{} c$
\item
  si $a \leq{} b$ et $b \leq{} a$,
  alors $a = b$
\item
  $a \leq{} b$ ou $b \leq{} a$
\item
  si $a \leq{} b$ et $c \leq{} d$,
  alors $a+c \leq{} b+d$
\item
  si $0 \leq{} a \leq{} b$ et $0 \leq{} c \leq{} d$,
  alors $ac \leq{} bd$
\end{itemize}

% -----------------------------------------------------------------------------
\par\pagebreak[1]\par
\paragraph{Exercice 1}%
\hfill{\tiny 6062}%
\begingroup~

On ne divise pas des in\'egalit\'es!

Que peut-on dire de $\frac{x}{y}$ sachant que $2 \leq{} x < 3$ et $\alpha{} \leq{} y \leq{} 2$?
\endgroup

% -----------------------------------------------------------------------------
\par\pagebreak[1]\par
\paragraph{Exercice 2}%
\hfill{\tiny 2672}%
\begingroup~

\begin{enumerate}
\item
  Montrer que pour tous r\'eels $a, b$, on a
  \[ a b \leq{} \frac12 (a^2+b^2) \]
\item
  Montrer que pour tous r\'eels $a,b,c$, on a
  \[ ab + bc + c a \leq{} a^2 + b^2 + c^2 \]
\end{enumerate}
\endgroup

\section{Variations}

\Para{D\'efinition}

Une fonction $\Fn{f}{I}{\R}$ est \emph{croissante} si et seulement si
\[ \forall{} (x,y) \in{} I^2 \+ x \leq{} y \implies f(x) \leq{} f(y). \]

% -----------------------------------------------------------------------------
\par\pagebreak[1]\par
\paragraph{Exercice 3}%
\hfill{\tiny 6078}%
\begingroup~

La fonction $x \mapsto 1/x$ est-elle d\'ecroissante sur $\Rs$?
\endgroup

% -----------------------------------------------------------------------------
\par\pagebreak[1]\par
\paragraph{Exercice 4 (in\'egalit\'es classiques)}%
\hfill{\tiny 1555}%
\begingroup~

Montrer que
\begin{enumerate}
\item
  $x - \frac{x^3}{6} \leq{} \sin x \leq{} x$ si $x > 0$
\item
  $e^x \geq{} x + 1$ pour tout $x$
\item
  $\ln(x) \leq{} x - 1$ si $x > 0$
\end{enumerate}
\endgroup

% -----------------------------------------------------------------------------
\par\pagebreak[1]\par
\paragraph{Exercice 5}%
\hfill{\tiny 6791}%
\begingroup~

Si une fonction r\'eelle $f$ tend vers $+\infty$ en $+\infty$,
est-elle n\'ecessairement croissante au voisinage de $+\infty$?
\endgroup

\section{Valeur absolue}

\Para{Propri\'et\'es}
Soit $a$ et $b$ deux r\'eels.
On pose $c = (a+b)/2$ et $r = (b-a)/2$.
Alors pour tout r\'eel $x$, on a l'\'equivalence
\[ a \leq{} x \leq{} b \iff \abs{x-c} \leq{} r. \]

En particulier
\[ -b \leq{} x \leq{} b \iff \abs{x} \leq{} b. \]

\Para{Corollaire}
Une suite r\'eelle $(u_n)_{n\in \N}$ tend vers 0 si et seulement si $\abs{u_n} \to 0$.

\section{Int\'egrales}

\Para{Propri\'et\'e}
Soit $a$ et $b$ deux r\'eels tels que $a \leq{} b$.
Soit $f$ et $g$ deux fonctions continues (par morceaux) de $[a,b]$ dans $\R$.
Si $f \leq{} g$ (c.-\`a-d. si $\forall{} x \in{} [a,b]$, $f(x) \leq{} g(x)$),
alors \[ \int_a^b f(x) \D x \leq{} \int_a^b g(x) \D x. \]

% -----------------------------------------------------------------------------
\par\pagebreak[1]\par
\paragraph{Exercice 6}%
\hfill{\tiny 7258}%
\begingroup~

Si $f$ et $g$ sont d\'erivables et si $f(x) \leq{} g(x)$ pour tout $x \in{} \intF{a,b}$,
a-t-on n\'ecessairement $f'(x) \leq{} g'(x)$?
\endgroup

% -----------------------------------------------------------------------------
\par\pagebreak[1]\par
\paragraph{Exercice 7}%
\hfill{\tiny 2257}%
\begingroup~

Calculer directement, sans utiliser le th\'eor\`eme fondamental, c.-\`a-d. sans faire intervenir de primitive, l'int\'egrale
$\int_0^1 x^3 \D x$.
\endgroup

% -----------------------------------------------------------------------------
\par\pagebreak[1]\par
\paragraph{Exercice 8}%
\hfill{\tiny 8117}%
\begingroup~

On pose $\DS u_n = \int_0^{\pi/4} \tan^n (x) \D x$.
\begin{enumerate}
\item
  En d\'ecoupant judicieusement, montrer que $u_n \to 0$.
\item
  Retrouver le r\'esultat avec un changement de variables.
\item Nettement plus d\'elicat: on va chercher un \'equivalent de $u_n$.
  \begin{enumerate}
  \item
    Soit $v_n = 2n u_n$.
    Montrer que \[ v_n = \int_0^1 \frac{2 x^{1/n} \D x}{1+x^{2/n}} \]
  \item
    Montrer que $v_n \geq{} \frac{n}{n+1}$.
  \item
    Soit $\alpha{} \in{} \intO{0,1}$.
    Montrer que \[ v_n \leq{} 2 \alpha{} + \frac{2}{1+\alpha^{2/n}} \]
  \item
    En d\'eduire soigneusement que $v_n \to 1.$
  \item Conclure.
  \end{enumerate}
\end{enumerate}
\endgroup

\section{Continuit\'e}

\Para{D\'efinition}
Soit $\Fn{f}{\R}{\R}$ et $x \in{} \R$.
On dit que $f$ est continue en $x$ si et seulement si
\begin{multline*}
  \forall{} \epsilon{} > 0 \+ \exists{} \eta{} > 0 \+ \forall{} y \in{} \R{} \+ \\
  \abs{x-y} \leq{} \eta{} \implies \abs{f(x)-f(y)} \leq{} \epsilon.
\end{multline*}

\Para{Remarque}
Les deux premi\`eres in\'egalit\'es doivent \^etre strictes,
et les deux derni\`eres peuvent \^etre large ou strictes,
on obtient des d\'efinitions \'equivalentes.

\section{Partie enti\`ere}

\Para{D\'efinition 1}
Pour $x$ r\'eel, on pose
\[ \floor{x} = \max \Ensemble{n\in \Z}{n \leq{} x}. \]
Ce maximum existe car on affaire \`a une partie de $\Z$ non vide major\'ee.

\Para{D\'efinition 2}
Pour $x$ r\'eel, il existe un unique entier $n \in{} \Z$ tel que
\[ n \leq{} x < n + 1. \]
On note $\floor{x} = n$.

% -----------------------------------------------------------------------------
\par\pagebreak[1]\par
\paragraph{Exercice 9}%
\hfill{\tiny 1402}%
\begingroup~

Montrer que pour tous r\'eels $a, b$ on a
\[ \floor{a} + \floor{b} \leq{} \floor{a+b} \leq{} \floor{a}+\floor{b}+1 \]
Ces in\'egalit\'es sont-elles strictes?
\endgroup

% -----------------------------------------------------------------------------
\par\pagebreak[1]\par
\paragraph{Exercice 10}%
\hfill{\tiny 7270}%
\begingroup~

Montrer que pour tout r\'eel $x$ et pour tout entier naturel $n \geq{} 1$, on a
\[ \Floor{\frac{\floor{nx}}{n}} = \floor{x} \]
\endgroup

% -----------------------------------------------------------------------------
\par\pagebreak[1]\par
\paragraph{Exercice 11}%
\hfill{\tiny 7696}%
\begingroup~

Calculer (\`a la main)
\[ \left\lfloor \sum_{k=1}^{10^9} k^{-2/3} \right\rfloor \]
\endgroup

\section{Cauchy-Schwarz}

\Para{Proposition}

Soit $n$ un entier naturel et $a_1, \dots, a_n, b_1, \dots, b_n$ des r\'eels.
Alors
\[ \Abs{ \sum_{i=1}^n a_i b_i } \leq{} \sqrt{ \sum_{i=1}^n a_i^2 } \sqrt{ \sum_{i=1}^n b_i^2 } \]

De plus, cette in\'egalit\'e est une \'egalit\'e si et seulement si les vecteurs $\nUplet a1n$ et $\nUplet b1n$ sont colin\'eaires dans $\R^n$.

% -----------------------------------------------------------------------------
\par\pagebreak[1]\par
\paragraph{Exercice 12}%
\hfill{\tiny 5506}%
\begingroup~

Soit $x,y,z$ des r\'eels tels que $x^2+y^2+z^2 = 1$.
Quelle est la valeur maximale de $x+2y+3z$?
\endgroup

\end{document}
