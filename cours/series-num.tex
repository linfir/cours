% autogenerated by ytex.rs

\documentclass{scrartcl}

\usepackage[francais]{babel}
\usepackage{geometry}
\usepackage{scrpage2}
\usepackage{lastpage}
\usepackage{ragged2e}
\usepackage{multicol}
\usepackage{etoolbox}
\usepackage{xparse}
\usepackage{enumitem}
% \usepackage{csquotes}
\usepackage{amsmath}
\usepackage{amsfonts}
\usepackage{amssymb}
\usepackage{mathrsfs}
\usepackage{stmaryrd}
\usepackage{dsfont}
\usepackage{eurosym}
% \usepackage{numprint}
\usepackage[most]{tcolorbox}
% \usepackage{tikz}
% \usepackage{tkz-tab}
\usepackage[unicode]{hyperref}
\usepackage[ocgcolorlinks]{ocgx2}

\let\ifTwoColumns\iftrue
\def\Classe{$\Psi$2019--2020}

% reproducible builds
% LuaTeX: \pdfvariable suppressoptionalinfo 1023 \relax
\pdfinfoomitdate=1
\pdftrailerid{}

\newif\ifDisplaystyle
\everymath\expandafter{\the\everymath\ifDisplaystyle\displaystyle\fi}
\newcommand\DS{\displaystyle}

\clearscrheadfoot
\pagestyle{scrheadings}
\thispagestyle{empty}
\ohead{\Classe}
\ihead{\thepage/\pageref*{LastPage}}

\setlist[itemize,1]{label=\textbullet}
\setlist[itemize,2]{label=\textbullet}

\ifTwoColumns
  \geometry{margin=1cm,top=2cm,bottom=3cm,foot=1cm}
  \setlist[enumerate]{leftmargin=*}
  \setlist[itemize]{leftmargin=*}
\else
  \geometry{margin=3cm}
\fi

\makeatletter
\let\@author=\relax
\let\@date=\relax
\renewcommand\maketitle{%
    \begin{center}%
        {\sffamily\huge\bfseries\@title}%
        \ifx\@author\relax\else\par\medskip{\itshape\Large\@author}\fi
        \ifx\@date\relax\else\par\bigskip{\large\@date}\fi
    \end{center}\bigskip
    \ifTwoColumns
        \par\begin{multicols*}{2}%
        \AtEndDocument{\end{multicols*}}%
        \setlength{\columnsep}{5mm}
    \fi
}
\makeatother

\newcounter{ParaNum}
\NewDocumentCommand\Para{smo}{%
  \IfBooleanF{#1}{\refstepcounter{ParaNum}}%
  \paragraph{\IfBooleanF{#1}{{\tiny\arabic{ParaNum}~}}#2\IfNoValueF{#3}{ (#3)}}}

\newcommand\I{i}
\newcommand\mi{i}
\def\me{e}

\def\do#1{\expandafter\undef\csname #1\endcsname}
\docsvlist{Ker,sec,csc,cot,sinh,cosh,tanh,coth,th}
\undef\do

\DeclareMathOperator\ch{ch}
\DeclareMathOperator\sh{sh}
\DeclareMathOperator\th{th}
\DeclareMathOperator\coth{coth}
\DeclareMathOperator\cotan{cotan}
\DeclareMathOperator\argch{argch}
\DeclareMathOperator\argsh{argsh}
\DeclareMathOperator\argth{argth}

\let\epsilon=\varepsilon
\let\phi=\varphi
\let\leq=\leqslant
\let\geq=\geqslant
\let\subsetneq=\varsubsetneq
\let\emptyset=\varnothing

\newcommand{\+}{,\;}

\undef\C
\newcommand\ninf{{n\infty}}
\newcommand\N{\mathbb{N}}
\newcommand\Z{\mathbb{Z}}
\newcommand\Q{\mathbb{Q}}
\newcommand\R{\mathbb{R}}
\newcommand\C{\mathbb{C}}
\newcommand\K{\mathbb{K}}
\newcommand\Ns{\N^*}
\newcommand\Zs{\Z^*}
\newcommand\Qs{\Q^*}
\newcommand\Rs{\R^*}
\newcommand\Cs{\C^*}
\newcommand\Ks{\K^*}
\newcommand\Rp{\R^+}
\newcommand\Rps{\R^+_*}
\newcommand\Rms{\R^-_*}
\newcommand{\Rpinf}{\Rp\cup\Acco{+\infty}}

\undef\B
\newcommand\B{\mathscr{B}}

\undef\P
\DeclareMathOperator\P{\mathbb{P}}
\DeclareMathOperator\E{\mathbb{E}}
\DeclareMathOperator\Var{\mathbb{V}}

\DeclareMathOperator*\PetitO{o}
\DeclareMathOperator*\GrandO{O}
\DeclareMathOperator*\Sim{\sim}
\DeclareMathOperator\Tr{tr}
\DeclareMathOperator\Ima{Im}
\DeclareMathOperator\Ker{Ker}
\DeclareMathOperator\Sp{Sp}
\DeclareMathOperator\Diag{diag}
\DeclareMathOperator\Rang{rang}
\DeclareMathOperator*\Coords{Coords}
\DeclareMathOperator*\Mat{Mat}
\DeclareMathOperator\Pass{Pass}
\DeclareMathOperator\Com{Com}
\DeclareMathOperator\Card{Card}
\DeclareMathOperator\Racines{Racines}
\DeclareMathOperator\Vect{Vect}
\DeclareMathOperator\Id{Id}

\newcommand\DerPart[2]{\frac{\partial #1}{\partial #2}}

\def\T#1{{#1}^T}

\def\pa#1{({#1})}
\def\Pa#1{\left({#1}\right)}
\def\bigPa#1{\bigl({#1}\bigr)}
\def\BigPa#1{\Bigl({#1}\Bigr)}
\def\biggPa#1{\biggl({#1}\biggr)}
\def\BiggPa#1{\Biggl({#1}\Biggr)}

\def\pafrac#1#2{\pa{\frac{#1}{#2}}}
\def\Pafrac#1#2{\Pa{\frac{#1}{#2}}}
\def\bigPafrac#1#2{\bigPa{\frac{#1}{#2}}}
\def\BigPafrac#1#2{\BigPa{\frac{#1}{#2}}}
\def\biggPafrac#1#2{\biggPa{\frac{#1}{#2}}}
\def\BiggPafrac#1#2{\BiggPa{\frac{#1}{#2}}}

\def\cro#1{[{#1}]}
\def\Cro#1{\left[{#1}\right]}
\def\bigCro#1{\bigl[{#1}\bigr]}
\def\BigCro#1{\Bigl[{#1}\Bigr]}
\def\biggCro#1{\biggl[{#1}\biggr]}
\def\BiggCro#1{\Biggl[{#1}\Biggr]}

\def\abs#1{\mathopen|{#1}\mathclose|}
\def\Abs#1{\left|{#1}\right|}
\def\bigAbs#1{\bigl|{#1}\bigr|}
\def\BigAbs#1{\Bigl|{#1}\Bigr|}
\def\biggAbs#1{\biggl|{#1}\biggr|}
\def\BiggAbs#1{\Biggl|{#1}\Biggr|}

\def\acco#1{\{{#1}\}}
\def\Acco#1{\left\{{#1}\right\}}
\def\bigAcco#1{\bigl\{{#1}\bigr\}}
\def\BigAcco#1{\Bigl\{{#1}\Bigr\}}
\def\biggAcco#1{\biggl\{{#1}\biggr\}}
\def\BiggAcco#1{\Biggl\{{#1}\Biggr\}}

\def\ccro#1{\llbracket{#1}\rrbracket}
\def\Dcro#1{\llbracket{#1}\rrbracket}

\def\floor#1{\lfloor#1\rfloor}
\def\Floor#1{\left\lfloor{#1}\right\rfloor}

\def\sEnsemble#1#2{\mathopen\{#1\mid#2\mathclose\}}
\def\bigEnsemble#1#2{\bigl\{#1\bigm|#2\bigr\}}
\def\BigEnsemble#1#2{\Bigl\{#1\Bigm|#2\Bigr\}}
\def\biggEnsemble#1#2{\biggl\{#1\biggm|#2\biggr\}}
\def\BiggEnsemble#1#2{\Biggl\{#1\Biggm|#2\Biggr\}}
\let\Ensemble=\bigEnsemble

\newcommand\IntO[1]{\left]#1\right[}
\newcommand\IntF[1]{\left[#1\right]}
\newcommand\IntOF[1]{\left]#1\right]}
\newcommand\IntFO[1]{\left[#1\right[}

\newcommand\intO[1]{\mathopen]#1\mathclose[}
\newcommand\intF[1]{\mathopen[#1\mathclose]}
\newcommand\intOF[1]{\mathopen]#1\mathclose]}
\newcommand\intFO[1]{\mathopen[#1\mathclose[}

\newcommand\Fn[3]{#1\colon#2\to#3}
\newcommand\CC[1]{\mathscr{C}^{#1}}
\newcommand\D{\mathop{}\!\mathrm{d}}

\newcommand\longto{\longrightarrow}

\undef\M
\newcommand\M[3]{\mathrm{#1}_{#2}\pa{#3}}
\newcommand\MnR{\M{M}{n}{\R}}
\newcommand\MnC{\M{M}{n}{\C}}
\newcommand\MnK{\M{M}{n}{\K}}
\newcommand\GLnR{\M{GL}{n}{\R}}
\newcommand\GLnC{\M{GL}{n}{\C}}
\newcommand\GLnK{\M{GL}{n}{\K}}
\newcommand\DnR{\M{D}{n}{\R}}
\newcommand\DnC{\M{D}{n}{\C}}
\newcommand\DnK{\M{D}{n}{\K}}
\newcommand\SnR{\M{S}{n}{\R}}
\newcommand\AnR{\M{A}{n}{\R}}
\newcommand\OnR{\M{O}{n}{\R}}
\newcommand\SnRp{\mathrm{S}_n^+(\R)}
\newcommand\SnRpp{\mathrm{S}_n^{++}(\R)}

\newcommand\LE{\mathscr{L}(E)}
\newcommand\GLE{\mathscr{GL}(E)}
\newcommand\SE{\mathscr{S}(E)}
\renewcommand\OE{\mathscr{O}(E)}

\newcommand\ImplD{$\Cro\Rightarrow$}
\newcommand\ImplR{$\Cro\Leftarrow$}
\newcommand\InclD{$\Cro\subset$}
\newcommand\InclR{$\Cro\supset$}
\newcommand\notInclD{$\Cro{\not\subset}$}
\newcommand\notInclR{$\Cro{\not\supset}$}

\newcommand\To[1]{\xrightarrow[#1]{}}
\newcommand\Toninf{\To{\ninf}}

\newcommand\Norm[1]{\|#1\|}
\newcommand\Norme{{\Norm{\cdot}}}

\newcommand\Int[1]{\mathring{#1}}
\newcommand\Adh[1]{\overline{#1}}

\newcommand\Uplet[2]{{#1},\dots,{#2}}
\newcommand\nUplet[3]{(\Uplet{{#1}_{#2}}{{#1}_{#3}})}

\newcommand\Fonction[5]{{#1}\left|\begin{aligned}{#2}&\;\longto\;{#3}\\{#4}&\;\longmapsto\;{#5}\end{aligned}\right.}

\DeclareMathOperator\orth{\bot}
\newcommand\Orth[1]{{#1}^\bot}
\newcommand\PS[2]{\langle#1,#2\rangle}

\newcommand{\Tribu}{\mathscr{T}}
\newcommand{\Part}{\mathcal{P}}
\newcommand{\Pro}{\bigPa{\Omega,\Tribu}}
\newcommand{\Prob}{\bigPa{\Omega,\Tribu,\P}}

\newcommand\DEMO{$\spadesuit$}
\newcommand\DUR{$\spadesuit$}

\newenvironment{psmallmatrix}{\left(\begin{smallmatrix}}{\end{smallmatrix}\right)}

% -----------------------------------------------------------------------------

\undef\U
\undef\V
\newcommand\U{(u_n)_{n\in \N}}
\newcommand\V{(v_n)_{n\in \N}}
\newcommand\SU{\sum_n u_n}
\newcommand\SV{\sum_n v_n}

\begin{document}
\title{S\'eries num\'eriques}
\maketitle
\Displaystyletrue

% -----------------------------------------------------------------------------
\section{G\'en\'eralit\'es}

\Para{D\'efinitions}
\begin{itemize}
\item
  Soit $\U$ une suite num\'erique.
  La \emph{s\'erie num\'erique} de \emph{terme g\'en\'eral} $u_n$ est la suite $(S_n)_{n\in \N}$ o\`u $\forall n\in \N$, $S_n = \sum_{k=0}^n u_k$.
  On note $\SU$ au lieu de $(S_n)_{n\in \N}$.
\item
  Soit $\SU$ une s\'erie num\'erique.
  Pour $n\in \N$, on pose $S_n = \sum_{k=0}^n u_k$.
  $S_n$ s'appelle la \emph{somme partielle} d'ordre $n$ de la s\'erie $\SU$.
\end{itemize}

\Para{Remarque}

Techniquement, une s\'erie est donc \'egale \`a la suite de ses sommes partielles; toutefois, en pratique, on fera la distinction.

\Para{D\'efinition}

Soit $\SU$ une s\'erie num\'erique et $(S_n)_{n\in \N}$ ses sommes partielles.

On dit que la s\'erie $\SU$ est convergente (respectivement divergente, major\'ee, minor\'ee, born\'ee, croissante ou d\'ecroissante)
si et seulement si la suite $(S_n)_{n\in \N}$ l'est.

\Para{D\'efinition}

Soit $\SU$ une s\'erie num\'erique \emph{convergente}.

On appelle \emph{somme de la s\'erie} la limite $S$ de la suite de ses sommes partielles $(S_n)_{n\in \N}$, et on la note $\sum_{n=0}^{+\infty} u_n$.
Ainsi, $S = \sum_{n=0}^{+\infty} u_n = \lim_\ninf S_n = \lim_\ninf \sum_{k=0}^n u_k$

\Para{D\'efinition}

Soit $\SU$ une s\'erie num\'erique \emph{convergente}.
Notons $(S_n)_{n\in \N}$ ses sommes partielles et $S$ sa somme.
On appelle \emph{reste} d'ordre $n$ de la s\'erie la quantit\'e $R_n = S - S_n = \sum_{k=0}^{+\infty} u_k -\sum_{k=0}^n u_k = \sum_{k=n+1}^{+\infty} u_k = \sum_{k>n} u_k$.

\Para{Proposition}

Soit $\SU$ une s\'erie num\'erique convergente.
Notons $S$ sa somme, $(S_n)_{n\in \N}$ ses sommes partielles et $(R_n)_{n\in \N}$ ses restes.
Alors on a $\forall n\in \N, S_n + R_n = S$.
De plus, $R_n \Toninf 0$.

\Para{Exemples}
\begin{itemize}
\item
  \emph{S\'eries g\'eom\'etriques:}

  Soit $(u_n)_{n\in \N}$ une suite g\'eom\'etrique non nulle de raison $q\in \C$.
  Alors la s\'erie $\SU$ converge si et seulement si $\Abs{q}<1$.
\item
  \emph{S\'eries t\'elescopiques:}

  Soit $(a_n)_{n\in \N}$ une suite num\'erique et $u_n = a_{n+1} - a_n$.
  Alors la s\'erie $\SU$ converge si et seulement si la suite $(a_n)_{n\in \N}$ converge.
  On a alors $\sum_{n=0}^{+\infty} u_n = \lim_\ninf a_n - a_0$.
\end{itemize}

\Para{Proposition}

Soit $\SU$ une s\'erie num\'erique.

Si la s\'erie converge, alors $u_n \Toninf 0$.

\Para{Attention}

La r\'eciproque est \emph{FAUSSE}.
Le contre-exemple classique est la \emph{s\'erie harmonique}
$\sum_{n\geq1} \frac1n$ qui est \emph{divergente},
bien que $\frac1n \Toninf 0$.

\Para{D\'efinition}

On dit que la s\'erie num\'erique $\SU$ est \emph{grossi\`erement divergente} si et seulement si $u_n$ ne tend pas vers 0.

\subsection{Absolue convergence}

\Para{D\'efinition}

Soit $\SU$ une s\'erie num\'erique.
On dit que la s\'erie est \emph{absolument convergente} si et seulement si la s\'erie $\sum_n \Abs{u_n}$ converge.

\Para{Th\'eor\`eme}

Toute s\'erie absolument convergente est convergente.

\emph{Autrement dit:}
Soit $\U$ une suite num\'erique.
Si $\sum_n \Abs{u_n}$ converge, alors $\SU$ converge.

\Para{Remarque}

L'utilit\'e pratique de ce th\'eor\`eme r\'eside dans le fait qu'il est g\'en\'eralement plus simple d'\'etudier la s\'erie \`a termes positifs $\sum_n \Abs{u_n}$ que la s\'erie num\'erique g\'en\'erale $\SU$.

% -----------------------------------------------------------------------------
\section{S\'eries \`a termes positifs (SATP)}

\Para{D\'efinition}

Une \emph{s\'erie \`a termes positifs} est une s\'erie num\'erique $\SU$ telle que $\forall n\in \N, u_n\in\Rp$.

\Para{Proposition}

Une SATP converge si et seulement si la suite des ses sommes partielles est major\'ee.
De plus, si elle converge, sa somme est \'egale \`a la borne sup\'erieure
des sommes partielles.

\Para{Lemme}

Soit $\U$ et $\V$ deux suites r\'eelles.

On suppose que $\forall n\in \N, 0\leq u_n\leq v_n$.
\begin{itemize}
\item
  Si $\SV$ converge, alors $\SU$ converge \'egalement.
\item
  Si $\SU$ diverge, alors $\SV$ diverge \'egalement.
\end{itemize}

\Para{Proposition}

Soit $\SU$ et $\SV$ deux SATP.
On suppose que $u_n = \GrandO_\ninf(v_n)$.
Alors:
\begin{itemize}
\item
  Si $\SV$ converge, alors $\SU$ converge \'egalement.
\item
  Si $\SU$ diverge, alors $\SV$ diverge \'egalement.
\end{itemize}

\Para{Th\'eor\`eme}

Soit $\SU$ et $\SV$ deux SATP telles que \emph{$u_n \Sim_\ninf v_n$}.
Alors $\SU$ et $\SV$ ont \emph{m\^eme nature}.

\Para{Remarque}

On a en fait un r\'esultat plus pr\'ecis (mais hors-programme), cf. exercice~6.

\subsection{Comparaison s\'erie-int\'egrale}

\Para{Lemme}

Soit $I$ est un intervalle de $\R$
et $\Fn{f}{I}{\R}$ continue par morceaux \emph{d\'ecroissante}.
Alors si $[n-1,n+1]\subset I$, on a:
\[ f(n+1) \leq{} \int_n^{n+1} f(x) \D x \leq{} f(n), \]
\[ \int_n^{n+1} f(x) \D x \leq{} f(n) \leq{} \int_{n-1}^n f(x) \D x. \]

\Para{Lemme}

Soit $\Fn{f}{\Rp}{\R}$ continue par morceaux, \emph{positive} et \emph{d\'ecroissante}.
Alors la s\'erie de terme g\'en\'eral $u_n =\int_{n-1}^n f(x) \D x - f(n)$
est convergente.

\Para{Th\'eor\`eme}

Soit $n_0\in \N$ et $\Fn{f}{\intFO{n_0, +\infty}}{\R}$ continue par morceaux, \emph{positive} et \emph{d\'ecroissante}.
Alors la s\'erie $\sum_{n} f(n)$ converge si et seulement si la suite $\left( \int_{n_0}^n f(x) \D x \right)_n$ converge.

\Para{Corollaire}[pour les $5/2$]

La s\'erie $\sum_{n} f(n)$ converge si et seulement si l'int\'egrale $\int_{n_0}^{+\infty} f(x) \D x$ converge.

\Para{Proposition}[s\'eries de Riemann]

Soit $\alpha\in \R$.
La s\'erie $\sum_n \frac{1}{n^\alpha}$ converge si et seulement si $\alpha> 1$.

% -----------------------------------------------------------------------------
\section{S\'eries num\'eriques g\'en\'erales}

\Para{D\'efinition}

Une s\'erie \emph{semi-convergente} est une s\'erie convergente mais non absolument convergente.
Ainsi, une s\'erie num\'erique est:
\begin{itemize}
\item
  soit absolument convergente,
\item
  soit semi-convergente,
\item
  soit divergente.
\end{itemize}

\Para{D\'efinition}

Soit $\SU$ une s\'erie r\'eelle.
On dit que cette s\'erie est \emph{altern\'ee} si et seulement si $(-1)^n u_n$ est de signe constant.

\Para{Th\'eor\`eme}[crit\`ere sp\'ecial]

On consid\`ere la s\'erie r\'eelle $\SU$ o\`u:
\begin{itemize}
\item
  $u_n = (-1)^n a_n$
\item
  $(a_n)_{n\in \N}$ d\'ecroissante
\item
  $a_n \Toninf 0$
\end{itemize}

Alors la s\'erie $\SU$ converge.

De plus, on a la majoration suivante, valable pour tout $n\in \N\cup\Acco{-1}$:
\[ \Abs{R_n} = \left| \sum_{k=n+1}^{+\infty} u_k \right| \leq\Abs{u_{n+1}} \]
et $R_n$ est du signe de $u_{n+1}$.

% -----------------------------------------------------------------------------
\section{R\`egles pratiques}

\Para{Proposition}

Soit $\SU$ et $\SV$ deux s\'eries num\'eriques. On suppose que:
\begin{itemize}
\item
  $\SV$ converge \emph{absolument},
\item
  $u_n = \GrandO_\ninf(v_n)$, ou a fortiori $u_n = \PetitO_\ninf(v_n)$.
\end{itemize}

Alors $\SU$ converge absolument.

\Para{Proposition}

Soit $\SU$ et $\SV$ deux s\'eries r\'eelles.
On suppose que:
\begin{itemize}
\item
  $u_n \Sim_\ninf v_n$,
\item
  \`a partir d'un certain rang $v_n$ est de \emph{signe constant}.
\end{itemize}

Alors $\SU$ et $\SV$ ont m\^eme nature.

\Para{Proposition}[r\`egle $n^\alpha u_n$]

Soit $\SU$ une s\'erie num\'erique.
On suppose qu'il existe \emph{$\alpha>1$} tel que $n^\alpha u_n \Toninf 0$.

Alors la s\'erie $\SU$ est absolument convergente.

\Para{Proposition}[r\`egle de d'Alembert]

Soit $\SU$ une s\'erie num\'erique.
On suppose que:
\begin{itemize}
\item
  $\exists N\in \N,\forall n\geq N, u_n\neq0$
\item
  $\left|\frac{u_{n+1}}{u_n}\right| \Toninf \lambda\in\Rp \cup\Acco{+\infty}$
\end{itemize}

Alors:
\begin{itemize}
\item
  si $\lambda>1$, la s\'erie est grossi\`erement divergente.
\item
  si $\lambda<1$, la s\'erie est absolument convergente.
\item
  si $\lambda=1$, on ne peut pas conclure.
\end{itemize}

\Para{Remarque}

La r\`egle de Raabe-Duhamel (hors-programme) permet de pr\'eciser le cas $\lambda=1$, cf. exercice~13.

% -----------------------------------------------------------------------------
\section{Op\'erations}

\Para{Proposition}

L'ensemble des s\'eries num\'eriques est un espace vectoriel pour les lois usuelles.
L'ensemble des s\'eries num\'eriques convergentes est un sous-espace vectoriel de cet espace vectoriel.

\Para{D\'efinition}

Soit $\sum_n a_n$ et $\sum_n b_n$ deux s\'eries num\'eriques.

On appelle \emph{produit de Cauchy} de ces deux s\'eries la s\'erie
$\sum_n c_n$ o\`u $\forall n\in \N$,
\[ c_n =\sum_{k=0}^n a_k b_{n-k} =\sum_{l=0}^n a_{n-l} b_l \]

\Para{Th\'eor\`eme}

Soit $\sum_n a_n$ et $\sum_n b_n$ deux s\'eries num\'eriques \emph{absolument convergentes}.

Alors leur produit de Cauchy $\sum_n c_n$ est absolument convergent, et:
\[ \sum_{n=0}^{+\infty} c_n = \left(\sum_{n=0}^{+\infty} a_n \right) \left(\sum_{n=0}^{+\infty} b_n \right) \]

% -----------------------------------------------------------------------------
% \Displaystylefalse
\section{Exercices}

\subsection{S\'eries \`a termes positifs}

% -----------------------------------------------------------------------------
\par\pagebreak[1]\par
\paragraph{Exercice 1}%
\hfill{\tiny 5744}%
\begingroup~

Soit $u_n = 1/n$ si $n$ est un carr\'e parfait (0, 1, 4, 9, 16, etc.) et $u_n = 0$ sinon.
D\'eterminer la nature de la s\'erie de terme g\'en\'eral $u_n$.
\endgroup

% -----------------------------------------------------------------------------
\par\pagebreak[1]\par
\paragraph{Exercice 2 (s\'eries de Bertrand)}%
\hfill{\tiny 4815}%
\begingroup~

Les \emph{s\'eries de Bertrand} sont les s\'eries de la forme
\[ \sum_n \frac{1}{n^\alpha\ln^\beta n} \] o\`u $(\alpha,\beta)\in \R^2$.
Le but de cet exercice est de montrer que ces s\'eries convergent si et seulement si
\[ \alpha>1 \text{ ou } \pa{\alpha=1 \text{ et } \beta>1}. \]

Notons $u_n = \frac{1}{n^\alpha\ln^\beta n}$ pour $n\geq2$.
\begin{enumerate}
\item
  V\'erifier que dans le cas particulier $\beta= 0$, on retrouve une r\`egle connue.
\item
  Si $\alpha> 1$, montrer qu'il existe $\gamma> 1$ tel que $u_n = \PetitO\BigPa{\frac{1}{n^\gamma}}$. Conclure.
\item
  Si $\alpha< 1$, montrer qu'il existe $\gamma< 1$ tel que $\frac{1}{n^\gamma} = \PetitO(u_n)$. Conclure.
\item
  On suppose d\'esormais $\alpha= 1$. Calculer $\int\frac{\D x}{x \ln^\beta x}$, en faisant bien attention de distinguer le cas $\beta= 1$.
\item
  Si $\beta\leq0$, montrer que $u_n\geq\frac1n$. Conclure.
\item
  Si $\beta> 0$, montrer que $\int_n^{n+1} \frac{\D x}{x \ln^\beta x}\leq u_n\leq\int_{n-1}^n \frac{\D x}{x \ln^\beta x}$ pour tout $n\geq3$.
\item
  Conclure en distinguant les cas $0<\beta<1$, $\beta=1$ et $\beta>1$.
\end{enumerate}
\endgroup

% -----------------------------------------------------------------------------
\par\pagebreak[1]\par
\paragraph{Exercice 3}%
\hfill{\tiny 1593}%
\begingroup~

\'Etudier la nature des s\'eries $\sum_n u_n$ dans chacun des cas suivants:
\begin{enumerate}
\item
  $u_n =\sqrt{\frac{n+1}{n}}$
\item
  $u_n = \frac{\ln n}{2^n}$
\item
  $u_n = n^{1/n}-(n+1)^{1/n}$
\item
  $u_n = \frac{n\sqrt n}{2^n+\sqrt n}$
\item
  $u_n = \ln(\frac{1}{\sqrt n}) - \ln\bigl(\sin(\frac{1}{\sqrt n})\bigr)$
\item
  $u_n = \frac{1}{n\sqrt[n]{n}}$
\item
  $u_n = \arccos\left(\frac{n^3+1}{n^3+2}\right)$
\item
  $u_n = \frac{\ln(1 + a^n n^\alpha)}{n^\beta}$ o\`u $a > 0$
\item
  $u_n = n^\alpha e^{\beta\sqrt n}$
\item
  $u_n^{-1} = 1^\alpha+ 2^\alpha+ \cdots + n^\alpha$ o\`u $\alpha> 0$
\item
  $u_n^{-1} = 1 + \sqrt2 + \sqrt[3] 3 + \cdots + \sqrt[n] n$
\item
  $u_n = \cos\left(\arctan(n) + \frac{1}{n^\alpha}\right)$ o\`u $\alpha> 0$
\item
  $u_n = \bigl(\cos \frac1n\bigr)^{n^\alpha}$ o\`u $\alpha> 0$
\item
  $u_n = \frac{(\sqrt n)^{\ln n}}{(\ln n)^{\sqrt n}}$
\item
  $u_n = \ln\left(3\tan^2\bigl(\frac{\pi n^\alpha}{6n^\alpha+1}\bigr)\right)$ o\`u $\alpha> 0$
\item
  $u_n = \sin\bigl(\pi(2+\sqrt3)^n\bigr)$
\end{enumerate}
\endgroup

% -----------------------------------------------------------------------------
\par\pagebreak[1]\par
\paragraph{\href{https://psi.miomio.fr/exo/2228.pdf}{Exercice 4}}%
\hfill{\tiny 2228}%
\begingroup~

\begin{enumerate}
\item
  Soit $\sum{} u_n$ et $\sum{} v_n$ deux s\'eries \`a termes positifs convergentes.
  Peut-on affirmer que la s\'erie de terme g\'en\'eral $\max(u_n, v_n)$ est \'egalement convergente?
\item
  Le r\'esultat reste-il vrai sans l'hypoth\`ese de positivit\'e?
\end{enumerate}
\endgroup

% -----------------------------------------------------------------------------
\par\pagebreak[1]\par
\paragraph{Exercice 5}%
\hfill{\tiny 3359}%
\begingroup~

Soit $\sum{} u_n$ une SATP convergente.
Peut-on affirmer que la s\'erie de terme g\'en\'eral $v_n = \prod_{k=0}^n u_k$ est \'egalement convergente?
\endgroup

% -----------------------------------------------------------------------------
\par\pagebreak[1]\par
\paragraph{Exercice 6 (\'equivalents des restes ou des sommes partielles)}%
\hfill{\tiny 3026}%
\begingroup~

Soit $(u_n)$ et $(v_n)$ deux s\'eries num\'eriques \`a termes positifs.
On suppose que $u_n \sim v_n$ quand $n \to +\infty$.
\begin{enumerate}
\item
  On suppose que $\sum{} u_n$ converge;
  on sait qu'alors $\sum{} v_n$ converge \'egalement.
  Montrer que les restes des deux s\'eries sont \'equivalents, c.-\`a-d.:
  \[ \sum_{k=n+1}^{+\infty} u_k \,\, \Sim_\ninf \,\, \sum_{k=n+1}^{+\infty} v_k \]
\item
  On suppose que $\sum{} u_n$ diverge;
  on sait qu'alors $\sum{} v_n$ diverge \'egalement.
  Montrer que les sommes partielles des deux s\'eries sont \'equivalentes, c.-\`a-d.:
  \[ \sum_{k=0}^n u_k \,\, \Sim_\ninf \,\, \sum_{k=0}^n v_k \]
\end{enumerate}
\endgroup

% -----------------------------------------------------------------------------
\par\pagebreak[1]\par
\paragraph{\href{https://psi.miomio.fr/exo/2957.pdf}{Exercice 7}}%
\hfill{\tiny 2957}%
\begingroup~

D\'emontrer que chacune des s\'eries $\sum_n u_n$ converge,
et calculer leur somme:
\begin{enumerate}
\item
  $u_n =\sqrt{n}+a\sqrt{n+1}+b\sqrt{n+2}$
\item
  $u_n = \arctan\Pa{\frac{1}{n^2+n+1}}$
\item
  $u_n = \ln\left(1-\frac{1}{n^2}\right)$
\item
  $u_n = \ln\left(\cos\frac{\theta}{2^n}\right)$ o\`u $\theta\in\IntO{0,\frac\pi2}$
\item
  $u_n = \frac{(-1)^n}{3^n} \cos^3(3^n\theta)$
\item
  $u_n = \frac{\sqrt{n+1}-2\sqrt{n}+1}{2^{n+1}}$
\item
  $u_n = \frac{4n}{n^4+2n^2+9}$
\item
  $u_n = \frac{4n-3}{n(n^2-4)}$ pour $n\geq3$
\item
  $u_n = \frac{1}{(2n-1)(2n)(2n+1)(2n+2)}$
\item
  $u_n = \frac{2n^3-3n^2+1}{(n+2)!}$
\item
  $u_n = \frac{1}{(n+1)^2(n+2)^2}$
\item
  $u_n =\int_{\frac{1}{n+1}}^{\frac{1}{n}} \frac{e^{\sqrt x}}{\sqrt x} \D x$ pour $n\geq1$
\end{enumerate}
\endgroup

% -----------------------------------------------------------------------------
\par\pagebreak[1]\par
\paragraph{Exercice 8}%
\hfill{\tiny 6845}%
\begingroup~

\begin{enumerate}
\item
  Soit $\alpha{} < 1$.

  D\'eterminer un \'equivalent de $S_n = \sum_{k=1}^n \frac{1}{k^\alpha}$.
\item
  Soit $\alpha{} > 1$.

  D\'eterminer un \'equivalent de $R_n = \sum_{k=n+1}^{+\infty} \frac{1}{k^\alpha}$.
\end{enumerate}
\endgroup

% -----------------------------------------------------------------------------
\par\pagebreak[1]\par
\paragraph{\href{https://psi.miomio.fr/exo/0806.pdf}{Exercice 9} (d\'eveloppement asymptotique de la s\'erie harmonique)}%
\hfill{\tiny 0806}%
\begingroup~

La \emph{s\'erie harmonique} est la s\'erie $\sum_n \frac1n$.
\begin{enumerate}
\item
  Montrer de plusieurs fa\c cons qu'il s'agit d'une s\'erie divergente:
  \begin{itemize}
  \item
    en comparant la s\'erie \`a une int\'egrale
  \item
    en minorant $\sum_{k=2^n}^{2^{n+1} - 1} \frac1k$
  \item
    en remarquant que $\frac1n \sim \ln n - \ln (n-1)$, et que cette expression est t\'elescopique
  \item
    une autre m\'ethode?
  \end{itemize}
\item
  On note $H_n = \sum_{k=1}^n \frac1k$ les sommes partielles de la s\'erie harmonique.
  En reprenant la comparaison s\'erie int\'egrale de la question 1a,
  montrer que $H_n \sim \ln n$.
  \emph{Remarque:} on pourrait aussi utiliser la question 1c.
\item
  Montrer qu'il existe une unique suite $(u_n)_{n\in \N^*}$
  telle que $\forall n\geq1$, $\sum_{k=1}^n u_k = H_n - \ln n$ et la d\'eterminer.
\item
  D\'eterminer un \'equivalent de $u_n$, puis montrer que la s\'erie $\sum u_n$ converge.
\item
  En d\'eduire que la suite $H_n - \ln n$ converge \'egalement.
  On note $\gamma$ sa limite; il s'agit de la \emph{constante d'Euler}, dont une valeur approch\'ee est $\gamma\simeq 0.577$.
  Ainsi \[ \tcboxmath{ H_n = \ln n + \gamma{} + \PetitO\limits_\ninf(1) } \]
\item
  Pour $\alpha> 1$, montrer que $\sum_{k=n+1}^{+\infty} \frac{1}{k^\alpha} \sim \frac{1}{(\alpha-1) n^{\alpha-1}}$.
  On pourra utiliser une comparaison s\'erie-int\'egrale.
\item
  D\'eterminer un \'equivalent du reste $R_n = \sum_{k = n+1}^{+\infty} u_k$ de la s\'erie $\sum u_n$.
\item
  Exprimer $R_n$ en fonction de $H_n$, $\ln n$ et $\gamma$.
  En d\'eduire que $H_n = \ln n + \gamma+ \frac{1}{2n} + \PetitO(\frac{1}{n})$.
\item
  Montrer qu'il existe une unique suite $(v_n)_{n\in \N^*}$ telle que $\forall n\geq1$, $\sum_{k=1}^n v_k = H_n - \ln n - \frac{1}{2n}$ et la d\'eterminer.
\item
  D\'eterminer un \'equivalent de $v_n$ quand $n \to +\infty$.
\item
  En d\'eduire un \'equivalent du reste $T_n = \sum_{k=n+1}^{+\infty} v_k$.
\item
  Montrer que:
  \[ H_n = \ln n +\gamma+ \frac{1}{2n} - \frac{1}{12n^2} + \PetitO_\ninf\BigPa{\frac{1}{n^2}}. \]
\end{enumerate}
\endgroup

% -----------------------------------------------------------------------------
\par\pagebreak[1]\par
\paragraph{Exercice 10}%
\hfill{\tiny 7341}%
\begingroup~

Soit $u_n = \frac{n-a\floor{n/a}}{n(n+1)}$ o\`u $a\in\Ns$.
D\'eterminer la nature et la somme de la s\'erie de terme g\'en\'eral $u_n$.
\endgroup

% -----------------------------------------------------------------------------
\par\pagebreak[1]\par
\paragraph{Exercice 11 (r\`egle de Cauchy)}%
\hfill{\tiny 6879}%
\begingroup~

La \emph{r\`egle de Cauchy} est la suivante:

Soit $\sum u_n$ une s\'erie \`a termes positifs. On suppose que $\sqrt[n]{u_n} \Toninf \ell\in \R\cup\Acco{+\infty}$. Alors:
\begin{itemize}
\item
  si $\ell< 1$, la s\'erie converge;
\item
  si $\ell> 1$, la s\'erie diverge;
\item
  si $\ell= 1$, on ne peut pas conclure.
\end{itemize}

Questions:
\begin{enumerate}
\item
  Si $\ell< 1$, montrer qu'il existe $a\in\intFO{0,1}$ tel que $u_n = \PetitO(a^n)$. Conclure.
\item
  Si $\ell> 1$, montrer qu'il existe $a > 1$ tel que $a^n = \PetitO(u_n)$. Conclure.
\item
  Trouver un exemple de suite $(u_n)$ convergente telle que $\ell= 1$.
  Trouver \'egalement un exemple de suite $(u_n)$ divergente telle que $\ell= 1$.
  Conclure.
\item
  D\'emontrer la r\`egle de Cauchy.
\item
  \emph{Application:} D\'eterminer la nature des s\'eries $\sum u_n$ dans les cas suivants.
  \begin{enumerate}
  \item
    $u_n = \left(\frac{n-1}{2n+1}\right)^n$
  \item
    $u_n = \left({a+\frac1n}\right)^n$
  \end{enumerate}
\item
  Soit $(u_n)$ une suite de r\'eels strictements positifs telle que $\frac{u_{n+1}}{u_n} \Toninf\ell\in \R\cup\Acco{+\infty}$.
  Montrer que $\sqrt[n]{u_n} \Toninf\ell$.
  En d\'eduire que la r\`egle de Cauchy est \emph{plus forte} que la r\`egle de d'Alembert.
\end{enumerate}
\endgroup

% -----------------------------------------------------------------------------
\par\pagebreak[1]\par
\paragraph{Exercice 12 (comparaison logarithmique)}%
\hfill{\tiny 1510}%
\begingroup~

Soit $\sum_n u_n$ et $\sum_n v_n$ deux s\'eries r\'eelles.
On suppose qu'il existe $N\in \N$ tel que pour tout $n\geq N$:
\begin{itemize}
\item
  $u_n > 0$
\item
  $v_n > 0$
\item
  $\frac{u_{n+1}}{u_n}\leq\frac{v_{n+1}}{v_n}$
\end{itemize}

Montrer que:
\begin{itemize}
\item
  Si $\sum_n v_n$ converge, alors $\sum_n u_n$ converge \'egalement.
\item
  Si $\sum_n u_n$ diverge, alors $\sum_n v_n$ diverge \'egalement.
\end{itemize}
\endgroup

% -----------------------------------------------------------------------------
\par\pagebreak[1]\par
\paragraph{\href{https://psi.miomio.fr/exo/9274.pdf}{Exercice 13} (r\`egle de Raabe-Duhamel)}%
\hfill{\tiny 9274}%
\begingroup~

La \emph{r\`egle de Raabe-Duhamel} est la suivante:

Soit $\sum u_n$ une s\'erie \`a termes strictement positifs.
On suppose que $\exists \alpha\in \R$ tel que
\[ \frac{u_{n+1}}{u_n} = 1 - \frac{\alpha}{n} + O\Pafrac{1}{n^2}. \]
Alors la s\'erie $\sum u_n$ converge si et seulement si $\alpha{} > 1$.
Plus pr\'ecis\'ement, il existe une constante $K > 0$ telle que $u_n \sim \frac{K}{n^\alpha}$.

Cette r\`egle pr\'ecise le cas douteux de la r\`egle de d'Alembert.
\begin{enumerate}
\item
  Soit $(u_n)$ une suite de r\'eels strictements positifs telle que $\frac{u_{n+1}}{u_n} = 1 - \frac{\alpha}{n} + \GrandO(\frac{1}{n^2})$.
  On pose $v_n = n^\alpha u_n$ et $w_n = \ln\left(\frac{v_{n+1}}{v_n}\right)$.
  Montrer que $w_n = \GrandO(\frac{1}{n^2})$.
\item
  En d\'eduire que la suite de terme g\'en\'eral $\ln v_n$ converge.
\item
  En d\'eduire l'existence d'une constante $K > 0$ telle que $u_n \sim \frac{K}{n^\alpha}$.
\item
  D\'emontrer la r\`egle de Raabe-Duhamel.
\item
  Appliquer le r\'esultat pr\'ec\'edent \`a:
  \begin{enumerate}
  \item
    $u_n = \frac{n\cdot n!}{(a+1)(a+2)\cdots(a+n)}$ o\`u $a>0$
  \item
    $u_n = n! e^n n^{-n}$
  \end{enumerate}
\item
  Montrer que le r\`egle s'applique encore si l'on suppose seulement que
  $\frac{u_{n+1}}{u_n} = 1 - \frac{\alpha}{n} + \epsilon_n$, o\`u
  $\sum\Abs{\epsilon_n}$ est une s\'erie convergente,
  ce qui est bien le cas si $\epsilon_n = \GrandO(\frac{1}{n^2})$.
\end{enumerate}
\endgroup

% -----------------------------------------------------------------------------
\par\pagebreak[1]\par
\paragraph{Exercice 14}%
\hfill{\tiny 3194}%
\begingroup~

Soit $(a_n)_{n\in \N}$ une suite r\'eelle positive.
On d\'efinit la suite $(u_n)$ par $u_0 > 0$ et $\forall n\in \N$, $u_{n+1} = u_n + \frac{a_n}{u_n}$.
Montrer que la suite $(u_n)$ converge si et seulement si la s\'erie $\sum a_n$ converge.
\endgroup

% -----------------------------------------------------------------------------
\par\pagebreak[1]\par
\paragraph{Exercice 15 (crit\`ere de condensation de Cauchy)}%
\hfill{\tiny 2114}%
\begingroup~

\begin{enumerate}
\item
  Soit $(a_n)_{n\in \N^*}$ une suite r\'eelle positive d\'ecroissante.
  On note $b_n = 2^n a_{2^n}$ pour $n\in \N$.
  \begin{enumerate}
  \item
    Montrer que $\forall n\in \N$, $\sum_{k = 1}^{2^{n+1}-1} a_k\leq\sum_{k=0}^n b_k$
  \item
    Montrer que $\forall n\in \N$, $\sum_{k = 0}^n b_k \leq2\left( \sum_{k=1}^{2^n} a_k \right) - a_{2^n}$
  \item
    En d\'eduire que les s\'eries $\sum_n a_n$ et $\sum_n b_n$ sont de m\^eme nature.
  \end{enumerate}
\item
  Montrer le \emph{crit\`ere de condensation de Cauchy}:

  Soit $(a_n)_{n\in \N}$ une suite r\'eelle positive, d\'ecroissante \`a partir d'un certain rang.
  Alors les s\'eries $\sum_n a_n$ et $\sum_n 2^n a_{2^n}$ sont de m\^eme nature.
\item
  \emph{Application:} retrouver les r\'esultats sur les s\'eries de Riemann, puis sur les s\'eries de Bertrand.
\end{enumerate}
\endgroup

% -----------------------------------------------------------------------------
\par\pagebreak[1]\par
\paragraph{\href{https://psi.miomio.fr/exo/0066.pdf}{Exercice 16} (formule de Stirling)}%
\hfill{\tiny 0066}%
\begingroup~

\begin{enumerate}
\item
  Soit $a_n = \ln(n!)$.
  \begin{enumerate}
  \item
    Montrer que $a_n =\sum_{k=2}^n \ln k$.
  \item
    Montrer que $\forall k\geq2$, $\int_{k-1}^k \ln x \D x\leq\ln k\leq\int_k^{k+1} \ln x \D x$.
  \item
    En d\'eduire un encadrement de $a_n$, puis que $a_n \sim n\ln n$.
  \end{enumerate}
\item
  Soit $b_n = a_n - n\ln n = \ln(n!) - n\ln n$.
  \begin{enumerate}
  \item
    Montrer que $b_{n+1} - b_n \Toninf -1$.
  \item
    En utilisant le th\'eor\`eme de Ces\`aro, en d\'eduire que $b_n \sim -n$.
  \end{enumerate}
\item
  Soit $c_n = b_n - (-n) = \ln(n!) - n\ln n + n$.
  \begin{enumerate}
  \item
    Montrer que $c_{n+1} - c_n \sim \frac1{2n}$.
  \item
    En d\'eduire que $c_n \sim \frac12 \ln n$.
  \end{enumerate}
\item
  Soit $d_n = c_n - \frac12 \ln n = \ln(n!) - n\ln n + n - \frac12 \ln n$.
  \begin{enumerate}
  \item
    Montrer que $d_{n+1} - d_n = \GrandO_\ninf (\frac{1}{n^2})$.
  \item
    En d\'eduire que la s\'erie de terme g\'en\'eral $\sum_n (d_{n+1}-d_n)$ converge.
  \item
    En d\'eduire que la suite $(d_n)$ converge. On note $\ell$ sa limite.
    Montrer que:
    \[ \ln(n!) = n\ln n - n + \frac12 \ln n +\ell+ \PetitO_\ninf(1) \]
  \end{enumerate}
\item
  En d\'eduire qu'il existe une constante $K > 0$, que l'on exprimera en fonction de $\ell$, telle que:
  \[ n! \sim K \sqrt n \Pafrac{n}{e}^n \]
\item
  On pose $W_n =\int_0^{\frac\pi2} \sin^n t \D t$.
  Il s'agit bien \'evidemment des \emph{int\'egrales de Wallis}.
  \begin{enumerate}
  \item
    Montrer que pour tout $n\in \N$, on a: $W_{n+2} = \frac{n+1}{n+2} W_n$.
    \emph{Indication}: on pourra partir de $W_{n+2}$ et faire un int\'egration par parties
    bien choisie.
  \item
    En d\'eduire que pour tout $n\in \N$, on a:
    $W_{2n} = \frac{(2n)!\pi}{(n!)^2 2^{2n+1}}$.
  \item
    En d\'eduire que $W_{2n} \sim \frac{\pi}{K\sqrt{2n}}$.
  \item
    \'Etablir, pour tout entier naturel $n$, l'encadrement $W_{n+2}\leq W_{n+1}\leq W_n$.
    En d\'eduire que $W_{n+1} \sim W_n$.
  \item
    Soit $I_n = (n+1) W_{n+1} W_n$.
    Montrer que la suite $(I_n)$ est constante; calculer $I_0$.
    En d\'eduire que $W_n \sim\sqrt{\frac{\pi}{2n}}$.
  \item
    D\'eduire de c) et e) la valeur de $K$.
  \end{enumerate}
\item
  En d\'eduire que
  \[ \tcboxmath{ n! \Sim_\ninf \sqrt{2\pi n} \; \BigPa{\frac{n}{e}}^n } \]
  Ce r\'esultat est connu sous le nom de \emph{formule de Stirling}.
\end{enumerate}
\endgroup

% -----------------------------------------------------------------------------
\par\pagebreak[1]\par
\paragraph{Exercice 17}%
\hfill{\tiny 3779}%
\begingroup~

Soit $u_0 \in{} \intO{0,\pi}$ et pour tout $n\in \N$, $u_{n+1} = \sin(u_n)$.
On cherche \`a d\'eterminer la nature de la s\'erie de terme g\'en\'eral $\frac{u_n}{\sqrt n}$.
\begin{enumerate}
\item
  Montrer que $u_n \to 0$.
\item
  D\'eterminer un r\'eel $\alpha$ tel que $u_{n+1} - u_n$ ait une limite finie non nulle.
\item
  En d\'eduire un \'equivalent de $u_n$; on utilisera le th\'eor\`eme de Ces\`aro (ou celui des petits pas).
\item
  Conclure.
\end{enumerate}
\endgroup

% -----------------------------------------------------------------------------
\par\pagebreak[1]\par
\paragraph{Exercice 18}%
\hfill{\tiny 9555}%
\begingroup~

Expliquer comment il est possible d'empiler~$n$ pi\`eces identiques de sorte que celle du haut soit en \emph{porte-\`a-faux} avec celle du bas, c.-\`a-d. que la projection la pi\`ece du haut sur le plan horizontal ne touche pas la pi\`ece du bas.
\endgroup

\subsection{S\'eries num\'eriques (non absolument convergentes)}

% -----------------------------------------------------------------------------
\par\pagebreak[1]\par
\paragraph{\href{https://psi.miomio.fr/exo/9169.pdf}{Exercice 19}}%
\hfill{\tiny 9169}%
\begingroup~

\'Etudier la nature des s\'eries $\sum_n u_n$ dans chacun des cas suivants:
\begin{enumerate}
\item
  $u_n = \frac{(-1)^n}{\tan(\frac1n)}$
\item
  $u_n = \cos\left(\pi\sqrt{n^2+n+1}\right)$
\item
  $u_n = \frac{(-1)^n}{n+\sin n}$
\item
  $u_n = \ln\left(1+\frac{(-1)^n}{n^\alpha}\right)$ o\`u $\alpha> 0$
\item
  $u_n = \frac{(-1)^n}{n^\alpha+ (-1)^n n^\beta}$ o\`u $\alpha> 0$, $\beta> 0$ et $\alpha\neq \beta$
\item
  $u_n = \frac{\sin(\ln n)}{n}$
\item
  $u_n = \cos\left( \pi n^2 \ln(\frac{n}{n-1}) \right)$
\item
  $u_n = \frac{n(-1)^n}{(2n+1)(3n+1)}$, et calculer la somme
\item
  $u_n = \frac{(-1)^{\floor{\sqrt n}}}{n^\alpha}$
\end{enumerate}
\endgroup

% -----------------------------------------------------------------------------
\par\pagebreak[1]\par
\paragraph{Exercice 20}%
\hfill{\tiny 8780}%
\begingroup~

On consid\`ere la s\'erie harmonique altern\'ee $\sum_n u_n$ o\`u $u_n = \frac{(-1)^n}{n+1}$
\begin{enumerate}
\item
  Montrer qu'il s'agit d'une s\'erie convergente.
\item
  Montrer que les sommes partielles de cette s\'erie peuvent s'exprimer
  en fonction des sommes partielles de la s\'erie harmonique.
  Plus pr\'ecis\'ement, montrer que:
  \[ \sum_{n = 0}^{2N-1} u_n = H_{2N} - H_N \]
\item
  En d\'eduire la valeur de la somme $\sum_{n=0}^{+\infty} u_n$.

  On s'int\'eresse d\'esormais \`a la s\'erie $\sum_n v_n$,
  o\`u la suite $(v_n)$ est une permutation de la suite $(u_n)$
  obtenue en prenant dans l'ordre un terme positif puis deux termes n\'egatifs, et ainsi de suite:
  $v_0 = 1$, $v_1 = - \frac12$, $v_2 = -\frac14$, $v_3 = \frac13$, $v_4 = -\frac16$, $v_5 = -\frac18$,
  $v_6 = \frac15$, $v_7 = -\frac{1}{10}$, $v_8 = -\frac{1}{12}$, etc...
\item
  V\'erifiez que vous avez compris la d\'efinition de la suite $(v_n)$: calculer $v_{3n}, v_{3n+1}, v_{3n+2}$.
\item
  Notons $S_n =\sum_{k=0}^n v_k$.
  Exprimer $S_{3n-1}$ sous forme d'une s\'erie et en d\'eduire que la suite $(S_{3n-1})$ converge.
\item
  En d\'eduire que la s\'erie $\sum_n v_n$ converge.
\item
  Exprimer $S_{3n-1}$ en fonction de $H_p$.
\item
  En d\'eduire la valeur de la somme $\sum_{n=0}^{+\infty} v_n$.
\item
  Commenter.
\end{enumerate}
\endgroup

% -----------------------------------------------------------------------------
\par\pagebreak[1]\par
\paragraph{\href{https://psi.miomio.fr/exo/2646.pdf}{Exercice 21} (transformation d'Abel)}%
\hfill{\tiny 2646}%
\begingroup~

Soit $(a_n)$ et $(b_n)$ deux suites num\'eriques.
On pose $B_n = \sum_{k=0}^n b_k$, et $S_n = \sum_{k=0}^n a_k b_k$.
\begin{enumerate}
\item
  Montrer que $S_n = a_{n+1} B_n + \sum_{k=0}^n (a_k - a_{k+1}) B_k$.
  Cette \'egalit\'e porte le nom de \emph{transformation d'Abel}.
\item
  En d\'eduire que si
  \begin{itemize}
  \item
    la suite $(B_n)$ est born\'ee, et
  \item
    la suite $(a_n)$ est une suite r\'eelle positive d\'ecroissante de limite nulle,
  \end{itemize}
  alors la s\'erie $\sum a_n b_n$ converge.
\item
  Montrer que le crit\`ere sp\'ecial est un cas particulier de le la r\`egle pr\'ec\'edente.
\item
  \emph{Applications:}
  \begin{enumerate}
  \item
    Si $a_n$ est une suite r\'eelle positive d\'ecroissante de limite nulle, et $t\in \R\setminus2\pi \Z$, montrer que la s\'erie $\sum_n a_n e^{int}$ converge.
  \item
    \'Etudier pour $\alpha\in\intOF{0,1}$, $\beta{} \notin{} 2\pi \Z$, $\gamma\in \R$ la s\'erie $\sum_n \frac{\sin(n\beta+\gamma)}{n^\alpha}$.
  \item
    \'Etudier pour $\alpha>0$ la s\'erie $\sum_n \frac{\sin n}{n^\alpha{} + \sin n}$.
  \end{enumerate}
\end{enumerate}
\endgroup

% -----------------------------------------------------------------------------
\par\pagebreak[1]\par
\paragraph{Exercice 22 (dur)}%
\hfill{\tiny 8817}%
\begingroup~

On pose $u_n =\sum_{k=n}^{+\infty} \frac{(-1)^k}{\sqrt{k+1}}$.
\'Etudier la convergence de la s\'erie $\sum{} u_n$.
\endgroup

\end{document}
