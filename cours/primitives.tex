% autogenerated by ytex.rs

\documentclass{scrartcl}

\usepackage[francais]{babel}
\usepackage{geometry}
\usepackage{scrpage2}
\usepackage{lastpage}
\usepackage{ragged2e}
\usepackage{multicol}
\usepackage{etoolbox}
\usepackage{xparse}
\usepackage{enumitem}
\usepackage{csquotes}
\usepackage{amsmath}
\usepackage{amsfonts}
\usepackage{amssymb}
\usepackage{mathrsfs}
\usepackage{stmaryrd}
\usepackage{dsfont}
\usepackage{eurosym}
% \usepackage{numprint}
% \usepackage[most]{tcolorbox}
% \usepackage{tikz}
% \usepackage{tkz-tab}
\usepackage[unicode]{hyperref}
\usepackage[ocgcolorlinks]{ocgx2}

\let\ifTwoColumns\iffalse
\def\Classe{$\Psi$2019--2020}

% reproducible builds
% LuaTeX: \pdfvariable suppressoptionalinfo 1023 \relax
\pdfinfoomitdate=1
\pdftrailerid{}

\newif\ifDisplaystyle
\everymath\expandafter{\the\everymath\ifDisplaystyle\displaystyle\fi}
\newcommand\DS{\displaystyle}

\clearscrheadfoot
\pagestyle{scrheadings}
\thispagestyle{empty}
\ohead{\Classe}
\ihead{\thepage/\pageref*{LastPage}}

\setlist[itemize,1]{label=\textbullet}
\setlist[itemize,2]{label=\textbullet}

\ifTwoColumns
  \geometry{margin=1cm,top=2cm,bottom=3cm,foot=1cm}
  \setlist[enumerate]{leftmargin=*}
  \setlist[itemize]{leftmargin=*}
\else
  \geometry{margin=3cm}
\fi

\makeatletter
\let\@author=\relax
\let\@date=\relax
\renewcommand\maketitle{%
    \begin{center}%
        {\sffamily\huge\bfseries\@title}%
        \ifx\@author\relax\else\par\medskip{\itshape\Large\@author}\fi
        \ifx\@date\relax\else\par\bigskip{\large\@date}\fi
    \end{center}\bigskip
    \ifTwoColumns
        \par\begin{multicols*}{2}%
        \AtEndDocument{\end{multicols*}}%
        \setlength{\columnsep}{5mm}
    \fi
}
\makeatother

\newcounter{ParaNum}
\NewDocumentCommand\Para{smo}{%
  \IfBooleanF{#1}{\refstepcounter{ParaNum}}%
  \paragraph{\IfBooleanF{#1}{{\tiny\arabic{ParaNum}~}}#2\IfNoValueF{#3}{ (#3)}}}

\newcommand\I{i}
\newcommand\mi{i}
\def\me{e}

\def\do#1{\expandafter\undef\csname #1\endcsname}
\docsvlist{Ker,sec,csc,cot,sinh,cosh,tanh,coth,th}
\undef\do

\DeclareMathOperator\ch{ch}
\DeclareMathOperator\sh{sh}
\DeclareMathOperator\th{th}
\DeclareMathOperator\coth{coth}
\DeclareMathOperator\cotan{cotan}
\DeclareMathOperator\argch{argch}
\DeclareMathOperator\argsh{argsh}
\DeclareMathOperator\argth{argth}

\let\epsilon=\varepsilon
\let\phi=\varphi
\let\leq=\leqslant
\let\geq=\geqslant
\let\subsetneq=\varsubsetneq
\let\emptyset=\varnothing

\newcommand{\+}{,\;}

\undef\C
\newcommand\ninf{{n\infty}}
\newcommand\N{\mathbb{N}}
\newcommand\Z{\mathbb{Z}}
\newcommand\Q{\mathbb{Q}}
\newcommand\R{\mathbb{R}}
\newcommand\C{\mathbb{C}}
\newcommand\K{\mathbb{K}}
\newcommand\Ns{\N^*}
\newcommand\Zs{\Z^*}
\newcommand\Qs{\Q^*}
\newcommand\Rs{\R^*}
\newcommand\Cs{\C^*}
\newcommand\Ks{\K^*}
\newcommand\Rp{\R^+}
\newcommand\Rps{\R^+_*}
\newcommand\Rms{\R^-_*}
\newcommand{\Rpinf}{\Rp\cup\Acco{+\infty}}

\undef\B
\newcommand\B{\mathscr{B}}

\undef\P
\DeclareMathOperator\P{\mathbb{P}}
\DeclareMathOperator\E{\mathbb{E}}
\DeclareMathOperator\Var{\mathbb{V}}

\DeclareMathOperator*\PetitO{o}
\DeclareMathOperator*\GrandO{O}
\DeclareMathOperator*\Sim{\sim}
\DeclareMathOperator\Tr{tr}
\DeclareMathOperator\Ima{Im}
\DeclareMathOperator\Ker{Ker}
\DeclareMathOperator\Sp{Sp}
\DeclareMathOperator\Diag{diag}
\DeclareMathOperator\Rang{rang}
\DeclareMathOperator*\Coords{Coords}
\DeclareMathOperator*\Mat{Mat}
\DeclareMathOperator\Pass{Pass}
\DeclareMathOperator\Com{Com}
\DeclareMathOperator\Card{Card}
\DeclareMathOperator\Racines{Racines}
\DeclareMathOperator\Vect{Vect}
\DeclareMathOperator\Id{Id}

\newcommand\DerPart[2]{\frac{\partial #1}{\partial #2}}

\def\T#1{{#1}^T}

\def\pa#1{({#1})}
\def\Pa#1{\left({#1}\right)}
\def\bigPa#1{\bigl({#1}\bigr)}
\def\BigPa#1{\Bigl({#1}\Bigr)}
\def\biggPa#1{\biggl({#1}\biggr)}
\def\BiggPa#1{\Biggl({#1}\Biggr)}

\def\pafrac#1#2{\pa{\frac{#1}{#2}}}
\def\Pafrac#1#2{\Pa{\frac{#1}{#2}}}
\def\bigPafrac#1#2{\bigPa{\frac{#1}{#2}}}
\def\BigPafrac#1#2{\BigPa{\frac{#1}{#2}}}
\def\biggPafrac#1#2{\biggPa{\frac{#1}{#2}}}
\def\BiggPafrac#1#2{\BiggPa{\frac{#1}{#2}}}

\def\cro#1{[{#1}]}
\def\Cro#1{\left[{#1}\right]}
\def\bigCro#1{\bigl[{#1}\bigr]}
\def\BigCro#1{\Bigl[{#1}\Bigr]}
\def\biggCro#1{\biggl[{#1}\biggr]}
\def\BiggCro#1{\Biggl[{#1}\Biggr]}

\def\abs#1{\mathopen|{#1}\mathclose|}
\def\Abs#1{\left|{#1}\right|}
\def\bigAbs#1{\bigl|{#1}\bigr|}
\def\BigAbs#1{\Bigl|{#1}\Bigr|}
\def\biggAbs#1{\biggl|{#1}\biggr|}
\def\BiggAbs#1{\Biggl|{#1}\Biggr|}

\def\acco#1{\{{#1}\}}
\def\Acco#1{\left\{{#1}\right\}}
\def\bigAcco#1{\bigl\{{#1}\bigr\}}
\def\BigAcco#1{\Bigl\{{#1}\Bigr\}}
\def\biggAcco#1{\biggl\{{#1}\biggr\}}
\def\BiggAcco#1{\Biggl\{{#1}\Biggr\}}

\def\ccro#1{\llbracket{#1}\rrbracket}
\def\Dcro#1{\llbracket{#1}\rrbracket}

\def\floor#1{\lfloor#1\rfloor}
\def\Floor#1{\left\lfloor{#1}\right\rfloor}

\def\sEnsemble#1#2{\mathopen\{#1\mid#2\mathclose\}}
\def\bigEnsemble#1#2{\bigl\{#1\bigm|#2\bigr\}}
\def\BigEnsemble#1#2{\Bigl\{#1\Bigm|#2\Bigr\}}
\def\biggEnsemble#1#2{\biggl\{#1\biggm|#2\biggr\}}
\def\BiggEnsemble#1#2{\Biggl\{#1\Biggm|#2\Biggr\}}
\let\Ensemble=\bigEnsemble

\newcommand\IntO[1]{\left]#1\right[}
\newcommand\IntF[1]{\left[#1\right]}
\newcommand\IntOF[1]{\left]#1\right]}
\newcommand\IntFO[1]{\left[#1\right[}

\newcommand\intO[1]{\mathopen]#1\mathclose[}
\newcommand\intF[1]{\mathopen[#1\mathclose]}
\newcommand\intOF[1]{\mathopen]#1\mathclose]}
\newcommand\intFO[1]{\mathopen[#1\mathclose[}

\newcommand\Fn[3]{#1\colon#2\to#3}
\newcommand\CC[1]{\mathscr{C}^{#1}}
\newcommand\D{\mathop{}\!\mathrm{d}}

\newcommand\longto{\longrightarrow}

\undef\M
\newcommand\M[3]{\mathrm{#1}_{#2}\pa{#3}}
\newcommand\MnR{\M{M}{n}{\R}}
\newcommand\MnC{\M{M}{n}{\C}}
\newcommand\MnK{\M{M}{n}{\K}}
\newcommand\GLnR{\M{GL}{n}{\R}}
\newcommand\GLnC{\M{GL}{n}{\C}}
\newcommand\GLnK{\M{GL}{n}{\K}}
\newcommand\DnR{\M{D}{n}{\R}}
\newcommand\DnC{\M{D}{n}{\C}}
\newcommand\DnK{\M{D}{n}{\K}}
\newcommand\SnR{\M{S}{n}{\R}}
\newcommand\AnR{\M{A}{n}{\R}}
\newcommand\OnR{\M{O}{n}{\R}}
\newcommand\SnRp{\mathrm{S}_n^+(\R)}
\newcommand\SnRpp{\mathrm{S}_n^{++}(\R)}

\newcommand\LE{\mathscr{L}(E)}
\newcommand\GLE{\mathscr{GL}(E)}
\newcommand\SE{\mathscr{S}(E)}
\renewcommand\OE{\mathscr{O}(E)}

\newcommand\ImplD{$\Cro\Rightarrow$}
\newcommand\ImplR{$\Cro\Leftarrow$}
\newcommand\InclD{$\Cro\subset$}
\newcommand\InclR{$\Cro\supset$}
\newcommand\notInclD{$\Cro{\not\subset}$}
\newcommand\notInclR{$\Cro{\not\supset}$}

\newcommand\To[1]{\xrightarrow[#1]{}}
\newcommand\Toninf{\To{\ninf}}

\newcommand\Norm[1]{\|#1\|}
\newcommand\Norme{{\Norm{\cdot}}}

\newcommand\Int[1]{\mathring{#1}}
\newcommand\Adh[1]{\overline{#1}}

\newcommand\Uplet[2]{{#1},\dots,{#2}}
\newcommand\nUplet[3]{(\Uplet{{#1}_{#2}}{{#1}_{#3}})}

\newcommand\Fonction[5]{{#1}\left|\begin{aligned}{#2}&\;\longto\;{#3}\\{#4}&\;\longmapsto\;{#5}\end{aligned}\right.}

\DeclareMathOperator\orth{\bot}
\newcommand\Orth[1]{{#1}^\bot}
\newcommand\PS[2]{\langle#1,#2\rangle}

\newcommand{\Tribu}{\mathscr{T}}
\newcommand{\Part}{\mathcal{P}}
\newcommand{\Pro}{\bigPa{\Omega,\Tribu}}
\newcommand{\Prob}{\bigPa{\Omega,\Tribu,\P}}

\newcommand\DEMO{$\spadesuit$}
\newcommand\DUR{$\spadesuit$}

\newenvironment{psmallmatrix}{\left(\begin{smallmatrix}}{\end{smallmatrix}\right)}

% -----------------------------------------------------------------------------

\usepackage{booktabs}

\geometry{margin=2.5cm}

\begin{document}
\title{Calcul de primitives}
\maketitle

Il y a longtemps, tr\`es longtemps, vous avez appris \`a d\'eriver une fonction.
C'est utile, mais pas tr\`es amusant: il suffit de conna\^itre les d\'eriv\'ees usuelles et d'appliquer les r\`egles sur les sommes, produits, quotients et compos\'ees.
C'est tr\`es m\'ecanique...

Primitiver une fonction, c'est une toute autre histoire.
Il faut toujours conna\^itre les primitives usuelles (c'est juste le tableau des d\'eriv\'ees, vu \`a l'envers), et il y a encore une r\`egle pour la somme.
En revanche, il n'y a pas de r\`egle g\'en\'erale qui donne la primitive d'un produit, d'un quotient ou d'une compos\'ee.

Mais comment faire pour calculer une primitive?
Au fil du temps, on a trouv\'e un certain nombre de m\'ethodes\footnote{Une m\'ethode, c'est une astuce qui sert au moins deux fois.} qui s'appliquent dans certains cas particuliers.
C'est d'ailleurs l'objet de ce chapitre.
Apr\`es, vous \^etes tout seuls.
Il faudra essayer, deviner, combiner les m\'ethodes, trouver des astuces diaboliques, s'apercevoir que \c ca ne marche pas, recommencer, recommencer encore...
Bref, c'est comme un puzzle, il faudra r\'efl\'echir.

Existe-t-il une m\'ethode ultime permettant de calculer tou\-tes les primitives?
Et Wolfram Alpha (ou Maple, Mathematica, Maxima, Sympy, Sage, etc.), au fait: la conna\^it-il?

\`A ces questions, la r\'eponse est NON; et ce non ne signifie pas \og on a beaucoup cherch\'e, et on n'a rien trouv\'e\fg, mais un bien plus fort \og on a \emph{d\'emontr\'e} qu'il n'existe pas de m\'ethode universelle\fg.
Certaines primitives \emph{ne peuvent pas} s'exprimer \`a partir\footnote{Plus pr\'ecis\'ement, en partant des fonctions usuelles et en faisant un nombre fini de sommes, diff\'erences, produits, quotients ou compositions.} des fonctions usuelles.

Par exemple, si on demande $\DS \int\frac{\D x}{\ln x}$ \`a Wolfram Alpha, il r\'epondra \verb|li(x)|.
Or la fonction $\mathrm{li}$, dite logarithme int\'egral est par d\'efinition.. une primitive de $1/\ln(x)$!
En un sens, c'est de la triche, mais il n'existe pas de meilleure r\'eponse.
Et si on lui demande une fonction assez compliqu\'ee comme
\[ \int\frac{\sin^3\Pa{1+\frac{1}{\ln t}}}{\cos t + \sqrt{e^t + 1}} \D t, \]
il ne r\'epondra rien.

\Para{Rappel}
Soit $I$ un intervalle de $\R$, et $f, F$ deux fonctions de $I$ dans $\K$.
On dit que $F$ est une primitive de $f$ sur $I$ si et seulement si
$F$ est d\'erivable sur $I$ et $F' = f$.
On note $\int{} f = F + C^{\text{ste}}$.

\section{Fonctions usuelles}

\begingroup
  \def\Icos{$\intO{k\pi-\pi/2, k\pi+\pi/2}$ o\`u~$k\in \Z$}
  \def\Isin{$\intO{k\pi, (k+1)\pi}$ o\`u~$k\in \Z$}
  \def\IRs{$\R^*_-$, $\R^*_+$}
  \renewcommand\arraystretch{1.5}
  \begin{tabular}{ccc}
    \toprule
    Fonction $f$& Intervalle de d\'efinition& Primitive de $f$\cr
    \midrule
    $f(x)$& $I$& $\int f(x)\D x = F(x)$\cr
    $f(\alpha x)$ o\`u $\alpha\neq0$& $I$& $\frac1\alpha F(\alpha x)$\cr
    \midrule
    $e^x$ & $\R$& $e^x$\cr
    $\ln x$ & $\Rps$ & $x\ln(x)-x$\cr
    \midrule
    $x^\alpha$ o\`u $\alpha\in \C\setminus\acco{-1}$& $\Rps$& $\frac{x^{\alpha+1}}{\alpha+1}$\cr
    $x^n$ o\`u $n\in \N$& $\R$& $\frac{x^{n+1}}{n+1}$\cr
    $x^n$ o\`u $n\in \Z$, $n\leq-2$& \IRs& $\frac{x^{n+1}}{n+1}$\cr
    $x^{-1} = \frac1x$& \IRs& $\ln\Abs{x}$\cr
    \midrule
    $\cos x$& $\R$& $\sin x$\cr
    $\sin x$& $\R$& $-\cos x$\cr
    $\ch x$& $\R$& $\sh x$\cr
    $\sh x$& $\R$& $\ch x$\cr
    % $\tan x$& \Icos& $-\ln\abs{\cos x}$\cr
    % $\cotan x$& \Isin& $\ln\abs{\sin x}$\cr
    % $\th x$& $\R$& $\ln\ch x$\cr
    % $\coth x$& \IRs& $\ln\abs{\sh x}$\cr
    % \midrule
    $\frac{1}{\cos^2 x} = 1 + \tan^2 x$& \Icos& $\tan x$\cr
    % $\frac{1}{\sin^2 x} = 1 + \cotan^2 x$& \Isin& $-\cotan x$\cr
    % $\frac{1}{\ch^2 x} = 1 - \th^2 x$& $\R$& $\th x$\cr
    % $\frac{1}{\sh^2 x} = \coth^2 x - 1$& \IRs& $-\coth x$\cr
    \midrule
    $\frac{1}{1+x^2}$& $\R$& $\arctan x$\cr
    % $\frac{1}{1-x^2}$& $]-1,1[$& $\argth x$\cr
    % $\frac{1}{1-x^2}$& $]-\infty,-1[$, $]-1,1[$, $]1,+\infty[$& $\frac12\ln\Abs{\frac{1+x}{1-x}}$\cr
    % $\frac{1}{\sqrt{1+x^2}}$& $\R$& $\argsh x = \ln\Pa{x+\sqrt{x^2+1}}$\cr
    % $\frac{1}{\sqrt{1-x^2}}$& $]-1,1[$& $\arcsin x$ ou  $-\arccos x$\cr
    % $\frac{1}{\sqrt{x^2-1}}$& $]1,+\infty[$& $\argch x = \ln\Pa{x+\sqrt{x^2-1}}$\cr
    % $\frac{1}{\sqrt{x^2-1}}$& $]-\infty,-1[$, $]1,+\infty[$& $\ln\Abs{x+\sqrt{x^2-1}}$\cr
    % \midrule
    $\frac{1}{a^2+x^2}$ o\`u $a \in \R^*$& $\R$& $\frac1a \arctan\Pafrac xa$\cr
    % $\frac{1}{a^2-x^2}$ o\`u $a > 0$& $]-\infty,-a[$, $]-a,a[$, $]a,+\infty[$& $\frac1{2a} \ln\Abs{\frac{a+x}{a-x}}$\cr
    % $\frac{1}{\sqrt{a^2+x^2}}$ o\`u $a \in\Rs$& $\R$& $\argsh\Pafrac xa$\cr
    % $\frac{1}{\sqrt{a^2-x^2}}$ o\`u $a > 0$& $]-a,a[$& $\arcsin\Pafrac xa$ ou $-\arccos\Pafrac xa$\cr
    % $\frac{1}{\sqrt{x^2-a^2}}$ o\`u $a > 0$& $]-\infty,-a[$, $]a,+\infty[$& $\argch\Pafrac xa$\cr
    \bottomrule
  \end{tabular}
\endgroup

\Displaystyletrue

\section{M\'ethodes g\'en\'erales}

\subsection{Changement de variables}

Soit $f$ une fonction de classe $\CC0$ et $\phi$ de classe $\CC1$.
Le changement de variable $x = \phi(y)$ s'\'ecrit:
\[ \int{} f(x) \D x = \int{} f(\phi(y)) \phi'(y) \D y \]

Si l'on souhaite faire un changement de variables \enquote{dans l'autre sens}, c.-\`a-d. de la forme $y = \psi(x)$,
on doit s'assurer que $\psi$ est bijective et que $\psi^{-1}$ est de classe $\CC1$;
cela est garanti si $\psi$ est une bijection de classe $\CC1$ dont la d\'eriv\'ee ne s'annule pas.

\Para{Exemples}
$\int\frac{\D x}{x^2 + a^2}$,
$\int\tan x \D x$,
$\int\frac{\arcsin x}{\sqrt{1-x^2}} \D x$,
$\int\frac{\D x}{\sqrt{1+\sqrt{1+x}}}$.

\subsection{Int\'egration par parties}

Si $f$ est de classe $\CC0$ et $g$ est de classe $\CC1$, on a:
\[ \int fg = Fg - \int Fg' \qquad \text{ o\`u } F = \int f \]

\Para{Exemples}
$\int\ln x \D x$,
$\int\arcsin x \D x$,
$\int\ln\Pa{1+x^2} \D x$,
$\int\frac{x \D x}{\cos^2 x}$.

\subsection{G\'en\'eralisation: int\'egration par parties multiple}

Si $f$ et $g$ sont de classe $\CC n$:

\begin{align*}
  \int{} f^{(n)} g
  &= \sum_{k=0}^{n-1} (-1)^k f^{(n-1-k)} g^{(k)} + (-1)^n \int fg^{(n)} \\
  &= f^{(n-1)} g - f^{(n-2)}g' + f^{(n-3)}g'' - f^{(n-4)}g^{(3)} + \cdots
  + (-1)^{n-1} f g^{(n-1)} + (-1)^n \int fg^{(n)}
\end{align*}

\Para{Exemples}
$\int(x^3 + x^2 + 1) e^x \D x$,
$\int_{-1}^1 (1-x^2)^n \D x$.

\section{Fonctions polyn\^ome-exponentielle}

\subsection{$\int P(x) e^{\alpha x} \D x$}

On suppose $P \in \C[X]$, $\alpha\in \C^*$. Soit $d$ le degr\'e de $P$.
On dispose de deux m\'ethodes:
\begin{itemize}
\item
  int\'egration par parties: on effectue $d$ int\'egrations par parties (ou une int\'egration par parties multiple) en d\'erivant le polyn\^ome et en primitivant l'exponentielle;
\item
  identification (pr\'ef\'erable quand $d \geq3$): on cherche $Q \in \C_d[X]$ tel que \[ \int P(x) e^{\alpha x} \D x = Q(X) e^{\alpha x} + C, \] c.-\`a-d. $\alpha Q(X) + Q'(X) = P(X)$.
\end{itemize}

\Para{Exemple}
$\int(x^2-x+3) e^{2x} \D x$.

\subsection{$\int e^{\alpha x} \cos(\beta x) \D x$}

On suppose $(\alpha,\beta) \in \R^2 \setminus\Acco{(0,0)}$.
On calcule $\int e^{(\alpha+ i\beta) x} \D x$, et l'on s\'epare partie r\'eelle et partie imaginaire.

\Para{Exemple}
$\int e^{2x} \sin{3x} \D x$.

\subsection{$\int P(x) \cos(\beta x) \D x$}

On suppose $P \in \R_d[X]$, $\beta\in \R^*$.
On dispose de trois m\'ethodes:
\begin{itemize}
\item
  int\'egration par parties;
\item
  identification: on cherche $\int P(x) \cos(\beta x) \D x$ sous la forme $A(x) \cos(\beta x) + B(x) \sin(\beta x)$ o\`u $(A,B) \in \R_d[X]^2$;
\item
  passage par l'exponentielle complexe: on commence par calculer $\int P(x) e^{i\beta x} \D x$.
\end{itemize}

\Para{Exemple}
$\int x^3 \sin x \D x$.

\subsection{$\int P(x) e^{\alpha x} \cos(\beta x) \D x$}

On suppose $P \in \R[X]$, $(\alpha,\beta) \in \R^2 \setminus\Acco{(0,0)}$.
On dispose de deux m\'ethodes:
\begin{itemize}
\item
  passage \`a l'exponentielle complexe puis int\'egration par parties;
\item
  identification.
\end{itemize}

\Para{Exemple}
$\int x^3 e^x \cos x \D x$.

\section{Fonctions rationnelles}

Pour toute fonction rationnelle $R \in \R(X)$ dont on sait factoriser le d\'enominateur, on dispose d'un algorithme permettant de calculer $\int R(x) \D x$.

\Para{Astuce}
Si $R$ est impaire, on peut \'ecrire \[ R(X) = X \frac{P(X^2)}{Q(X^2)}, \] de sorte que si l'on fait le changement de variables $y=x^2$, on a \[ \int R(x) \D x = \frac12 \int\frac{P(y)}{Q(y)} \D y. \]
Il reste \`a int\'egrer $S(Y) = \frac{P(Y)}{Q(Y)}$, qui est plus simple que $R$.
Ainsi, on a abaiss\'e le degr\'e et donc simplifi\'e le calcul.

\subsection{D\'ecomposition en \'el\'ements simples}

\Para{Exemples}[p\^oles simples r\'eels]
\[ \frac{1}{X^2-1}, \quad
  \frac{X+1}{X+2}, \quad
\frac{2X+8}{X^2+8X+15}, \]
\[ \frac{X^4+X+1}{X(X^2-1)}, \quad
\frac{X^5+6X^4+9X^3+6}{X(X+1)(X+2)(X+3)}. \]

\Para{Exemples}[p\^oles multiples r\'eels]
\[ \frac{1}{X^2(X+1)}, \quad \frac{X^3+2}{(X^2-1)^2}. \]

\Para{Exemples}[p\^oles simples complexes]
\[ \frac{1}{X(X^2+1)}, \quad
  \frac{-X^3+2X^2+1}{X^3+X}, \quad
\frac{2X^3-4X+2}{X^3+2X}. \]

\Para{Exemple}[p\^oles multiples complexes]
\[ \frac{X^3+3X+1}{X^5+2X^3+X}. \]

\subsection{\'El\'ements simples de premi\`ere esp\`ece}

On cherche \`a calculer \[ \int\frac{c}{(ax+b)^n} \D x. \]
Ce cas est tr\`es facile.
Attention toutefois au cas $n = 1$ si $(a,b) \notin \R^2$: il faut alors multiplier par la quantit\'e conjugu\'ee et se ramener \`a un \'el\'ement simple de seconde esp\`ece.

\subsection{\'El\'ements simples de seconde esp\`ece}

On cherche \`a calculer \[ \int\frac{dx+e}{(ax^2+bx+c)^n} \D x, \]
o\`u $(a,b,c)\in \R^3$, $a\neq0$, $\Delta=b^2-4ac<0$.

On proc\`ede par \'etapes.

\subsubsection{Premi\`ere \'etape}

On effectue le changement de variables $y = x+\frac{b}{2a}$, de sorte que
\[ \int\frac{dx+e}{(ax^2+bx+c)^n} \D x = \int\frac{dy+e'}{(ay^2+c')^n} \D y. \]
Notons que l'on a n\'ecessairement $c'>0$.

\subsubsection{Deuxi\`eme \'etape}

On s\'epare l'int\'egrale en deux, de sorte que l'on doit calculer:
\begin{itemize}
\item
  $I = \int\frac{dy}{(ay^2+c')^n} \D y$, qui est imm\'ediate \`a calculer, car l'int\'egrande est de la forme $\frac{u'}{u^n}$;
\item
  $J = \int\frac{e'}{(ay^2+c')^n} \D y$, qui se ram\`ene imm\'ediatement \`a $K = \int\frac{\D y}{(y^2+\alpha^2)^n}$.
\end{itemize}

\subsubsection{Troisi\`eme \'etape}

Si $n=1$, c'est gagn\'e, on conna\^it une primitive:
\[ \frac{1}{\alpha}\arctan\Pafrac{y}{\alpha}. \]
Sinon, on fait le changement de variables $y=t\alpha$, de fa\c con \`a se ramener \`a
\[ J_n = \int\frac{\D t}{(t^2+1)^n}. \]
Avec une int\'egration par parties, on \'etablit une relation de r\'ecurrence entre les $J_n$ permettant
de calculer $J_2$, $J_3$, etc...

\subsubsection{Conclusion} On exprime en fonction de la variable initiale, puis on regroupe le tout.

\Para{Exemples}
$\int\frac{\D x}{x(x^2+1)^3}$,
$\int\frac{x^3 \D x}{x^2+2x+2}$,
$\int\frac{\D x}{x^3+1}$.

% -----------------------------------------------------------------------------
\section{Fonctions de la forme $R(\cos\theta,\sin\theta)$}
On suppose $R \in \R(X,Y)$.

\subsection{Cas particulier: $\int\cos^m \theta\sin^n \theta\D \theta$}

On a une m\'ethode qui marche toujours: lin\'eariser.
Cependant, on peut parfois \^etre plus malin.
Si $m$ est impair, on peut poser $x = \sin \theta$;
si $n$ est impair, on peut poser $x = \cos \theta$.

\Para{Exemple}
$\int\sin^7 \theta\cos^4 \theta\D \theta$.

\subsection{Cas g\'en\'eral}

On a une m\'ethode qui marche toujours: la tangente de l'arc moiti\'e, c.-\`a-d. le changement de variable $x = \tan\frac{\theta}{2}$.
Cela nous ram\`ene \`a primitiver une fraction rationnelle, mais les calculs sont souvent horribles.
On peut \^etre plus malin.

La \emph{r\`egle de Bioche} r\'epond au probl\`eme.
On forme l'\'el\'ement diff\'erentiel $\omega(\theta) = R(\cos\theta,\sin\theta) \D \theta$.
\begin{itemize}
\item
  si $\omega(-\theta)  = \omega(\theta)$, alors on pose $x = \cos\theta$;
\item
  si $\omega(\pi-\theta) = \omega(\theta)$, alors on pose $x = \sin\theta$;
\item
  si $\omega(\pi+\theta) = \omega(\theta)$, alors on pose $x = \tan\theta$;
\item
  sinon, on pose $x = \tan\frac{\theta}{2}$.
  Rappel: on a alors $\sin\theta= \frac{2x}{1+x^2}$ et $\cos\theta=\frac{1-x^2}{1+x^2}$.
\end{itemize}
Dans tous les cas, on se ram\`ene alors \`a primitiver une fraction rationnelle.

\Para{Exemple}
$\int{} \frac{\sin^2 x}{2+\cos x} \D x$,
$\int{} \frac{\D x}{2+\cos x}$.

\subsection{Fonctions de la forme $R(\ch\theta, \sh\theta)$}
On suppose $R \in \R(X,Y)$.

Il suffit d'appliquer la r\`egle de Bioche \`a $R(\cos\theta, \sin\theta)$, puis:
\begin{itemize}
\item
  au lieu de poser $x = \cos\theta$, on pose $x = \ch\theta$;
\item
  au lieu de poser $x = \sin\theta$, on pose $x = \sh\theta$;
\item
  au lieu de poser $x = \tan\theta$, on pose $x = \th\theta$;
\item
  au lieu de poser $x = \tan\frac\theta2$, on pose $x = \th\frac\theta2$ ou bien $x = e^\theta$.
\end{itemize}

\Para{Exemple}
$\int\frac{\sh^2 x}{\ch x(2\sh^3 x + 3\ch^3 x)} \D x$.

% -----------------------------------------------------------------------------
\section{Fonctions de la forme $R\Pa{e^{\alpha x}}$}
On suppose $R \in \R(X)$.
On pose $y = e^{\alpha x}$.

\Para{Exemple}
$\int\frac{\D x}{(e^x + 2)^2}$.

Le m\^eme changement marche parfois m\^eme si $R$ n'est pas rationnelle, par exemple pour:
$\int\sqrt{e^x + 1} \D x$.

% -----------------------------------------------------------------------------
\section{Fonctions de la forme $R\Pa{x,\sqrt[n]{\frac{ax+b}{cx+d}}}$}
On suppose $R \in \R(X,Y)$.
On pose $y = \sqrt[n]{\frac{ax+b}{cx+d}}$.

\Para{Exemples}
$\int\sqrt{\frac{x}{(1-x)^3}} \D x$,
$\int\frac{1}{\sqrt{x}+\sqrt{x+1}}$,
$\int{\frac{\D x}{\sqrt{x} + \sqrt[3]{x}}}$.

% -----------------------------------------------------------------------------
\section{Int\'egrales ab\'eliennes $R\Pa{x, \sqrt{ax^2+bx+c}}$}
On suppose $R \in \R(X,Y)$.

Notons $\Delta= b^2 - 4ac$.
\begin{itemize}
\item
  $\Delta= 0$: la racine carr\'ee dispara\^it, il ne reste qu'une valeur absolue,
  qui disparait \'egalement si on restreint l'intervalle.
\item
  $a<0$, $\Delta<0$: impossible car $\forall x \in \R$, $ax^2+bx+c < 0$.
\item
  $a>0$, $\Delta<0$: par un changement de variable affine,
  on se ram\`ene \`a $G(t,\sqrt{1+t^2})$, puis on pose $t = \sh\theta$.
\item
  $a<0$, $\Delta>0$: par un changement de variable affine,
  on se ram\`ene \`a $G(t,\sqrt{1-t^2})$, puis on pose $t = \sin\theta$ ou $t = \cos\theta$.
\item
  $a>0$, $\Delta>0$: par un changement de variable affine,
  on se ram\`ene \`a $G(t,\sqrt{t^2-1})$, puis on pose $t = \ch\theta$.
\end{itemize}

\Para{Exemple}
$\int\frac{x \D x}{\sqrt{x^2+x+1}}$,
$\int\frac{x \D x}{1 + \sqrt{x^2+x+1}}$.

%\end{multicols}
\newpage

\end{document}
