% autogenerated by ytex.rs

\documentclass{scrartcl}

\usepackage[francais]{babel}
\usepackage{geometry}
\usepackage{scrpage2}
\usepackage{lastpage}
\usepackage{ragged2e}
\usepackage{multicol}
\usepackage{etoolbox}
\usepackage{xparse}
\usepackage{enumitem}
% \usepackage{csquotes}
\usepackage{amsmath}
\usepackage{amsfonts}
\usepackage{amssymb}
\usepackage{mathrsfs}
\usepackage{stmaryrd}
\usepackage{dsfont}
\usepackage{eurosym}
% \usepackage{numprint}
% \usepackage[most]{tcolorbox}
% \usepackage{tikz}
% \usepackage{tkz-tab}
\usepackage[unicode]{hyperref}
\usepackage[ocgcolorlinks]{ocgx2}

\let\ifTwoColumns\iftrue
\def\Classe{$\Psi$2019--2020}

% reproducible builds
% LuaTeX: \pdfvariable suppressoptionalinfo 1023 \relax
\pdfinfoomitdate=1
\pdftrailerid{}

\newif\ifDisplaystyle
\everymath\expandafter{\the\everymath\ifDisplaystyle\displaystyle\fi}
\newcommand\DS{\displaystyle}

\clearscrheadfoot
\pagestyle{scrheadings}
\thispagestyle{empty}
\ohead{\Classe}
\ihead{\thepage/\pageref*{LastPage}}

\setlist[itemize,1]{label=\textbullet}
\setlist[itemize,2]{label=\textbullet}

\ifTwoColumns
  \geometry{margin=1cm,top=2cm,bottom=3cm,foot=1cm}
  \setlist[enumerate]{leftmargin=*}
  \setlist[itemize]{leftmargin=*}
\else
  \geometry{margin=3cm}
\fi

\makeatletter
\let\@author=\relax
\let\@date=\relax
\renewcommand\maketitle{%
    \begin{center}%
        {\sffamily\huge\bfseries\@title}%
        \ifx\@author\relax\else\par\medskip{\itshape\Large\@author}\fi
        \ifx\@date\relax\else\par\bigskip{\large\@date}\fi
    \end{center}\bigskip
    \ifTwoColumns
        \par\begin{multicols*}{2}%
        \AtEndDocument{\end{multicols*}}%
        \setlength{\columnsep}{5mm}
    \fi
}
\makeatother

\newcounter{ParaNum}
\NewDocumentCommand\Para{smo}{%
  \IfBooleanF{#1}{\refstepcounter{ParaNum}}%
  \paragraph{\IfBooleanF{#1}{{\tiny\arabic{ParaNum}~}}#2\IfNoValueF{#3}{ (#3)}}}

\newcommand\I{i}
\newcommand\mi{i}
\def\me{e}

\def\do#1{\expandafter\undef\csname #1\endcsname}
\docsvlist{Ker,sec,csc,cot,sinh,cosh,tanh,coth,th}
\undef\do

\DeclareMathOperator\ch{ch}
\DeclareMathOperator\sh{sh}
\DeclareMathOperator\th{th}
\DeclareMathOperator\coth{coth}
\DeclareMathOperator\cotan{cotan}
\DeclareMathOperator\argch{argch}
\DeclareMathOperator\argsh{argsh}
\DeclareMathOperator\argth{argth}

\let\epsilon=\varepsilon
\let\phi=\varphi
\let\leq=\leqslant
\let\geq=\geqslant
\let\subsetneq=\varsubsetneq
\let\emptyset=\varnothing

\newcommand{\+}{,\;}

\undef\C
\newcommand\ninf{{n\infty}}
\newcommand\N{\mathbb{N}}
\newcommand\Z{\mathbb{Z}}
\newcommand\Q{\mathbb{Q}}
\newcommand\R{\mathbb{R}}
\newcommand\C{\mathbb{C}}
\newcommand\K{\mathbb{K}}
\newcommand\Ns{\N^*}
\newcommand\Zs{\Z^*}
\newcommand\Qs{\Q^*}
\newcommand\Rs{\R^*}
\newcommand\Cs{\C^*}
\newcommand\Ks{\K^*}
\newcommand\Rp{\R^+}
\newcommand\Rps{\R^+_*}
\newcommand\Rms{\R^-_*}
\newcommand{\Rpinf}{\Rp\cup\Acco{+\infty}}

\undef\B
\newcommand\B{\mathscr{B}}

\undef\P
\DeclareMathOperator\P{\mathbb{P}}
\DeclareMathOperator\E{\mathbb{E}}
\DeclareMathOperator\Var{\mathbb{V}}

\DeclareMathOperator*\PetitO{o}
\DeclareMathOperator*\GrandO{O}
\DeclareMathOperator*\Sim{\sim}
\DeclareMathOperator\Tr{tr}
\DeclareMathOperator\Ima{Im}
\DeclareMathOperator\Ker{Ker}
\DeclareMathOperator\Sp{Sp}
\DeclareMathOperator\Diag{diag}
\DeclareMathOperator\Rang{rang}
\DeclareMathOperator*\Coords{Coords}
\DeclareMathOperator*\Mat{Mat}
\DeclareMathOperator\Pass{Pass}
\DeclareMathOperator\Com{Com}
\DeclareMathOperator\Card{Card}
\DeclareMathOperator\Racines{Racines}
\DeclareMathOperator\Vect{Vect}
\DeclareMathOperator\Id{Id}

\newcommand\DerPart[2]{\frac{\partial #1}{\partial #2}}

\def\T#1{{#1}^T}

\def\pa#1{({#1})}
\def\Pa#1{\left({#1}\right)}
\def\bigPa#1{\bigl({#1}\bigr)}
\def\BigPa#1{\Bigl({#1}\Bigr)}
\def\biggPa#1{\biggl({#1}\biggr)}
\def\BiggPa#1{\Biggl({#1}\Biggr)}

\def\pafrac#1#2{\pa{\frac{#1}{#2}}}
\def\Pafrac#1#2{\Pa{\frac{#1}{#2}}}
\def\bigPafrac#1#2{\bigPa{\frac{#1}{#2}}}
\def\BigPafrac#1#2{\BigPa{\frac{#1}{#2}}}
\def\biggPafrac#1#2{\biggPa{\frac{#1}{#2}}}
\def\BiggPafrac#1#2{\BiggPa{\frac{#1}{#2}}}

\def\cro#1{[{#1}]}
\def\Cro#1{\left[{#1}\right]}
\def\bigCro#1{\bigl[{#1}\bigr]}
\def\BigCro#1{\Bigl[{#1}\Bigr]}
\def\biggCro#1{\biggl[{#1}\biggr]}
\def\BiggCro#1{\Biggl[{#1}\Biggr]}

\def\abs#1{\mathopen|{#1}\mathclose|}
\def\Abs#1{\left|{#1}\right|}
\def\bigAbs#1{\bigl|{#1}\bigr|}
\def\BigAbs#1{\Bigl|{#1}\Bigr|}
\def\biggAbs#1{\biggl|{#1}\biggr|}
\def\BiggAbs#1{\Biggl|{#1}\Biggr|}

\def\acco#1{\{{#1}\}}
\def\Acco#1{\left\{{#1}\right\}}
\def\bigAcco#1{\bigl\{{#1}\bigr\}}
\def\BigAcco#1{\Bigl\{{#1}\Bigr\}}
\def\biggAcco#1{\biggl\{{#1}\biggr\}}
\def\BiggAcco#1{\Biggl\{{#1}\Biggr\}}

\def\ccro#1{\llbracket{#1}\rrbracket}
\def\Dcro#1{\llbracket{#1}\rrbracket}

\def\floor#1{\lfloor#1\rfloor}
\def\Floor#1{\left\lfloor{#1}\right\rfloor}

\def\sEnsemble#1#2{\mathopen\{#1\mid#2\mathclose\}}
\def\bigEnsemble#1#2{\bigl\{#1\bigm|#2\bigr\}}
\def\BigEnsemble#1#2{\Bigl\{#1\Bigm|#2\Bigr\}}
\def\biggEnsemble#1#2{\biggl\{#1\biggm|#2\biggr\}}
\def\BiggEnsemble#1#2{\Biggl\{#1\Biggm|#2\Biggr\}}
\let\Ensemble=\bigEnsemble

\newcommand\IntO[1]{\left]#1\right[}
\newcommand\IntF[1]{\left[#1\right]}
\newcommand\IntOF[1]{\left]#1\right]}
\newcommand\IntFO[1]{\left[#1\right[}

\newcommand\intO[1]{\mathopen]#1\mathclose[}
\newcommand\intF[1]{\mathopen[#1\mathclose]}
\newcommand\intOF[1]{\mathopen]#1\mathclose]}
\newcommand\intFO[1]{\mathopen[#1\mathclose[}

\newcommand\Fn[3]{#1\colon#2\to#3}
\newcommand\CC[1]{\mathscr{C}^{#1}}
\newcommand\D{\mathop{}\!\mathrm{d}}

\newcommand\longto{\longrightarrow}

\undef\M
\newcommand\M[3]{\mathrm{#1}_{#2}\pa{#3}}
\newcommand\MnR{\M{M}{n}{\R}}
\newcommand\MnC{\M{M}{n}{\C}}
\newcommand\MnK{\M{M}{n}{\K}}
\newcommand\GLnR{\M{GL}{n}{\R}}
\newcommand\GLnC{\M{GL}{n}{\C}}
\newcommand\GLnK{\M{GL}{n}{\K}}
\newcommand\DnR{\M{D}{n}{\R}}
\newcommand\DnC{\M{D}{n}{\C}}
\newcommand\DnK{\M{D}{n}{\K}}
\newcommand\SnR{\M{S}{n}{\R}}
\newcommand\AnR{\M{A}{n}{\R}}
\newcommand\OnR{\M{O}{n}{\R}}
\newcommand\SnRp{\mathrm{S}_n^+(\R)}
\newcommand\SnRpp{\mathrm{S}_n^{++}(\R)}

\newcommand\LE{\mathscr{L}(E)}
\newcommand\GLE{\mathscr{GL}(E)}
\newcommand\SE{\mathscr{S}(E)}
\renewcommand\OE{\mathscr{O}(E)}

\newcommand\ImplD{$\Cro\Rightarrow$}
\newcommand\ImplR{$\Cro\Leftarrow$}
\newcommand\InclD{$\Cro\subset$}
\newcommand\InclR{$\Cro\supset$}
\newcommand\notInclD{$\Cro{\not\subset}$}
\newcommand\notInclR{$\Cro{\not\supset}$}

\newcommand\To[1]{\xrightarrow[#1]{}}
\newcommand\Toninf{\To{\ninf}}

\newcommand\Norm[1]{\|#1\|}
\newcommand\Norme{{\Norm{\cdot}}}

\newcommand\Int[1]{\mathring{#1}}
\newcommand\Adh[1]{\overline{#1}}

\newcommand\Uplet[2]{{#1},\dots,{#2}}
\newcommand\nUplet[3]{(\Uplet{{#1}_{#2}}{{#1}_{#3}})}

\newcommand\Fonction[5]{{#1}\left|\begin{aligned}{#2}&\;\longto\;{#3}\\{#4}&\;\longmapsto\;{#5}\end{aligned}\right.}

\DeclareMathOperator\orth{\bot}
\newcommand\Orth[1]{{#1}^\bot}
\newcommand\PS[2]{\langle#1,#2\rangle}

\newcommand{\Tribu}{\mathscr{T}}
\newcommand{\Part}{\mathcal{P}}
\newcommand{\Pro}{\bigPa{\Omega,\Tribu}}
\newcommand{\Prob}{\bigPa{\Omega,\Tribu,\P}}

\newcommand\DEMO{$\spadesuit$}
\newcommand\DUR{$\spadesuit$}

\newenvironment{psmallmatrix}{\left(\begin{smallmatrix}}{\end{smallmatrix}\right)}

% -----------------------------------------------------------------------------

\renewcommand{\Tribu}{\Part(\Omega)}

\begin{document}
\title{Variables al\'eatoires dans un univers fini}
\maketitle

% -----------------------------------------------------------------------------
\section{G\'en\'eralit\'es}

\Para{D\'efinition}[variable al\'eatoire]

Soit $\Prob$ un espace probabilis\'e fini.
Une \emph{variable al\'eatoire} est une application $\Fn{X}{\Omega}{E}$.
Lorsque $E\subset \R$, on parle de \emph{variable al\'eatoire r\'eelle}.

\Para{D\'efinition}[fonction d'une variable al\'eatoire]

Soit $\Prob$ un espace probabilis\'e.
Soit $X$ une variable al\'eatoire \`a valeurs dans $E$.
Soit $\Fn{f}{E}{F}$ une application quelconque.
L'application
\[ \Fonction{f\circ X}{\Omega}{F}{\omega}{f(X(\omega))} \]
est une variable al\'eatoire \`a valeurs dans $F$.
L'usage veut qu'on la note abusivement $f(X)$ au lieu de $f\circ X$.

\Para{D\'efinition}[\'ev\'enements associ\'es \`a une variable al\'eatoire]

Soit $\Prob$ un espace probabilis\'e.
Soit $X$ une variable al\'eatoire \`a valeurs dans $E$.
Pour tout $A\subset E$, on d\'efinit l'\'ev\'enement $\{ X \in A \}$ comme \'etant
\[ X^{-1}(A) = \Ensemble{\omega\in \Omega}{X(w) \in A}. \]
\begin{itemize}
\item
  On notera plus simplement \og$X\in A$\fg{} pour \og$\{ X\in A \}$\fg.
\item
  Si $A = \Acco{a}$, on notera \og$X=a$\fg{} pour \og$\{ X\in A \}$\fg.
\item
  Si $E = \R$ et $A = \intOF{-\infty,a}$, on notera \og$X\leq a$\fg{} pour \og$\{ X\in A \}$\fg, etc.
\end{itemize}

\Para{D\'efinition}[loi d'une variable al\'eatoire]

Soit $\Prob$ un espace probabilis\'e fini.
Soit $X$ une variable al\'eatoire \`a valeurs dans l'ensemble $E$ fini.
L'application
\[ \Fonction{\P_X}{\Part(E)}{[0,1]}{A}{\P(X\in A)} \]
est une probabilit\'e sur $E$, appel\'ee \emph{loi de la variable $X$}.

\Para{Proposition}

Soit $\Prob$ un espace probabilis\'e fini.
Soit $X$ une variable al\'eatoire \`a valeurs dans $E$.
La loi de $X$ est uniquement d\'etermin\'ee par les valeurs $\P(X=x)$ pour $x\in X(\Omega)$.

% -----------------------------------------------------------------------------
\section{Lois usuelles}

\Para{D\'efinition}[loi uniforme]

Une variable $X$ \`a valeurs dans $E$ fini suit la loi uniforme sur $E$
si pour tout $x\in E$, \[ \P(X=x) = \frac{1}{\Card{E}}. \]
On note $X \sim \mathscr{U}(E)$.

\Para{D\'efinition}[loi de Bernoulli]

Une variable $X$ \`a valeurs dans $\{0,1\}$ suit la loi de Bernoulli de param\`etre $p$
si \[ \P(X=1) = p \quad \text{et} \quad \P(X=0) = 1-p. \]
On note $X \sim \mathscr{B}(p)$.

\Para{D\'efinition}[loi binomiale]

Une variable $X$ \`a valeurs dans $\Dcro{0,n}$ suit la loi de binomiales de param\`etres $n$ et $p$
si pour tout $k\in\Dcro{0,n}$, \[ \P(X=k) = \binom{n}{k} p^k (1-p)^{n-k}. \]
On note $X \sim \mathscr{B}(n,p)$.

% -----------------------------------------------------------------------------
\section{Esp\'erance}

\Para{D\'efinition}[esp\'erance d'une variable al\'eatoire]

Soit $X$ une valeur al\'eatoire \`a valeurs dans $E\subset \R$.
L'\emph{esp\'erance de $X$} est
\[ \E(X) = \sum_{\omega\in \Omega} X(\omega) \, \P\bigPa{\acco{\omega}}. \]

\Para{Th\'eor\`eme}[formule de transfert]

Si $X$ est une variable al\'eatoire r\'eelle, alors
\[ \E(X) = \sum_{x\in E} x \, \P(X=x). \]

Plus g\'en\'eralement,
si $X$ est une variable al\'eatoire \`a valeurs dans $E$
et $\Fn{f}{E}{\R}$, alors
\[ \E\bigPa{f(X)} = \sum_{x\in E} f(x) \, \P(X=x). \]

\Para{D\'efinition}

Une variable al\'eatoire est dite \emph{centr\'ee} si son esp\'erance est nulle.

\Para{Proposition}[propri\'et\'es de l'esp\'erance]

Soit $X$ et $Y$ deux variables al\'eatoires r\'eelles et $(a,b) \in \R^2$.
\begin{itemize}
\item
  $\E(aX+bY) = a\,\E(X)+b\,\E(Y)$.
\item
  Si $X$ est \`a valeurs positives, alors $\E(X)\geq0$.
\item
  Si $\P(X\leq Y)=1$, alors $\E(X)\leq \E(Y)$.
\end{itemize}

\Para{Proposition}[in\'egalit\'e de Markov]

Soit $X$ une variable al\'eatoire r\'eelle.
Pour tout $a > 0$, on a
\[ \P(X\geq a) \leq\frac{\E(X)}{a} \]

% -----------------------------------------------------------------------------
\section{Ind\'ependance}

\Para{D\'efinition}

Soit $\Prob$ un espace probabilis\'e fini.
Soit $X$ une variable al\'eatoire \`a valeurs dans $E$ et $Y$ une variable al\'eatoire \`a valeurs dans $F$.
On dit que $X$ et $Y$ sont \emph{ind\'ependantes} si et seulement si
\begin{multline*}
  \forall x\in E \+ \forall y\in F \+ \\
  \P\bigPa{ (X=x)\cap(Y=y) } = \P(X=x) \, \P(Y=y).
\end{multline*}

\Para{Th\'eor\`eme}

Soit $\Prob$ un espace probabilis\'e fini.
Soit $X$ une variable al\'eatoire \`a valeurs dans $E$ et $Y$ une variable al\'eatoire \`a valeurs dans $F$.
Les variables al\'eatoires $X$ et $Y$ sont ind\'ependantes si et seulement si
\begin{multline*}
  \forall A\subset E \+\forall B\subset F \+ \\
  \P\bigPa{ (X\in A)\cap(Y\in B) } = \P(X\in A) \, \P(Y\in B).
\end{multline*}

\Para{Th\'eor\`eme}

Soit $\Prob$ un espace probabilis\'e fini.
Soit $X$ une variable al\'eatoire \`a valeurs dans $E$ et $Y$ une variable al\'eatoire \`a valeurs dans $F$.
Si $X$ et $Y$ sont ind\'ependantes, il en va de m\^eme pour les variables al\'eatoires $f(X)$ et $g(Y)$
o\`u $\Fn{f}{E}{E'}$ et $\Fn{g}{F}{F'}$ sont deux applications quelconques.

\Para{Th\'eor\`eme}

Soit $X$ et $Y$ deux variables al\'eatoires ind\'ependantes.
Alors \[ \E(XY) = \E(X) \, \E(Y). \]

% -----------------------------------------------------------------------------
\section{Variance}

\Para{D\'efinition}[variance d'une variable al\'eatoire]

Soit $X$ une variable al\'eatoire r\'eelle.
Sa \emph{variance} est $\E\bigPa{(X-\E(X))^2}$ et on la note $\Var(X)$.
Son \emph{\'ecart-type} est $\sigma(X) =\sqrt{\Var(X)}$.

\Para{Proposition}[in\'egalit\'e de Bienaym\'e-Tchebychev]

Soit $X$ une variable al\'eatoire r\'eelle.
Pour tout $\alpha>0$, on a
\[ \P{} \bigPa{ \abs{X-\E(X)} \geq{} \alpha} \leq{} \frac{\Var(X)}{\alpha^2} \]

\Para{D\'efinition}[covariance de deux variables al\'eatoires]

Soit $X$ et $Y$ deux variables al\'eatoires r\'eelles.
Leur \emph{covariance} est $\E\bigPa{ (X-\E(X)) (Y-\E(Y)) }$ et on la note $\mathrm{Cov}(X,Y)$.

\Para{Propri\'et\'es}

Soit $X$ et $Y$ deux variables al\'eatoires r\'eelles.
\begin{itemize}
\item
  $\Var(aX+b) = a^2\Var(X)$.
\item
  Si $X$ et $Y$ sont ind\'ependantes, alors $\Var(X+Y) = \Var(X)+\Var(Y)$.
\item
  $\Var(X+Y) = \Var(X) + \Var(Y) + 2\mathrm{Cov}(X,Y)$.
\end{itemize}

\Para{D\'efinition}[corr\'elation]

Soit $X$ et $Y$ deux variables al\'eatoires r\'eelles de variance non nulles.
Leur \emph{coefficient de corr\'elation} est
\[ \rho(X,Y) = \frac{\mathrm{Cov}(X,Y)}{\sigma(X)\sigma(Y)} \]

\Para{Propri\'et\'es}

Soit $X$ et $Y$ deux variables al\'eatoires r\'eelles de variance non nulles.
\begin{itemize}
\item
  $\rho(X,Y) \in{} \intF{-1,1}$
\item
  $\rho(X,Y) = \pm1$ si et seulement si il existe deux r\'eels $a$ et $b$ tels que l'\'ev\'enement $(Y=aX+b)$ soit certain.
\item
  Si $X$ et $Y$ sont ind\'ependantes, alors $\rho(X,Y) = 0$. On dit qu'elles sont d\'ecorr\'el\'ees.
  La r\'eciproque est fausse.
\end{itemize}

% -----------------------------------------------------------------------------
\section{Exercices}

% -----------------------------------------------------------------------------
\par\pagebreak[1]\par
\paragraph{Exercice 1 (lois usuelles)}%
\hfill{\tiny 2631}%
\begingroup~

\begin{enumerate}
\item
  Soit $X$ une variable al\'eatoire suivant une loi uniforme sur $\Dcro{a,b}$.
  D\'eterminer son esp\'erance et sa variance.
\item
  M\^eme questions pour la loi de Bernoulli.
\item
  M\^eme questions pour la loi binomiale.
\end{enumerate}
\endgroup

% -----------------------------------------------------------------------------
\par\pagebreak[1]\par
\paragraph{Exercice 2}%
\hfill{\tiny 6817}%
\begingroup~

Soit $X_1, \dots, X_n$ des variables al\'eatoires ind\'ependantes de loi de Bernoulli de param\`etre $p$.
On pose $X = \sum_{k=1}^n X_k$.
\begin{enumerate}
\item
  D\'eterminer la loi de $X$
\item
  Retrouver l'esp\'erance et la variance de la loi binomiale.
\end{enumerate}
\endgroup

% -----------------------------------------------------------------------------
\par\pagebreak[1]\par
\paragraph{Exercice 3}%
\hfill{\tiny 1014}%
\begingroup~

D'apr\`es les statistiques, un pi\'eton marchant sur la bande d'arr\^et
d'urgence d'une autoroute a une probabilit\'e de $8\%$ d'\^etre renvers\'e
par une voiture chaque minute.
Monsieur X, quelque peu atteint, a d\'ecid\'e de faire avec un ami
le pari stupide suivant:
s'il r\'eussit \`a marcher un quart d'heure sur la bande d'arr\^et d'urgence
sans \^etre victime d'un accident, son ami lui devra $200$ Euros.
\`A combien monsieur $X$ estime-t-il sa vie?
\endgroup

% -----------------------------------------------------------------------------
\par\pagebreak[1]\par
\paragraph{Exercice 4}%
\hfill{\tiny 6513}%
\begingroup~

Pour d\'eterminer la note de fin d'ann\'ee, un professeur proc\`ede ainsi:
il lance deux d\'es, et consid\`ere la plus petite valeur obtenue.
Il d\'efinit alors la variable al\'eatoire $N$, valant $3$ fois la
plus petite valeur obtenue.
D\'ecrire la loi de $N$, puis calculer son esp\'erance et son \'ecart-type.
\endgroup

% -----------------------------------------------------------------------------
\par\pagebreak[1]\par
\paragraph{Exercice 5}%
\hfill{\tiny 0978}%
\begingroup~

Monsieur Duchmol affirme que, gr\^ace \`a son ordinateur, il peut pr\'edire
le sexe des enfants \`a na\^itre. Pour cette pr\'ediction, il ne demande que
5 Euros, destin\'es \`a couvrir les frais de gestion; de plus, pour
\og prouver\fg{} sa bonne foi, il s'engage \`a rembourser int\'egralement
en cas de pr\'ediction erron\'ee.
\begin{enumerate}
\item
  Soit $X$ le gain de monsieur Duchmol; \'ecrire la loi de probabilit\'e de $X$.
\item
  Si monsieur Duchmol trouve 1000 na\"ifs, combien peut-il esp\'erer gagner?
\end{enumerate}
\endgroup

% -----------------------------------------------------------------------------
\par\pagebreak[1]\par
\paragraph{Exercice 6}%
\hfill{\tiny 9473}%
\begingroup~

Casimir a une technique bien particuli\`ere pour choisir quel nouveau CD il va acheter.
Il commence par choisir un CD au hasard, et l'ach\`ete s'il lui pla\^it,
et le repose dans le cas contraire.
Or Casimir est difficile: il n'aime que $1\%$ des CDs.
Bien s\^ur, tant qu'il n'a pas trouv\'e de CD \`a sa convenance, il recommence l'op\'eration.
\begin{enumerate}
\item
  Soit $N$ le nombre de CDs que Casimir regarde avant de se d\'ecider.
  Calculer $\P(N=k)$.
\item
  D\'eterminer l'esp\'erance et l'\'ecart-type de $N$.
\item
  En fait, Casimir, toujours curieux, lance deux d\'es quand le CD ne lui pla\^it pas.
  S'il obtient deux as, il prend le CD quand m\^eme, se disant qu'il y a l\`a
  un signe. Que vaut maintenant l'esp\'erance de $N$?
\end{enumerate}
\endgroup

% -----------------------------------------------------------------------------
\par\pagebreak[1]\par
\paragraph{Exercice 7}%
\hfill{\tiny 2517}%
\begingroup~

Monica et Chandler jouent au baby-foot.
Monica, plus dou\'ee que Chandler au baby-foot,
gagne chaque manche avec une probabilit\'e de $80\%$.

Quelle est la probabilit\'e pour que Monica gagne la partie, i.e.
gagne $5$ manches avant Chandler?
\endgroup

% -----------------------------------------------------------------------------
\par\pagebreak[1]\par
\paragraph{Exercice 8}%
\hfill{\tiny 8023}%
\begingroup~

Alice et Bob jouent aux d\'es.
Normalement, Alice utilise un d\'e \'equilibr\'e, mais il lui arrive de tricher,
environ une fois sur cent, en utilisant un d\'e pip\'e qui sort un as deux fois
sur trois. Une fois que la partie est commenc\'ee, Alice ne peut plus
changer de d\'e sans que Bob s'en aper\c coive.

Sachant qu'Alice a obtenu 9 as sur les 20 derniers lancers, quelle est la
probabilit\'e pour que le prochain lancer donne un as?
\endgroup

% -----------------------------------------------------------------------------
\par\pagebreak[1]\par
\paragraph{Exercice 9 (lois usuelles)}%
\hfill{\tiny 4017}%
\begingroup~

\begin{enumerate}
\item
  Un automobiliste rencontre successivement 5 feux de circulation ind\'ependants sur le boulevard de Strasbourg.
  La probabilit\'e qu'un feu soit vert est de $1/2$.
  On note $X$ le nombre de feux verts.
  D\'eterminer la loi de $X$, son esp\'erance et sa variance.
\item
  Un parking souterrain contient 20 scooters \`a trois roues, 20 motos et 20 voitures.
  On choisit un v\'ehicule au hasard, et on note $X$ le nombre de roues de ce v\'ehicule.
  D\'eterminer la loi de $X$, son esp\'erance, et sa variance.
\item
  Une \'etude statistique a permis de d\'eterminer que 10\% de la population est gauch\`ere.
  Quelle est la probabilit\'e qu'un groupe de 8 personnes contienne un seul gaucher?
  Au plus deux gauchers?
\item
  Le stock d'un fournisseur de lasagnes contient une proportion $p = 49/1000$
  de barquettes de lasagnes \`a base de viande de cheval.
  Un contr\^oleur examine des barquettes de lasagnes chez ce fournisseur.
  Combien doit-il contr\^oler de barquettes en moyenne pour qu'il trouve au moins une barquette \`a base de viande de cheval?
\item
  Une jarre contient 12 scorpions, 27 araign\'ees et 56 blattes.
  On choisit une araign\'ee au hasard parmi les 27 araign\'ees dans la jarre.
  $X$ est le nombre de pattes de l'animal choisi.
  D\'eterminer la loi de $X$.
\end{enumerate}
\endgroup

% -----------------------------------------------------------------------------
\par\pagebreak[1]\par
\paragraph{Exercice 10}%
\hfill{\tiny 8832}%
\begingroup~

Harry P., apprenti-sorcier de son \'etat, sort en moyenne deux soirs par semaine.
Comme les lendemains matins sont plut\^ot difficile, il soulage alors ses maux de t\^ete par un sortil\`ege.
Cependant, ainsi que Hermione G. l'en avait averti, ce sortil\`ege poss\`ede un effet secondaire parfois g\^enant:
une fois sur cent, al\'eatoirement, le sorcier se retrouve transform\'e pour la journ\'ee en une ic\^one disco, dans son cas un \emph{Village People}.

Quelle est la probabilit\'e que Harry P., sur le cours d'une ann\'ee enti\`ere, se retrouve au moins 3 jours sous cette forme?
\endgroup

% -----------------------------------------------------------------------------
\par\pagebreak[1]\par
\paragraph{Exercice 11 (loi hyperg\'eom\'etrique)}%
\hfill{\tiny 1725}%
\begingroup~

Une urne contient $a$ boules blanches et $b$ boules noires.
On tire une poign\'ee de $n$ boules dans l'urne, avec $(a,b) \in(\N^*)^2$ et $n \in\Dcro{1,a+b}$.
On appelle $X$ le nombre de boules blanches dans la poign\'ee.
\begin{enumerate}
\item
  D\'eterminer le support de $X$.
\item
  D\'eterminer la loi de $X$.
\item
  Calculer l'esp\'erance de $X$.
\item
  Calculer l'esp\'erance de $X(X-1)$ puis la variance de $X$.
\item
  Comparer l'esp\'erance et la variance de $X$ \`a celle d'une loi binomiale de param\`etres
  $n$ et $a/(a+b)$. Commentaire?
\end{enumerate}
\endgroup

% -----------------------------------------------------------------------------
\par\pagebreak[1]\par
\paragraph{Exercice 12 (marche al\'eatoire)}%
\hfill{\tiny 4831}%
\begingroup~

Un mobile se d\'eplace de fa\c con al\'eatoire sur un axe gradu\'e.
\`A l'instant $0$, il est \`a l'origine.
\`A chaque instant entier, il se d\'eplace d'une unit\'e vers la droite avec la probabilit\'e $p\in\intO{0,1}$
ou d'un pas vers la gauche avec la probabilit\'e $q=1-p$,
et ce de fa\c con ind\'ependante.
On note $X_n$ son abscisse apr\`es $n$ pas.
\begin{enumerate}
\item
  Soit $D_n$ la variable al\'eatoire \'egale au nombre de pas vers la droite.
  Quelle est la loi de $D_n$? Exprimer $X_n$ en fonction de $D_n$.
\item
  En d\'eduire l'esp\'erance et la variance de $X_n$.
  Pour quelle valeur de $p$ la variable $X_n$ est-elle centr\'ee?
  Interpr\'eter.
\item
  Reprendre l'exercice avec une autre m\'ethode:
  on note, pour $n\geq1$, $Y_n = X_n - X_{n-1}$.

  \begin{enumerate}
  \item
    D\'eterminer la loi de $Y_n$.
  \item
    Justifier l'ind\'ependance de $Y_1, \dots, Y_n$.
  \item
    En d\'eduire l'esp\'erance et la variance de $X_n$.
  \end{enumerate}
\end{enumerate}
\endgroup

% -----------------------------------------------------------------------------
\par\pagebreak[1]\par
\paragraph{Exercice 13}%
\hfill{\tiny 7765}%
\begingroup~

Un microorganisme se reproduit par divisions \`a intervalles r\'eguliers.
Apr\`es une division, la division suivante se produit soit une heure apr\`es
(division courte, avec la probabilit\'e $p$)
soit deux heures apr\`es (division lente, avec la probabilit\'e $q=1-p$),
ind\'ependamment de l'histoire du microorganisme.

On isole un microorganisme,
et on note $X_n$ le nombre de microorganismes pr\'esents au bout de $n$ heures.
On cherche \`a calculer $\E(X_n)$.
\begin{enumerate}
\item
  Pour cela, notons $B_n$ le nombre de microorganismes pr\'esents au temps $n$
  qui sont au milieu d'une division longue, et $A_n = X_n - B_n$.
  Exprimer $\E(A_{n+1})$ et $\E(B_{n+1})$ en fonction de $\E(A_n)$ et de $\E(B_n)$.
\item
  D\'eterminer $\E(X_n)$.
\end{enumerate}
\endgroup

% -----------------------------------------------------------------------------
\par\pagebreak[1]\par
\paragraph{Exercice 14}%
\hfill{\tiny 0198}%
\begingroup~

Deux avions $A_1$ et $A_2$ poss\`edent respectivement deux et quatre moteurs.
Chaque moteur a la probabilit\'e $p\in\intO{0,1}$ de tomber en panne
et les moteurs sont ind\'ependants les uns des autres.
Les deux avions partent pour un m\^eme trajet.
Chacun des avions arrive \`a destination si et seulement si strictement plus de la moiti\'e de ses moteurs reste en \'etat de marche.
Vous partez pour cette destination.
Quel avion choisissez-vous?
\endgroup

% -----------------------------------------------------------------------------
\par\pagebreak[1]\par
\paragraph{Exercice 15}%
\hfill{\tiny 4329}%
\begingroup~

Soit $X$ une variable al\'eatoire suivant une loi binomiale $\mathscr{B}(n,p)$.
Calculer $\E(2^X)$ et $\E(\frac{1}{1+X})$.
\endgroup

% -----------------------------------------------------------------------------
\par\pagebreak[1]\par
\paragraph{Exercice 16}%
\hfill{\tiny 0737}%
\begingroup~

Une urne contient deux boules blanches et $n-2$ boules noires.
On tire les boules successivement, sans remise.
On appelle $X$ le rang de sortie de la premi\`ere boule blanche,
$Y$ le nombre de boules noires restantes \`a ce moment dans l'urne
et $Z$ le rang de sortie de la seconde boule blanche.
\begin{enumerate}
\item
  D\'eterminer la loi de $X$ et son esp\'erance.
\item
  Exprimer $Y$ en fonction de $X$ et calculer $\E(Y)$.
\item
  Trouver un lien entre $Z$ et $X$ et en d\'eduire la loi de~$Z$.
\end{enumerate}
\endgroup

\end{document}
