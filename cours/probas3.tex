% autogenerated by ytex.rs

\documentclass{scrartcl}

\usepackage[francais]{babel}
\usepackage{geometry}
\usepackage{scrpage2}
\usepackage{lastpage}
\usepackage{multicol}
\usepackage{etoolbox}
\usepackage{xparse}
\usepackage{enumitem}
% \usepackage{csquotes}
\usepackage{amsmath}
\usepackage{amsfonts}
\usepackage{amssymb}
\usepackage{mathrsfs}
\usepackage{stmaryrd}
\usepackage{dsfont}
\usepackage{eurosym}
% \usepackage{numprint}
% \usepackage[most]{tcolorbox}
% \usepackage{tikz}
% \usepackage{tkz-tab}
\usepackage[unicode]{hyperref}
\usepackage[ocgcolorlinks]{ocgx2}

\let\ifTwoColumns\iftrue
\def\Classe{$\Psi$2019--2020}

% reproducible builds
% LuaTeX: \pdfvariable suppressoptionalinfo 1023 \relax
\pdfinfoomitdate=1
\pdftrailerid{}

\newif\ifDisplaystyle
\everymath\expandafter{\the\everymath\ifDisplaystyle\displaystyle\fi}
\newcommand\DS{\displaystyle}

\clearscrheadfoot
\pagestyle{scrheadings}
\thispagestyle{empty}
\ohead{\Classe}
\ihead{\thepage/\pageref*{LastPage}}

\setlist[itemize,1]{label=\textbullet}
\setlist[itemize,2]{label=\textbullet}

\ifTwoColumns
  \geometry{margin=1cm,top=2cm,bottom=3cm,foot=1cm}
  \setlist[enumerate]{leftmargin=*}
  \setlist[itemize]{leftmargin=*}
\else
  \geometry{margin=3cm}
\fi

\makeatletter
\let\@author=\relax
\let\@date=\relax
\renewcommand\maketitle{%
    \begin{center}%
        {\sffamily\huge\bfseries\@title}%
        \ifx\@author\relax\else\par\medskip{\itshape\Large\@author}\fi
        \ifx\@date\relax\else\par\bigskip{\large\@date}\fi
    \end{center}\bigskip
    \ifTwoColumns
        \par\begin{multicols*}{2}%
        \AtEndDocument{\end{multicols*}}%
        \setlength{\columnsep}{5mm}
    \fi
}
\makeatother

\newcounter{ParaNum}
\NewDocumentCommand\Para{smo}{%
  \IfBooleanF{#1}{\refstepcounter{ParaNum}}%
  \paragraph{\IfBooleanF{#1}{{\tiny\arabic{ParaNum}~}}#2\IfNoValueF{#3}{ (#3)}}}

\newcommand\I{i}
\newcommand\mi{i}
\def\me{e}

\def\do#1{\expandafter\undef\csname #1\endcsname}
\docsvlist{Ker,sec,csc,cot,sinh,cosh,tanh,coth,th}
\undef\do

\DeclareMathOperator\ch{ch}
\DeclareMathOperator\sh{sh}
\DeclareMathOperator\th{th}
\DeclareMathOperator\coth{coth}
\DeclareMathOperator\cotan{cotan}
\DeclareMathOperator\argch{argch}
\DeclareMathOperator\argsh{argsh}
\DeclareMathOperator\argth{argth}

\let\epsilon=\varepsilon
\let\phi=\varphi
\let\leq=\leqslant
\let\geq=\geqslant
\let\subsetneq=\varsubsetneq
\let\emptyset=\varnothing

\newcommand{\+}{,\;}

\undef\C
\newcommand\ninf{{n\infty}}
\newcommand\N{\mathbb{N}}
\newcommand\Z{\mathbb{Z}}
\newcommand\Q{\mathbb{Q}}
\newcommand\R{\mathbb{R}}
\newcommand\C{\mathbb{C}}
\newcommand\K{\mathbb{K}}
\newcommand\Ns{\N^*}
\newcommand\Zs{\Z^*}
\newcommand\Qs{\Q^*}
\newcommand\Rs{\R^*}
\newcommand\Cs{\C^*}
\newcommand\Ks{\K^*}
\newcommand\Rp{\R^+}
\newcommand\Rps{\R^+_*}
\newcommand\Rms{\R^-_*}
\newcommand{\Rpinf}{\Rp\cup\Acco{+\infty}}

\undef\B
\newcommand\B{\mathscr{B}}

\undef\P
\DeclareMathOperator\P{\mathbb{P}}
\DeclareMathOperator\E{\mathbb{E}}
\DeclareMathOperator\Var{\mathbb{V}}

\DeclareMathOperator*\PetitO{o}
\DeclareMathOperator*\GrandO{O}
\DeclareMathOperator*\Sim{\sim}
\DeclareMathOperator\Tr{tr}
\DeclareMathOperator\Ima{Im}
\DeclareMathOperator\Ker{Ker}
\DeclareMathOperator\Sp{Sp}
\DeclareMathOperator\Diag{diag}
\DeclareMathOperator\Rang{rang}
\DeclareMathOperator*\Coords{Coords}
\DeclareMathOperator*\Mat{Mat}
\DeclareMathOperator\Pass{Pass}
\DeclareMathOperator\Com{Com}
\DeclareMathOperator\Card{Card}
\DeclareMathOperator\Racines{Racines}
\DeclareMathOperator\Vect{Vect}
\DeclareMathOperator\Id{Id}

\newcommand\DerPart[2]{\frac{\partial #1}{\partial #2}}

\def\T#1{{#1}^T}

\def\pa#1{({#1})}
\def\Pa#1{\left({#1}\right)}
\def\bigPa#1{\bigl({#1}\bigr)}
\def\BigPa#1{\Bigl({#1}\Bigr)}
\def\biggPa#1{\biggl({#1}\biggr)}
\def\BiggPa#1{\Biggl({#1}\Biggr)}

\def\pafrac#1#2{\pa{\frac{#1}{#2}}}
\def\Pafrac#1#2{\Pa{\frac{#1}{#2}}}
\def\bigPafrac#1#2{\bigPa{\frac{#1}{#2}}}
\def\BigPafrac#1#2{\BigPa{\frac{#1}{#2}}}
\def\biggPafrac#1#2{\biggPa{\frac{#1}{#2}}}
\def\BiggPafrac#1#2{\BiggPa{\frac{#1}{#2}}}

\def\cro#1{[{#1}]}
\def\Cro#1{\left[{#1}\right]}
\def\bigCro#1{\bigl[{#1}\bigr]}
\def\BigCro#1{\Bigl[{#1}\Bigr]}
\def\biggCro#1{\biggl[{#1}\biggr]}
\def\BiggCro#1{\Biggl[{#1}\Biggr]}

\def\abs#1{\mathopen|{#1}\mathclose|}
\def\Abs#1{\left|{#1}\right|}
\def\bigAbs#1{\bigl|{#1}\bigr|}
\def\BigAbs#1{\Bigl|{#1}\Bigr|}
\def\biggAbs#1{\biggl|{#1}\biggr|}
\def\BiggAbs#1{\Biggl|{#1}\Biggr|}

\def\acco#1{\{{#1}\}}
\def\Acco#1{\left\{{#1}\right\}}
\def\bigAcco#1{\bigl\{{#1}\bigr\}}
\def\BigAcco#1{\Bigl\{{#1}\Bigr\}}
\def\biggAcco#1{\biggl\{{#1}\biggr\}}
\def\BiggAcco#1{\Biggl\{{#1}\Biggr\}}

\def\ccro#1{\llbracket{#1}\rrbracket}
\def\Dcro#1{\llbracket{#1}\rrbracket}

\def\floor#1{\lfloor#1\rfloor}
\def\Floor#1{\left\lfloor{#1}\right\rfloor}

\def\sEnsemble#1#2{\mathopen\{#1\mid#2\mathclose\}}
\def\bigEnsemble#1#2{\bigl\{#1\bigm|#2\bigr\}}
\def\BigEnsemble#1#2{\Bigl\{#1\Bigm|#2\Bigr\}}
\def\biggEnsemble#1#2{\biggl\{#1\biggm|#2\biggr\}}
\def\BiggEnsemble#1#2{\Biggl\{#1\Biggm|#2\Biggr\}}
\let\Ensemble=\bigEnsemble

\newcommand\IntO[1]{\left]#1\right[}
\newcommand\IntF[1]{\left[#1\right]}
\newcommand\IntOF[1]{\left]#1\right]}
\newcommand\IntFO[1]{\left[#1\right[}

\newcommand\intO[1]{\mathopen]#1\mathclose[}
\newcommand\intF[1]{\mathopen[#1\mathclose]}
\newcommand\intOF[1]{\mathopen]#1\mathclose]}
\newcommand\intFO[1]{\mathopen[#1\mathclose[}

\newcommand\Fn[3]{#1\colon#2\to#3}
\newcommand\CC[1]{\mathscr{C}^{#1}}
\newcommand\D{\mathop{}\!\mathrm{d}}

\newcommand\longto{\longrightarrow}

\undef\M
\newcommand\M[3]{\mathrm{#1}_{#2}\pa{#3}}
\newcommand\MnR{\M{M}{n}{\R}}
\newcommand\MnC{\M{M}{n}{\C}}
\newcommand\MnK{\M{M}{n}{\K}}
\newcommand\GLnR{\M{GL}{n}{\R}}
\newcommand\GLnC{\M{GL}{n}{\C}}
\newcommand\GLnK{\M{GL}{n}{\K}}
\newcommand\DnR{\M{D}{n}{\R}}
\newcommand\DnC{\M{D}{n}{\C}}
\newcommand\DnK{\M{D}{n}{\K}}
\newcommand\SnR{\M{S}{n}{\R}}
\newcommand\AnR{\M{A}{n}{\R}}
\newcommand\OnR{\M{O}{n}{\R}}
\newcommand\SnRp{\mathrm{S}_n^+(\R)}
\newcommand\SnRpp{\mathrm{S}_n^{++}(\R)}

\newcommand\LE{\mathscr{L}(E)}
\newcommand\GLE{\mathscr{GL}(E)}
\newcommand\SE{\mathscr{S}(E)}
\renewcommand\OE{\mathscr{O}(E)}

\newcommand\ImplD{$\Cro\Rightarrow$}
\newcommand\ImplR{$\Cro\Leftarrow$}
\newcommand\InclD{$\Cro\subset$}
\newcommand\InclR{$\Cro\supset$}
\newcommand\notInclD{$\Cro{\not\subset}$}
\newcommand\notInclR{$\Cro{\not\supset}$}

\newcommand\To[1]{\xrightarrow[#1]{}}
\newcommand\Toninf{\To{\ninf}}

\newcommand\Norm[1]{\|#1\|}
\newcommand\Norme{{\Norm{\cdot}}}

\newcommand\Int[1]{\mathring{#1}}
\newcommand\Adh[1]{\overline{#1}}

\newcommand\Uplet[2]{{#1},\dots,{#2}}
\newcommand\nUplet[3]{(\Uplet{{#1}_{#2}}{{#1}_{#3}})}

\newcommand\Fonction[5]{{#1}\left|\begin{aligned}{#2}&\;\longto\;{#3}\\{#4}&\;\longmapsto\;{#5}\end{aligned}\right.}

\DeclareMathOperator\orth{\bot}
\newcommand\Orth[1]{{#1}^\bot}
\newcommand\PS[2]{\langle#1,#2\rangle}

\newcommand{\Tribu}{\mathscr{T}}
\newcommand{\Part}{\mathcal{P}}
\newcommand{\Pro}{\bigPa{\Omega,\Tribu}}
\newcommand{\Prob}{\bigPa{\Omega,\Tribu,\P}}

\newcommand\DEMO{$\spadesuit$}
\newcommand\DUR{$\spadesuit$}

\newenvironment{psmallmatrix}{\left(\begin{smallmatrix}}{\end{smallmatrix}\right)}

% -----------------------------------------------------------------------------

\usepackage{dsfont}

\begin{document}
\title{Probabilit\'es discr\`etes}
\maketitle

% -----------------------------------------------------------------------------
\section{Tribus}

\Para{D\'efinition}

Soit $\Omega$ un ensemble.
Une \emph{tribu} (ou $\sigma$-alg\`ebre) sur $\Omega$ est une partie $\Tribu$ de $\Part(\Omega)$ v\'erifiant les propri\'et\'es suivantes:
\begin{itemize}
\item
  $\emptyset\in\Tribu$
\item
  $\forall A \in\Tribu$, $\Omega\setminus A \in\Tribu$
\item
  Si $(A_n)_{n\in \N}$ est une suite d'\'el\'ements de $\Tribu$, alors $\bigcup_{n\in \N} A_n\in\Tribu$.
\end{itemize}

\Para{Proposition}

Soit $\Omega$ un ensemble.
\begin{itemize}
\item
  $\{ \emptyset, \Omega\}$ est une tribu sur $\Omega$.
\item
  $\Part(\Omega)$ est une tribu sur $\Omega$.
\end{itemize}

\Para{D\'efinitions}

Soit $\Omega$ un ensemble non vide et $\Tribu$ une tribu sur $\Omega$.
\begin{itemize}
\item
  Le couple $\Pro$ s'appelle un \emph{espace probabilisable}.
\item
  Un \emph{\'ev\'enement} est un \'el\'ement de $\Tribu$.
\end{itemize}

\Para{Proposition}

Soit $\Pro$ un espace probabilisable.
\begin{itemize}
\item
  $\emptyset$ est un \'ev\'enement, dit \og \'ev\'enement impossible\fg.
\item
  $\Omega$ est un \'ev\'enement, dit \og \'ev\'enement certain\fg.
\item
  Si $A$ est un \'ev\'enement, $\bar A = \Omega\setminus A$ est un \'ev\'enement, dit \og non-$A$\fg.
\item
  Si $A$ et $B$ sont des \'ev\'enements, alors $A\setminus B$ est un \'ev\'enement.
\item
  Si $A_1, \dots, A_n$ sont des \'ev\'enements, alors

  \begin{itemize}
  \item
    $\bigcup_{i=1}^n A_i$ est un \'ev\'enement appel\'e \og disjonction de $A_1, \dots, A_n$\fg.
  \item
    $\bigcap_{i=1}^n A_i$ est un \'ev\'enement appel\'e \og conjonction de $A_1, \dots, A_n$\fg.
  \end{itemize}
\item
  Si $(A_n)_{n\in \N}$ est une suite d'\'ev\'enements, alors

  \begin{itemize}
  \item
    $\bigcup_{n\in \N} A_n$ est un \'ev\'enement appel\'e \og disjonction de $A_n$ pour $n\in \N$\fg.
  \item
    $\bigcap_{n\in \N} A_n$ est un \'ev\'enement appel\'e \og conjonction de $A_n$ pour $n\in \N$\fg.
  \end{itemize}
\end{itemize}

% -----------------------------------------------------------------------------
\section{Probabilit\'e}

\subsection{G\'en\'eralit\'es}

\Para{D\'efinition}

Soit $\Pro$ un espace probabilisable.
Soit $A$ et $B$ deux \'ev\'enements.
On dit que $A$ et $B$ sont \emph{incompatibles} si et seulement si $A\cap B = \emptyset$.

\Para{D\'efinition}

Soit $\Pro$ un espace probabilisable.
Une \emph{probabilit\'e} sur $\Pro$ est une application
$\Fn{\P}{\Tribu}{[0,1]}$ telle que
\begin{enumerate}
\item
  $\P(\Omega) = 1$
\item
  Pour toute suite d'\'ev\'enements $(A_n)_{n\in \N}$ deux \`a deux incompatibles,
  la s\'erie $\sum_n \P(A_n)$ converge et
  \[ \P\left( \bigcup_{n\in \N} A_n \right) = \sum_{n\in \N}\P(A_n) \]
  Cette propri\'et\'e s'appelle la \emph{$\sigma$-additivit\'e}.
\end{enumerate}

\Para{Remarque}

Pourquoi ne pas syst\'ematiquement choisir $\Tribu = \Part(\Omega)$,
ce qui \'eviterait de parler de tribu et simplifierait sensiblement les d\'efinitions?
Parce que cela ne fonctionne malheureusement pas, cf. le th\'eor\`eme ci-dessous.

\Para{Th\'eor\`eme}[hors-programme]

Soit $\Omega= \intF{0,1}$ et $\Tribu = \Part(\Omega)$.
Il n'existe pas de probabilit\'e $\P$ sur $\Pro$ telle que
\[ \forall(a,b) \in\intF{0,1}^2 \+ a\leq b \implies \P(\intF{a,b}) = b - a. \]

Par contre, il existe une tribu $\mathcal{B}$ (dite tribu des bor\'eliens) sur $\Omega$
contenant tous les intervalles de la forme $\intF{a,b}$
et une probabilit\'e $\P$ sur $(\Omega,\mathcal{B})$ telle que
\[ \forall(a,b) \in\intF{0,1}^2 \+ a\leq b \implies \P(\intF{a,b}) = b - a. \]

\Para{D\'efinition}

Un \emph{espace probabilis\'e} est un triplet $\Prob$
o\`u $\Pro$ est un espace probabilisable
et $\P$ une probabilit\'e $\P$ sur cet espace.

\Para{D\'efinitions}

Soit $\Prob$ un espace probabilis\'e.
Soit $A$ un \'ev\'enement.
\begin{itemize}
\item
  $A$ est dit \emph{n\'egligeable} si et seulement si $\P(A) = 0$.
\item
  $A$ est dit \emph{presque certain} si et seulement si $\P(A) = 1$.
\end{itemize}

\subsection{Propri\'et\'es}

\Para{Proposition}

Soit $\Prob$ un espace probabilis\'e, $A$ et $B$ deux \'ev\'enements.
On a
\begin{enumerate}
\item
  $\P(\emptyset) = 0$.
\item
  $\P(A\cup B) = \P(A) + \P(B)$ si $A\cap B = \emptyset$.
\item
  $\P(\bar A) = 1 - \P(A)$.
\item
  $\P(A\cup B) = \P(A) + \P(B) - \P(A\cap B)$.
\item
  $\P(A)\leq \P(B)$ si $A\subset B$.
\end{enumerate}

\Para{Proposition}[sous-additivit\'e finie]

Soit $\Prob$ un espace probabilis\'e.
Soit $\nUplet{A}{0}{n}$ des \'ev\'enements.
Alors
\[ \P{} \Pa{\bigcup_{k=0}^n A_k} \leq{} \sum_{k=0}^n \P(A_k) \]

\subsection{Th\'eor\`emes de continuit\'e}

\Para{D\'efinitions}

Soit $\Prob$ un espace probabilis\'e.
Soit $(A_n)_{n\in \N}$ est une suite d'\'ev\'enements, c.-\`a-d. une suite \`a valeurs dans $\Tribu$.
\begin{itemize}
\item
  On dit que $(A_n)_{n\in \N}$ est \emph{croissante} (au sens de l'inclusion)
  si et seulement si $\forall n\in \N$, $A_n \subset{} A_{n+1}$.
\item
  On dit que $(A_n)_{n\in \N}$ est \emph{d\'ecroissante} (au sens de l'inclusion)
  si et seulement si $\forall n\in \N$, $A_{n+1} \subset{} A_n$.
\end{itemize}

\Para{Th\'eor\`eme}[continuit\'e croissante]

Soit $\Prob$ un espace probabilis\'e.
Soit $(A_n)_{n\in \N}$ une suite croissante d'\'ev\'enements.
Alors
\[ \P(A_n) \Toninf \P\left(\bigcup_{n\in \N} A_n \right). \]

\Para{Th\'eor\`eme}[continuit\'e d\'ecroissante]

Soit $\Prob$ un espace probabilis\'e.
Soit $(A_n)_{n\in \N}$ une suite d\'ecroissante d'\'ev\'enements.
Alors
\[ \P(A_n) \Toninf \P\left(\bigcap_{n\in \N} A_n \right). \]

\Para{Th\'eor\`eme}[sous-additivit\'e]

Soit $\Prob$ un espace probabilis\'e.
Soit $(A_n)_{n\in \N}$ une suite d'\'ev\'enements.
On suppose que la s\'erie de terme g\'en\'eral $\P(A_n)$ converge.
Alors
\[ \P\Pa{\bigcup_{n\in \N} A_n} \leq{} \sum_{n\geq0} \P(A_n). \]

\subsection{D\'efinition d'une probabilit\'e sur un univers d\'enombrable}

\Para{Th\'eor\`eme}

Soit $\Omega$ un ensemble infini \emph{d\'enombrable}.
On peut \'ecrire $\Omega= \Ensemble{w_n}{n\in \N}$ o\`u $(\omega_n)$ est une suite injective.
Soit $(p_n)$ une suite de r\'eels positifs telle que la s\'erie $\sum_{n\in \N} p_n$ converge et soit de somme $1$.
Alors il existe une unique probabilit\'e sur $\bigl(\Omega, \Part(\Omega) \bigr)$ telle que
\[ \forall n\in \N\+ \P\bigl(\{ \omega_n \}\bigr) = p_n. \]
Autrement dit, une probabilit\'e sur un ensemble d\'enombrable est enti\`erement caract\'eris\'ee
par sa valeur sur les singletons.

% -----------------------------------------------------------------------------
\section{Ind\'ependance}

\Para{D\'efinition}

Soit $\Prob$ un espace probabilis\'e.
Soit $(A_i)_{n\in I}$ une famille d'\'ev\'enements.
\begin{itemize}
\item
  Les $(A_i)_{i\in I}$ sont \emph{deux \`a deux ind\'ependants}
  si et seulement si pour tous $(i,j)\in I^2$ tels que $i\neq j$, on a $\P(A_i\cap A_j)=\P(A_i)\P(A_j)$.
\item
  La famille $(A_i)_{i\in I}$ sont des \'ev\'enements \emph{mutuellement ind\'ependants}
  si pour toute partie finie $F\subset I$, on a
  \[ \P\left( \bigcap_{i\in F} A_i \right) = \prod_{i\in F} \P(A_i) \]
\end{itemize}

\Para{Proposition}

Soit $\Prob$ un espace probabilis\'e.
Soit $(A_n)_{n\in \N}$ des \'ev\'enements mutuellement ind\'ependants.
\begin{itemize}
\item
  Si $I\subset \N$, alors $(A_i)_{i\in I}$ sont des \'ev\'enements mutuellement ind\'ependants.
\item
  Si pour tout $i\in \N$, $B_i \in\bigl\{ \emptyset, A_i, \bar A_i,\Omega\bigr\}$,
  alors $(B_n)_{n\in \N}$ sont des \'ev\'enements mutuellement ind\'ependants.
\end{itemize}

% -----------------------------------------------------------------------------
\section{Probabilit\'e conditionnelle}

\Para{Proposition-D\'efinition}

Soit $\Prob$ un espace probabilis\'e.
Soit $A$ un \'ev\'enement tel que $\P(A)>0$.
L'application
\[ \Fonction{\P_A}{\Tribu}{[0,1]}{B}{\frac{\P(A\cap B)}{\P(A)}} \]
est une probabilit\'e sur $\Pro$
appel\'ee \emph{probabilit\'e conditionnellement \`a $A$},
ou \emph{probabilit\'e sachant $A$}.

On note $\P(B|A) = \P_A(B)$.

\Para{Proposition}[lien avec l'ind\'ependance]

Soit $\Prob$ un espace probabilis\'e.
Soit $A$ un \'ev\'enement de probabilit\'e non nulle
et $B$ un \'ev\'enement quelconque.
Alors les \'ev\'enements $A$ et $B$ sont ind\'ependants si et seulement si $\P(B|A)=\P(B)$.

\Para{D\'efinitions}

Soit $\Pro$ un espace probabilisable.
Soit $(A_n)_{n\in \N}$ une famille d'\'ev\'enements.
\begin{enumerate}
\item
  On dit que $(A_n)$ est un \emph{syst\`eme complet d'\'ev\'enements} si et seulement si
  \begin{itemize}
  \item
    $\bigcup_{n\in \N} A_n = \Omega$;
  \item
    $\forall(n,p)\in \N^2$, $n\neq p \implies A_n\cap A_p = \emptyset$.
  \end{itemize}

\item
  On dit que $(A_n)$ est un \emph{syst\`eme quasi-complet d'\'ev\'enements} si et seulement si
  \begin{itemize}
  \item
    $\P(\bigcup_{n\in \N} A_n) = 1$;
  \item
    $\forall(n,p)\in \N^2$, $n\neq p \implies \P(A_n\cap A_p) = 0$.
  \end{itemize}
\end{enumerate}

\Para{Formule des probabilit\'es totales}

Soit $\Prob$ un espace probabilis\'e.
Soit $(A_n)_{n\in \N}$ un syst\`eme quasi-complet d'\'ev\'enements de probabilit\'es non nulles.
Pour tout \'ev\'enement $B$, on a
\[ \P(B) = \sum_{n\geq0} \P(B|A_n)\P(A_n). \]

\Para{Formule de Bayes}

Soit $\Prob$ un espace probabilis\'e.
Soit $(A_n)_{n\in \N}$ un syst\`eme quasi-complet d'\'ev\'enements de probabilit\'es non nulles.
Pour tout \'ev\'enement $B$ de probabilit\'e non nulle,
et pour tout $k\in \N$, on a
\[ \P(A_k|B) = \frac{ \P(B|A_k)\P(A_k) }{ \DS \sum_{n\geq0} \P(B|A_n)\P(A_n) }. \]

% -----------------------------------------------------------------------------
\section{Exercices}

% -----------------------------------------------------------------------------
\par\pagebreak[1]\par
\paragraph{Exercice 1}%
\hfill{\tiny 1929}%
\begingroup~

Soit $\Omega$ un ensemble,
$Z$ une partie de $\Omega$,
$(X_i)_{i\in I}$ et $(Y_i)_{i\in I}$ deux familles de parties de $\Omega$.
\begin{enumerate}
\item
  $\Pa{ \bigcup_{i\in I} X_i } \cup\Pa{ \bigcup_{i\in I} Y_i } = \bigcup_{i\in I} (X_i\cup Y_i)$
\item
  $\Pa{ \bigcap_{i\in I} X_i } \cap\Pa{ \bigcap_{i\in I} Y_i } = \bigcap_{i\in I} (X_i\cap Y_i)$
\item
  $\Omega\setminus\Pa{ \bigcup_{i\in I} X_i } = \bigcap_{i\in I} (\Omega\setminus X_i)$
\item
  $\Omega\setminus\Pa{ \bigcap_{i\in I} X_i } = \bigcup_{i\in I} (\Omega\setminus X_i)$
\item
  $\bigcup_{i\in I} (X_i\setminus Z) = \Pa{ \bigcup_{i\in I} X_i } \setminus Z$
\item
  $\bigcup_{i\in I} (Z\setminus X_i) = Z \setminus\Pa{ \bigcap_{i\in I} X_i }$
\item
  $\bigcap_{i\in I} (X_i\setminus Y_i) = \Pa{ \bigcap_{i\in I} X_i } \setminus\Pa{ \bigcup_{i\in I} Y_i }$
\item
  Si $I =\N$, montrer que

  \begin{enumerate}
  \item
    $\bigcup_{n\geq0} \Pa{ \bigcup_{k=0}^n A_k } = \bigcup_{n\geq0} A_n$
  \item
    $\bigcap_{n\geq0} \Pa{ \bigcap_{k=0}^n A_k } = \bigcap_{n\geq0} A_n$
  \end{enumerate}
\end{enumerate}
\endgroup

% -----------------------------------------------------------------------------
\par\pagebreak[1]\par
\paragraph{Exercice 2}%
\hfill{\tiny 7745}%
\begingroup~

Soit $\Omega= \{ a,b,c \}$.
\begin{enumerate}
\item
  D\'eterminer la plus petite tribu contenant $\{ a \}$.
\item
  D\'eterminer la plus petite tribu contenant $\{ a \}$ et $\{ b \}$.
\end{enumerate}
\endgroup

% -----------------------------------------------------------------------------
\par\pagebreak[1]\par
\paragraph{Exercice 3}%
\hfill{\tiny 7776}%
\begingroup~

Soit $\Omega$ un ensemble, $\Tribu$ une tribu sur $\Omega$ et $\Omega'$ une partie de $\Omega$.
On pose $\Tribu' = \Ensemble{A\cap \Omega'}{A\in\Tribu}$.
Montrer que $\Tribu'$ est une tribu sur $\Omega'$.
\endgroup

% -----------------------------------------------------------------------------
\par\pagebreak[1]\par
\paragraph{\href{https://psi.miomio.fr/exo/2610.pdf}{Exercice 4}}%
\hfill{\tiny 2610}%
\begingroup~

Soit $\Omega$ un ensemble et $\Fn{f}{\Lambda}{\Omega}$ une application.
\emph{Rappel:} pour tout $X\subset \Omega$, on note $f^{-1}(X) = \Ensemble{x\in \Lambda}{f(x)\in X}$.
D\'emontrer les formules suivantes, o\`u
$Y$ est une partie de $\Omega$ et $(X_i)_{i\in I}$ une famille de parties de $\Omega$:
\begin{enumerate}
\item
  $f^{-1}(\Omega\setminus Y) = \Lambda\setminus f^{-1} (Y)$
\item
  $f^{-1} \Pa{ \bigcup_{i\in I} X_i } = \bigcup_{i\in I} f^{-1} (X_i)$
\item
  $f^{-1} \Pa{ \bigcap_{i\in I} X_i } = \bigcap_{i\in I} f^{-1} (X_i)$
\end{enumerate}
\endgroup

% -----------------------------------------------------------------------------
\par\pagebreak[1]\par
\paragraph{Exercice 5 (image r\'eciproque)}%
\hfill{\tiny 0289}%
\begingroup~

Soit $\Fn{f}{\Lambda}{\Omega}$ une application et $\Tribu$ une tribu sur $\Omega$.
On pose $\mathcal{U} = \Ensemble{ f^{-1}(A) }{ A\in\Tribu }$.
Montrer que $\mathcal{U}$ est une tribu sur $\Lambda$.
\endgroup

% -----------------------------------------------------------------------------
\par\pagebreak[1]\par
\paragraph{Exercice 6}%
\hfill{\tiny 7219}%
\begingroup~

Existe-t-il un espace probabilit\'e $\Prob$ sur $\Omega{} = \Z$
tel que $\forall{} n \in{} \Z$, $\P(\acco{n}) = 0$?

\emph{Remarque:} si l'on remplace $\Z$ par $\R$, la r\'eponse est positive.
\endgroup

% -----------------------------------------------------------------------------
\par\pagebreak[1]\par
\paragraph{Exercice 7}%
\hfill{\tiny 2201}%
\begingroup~

Soit $\Omega$ un ensemble infini.
\begin{enumerate}
\item
  Soit $\Tribu = \Ensemble{A\subset \Omega}{ \text{$A$ ou $\bar A$ est au plus d\'enombrable} }$.
  Montrer que $\Tribu$ est une tribu sur $\Omega$.
\item
  Soit $\mathcal{U} = \Ensemble{A\subset \Omega}{ \text{$A$ ou $\bar A$ est fini} }$.
  Montrer que $\mathcal{U}$ n'est pas une tribu sur $\Omega$.
\end{enumerate}
\endgroup

% -----------------------------------------------------------------------------
\par\pagebreak[1]\par
\paragraph{Exercice 8}%
\hfill{\tiny 9789}%
\begingroup~

Soit $\Prob$ un espace probabilis\'e
et $(A_n)_{n\in \N}$ une suite d'\'ev\'enements presque certains.
Montrer que $\bigcap_{n\in \N} A_n$ est presque certain.
Expliquer ce r\'esultat en langage courant.
\endgroup

% -----------------------------------------------------------------------------
\par\pagebreak[1]\par
\paragraph{Exercice 9}%
\hfill{\tiny 6195}%
\begingroup~

Soit $(A_n)_{n\in \N}$ une suite d'\'ev\'enements telle que
\[ \forall(n,p)\in \N^2 \+ n\neq p \implies \P(A_n\cap A_p) = 0. \]
Montrer que la s\'erie $\sum_n \P(A_n)$ converge et que
\[ \P(\bigcup_{n\in \N} A_n) = \sum_{n\in \N} \P(A_n). \]
\endgroup

% -----------------------------------------------------------------------------
\par\pagebreak[1]\par
\paragraph{Exercice 10}%
\hfill{\tiny 3875}%
\begingroup~

Soit $\Omega{} = \cro{0,1}^\N$.
On pose $A_n = \Ensemble{\omega\in \Omega} { \omega_n = 1 }$.
On admet qu'il existe une tribu $\Tribu$ sur $\Omega$ qui contient tous les
$A_n$ et une probabilit\'e $\P$ sur $\Pro$ telle que $\P(A_n) = \frac12$.
Les outils permettant de prouver cela ne sont pas au programme en CPGE.
\begin{enumerate}
\item
  Expliquer pourquoi $\Prob$ mod\'elise le lanc\'e d'une infinit\'e de pi\`eces \'equilibr\'ees ind\'ependantes.
\item
  Soit $A = \bigcap_{n\in \N} A_n$.
  Que repr\'esente $A$?
\item
  Montrer que $A\neq\emptyset$ mais que $\P(A) = 0$.
\end{enumerate}
\endgroup

% -----------------------------------------------------------------------------
\par\pagebreak[1]\par
\paragraph{Exercice 11}%
\hfill{\tiny 2449}%
\begingroup~

Soit $\Prob$ un espace probabilis\'e.
Soit $(A_n)_{n\in \N}$ une suite d'\'ev\'enements ind\'ependants
tels que $\forall{} n\in \N$, $\P(A_n) = p$.

D\'eterminer $\P(A)$ o\`u $\DS A = \bigcap_{n\in \N} A_n$.
\endgroup

% -----------------------------------------------------------------------------
\par\pagebreak[1]\par
\paragraph{Exercice 12}%
\hfill{\tiny 1862}%
\begingroup~

On consid\`ere un singe devant un clavier d'ordinateur,
qui appuie sur les touches au hasard, une touche chaque seconde, pour l'\'eternit\'e.
Montrer qu'il est presque certain que ce singe produira l'int\'egrale de George R. R. Martin.
\endgroup

% -----------------------------------------------------------------------------
\par\pagebreak[1]\par
\paragraph{Exercice 13}%
\hfill{\tiny 7077}%
\begingroup~

Soit $\Omega{} = \intF{0,1}$, $\B$ la tribu des bor\'eliens sur $\Omega$ \'evoqu\'ee plus haut et $\P$
telle que $\P\bigl( \intF{a,b} \bigr) = b-a$ si $0\leq a\leq b\leq1$.
\begin{enumerate}
\item
  Montrer que $\P(\Q\cap[0,1]) = 0$.
\item
  Soit $X$ un r\'eel choisi al\'eatoirement, de fa\c con uniforme dans $[0,1]$.
  Montrer que $X$ est presque s\^urement irrationnel.
\end{enumerate}

% -----------------------------------------------------------------------------
\endgroup

\section{Exercices plus avanc\'es}

% -----------------------------------------------------------------------------
\par\pagebreak[1]\par
\paragraph{Exercice 14 (limite inf\'erieure et sup\'erieure d'une suite r\'eelle)}%
\hfill{\tiny 0442}%
\begingroup~

Soit $(u_n)_{n\in \N}$ une suite r\'eelle.
\begin{enumerate}
\item
  Soit $\overline\R=\R\cup\{\pm\infty\}$.
  Montrer que toute suite monotone \`a valeur dans $\overline\R$
  a une limite dans $\overline\R$.
\item
  On pose $v_n = \inf \Ensemble{ u_k }{ k\geq n }$.
  Montrer que $(v_n)$ existe dans $\R\cup\{ -\infty\}$ et que $(v_n)$ est croissante.

  On note alors $\liminf_\ninf u_n$ la limite de la suite $(v_n)$ dans $\overline \R$,
  et on l'appelle \emph{limite inf\'erieure} de la suite $(u_n)_{n\in \N}$.
\item
  On pose $w_n = \sup \Ensemble{ u_k }{ k\geq n }$.
  Montrer que $(w_n)$ existe dans $\R\cup\{ +\infty\}$ et que $(w_n)$ est d\'ecroissante.

  On note alors $\limsup_\ninf u_n$ la limite de la suite $(w_n)$ dans $\overline \R$,
  et on l'appelle \emph{limite sup\'erieure} de la suite $(u_n)_{n\in \N}$.
\item
  D\'eterminer $\liminf_\ninf (-1)^n$ et $\limsup_\ninf (-1)^n$.
\item
  Montrer que $\liminf u_n \leq\limsup u_n$.
\item
  Montrer que la suite $(u_n)$ admet une limite dans $\overline \R$
  si et seulement si $\liminf u_n = \limsup u_n$, et que dans ce cas les trois limites sont \'egales.
\item
  Montrer que si $(u_n)$ est une suite born\'ee, alors $\liminf u_n$ et $\limsup u_n$ sont dans $\R$.
\item
  Montrer que:

  \begin{enumerate}
  \item
    $\limsup u_n = +\infty$ si et seulement si $(u_n)$ n'est pas major\'ee.
  \item
    $\liminf u_n = -\infty$ si et seulement si $(u_n)$ n'est pas minor\'ee.
  \item
    $\limsup u_n = -\infty$ si et seulement si $u_n \to -\infty$.
  \item
    $\liminf u_n = +\infty$ si et seulement si $u_n \to +\infty$.
  \end{enumerate}
\end{enumerate}
\endgroup

% -----------------------------------------------------------------------------
\par\pagebreak[1]\par
\paragraph{Exercice 15 (limite inf\'erieure et sup\'erieure d'\'ev\'enements)}%
\hfill{\tiny 9409}%
\begingroup~

Soit $\Prob$ un espace probabilis\'e
et $(A_n)_{n\in \N}$ une suite d'\'ev\'enements.
\begin{enumerate}
\item
  On pose $\DS B = \bigcup_{n\in \N} \Pa{ \bigcap_{p=n}^{+\infty} A_p }$.

  \begin{enumerate}
  \item
    Montrer que $B$ est un \'ev\'enement.
  \item
    D\'ecrire $B$ en langage courant.

    $B$ s'appelle la \emph{limite inf\'erieure} de la suite d'\'ev\'enements $(A_n)_{n\in \N}$
  \end{enumerate}
\item
  On pose $\DS C = \bigcap_{n\in \N} \Pa{ \bigcup_{p=n}^{+\infty} A_p }$.

  \begin{enumerate}
  \item
    Montrer que $C$ est un \'ev\'enement.
  \item
    D\'ecrire $C$ en langage courant.

    $C$ s'appelle la \emph{limite sup\'erieure} de la suite d'\'ev\'enements $(A_n)_{n\in \N}$.
  \end{enumerate}
\item
  Montrer que $B\subset C$.
\item
  On rappelle que pour $X\in\Tribu$, la fonction indicatrice de $X$ est d\'efinie par
  \[ \Fonction{\mathds{1}_X}{\Omega}{\R}{\omega}{\begin{cases}
        1 & \text{si }\omega\in X  \\
        0 & \text{si }\omega\notin X
  \end{cases}} \]

  \begin{enumerate}
  \item
    Montrer que $\mathds{1}_B = \liminf \mathds{1}_{A_n}$.
  \item
    Montrer que $\mathds{1}_C = \limsup \mathds{1}_{A_n}$.
  \end{enumerate}
\end{enumerate}
\endgroup

% -----------------------------------------------------------------------------
\par\pagebreak[1]\par
\paragraph{Exercice 16 (lemme de Fatou)}%
\hfill{\tiny 4067}%
\begingroup~

Soit $\Prob$ un espace probabilis\'e.
Soit $(A_n)_{n\in \N}$ une suite d'\'ev\'enements.
\begin{enumerate}
\item
  Montrer que $\P\Pa{\liminf_\ninf A_n} \leq\liminf_\ninf \P(A_n)$.
\item
  Montrer que $\P\Pa{\limsup_\ninf A_n} \geq\limsup_\ninf \P(A_n)$.
\end{enumerate}
\endgroup

% -----------------------------------------------------------------------------
\par\pagebreak[1]\par
\paragraph{\href{https://psi.miomio.fr/exo/6957.pdf}{Exercice 17} (lemme de Borel-Cantelli 1)}%
\hfill{\tiny 6957}%
\begingroup~

Soit $\Prob$ un espace probabilis\'e.
Soit $(A_n)_{n\in \N}$ une suite d'\'ev\'enements.
On suppose que la somme des probabilit\'es des $A_n$ est finie,
c.-\`a-d. \[ \sum_{n\in \N} \P(A_n) < \infty{} \]
\begin{enumerate}
\item
  Montrer que $\P(\limsup A_n) = 0$.
\item
  En d\'eduire que presque certainement,
  seul un nombre fini d'\'ev\'enements $A_n$ sont r\'ealis\'es.
\end{enumerate}
\endgroup

% -----------------------------------------------------------------------------
\par\pagebreak[1]\par
\paragraph{\href{https://psi.miomio.fr/exo/1582.pdf}{Exercice 18} (lemme de Borel-Cantelli 2)}%
\hfill{\tiny 1582}%
\begingroup~

\newcommand\eqdef{\mathrel{\stackrel{\makebox[0pt]{\mbox{\normalfont\tiny def}}}{=}}}
Soit $\Prob$ un espace probabilis\'e.
Soit $(A_n)_{n\in \N}$ une suite d'\'ev\'enements \emph{ind\'ependants}.
On suppose que la somme des probabilit\'es des $A_n$ est infinie,
c.-\`a-d. \[ \sum_{n\in \N} \P(A_n) \text{ diverge} \]
\begin{enumerate}
\item
  Soit $(u_n)$ une suite \`a valeurs dans $\intF{0,1}$ tels que $\sum_n u_n$ diverge.
  Montrer que \[ \prod_{n\geq0} (1 - u_n) \eqdef \lim_\ninf \prod_{k=0}^n (1-u_k) = 0. \]
\item
  Montrer que $\P(\limsup A_n) = 1$.
\item
  En d\'eduire que presque certainement,
  une infinit\'e d'\'ev\'enements $A_n$ sont r\'ealis\'es.
\end{enumerate}
\endgroup

% -----------------------------------------------------------------------------
\par\pagebreak[1]\par
\paragraph{Exercice 19}%
\hfill{\tiny 2910}%
\begingroup~

Pour $n\in \N^*$, on consid\`ere le jeu $J_n$ suivant:
dans une urne, il y a $n^2$ boules, dont une seule noire.
Avant de jouer, on doit miser $1$\euro.
On tire une boule au hasard, et si on tire la boule noire, on gagne $n^2$\euro, rien sinon.

On note $X_n$ le gain (positif ou n\'egatif) du jeu $J_n$,
et $S_n$ le gain d'une personne qui jouerait successivement aux jeux $J_1, \dots, J_n$.
\begin{enumerate}
\item
  \begin{enumerate}
  \item
    D\'eterminer $\E(X_n)$ pour $n\in \N^*$.
  \item
    En d\'eduire $\E(S_n)$ pour $n\in \N^*$.
  \end{enumerate}
\item
  On suppose que l'on continue ind\'efiniment de jouer aux jeux $J_1, \dots, J_n, \dots$

  \begin{enumerate}
  \item
    En utilisant le lemme de Borel-Cantelli 1, montrer que,
    presque s\^urement, on ne gagnera qu'\`a un nombre fini de jeux.
    On pourra noter $A_n$ l'\'ev\'enement \og on gagne dans le jeu $J_n$\fg.
  \item
    On note $B$ l'\'ev\'enement \og$S_n \to -\infty$\fg.
    En d\'eduire que $\P(B) = 1$.
  \end{enumerate}
\item
  Commentaires?
\end{enumerate}
\endgroup

\end{document}
