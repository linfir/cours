% autogenerated by ytex.rs

\documentclass{scrartcl}

\usepackage[francais]{babel}
\usepackage{geometry}
\usepackage{scrpage2}
\usepackage{lastpage}
\usepackage{multicol}
\usepackage{etoolbox}
\usepackage{xparse}
\usepackage{enumitem}
% \usepackage{csquotes}
\usepackage{amsmath}
\usepackage{amsfonts}
\usepackage{amssymb}
\usepackage{mathrsfs}
\usepackage{stmaryrd}
\usepackage{dsfont}
\usepackage{eurosym}
% \usepackage{numprint}
% \usepackage[most]{tcolorbox}
% \usepackage{tikz}
% \usepackage{tkz-tab}
\usepackage[unicode]{hyperref}
\usepackage[ocgcolorlinks]{ocgx2}

\let\ifTwoColumns\iftrue
\def\Classe{$\Psi$2019--2020}

% reproducible builds
% LuaTeX: \pdfvariable suppressoptionalinfo 1023 \relax
\pdfinfoomitdate=1
\pdftrailerid{}

\newif\ifDisplaystyle
\everymath\expandafter{\the\everymath\ifDisplaystyle\displaystyle\fi}
\newcommand\DS{\displaystyle}

\clearscrheadfoot
\pagestyle{scrheadings}
\thispagestyle{empty}
\ohead{\Classe}
\ihead{\thepage/\pageref*{LastPage}}

\setlist[itemize,1]{label=\textbullet}
\setlist[itemize,2]{label=\textbullet}

\ifTwoColumns
  \geometry{margin=1cm,top=2cm,bottom=3cm,foot=1cm}
  \setlist[enumerate]{leftmargin=*}
  \setlist[itemize]{leftmargin=*}
\else
  \geometry{margin=3cm}
\fi

\makeatletter
\let\@author=\relax
\let\@date=\relax
\renewcommand\maketitle{%
    \begin{center}%
        {\sffamily\huge\bfseries\@title}%
        \ifx\@author\relax\else\par\medskip{\itshape\Large\@author}\fi
        \ifx\@date\relax\else\par\bigskip{\large\@date}\fi
    \end{center}\bigskip
    \ifTwoColumns
        \par\begin{multicols*}{2}%
        \AtEndDocument{\end{multicols*}}%
        \setlength{\columnsep}{5mm}
    \fi
}
\makeatother

\newcounter{ParaNum}
\NewDocumentCommand\Para{smo}{%
  \IfBooleanF{#1}{\refstepcounter{ParaNum}}%
  \paragraph{\IfBooleanF{#1}{{\tiny\arabic{ParaNum}~}}#2\IfNoValueF{#3}{ (#3)}}}

\newcommand\I{i}
\newcommand\mi{i}
\def\me{e}

\def\do#1{\expandafter\undef\csname #1\endcsname}
\docsvlist{Ker,sec,csc,cot,sinh,cosh,tanh,coth,th}
\undef\do

\DeclareMathOperator\ch{ch}
\DeclareMathOperator\sh{sh}
\DeclareMathOperator\th{th}
\DeclareMathOperator\coth{coth}
\DeclareMathOperator\cotan{cotan}
\DeclareMathOperator\argch{argch}
\DeclareMathOperator\argsh{argsh}
\DeclareMathOperator\argth{argth}

\let\epsilon=\varepsilon
\let\phi=\varphi
\let\leq=\leqslant
\let\geq=\geqslant
\let\subsetneq=\varsubsetneq
\let\emptyset=\varnothing

\newcommand{\+}{,\;}

\undef\C
\newcommand\ninf{{n\infty}}
\newcommand\N{\mathbb{N}}
\newcommand\Z{\mathbb{Z}}
\newcommand\Q{\mathbb{Q}}
\newcommand\R{\mathbb{R}}
\newcommand\C{\mathbb{C}}
\newcommand\K{\mathbb{K}}
\newcommand\Ns{\N^*}
\newcommand\Zs{\Z^*}
\newcommand\Qs{\Q^*}
\newcommand\Rs{\R^*}
\newcommand\Cs{\C^*}
\newcommand\Ks{\K^*}
\newcommand\Rp{\R^+}
\newcommand\Rps{\R^+_*}
\newcommand\Rms{\R^-_*}
\newcommand{\Rpinf}{\Rp\cup\Acco{+\infty}}

\undef\B
\newcommand\B{\mathscr{B}}

\undef\P
\DeclareMathOperator\P{\mathbb{P}}
\DeclareMathOperator\E{\mathbb{E}}
\DeclareMathOperator\Var{\mathbb{V}}

\DeclareMathOperator*\PetitO{o}
\DeclareMathOperator*\GrandO{O}
\DeclareMathOperator*\Sim{\sim}
\DeclareMathOperator\Tr{tr}
\DeclareMathOperator\Ima{Im}
\DeclareMathOperator\Ker{Ker}
\DeclareMathOperator\Sp{Sp}
\DeclareMathOperator\Diag{diag}
\DeclareMathOperator\Rang{rang}
\DeclareMathOperator*\Coords{Coords}
\DeclareMathOperator*\Mat{Mat}
\DeclareMathOperator\Pass{Pass}
\DeclareMathOperator\Com{Com}
\DeclareMathOperator\Card{Card}
\DeclareMathOperator\Racines{Racines}
\DeclareMathOperator\Vect{Vect}
\DeclareMathOperator\Id{Id}

\newcommand\DerPart[2]{\frac{\partial #1}{\partial #2}}

\def\T#1{{#1}^T}

\def\pa#1{({#1})}
\def\Pa#1{\left({#1}\right)}
\def\bigPa#1{\bigl({#1}\bigr)}
\def\BigPa#1{\Bigl({#1}\Bigr)}
\def\biggPa#1{\biggl({#1}\biggr)}
\def\BiggPa#1{\Biggl({#1}\Biggr)}

\def\pafrac#1#2{\pa{\frac{#1}{#2}}}
\def\Pafrac#1#2{\Pa{\frac{#1}{#2}}}
\def\bigPafrac#1#2{\bigPa{\frac{#1}{#2}}}
\def\BigPafrac#1#2{\BigPa{\frac{#1}{#2}}}
\def\biggPafrac#1#2{\biggPa{\frac{#1}{#2}}}
\def\BiggPafrac#1#2{\BiggPa{\frac{#1}{#2}}}

\def\cro#1{[{#1}]}
\def\Cro#1{\left[{#1}\right]}
\def\bigCro#1{\bigl[{#1}\bigr]}
\def\BigCro#1{\Bigl[{#1}\Bigr]}
\def\biggCro#1{\biggl[{#1}\biggr]}
\def\BiggCro#1{\Biggl[{#1}\Biggr]}

\def\abs#1{\mathopen|{#1}\mathclose|}
\def\Abs#1{\left|{#1}\right|}
\def\bigAbs#1{\bigl|{#1}\bigr|}
\def\BigAbs#1{\Bigl|{#1}\Bigr|}
\def\biggAbs#1{\biggl|{#1}\biggr|}
\def\BiggAbs#1{\Biggl|{#1}\Biggr|}

\def\acco#1{\{{#1}\}}
\def\Acco#1{\left\{{#1}\right\}}
\def\bigAcco#1{\bigl\{{#1}\bigr\}}
\def\BigAcco#1{\Bigl\{{#1}\Bigr\}}
\def\biggAcco#1{\biggl\{{#1}\biggr\}}
\def\BiggAcco#1{\Biggl\{{#1}\Biggr\}}

\def\ccro#1{\llbracket{#1}\rrbracket}
\def\Dcro#1{\llbracket{#1}\rrbracket}

\def\floor#1{\lfloor#1\rfloor}
\def\Floor#1{\left\lfloor{#1}\right\rfloor}

\def\sEnsemble#1#2{\mathopen\{#1\mid#2\mathclose\}}
\def\bigEnsemble#1#2{\bigl\{#1\bigm|#2\bigr\}}
\def\BigEnsemble#1#2{\Bigl\{#1\Bigm|#2\Bigr\}}
\def\biggEnsemble#1#2{\biggl\{#1\biggm|#2\biggr\}}
\def\BiggEnsemble#1#2{\Biggl\{#1\Biggm|#2\Biggr\}}
\let\Ensemble=\bigEnsemble

\newcommand\IntO[1]{\left]#1\right[}
\newcommand\IntF[1]{\left[#1\right]}
\newcommand\IntOF[1]{\left]#1\right]}
\newcommand\IntFO[1]{\left[#1\right[}

\newcommand\intO[1]{\mathopen]#1\mathclose[}
\newcommand\intF[1]{\mathopen[#1\mathclose]}
\newcommand\intOF[1]{\mathopen]#1\mathclose]}
\newcommand\intFO[1]{\mathopen[#1\mathclose[}

\newcommand\Fn[3]{#1\colon#2\to#3}
\newcommand\CC[1]{\mathscr{C}^{#1}}
\newcommand\D{\mathop{}\!\mathrm{d}}

\newcommand\longto{\longrightarrow}

\undef\M
\newcommand\M[3]{\mathrm{#1}_{#2}\pa{#3}}
\newcommand\MnR{\M{M}{n}{\R}}
\newcommand\MnC{\M{M}{n}{\C}}
\newcommand\MnK{\M{M}{n}{\K}}
\newcommand\GLnR{\M{GL}{n}{\R}}
\newcommand\GLnC{\M{GL}{n}{\C}}
\newcommand\GLnK{\M{GL}{n}{\K}}
\newcommand\DnR{\M{D}{n}{\R}}
\newcommand\DnC{\M{D}{n}{\C}}
\newcommand\DnK{\M{D}{n}{\K}}
\newcommand\SnR{\M{S}{n}{\R}}
\newcommand\AnR{\M{A}{n}{\R}}
\newcommand\OnR{\M{O}{n}{\R}}
\newcommand\SnRp{\mathrm{S}_n^+(\R)}
\newcommand\SnRpp{\mathrm{S}_n^{++}(\R)}

\newcommand\LE{\mathscr{L}(E)}
\newcommand\GLE{\mathscr{GL}(E)}
\newcommand\SE{\mathscr{S}(E)}
\renewcommand\OE{\mathscr{O}(E)}

\newcommand\ImplD{$\Cro\Rightarrow$}
\newcommand\ImplR{$\Cro\Leftarrow$}
\newcommand\InclD{$\Cro\subset$}
\newcommand\InclR{$\Cro\supset$}
\newcommand\notInclD{$\Cro{\not\subset}$}
\newcommand\notInclR{$\Cro{\not\supset}$}

\newcommand\To[1]{\xrightarrow[#1]{}}
\newcommand\Toninf{\To{\ninf}}

\newcommand\Norm[1]{\|#1\|}
\newcommand\Norme{{\Norm{\cdot}}}

\newcommand\Int[1]{\mathring{#1}}
\newcommand\Adh[1]{\overline{#1}}

\newcommand\Uplet[2]{{#1},\dots,{#2}}
\newcommand\nUplet[3]{(\Uplet{{#1}_{#2}}{{#1}_{#3}})}

\newcommand\Fonction[5]{{#1}\left|\begin{aligned}{#2}&\;\longto\;{#3}\\{#4}&\;\longmapsto\;{#5}\end{aligned}\right.}

\DeclareMathOperator\orth{\bot}
\newcommand\Orth[1]{{#1}^\bot}
\newcommand\PS[2]{\langle#1,#2\rangle}

\newcommand{\Tribu}{\mathscr{T}}
\newcommand{\Part}{\mathcal{P}}
\newcommand{\Pro}{\bigPa{\Omega,\Tribu}}
\newcommand{\Prob}{\bigPa{\Omega,\Tribu,\P}}

\newcommand\DEMO{$\spadesuit$}
\newcommand\DUR{$\spadesuit$}

\newenvironment{psmallmatrix}{\left(\begin{smallmatrix}}{\end{smallmatrix}\right)}

% -----------------------------------------------------------------------------

\newcommand\Epz{E\priveZE}
\newcommand\priveZE{\setminus\acco{0_E}}

\begin{document}
\title{R\'eduction des endomorphismes}
\maketitle

% \ExoIdtrue
% \tableofcontents
% \newpage

\Para{Notations}
\begin{itemize}
\item $\K$ d\'esigne un corps (ici $\R$ ou $\C$);
\item $E$ d\'esigne un $\K$-espace vectoriel;
\item $\LE$ d\'esigne l'ensemble des endomorphismes de $E$,
  c.-\`a-d. l'ensemble des applications lin\'eaires de $E$ dans $E$.
\end{itemize}

% -----------------------------------------------------------------------------
\section{\'El\'ements propres}

% -----------------------------------------------------------------------------
\subsection{Sous-espaces vectoriels stables}

\Para{D\'efinitions}

Soit $F$ un sous-espace vectoriel de $E$ et $u\in\LE$.
\begin{itemize}
\item
  On dit que $F$ est \emph{$u$-stable}, ou \emph{stable par $u$}, si et seulement si $u(F)\subset F$.
\item
  Si $F$ est $u$-stable, il existe un unique endomorphisme $v\in\mathscr{L}(F)$
  tel que $\forall x\in F$, $u(x) = v(x)$.
  L'endomorphisme $v$ s'appelle l'\emph{endomorphisme induit} par $u$ sur $F$.
\end{itemize}

\Para{Lemme}

Si $f$ et $g$ sont des endomorphismes de $E$ \emph{qui commutent},
alors $\Ker g$ et $\Ima g$ sont stables par $f$.

% -----------------------------------------------------------------------------
\subsection{Motivation}

R\'eduire un endomorphisme $u\in\LE$,
c'est d\'ecomposer l'espace $E$ comme somme directe de sous-espaces stables par $u$,
$E = \bigoplus_{i\in I} E_i$ o\`u les $E_i$ sont stables par $u$.

Ainsi, l'\'etude de $u$ se ram\`ene \`a l'\'etude des endomorphismes $u_i$ induits par $u$ sur $E_i$.

L'id\'ee directrice est que les $u_i$ sont plus \emph{simples} que $u$;
dans le cas favorable, les $u_i$ sont des homoth\'eties, c.-\`a-d. $u_i =\lambda_i \Id_{E_i}$.

\Para{Exemples}
\begin{itemize}
\item projecteurs;
\item sym\'etries.
\end{itemize}

% -----------------------------------------------------------------------------
\subsection{Bases adapt\'ees}

\Para{D\'efinition}

Soit $E$ un $\K$-espace vectoriel de dimension finie $n$ et $F$ un sous-espace vectoriel de $E$.
La base $\B = \nUplet e1n$ est dite \emph{adapt\'ee} \`a $F$
si et seulement s'il existe $p\in\Dcro{0,n}$
tel que $\nUplet e1p$ soit une base de $F$.

\Para{D\'efinition}

Soit $E$ un $\K$-espace vectoriel de dimension finie, $\Uplet{F_1}{F_p}$ des sous-espaces vectoriels suppl\'ementaires de $E$ et $\B$ une base de $E$.
La base $\B$ est dite \emph{adapt\'ee} \`a la d\'ecomposition $E =\bigoplus_{k=1}^p F_k$ si et seulement si $\B$ peut s'\'ecrire comme la concat\'enation de $\Uplet{\B_1}{\B_p}$ o\`u $\B_k$ est une base de $F_k$ pour tout $k\in\Dcro{1,p}$.

\Para{Proposition}

Soit $E$ un $\K$-espace vectoriel de dimension finie et $\B$ une base de $E$.
On suppose que $\B$ s'\'ecrit comme la concat\'enation de $\Uplet{\B_1}{\B_p}$.
Pour $k\in\Dcro{1,p}$, notons $F_k$ le sous-espace vectoriel engendr\'e par $\B_k$.
Alors les sous-espaces vectoriels $\Uplet{F_1}{F_p}$ sont suppl\'ementaires.

% \Para{Proposition}

% Soit $E$, $F$ deux $\K$-espaces vectoriels.
% Soit $\Uplet{E_1}{E_p}$ des sous-espaces vectoriels de $E$ qui sont suppl\'ementaires.
% L'application \[ \Fonction{\varphi}{\mathscr{L}(E,F)}{\prod_{k=1}^p \mathscr{L}(E_k,F)}{u}{(u_{\vert E_1}, ..., u_{\vert E_p})} \] est alors un isomorphisme.

% Autrement dit, si l'on se donne des applications lin\'eaires $\Fn{u_k}{E_k}{F}$ pour $1\leq k\leq p$, il existe une unique application $\Fn{u}{E}{F}$ telle que sa restriction \`a $E_k$ soit $u_k$ pour tout $1\leq k\leq p$.

% Autrement dit, la donn\'ee d'une application lin\'eaire $E \to F$
% est \'equivalente \`a la donn\'ee d'applications lin\'eaires $E_k \to F$ pour tout $1\leq k\leq p$.

% -----------------------------------------------------------------------------
\subsection{\'Elements propres}

\Para{D\'efinitions}

Soit $u\in\LE$.
\begin{itemize}
\item On dit que $\lambda\in \K$ est une \emph{valeur propre} de $u$ si et seulement si il existe $x\in\Epz$ tel que $u(x) =\lambda x$.
\item On dit que $x\in E$ est un \emph{vecteur propre} de $u$ associ\'e \`a la valeur propre $\lambda$ si et seulement si $x\in\Epz$ et $u(x) =\lambda x$.
\item On appelle \emph{spectre} de $u$, not\'e $\Sp(u)$, l'ensemble des valeurs propres de $u$.
\item On appelle \emph{sous-espace propre} de $u$ associ\'e \`a la valeur propre $\lambda$ le sous-espace vectoriel $E_\lambda(u) = \Ker(u -\lambda\Id_E)$.
\end{itemize}

\Para{Proposition}

Soit $u\in\LE$. Les conditions suivantes sont \'equivalentes:
\begin{itemize}
\item $\lambda\in\Sp u$;
\item $\exists x\in\Epz$ tel que $(u -\lambda\Id_E)(x) = 0_E$;
\item $E_\lambda(u) = \Ker(u -\lambda\Id_E)\neq\Acco{0_E}$;
\item $u -\lambda\Id_E$ n'est pas injective.
\end{itemize}

\Para{Proposition}

Soit $u\in\LE$ et $x\in\Epz$.
Alors $x$ est un vecteur propre de $u$ si et seulement si la droite $\K x$ est stable par $u$.

\Para{Proposition}

Soit $u\in\LE$ et $\lambda\in\Sp u$.
Alors l'ensemble des vecteurs propres de $u$ associ\'es \`a la valeur propre $\lambda$ est $\Ker(u-\lambda\Id_E) \priveZE$.

\Para{Proposition}

Soit $u\in\LE$, $\lambda$ une valeur propre de $u$ et $E_\lambda$ le sous-espace propre correspondant.
Alors $E_\lambda$ est stable par $u$.
De plus, l'endomorphisme induit par $u$ sur $E_\lambda$ est l'homoth\'ethie de rapport $\lambda$.
% Autrement dit, $u_{\vert E_\lambda} =\lambda\Id_{E_\lambda}$.

\Para{Lemme}

Soit $u$ et $v$ deux endomorphismes de $E$.
Si $u$ et $v$ commutent, les sous-espaces propres de $u$ sont stables par $v$.

% -----------------------------------------------------------------------------
\section{Polyn\^omes d'endomorphismes}

\subsection{Polyn\^omes d'endomorphismes}

\Para{D\'efinitions}

Soit $E$ un $\K$-espace vectoriel et $u\in\LE$.
On d\'efinit par r\'ecurrence $u^n$ pour $n\in \N$ par
\[ u^0 = \Id_E \quad \text{et} \quad \forall n\in \N{} \+  u^{n+1} = u\circ u^n. \]
Pour $\DS P = \sum_{k=0}^d a_k X^k$,
on pose $\DS P(u) = \sum_{k=0}^d a_k u^k$.

\Para{Th\'eor\`eme}

Soit $E$ un $\K$-espace vectoriel et $u\in\LE$.
L'application \[ \Fonction{\phi_u}{\K[X]}{\LE}{P}{P(u)} \]
est un morphisme d'alg\`ebre.
Autrement dit, si $P$ et $Q$ sont deux polyn\^omes et $\lambda$ un scalaire, on a
\begin{enumerate}
\item $(\lambda P)(u) = \lambda P(u)$
\item $(P+Q)(u) = P(u) + Q(u)$
\item $(PQ)(u) = P(u)\circ Q(u)$
\end{enumerate}

\Para{Lemme}

Soit $u\in\LE$, $x\in E$, $\lambda\in \K$ et $P\in \K[X]$.
On suppose que $u(x) = \lambda x$.
Alors $P(u)(x) = P(\lambda) x$.

\Para{Th\'eor\`eme}

Soit $u\in\LE$.
Les sous-espaces propres de $u$ sont en somme directe.
Autrement dit, si $(x_i)_{i\in I}$ est une famille de vecteurs propres
associ\'es \`a des valeurs propres deux \`a deux distinctes,
alors la famille $(x_i)_{i\in I}$ est libre.

% -----------------------------------------------------------------------------
\subsection{Polyn\^omes de matrices}

\Para{D\'efinition}

Soit $A\in\MnK$ une matrice carr\'ee.
On d\'efinit par r\'ecurrence $A^n$ pour $n\in \N$ par
$A^0 = I_n$ et $\forall n\in \N$, $A^{n+1} = A\times A^n$.

Pour $\DS P = \sum_{k=0}^d a_k X^k$,
on pose $\DS P(A) = \sum_{k=0}^d a_k A^k$.

\Para{Proposition}

Soit $E$ un $\K$-espace vectoriel de dimension finie, $\B$ une base de $E$, $u\in\LE$,
$A = \Mat_\B(u)$ et $P\in \K[X]$.

Alors $\Mat_\B P(u) = P(A)$.

\Para{Corollaire}

Soit $A\in\MnK$.
L'application \[ \Fonction{\phi_A}{\K[X]}{\MnK}{P}{P(A)} \]
est \'egalement un morphisme d'alg\`ebre.
Autrement dit, si $P$ et $Q$ sont deux polyn\^omes et $\lambda$ un scalaire, on a
\begin{enumerate}
\item $(\lambda P)(A) = \lambda P(A)$
\item $(P+Q)(A) = P(A) + Q(A)$
\item $(PQ)(A) = P(A)\times Q(A)$
\end{enumerate}

\Para{Proposition}

Soit $A\in\MnK$, $Q\in\GLnK$ et $P\in \K[X]$.
Alors
\[ P \bigl( Q^{-1}AQ \bigr) = Q^{-1} \, P(A) \, Q. \]

% -----------------------------------------------------------------------------
\subsection{Polyn\^omes annulateurs}

\Para{D\'efinition}

Soit $u\in\LE$.
Un polyn\^ome $P\in \K[X]$ est dit \emph{polyn\^ome annulateur de $u$} s'il est non nul et si $P(u) = 0 \in\LE$.
On dit \'egalement que $P$ \emph{annule} $u$.

\Para{Exemples}
Soit $u$ endomorphisme de $E$.
\begin{itemize}
\item
  $u$ est un projecteur $\iff$ $X^2 - X$ annule $u$.
\item
  $u$ est une sym\'etrie $\iff$ $X^2 - 1$ annule $u$.
\end{itemize}

\Para{Th\'eor\`eme}

Soit $u\in\LE$ et $P$ un polyn\^ome annulateur de $u$.
Alors toutes les valeurs propres de $u$ sont des racines de $P$, c.-\`a-d. $\Sp(u)\subset\Racines(P)$.

\Para{Notation}

\`A partir de maintenant et jusqu'\`a la fin du chapitre, on supposera que $E$ est un espace vectoriel de dimension finie.

% -----------------------------------------------------------------------------
\subsection{Polyn\^ome caract\'eristique}

\Para{Lemme}

Soit $u\in\LE$.
L'application de $\K$ dans $\K$ d\'efinie par $\lambda{} \mapsto \det(\lambda\Id_E - u)$ est une fonction polyn\^omiale.

\Para{D\'efinition}

Soit $u\in\LE$.
On appelle \emph{polyn\^ome caract\'eristique} de $u$ et on note $\chi_u$ l'unique polyn\^ome \`a coefficients dans $\K$ tel que
\[ \forall \lambda\in \K{} \+ \chi_u(\lambda) = \det(\lambda\Id_E - u). \]

\Para{Proposition}

Soit $u\in\LE$.
Le spectre de $u$ est exactement l'ensemble des racines de $\chi_u$,
c.-\`a-d. \[ \Sp(u) = \Racines(\chi_u). \]

\Para{Th\'eor\`eme}

Soit $u\in\LE$, et $n = \dim E$. Alors:
\begin{itemize}
\item $\chi_u$ est un polyn\^ome unitaire de degr\'e $n$;
\item le coefficient en $X^{n-1}$ de $\chi_u$ vaut $- \Tr u$;
\item le coefficient constant de $\chi_u$ vaut $(-1)^n \det u$.
\end{itemize}

\Para{Corollaire}

Soit $u\in\LE$. On suppose que $\chi_u$ est scind\'e;
$\chi_u$ peut alors s'\'ecrire sous la forme
\[ \chi_u(X) = \prod_{k=1}^n (X-\alpha_k) \]
o\`u $\alpha_1, \dots, \alpha_n$ sont les valeurs propres r\'ep\'et\'ees avec multiplicit\'e.
On a alors
\[ \Tr u = \sum_{k=1}^n \alpha_k \quad\text{et}\quad \det u = \prod_{k=1}^n \alpha_k. \]

De fa\c con \'equivalente, si on note
\[ \chi_u(X) = \prod_{k=1}^p \pa{X-\lambda_k}^{m_k} \]
o\`u $\lambda_1, \dots, \lambda_p$ sont les valeurs propres deux \`a deux distinctes,
de multiplicit\'es respectives $m_1, \dots, m_p$, alors
\[ \Tr u = \sum_{k=1}^p m_k \lambda_k \quad\text{et}\quad \det u = \prod_{k=1}^p \lambda_k^{m_k}. \]

\Para{Proposition}

Soit $u\in\LE$.
Si $F$ est un sous-espace vectoriel stable par $u$,
et $v$ l'endomorphisme induit par $u$ sur $F$,
alors $\chi_v$ divise $\chi_u$.

\Para{Proposition}

Soit $u\in\LE$.
Si $F$ et $G$ sont des sous-espaces vectoriels $u$-stables tels que $E = F\oplus G$,
$v$ et $w$ les endomorphismes induits par $u$ sur $F$ et $G$ respectivement,
alors $\chi_u = \chi_v\cdot \chi_w$.

\Para{D\'efinition}

On appelle \emph{multiplicit\'e de la valeur propre $\lambda$} la multiplicit\'e de $\lambda$ comme racine de $\chi_u$.

\Para{Th\'eor\`eme}

Soit $u\in\LE$ et $\lambda$ une valeur propre de $u$,
$E_\lambda$ le sous-espace propre de $u$ associ\'e \`a la valeur propre $\lambda$
et $m_\lambda$ est la multiplicit\'e de la valeur propre $\lambda$.
Alors \[ 1 \leq{} \dim E_\lambda{} \leq{} m_\lambda. \]

\Para{Th\'eor\`eme}[Cayley-Hamilton]

Soit $u\in\LE$.
Alors le polyn\^ome caract\'eristique de $u$ est un polyn\^ome annulateur de $u$.

% -----------------------------------------------------------------------------
\section{Endomorphismes diagonalisables}

\Para{D\'efinitions}
\begin{itemize}
\item Un endomorphisme $u\in\LE$ est dit \emph{diagonalisable} si et seulement s'il
  existe une base $\B$ de $E$ telle que $\Mat_\B(u)$ est diagonale.
\item Une matrice $A$ de $\MnK$ est dite \emph{diagonalisable} si et seulement si
  elle est semblable \`a une matrice diagonale.
\end{itemize}

\Para{Proposition}

Soit $u\in\LE$ et $\B$ une base de $E$.
Alors $u$ est diagonalisable si et seulement si $\Mat_\B(u)$ est diagonalisable.

\vfil

\Para{Th\'eor\`eme fondamental}

Soit $u\in\LE$.
Pour toute valeur propre $\lambda$ de $u$, on note $m_\lambda$ sa multiplicit\'e et $E_\lambda$ le sous-espace propre associ\'e.
Les conditions suivantes sont \'equivalentes:
\begin{enumerate}
\item $u$ est diagonalisable, c.-\`a-d. il existe une base $\B$ de $E$ telle que $\Mat_\B(u)$ est diagonale;
\item $u$ admet un polyn\^ome annulateur scind\'e et $\forall \lambda\in\Sp u$, $\dim E_\lambda{} = m_\lambda$;
\item $\chi_u$ est scind\'e et $\forall \lambda\in\Sp u$, $\dim E_\lambda{} = m_\lambda$;
\item $\DS \sum_{\lambda\in\Sp u} \dim E_\lambda{} = \dim E$;
\item $E$ est somme directe des espaces propres de $u$,
  c.-\`a-d. $\DS E = \bigoplus_{\lambda\in\Sp u} E_\lambda$;
\item il existe une base $\B$ de $E$ form\'ee de vecteurs propres de $u$;
\item le polyn\^ome $\DS \prod_{\lambda\in\Sp u}(X-\lambda)$ annule $u$;
\item $u$ admet un polyn\^ome annulateur scind\'e \`a racines simples.
\end{enumerate}

\vfil

\Para{Corollaires}

Soit $u\in\LE$.
\begin{enumerate}
\item Si $\chi_u$ est un polyn\^ome scind\'e \`a racines simples, alors $u$ est diagonalisable.
  La r\'eciproque est fausse (contre-exemple: $u = \Id_E$).
\item Si $u$ est diagonalisable et que $F$ est stable par $u$, alors l'endomorphisme induit par $u$ sur $F$ est \'egalement diagonalisable.
\item Si $\DS E = \bigoplus_{k=1}^p F_k$ et que pour tout $k\in\ccro{1,p}$, $u_{\vert F_k}$ est une homoth\'etie, alors $u$ est diagonalisable.
  Pour la r\'eciproque, cf. 2. du th\'eor\`eme pr\'ec\'edent.
\end{enumerate}

\Para{Th\'eor\`eme}[cf. chapitre \emph{alg\`ebre bilin\'eaire}]

Toute matrice sym\'etrique r\'eelle est diagonalisable.

\Para{Remarque}[en pratique]

Une m\'ethode pour diagonaliser une matrice $A$ de $\MnK$:
\begin{itemize}
\item
  on calcule le polyn\^ome caract\'eristique $\chi_A$;
\item
  on factorise $\chi_A$: s'il n'est pas scind\'e, c'est que $A$ n'est pas diagonalisable dans $\K$;
\item
  pour chaque racine $\lambda$, on d\'etermine une base du sous-espace propre $E_\lambda$ en r\'esolvant le syst\`eme lin\'eaire $AX =\lambda X$.
  Si on trouve strictement moins de $m_\lambda$ vecteurs libres, c'est que $A$ n'est pas diagonalisable;
\item
  soit $P$ la matrice carr\'ee dont les colonnes sont les $n$ vecteurs propres trouv\'es.
  Soit $D$ la matrice diagonale form\'ee des valeurs propres correspondantes.
  On a alors $A = PDP^{-1}$ et $D = P^{-1}AP$.
\end{itemize}

\Para{Applications}
\begin{itemize}
\item
  calcul des puissances d'une matrice $A$, par exemple pour les cha\^ines de Markov;
\item
  suites lin\'eaires r\'ecurrentes \`a coefficients constants;
\item
  syst\`eme de suites lin\'eaires r\'ecurrences \`a coefficients constants;
\item
  syst\`emes diff\'erentiels lin\'eaires \`a coefficients constants;
\item
  etc.
\end{itemize}

% -----------------------------------------------------------------------------
\section{Endomorphismes trigonalisables}

\Para{D\'efinitions}
\begin{itemize}
\item Un endomorphisme $u\in\LE$ est dit \emph{trigonalisable} si et seulement s'il existe une base $\B$ de $E$ telle que $\Mat_\B(u)$ est triangulaire sup\'erieure.
\item Une matrice $A$ de $\MnK$ est dite \emph{trigonalisable} si et seulement si elle est semblable \`a une matrice triangulaire sup\'erieure.
\end{itemize}

\Para{Proposition}

Soit $u\in\LE$ et $\B$ une base de $E$.
Alors $u$ est trigonalisable si et seulement si $\Mat_\B(u)$ est trigonalisable.

\Para{Th\'eor\`eme}

Soit $u\in\LE$. Les conditions suivantes sont \'equivalentes:
\begin{enumerate}
\item $u$ est trigonalisable, c.-\`a-d. il existe $\B$ base de $E$ telle que $\Mat_\B(u)$ est triangulaire sup\'erieure;
\item $\chi_u$ est scind\'e;
\item $u$ admet un polyn\^ome annulateur scind\'e.
\end{enumerate}

\Para{Corollaire}
\begin{enumerate}
\item Tout endomorphisme d'un $\C$-espace vectoriel de dimension finie est trigonalisable.
\item Toute matrice carr\'ee est trigonalisable dans $\MnC$.
\end{enumerate}

% -----------------------------------------------------------------------------
\section{Un peu de topologie sur $\LE$}

\Para{Proposition}

Soit $u\in\LE$.
Il existe une suite num\'erique $(\epsilon_n)_{n\in \N}$ convergeant vers~$0$ telle que
pour tout $n\in \N$, l'endomorphisme $u +\epsilon_n \Id_E$ est inversible.

\Para{Corollaire}

$\GLE$ est dense dans $\LE$.
De fa\c con \'equivalente, $\GLnK$ est dense dans $\MnK$.

\Para{Applications}

cf. exercices~67, 25.

\Para{Th\'eor\`eme (H.P.)}

L'ensemble des matrices diagonalisables complexes est dense dans $\MnC$.

\Para{Th\'eor\`eme}[Cayley-Hamilton]

Soit $A\in\MnK$. Alors $\chi_A(A) = 0$.

% -----------------------------------------------------------------------------
\section{Exercices}

\subsection{Calculs explicites}

% -----------------------------------------------------------------------------
\par\pagebreak[1]\par
\paragraph{Exercice 1}%
\hfill{\tiny 2125}%
\begingroup~

Les matrices suivantes sont-elles diagonalisables?
Si oui, les diagonaliser.

$A = \begin{pmatrix} 2 & 0 & 1 \\ 1 & 1 & 1 \\ -2 & 0 & -1 \end{pmatrix}$,
$B = \begin{pmatrix} 5 & -17 & 25 \\ 2 & -9 & 16 \\ 1 & -5 & 9 \end{pmatrix}$, \\
$C = \begin{pmatrix} 0 & 1 & 0 \\ 1 & 0 & 1 \\ 0 & 1 & 0 \end{pmatrix}$,
$D = \begin{pmatrix} 11 & -5 & 5 \\ -5 & 3 & -3 \\ 5 & -3 & 3 \end{pmatrix}$.
\endgroup

% -----------------------------------------------------------------------------
\par\pagebreak[1]\par
\paragraph{Exercice 2}%
\hfill{\tiny 4370}%
\begingroup~

Trouver sans calcul les \'el\'ements propres
de $M = \begin{pmatrix} a & a \\ b & b \end{pmatrix}$.
\endgroup

% -----------------------------------------------------------------------------
\par\pagebreak[1]\par
\paragraph{Exercice 3}%
\hfill{\tiny 2059}%
\begingroup~

R\'esoudre le syst\`eme
\[
  \left\{ \, \begin{array}{@{}c@{}c@{}r@{}c@{}r@{}c@{}r}
      x' & {}={} &  x  & {}+{} & y          \\
      y' & {}={} & -x  & {}+{} & 2y & {}+{} & z \\
      z' & {}={} &  x  &       &    & {}+{} & z \\
  \end{array} \right.
\]
et trouver les solutions sur $\R$.
\endgroup

% -----------------------------------------------------------------------------
\par\pagebreak[1]\par
\paragraph{Exercice 4}%
\hfill{\tiny 4683}%
\begingroup~

\'Etudier la diagonalisibilit\'e de
$\begin{pmatrix} 0 & 0 & c^2 \\ 0 & b^2 & 0 \\ a^2 & 0 & 0 \end{pmatrix}$
o\`u $(a,b,c) \in \R^3$.
\endgroup

% -----------------------------------------------------------------------------
\par\pagebreak[1]\par
\paragraph{Exercice 5}%
\hfill{\tiny 3312}%
\begingroup~

Soit \[ A = \begin{pmatrix} 29 & 38 & -18 \\ -11 & -14 & 7 \\ 20 & 27 & -12 \end{pmatrix} \quad\text{et}\quad
B = \begin{pmatrix} 3 & 1 & 2 \\ 2 & 0 & 1 \\ 1 & 0 & 0 \end{pmatrix}. \]
Montrer que $A$ et $B$ ont m\^eme rang, m\^eme d\'eterminant et m\^eme trace
mais ne sont pas semblables.
\endgroup

% -----------------------------------------------------------------------------
\par\pagebreak[1]\par
\paragraph{Exercice 6}%
\hfill{\tiny 5206}%
\begingroup~

\begin{enumerate}
\item \`A l'aide de manipulations \'el\'ementaires, d\'eterminer le polyn\^ome caract\'eristique
  de la matrice suivante
  \[ A = \begin{pmatrix} 2001 & 1 & 5 \\ 3 & 2001 & 3 \\ 4 & 2 & 2001 \end{pmatrix}. \]
\item Est-elle diagonalisable?
\item Existe-t-il des matrices $B$ telles que $B^2 = A$?
  Si oui, combien?
\end{enumerate}
\endgroup

% -----------------------------------------------------------------------------
\par\pagebreak[1]\par
\paragraph{Exercice 7}%
\hfill{\tiny 9700}%
\begingroup~

Quel est le rang de $A = \begin{pmatrix} 1 & 1 & a \\ 0 & 2 & 0 \\ 0 & 0 & a \end{pmatrix}$?
Est-elle inversible? Diagonalisable?
\endgroup

% -----------------------------------------------------------------------------
\par\pagebreak[1]\par
\paragraph{Exercice 8}%
\hfill{\tiny 9374}%
\begingroup~

On pose
\[ N = \begin{pmatrix} 1 & 0 & 0 & 0 & 1 \\ 1 & 1 & 0 & 0 & 1 \\ 1 & 0 & 1 & 0 & 1 \\ 1 & 0 & 0 & 1 & 1 \\ 1 & 0 & 0 & 0 & 1 \end{pmatrix}. \]
D\'eterminer les \'el\'ements propres de $N$.
$N$ est-elle diagonalisable?
\endgroup

% -----------------------------------------------------------------------------
\par\pagebreak[1]\par
\paragraph{Exercice 9}%
\hfill{\tiny 8472}%
\begingroup~

On d\'efinit les matrices suivantes
\[ M = \begin{pmatrix} a & b & c & d \\ d & a & b & c \\ c & d & a & b \\ b & c & d & a \end{pmatrix}
  \qquad ; \qquad
A = \begin{pmatrix} 0 & 1 & 0 & 0 \\ 0 & 0 & 1 & 0 \\ 0 & 0 & 0 & 1 \\ 1 & 0 & 0 & 0 \end{pmatrix}. \]

Exprimer $M$ sous la forme $M = P(A)$ o\`u $P$ est un polyn\^ome et
en d\'eduire le d\'eterminant de $M$ \emph{sous forme factoris\'ee}.
\endgroup

% -----------------------------------------------------------------------------
\par\pagebreak[1]\par
\paragraph{Exercice 10}%
\hfill{\tiny 8450}%
\begingroup~

Trouver $A \in\M{M}{2}{\C}$ puis $A \in\M{M}{2}{\R}$ v\'erifiant
\[ A^2 + A + I_2 = \begin{pmatrix} 0 & 1 \\ 1 & 0 \end{pmatrix}. \]
\endgroup

% -----------------------------------------------------------------------------
\par\pagebreak[1]\par
\paragraph{\href{https://psi.miomio.fr/exo/0496.pdf}{Exercice 11} (Attila)}%
\hfill{\tiny 0496}%
\begingroup~

Soit $J$ la matrice de $\MnR$ d\'efinie par
\[ J = \begin{pmatrix} 1 & \cdots & 1 \\ \vdots & (1) & \vdots \\ 1 & \cdots & 1 \end{pmatrix}. \]
\begin{enumerate}
\item Montrer que $J$ est diagonalisable et d\'eterminer ses valeurs propres
  (et leurs multiplicit\'es).
\item En d\'eduire la valeur du d\'eterminant
  \[ \begin{vmatrix} a & & (b) \\ & \ddots & \\ (b) & & a \end{vmatrix}. \]
\end{enumerate}
\endgroup

% -----------------------------------------------------------------------------
\par\pagebreak[1]\par
\paragraph{Exercice 12}%
\hfill{\tiny 2608}%
\begingroup~

La matrice
\[ A = \begin{pmatrix} a & b & a & b & a & b \\ b & a & b & a & b & a \\ a & b & a & b & a & b \\ b & a & b & a & b & a \\ a & b & a & b & a & b \\ b & a & b & a & b & a \end{pmatrix} \]
avec $(a,b) \in \C^2$ est-elle diagonalisable?
Calculer $A^n$ pour $n \in \N$.
\endgroup

% -----------------------------------------------------------------------------
\par\pagebreak[1]\par
\paragraph{Exercice 13}%
\hfill{\tiny 2704}%
\begingroup~

Soit $M$ la matrice d\'efinie par \[ M = \begin{pmatrix} -1 & 3 & -8 \\ 1 & -2 & 7 \\ 1 & -2 & 6 \end{pmatrix}. \]
\begin{enumerate}
\item Montrer que $\forall \lambda\in \R$, on a $\chi_M = (X-1)^3$.
\item $M$ est-elle diagonalisable? trigonalisable?
\item Montrer que $M$ peut s'\'ecrire sous la forme $M = I_3 + N$ o\`u $N$ est semblable
  \`a une matrice triangulaire sup\'erieure ayant des 0 sur la diagonale.
\item V\'erifier que $N^3 = 0$.
\item En d\'eduire que $\frac{1}{n^2} M^n$ tend vers
  \[ A = \frac12 \begin{pmatrix} -1 & 1 & -3 \\ 2 & -2 & 6 \\ 1 & -1 & 3 \end{pmatrix}. \]
\item Pour $X \in \R^3$ fix\'e, montrer que $U \mapsto UX$ est continue,
  et en d\'eduire que $\lim\limits_\ninf \frac{1}{n^2} M^n X = AX$.

  Soit les suites $(x_n)$, $(y_n)$ et $(z_n)$ d\'efinies par
  $(x_0,y_0,z_0) \neq{} (0,0,0)$ et
  \[
    \left\{ \, \begin{array}{@{}r@{}c@{}r@{}c@{}r@{}c@{}r}
        x_{n+1} & {}={} & -x_n & {}+{} & 3y_n & {}-{} & 8z_n \\
        y_{n+1} & {}={} &  x_n & {}-{} & 2y_n & {}+{} & 7z_n \\
        z_{n+1} & {}={} &  x_n & {}-{} & 2y_n & {}+{} & 6z_n \\
    \end{array} \right.
  \]
\item Montrer que $\forall n \in \N$, $(x_n,y_n,z_n) \neq(0,0,0)$.
\item Montrer que si $x_0-y_0+3z_0 \neq0$, alors la s\'erie $\sum_n (x_n^2+y_n^2+z_n^2)^{-1/2}$ converge.
\item Que se passe-t-il si $x_0-y_0+3z_0 = 0$?
\end{enumerate}
\endgroup

% -----------------------------------------------------------------------------
\par\pagebreak[1]\par
\paragraph{\href{https://psi.miomio.fr/exo/1225.pdf}{Exercice 14}}%
\hfill{\tiny 1225}%
\begingroup~

Soit $A = \begin{pmatrix} 0 & 1 &  & (0) \\ 1 & \ddots & \ddots \\  & \ddots & \ddots & 1 \\ (0) &  & 1 & 0 \end{pmatrix}\in\MnR$.
\begin{enumerate}
\item Calculer $D_n(\theta) = \det\bigl(A + (2\cos\theta) I_n\bigr)$ par r\'ecurrence.
\item En d\'eduire les valeurs propres de $A$.
\end{enumerate}
\endgroup

\subsection{Puissances}

% -----------------------------------------------------------------------------
\par\pagebreak[1]\par
\paragraph{\href{https://psi.miomio.fr/exo/4842.pdf}{Exercice 15}}%
\hfill{\tiny 4842}%
\begingroup~

Calculer les puissances $p^{\text{i\`emes}}$ des matrices suivantes:

$A = \begin{pmatrix} -1 & a & -a \\ 1 & -1 & 0 \\ 1 & 0 & -1 \end{pmatrix}$;
$B = \begin{pmatrix} 1 & 1 & 1 \\ 2 & 2 & 2 \\ 0 & 0 & 1 \end{pmatrix}$;
$C = \begin{pmatrix} 1 & a & a^2 \\ a^{-1} & 1 & a \\ a^{-2} & a^{-1} & 1 \end{pmatrix}$.
\endgroup

% -----------------------------------------------------------------------------
\par\pagebreak[1]\par
\paragraph{Exercice 16}%
\hfill{\tiny 1484}%
\begingroup~

Soit $(u_n)_{n\in \N}$ une suite r\'eelle v\'erifiant la relation de r\'ecurrence
$\forall n\in \N$, $u_{n+3} = 6u_{n+2}-11u_{n+1}+6u_n$.
On pose $X_n = \begin{pmatrix} u_n \\ u_{n+1} \\ u_{n+2} \end{pmatrix}$.
\begin{enumerate}
\item Expliciter une matrice $A\in\mathrm{M}_3(\R)$ telle que $X_{n+1} = AX_n$.
\item Diagonaliser $A$.
\item En d\'eduire $u_n$ en fonction de $n$.
\end{enumerate}
\endgroup

% -----------------------------------------------------------------------------
\par\pagebreak[1]\par
\paragraph{\href{https://psi.miomio.fr/exo/6107.pdf}{Exercice 17}}%
\hfill{\tiny 6107}%
\begingroup~

Soit $A = \begin{pmatrix} -1 & 2 & 1 \\ 2 & -1 & -1 \\ -4 & 4 & 3 \end{pmatrix}$.
\begin{enumerate}
\item Calculer $A^n$.
\item Soit $U_0 = \begin{pmatrix} -2 \\ 4 \\ 1 \end{pmatrix}$ et $(U_n)_{n\in \N}$ d\'efinie par la relation $U_{n+1} = AU_n$.
  Calculer $U_n$ en fonction de $n$.
\item Soit $X(t) = \begin{pmatrix} x(t) \\ y(t) \\ z(t) \end{pmatrix}$.
  R\'esoudre le syst\`eme diff\'erentiel $\frac{\D X}{\D t} = AX$.
\end{enumerate}
\endgroup

% -----------------------------------------------------------------------------
\par\pagebreak[1]\par
\paragraph{Exercice 18}%
\hfill{\tiny 1761}%
\begingroup~

Soit $A = \begin{pmatrix} 1 & 0 & 0 \\ 0 & 1 & 1 \\ 1 & 0 & 1 \end{pmatrix}$.
En \'ecrivant $A = I_3 + B$, calculer $A^n$ pour $n\in \N$.
\endgroup

% -----------------------------------------------------------------------------
\par\pagebreak[1]\par
\paragraph{\href{https://psi.miomio.fr/exo/1731.pdf}{Exercice 19}}%
\hfill{\tiny 1731}%
\begingroup~

Soit $A\in\mathrm{M}_3(\R)$ telle que $\Sp A = \Acco{1,-2,2}$.
\begin{enumerate}
\item D\'eterminer $\chi_A$.
\item Montrer que $A^n$ peut s'\'ecrire sous la forme $A^n =\alpha_n A^2 +\beta_n A +\gamma_n I_3$ avec $(\alpha_n,\beta_n,\gamma_n)\in \R^3$.
\item On consid\`ere le polyn\^ome $P_n(X) =\alpha_n X^2 +\beta_n X +\gamma_n$.
  Montrer que $P_n(1) = 1$, $P_n(2) = 2^n$ et $P_n(-2) = (-2)^n$.
\item En d\'eduire les coefficients $\alpha_n$, $\beta_n$ et $\gamma_n$.
\end{enumerate}
\endgroup

% -----------------------------------------------------------------------------
\par\pagebreak[1]\par
\paragraph{Exercice 20}%
\hfill{\tiny 9531}%
\begingroup~

Calculer les puissances $p^{\text{i\`emes}}$ ($p\in \N$ puis $p\in \Z$) des matrices suivantes:

$A = \begin{pmatrix} 1 &  & (2) \\  & \ddots \\ (2) &  & 1 \end{pmatrix}$;
$B = \begin{pmatrix} 1 & 2 & 3 & 4 \\ 0 & 1 & 2 & 3 \\ 0 & 0 & 1 & 2 \\ 0 & 0 & 0 & 1 \end{pmatrix}$.
\endgroup

% -----------------------------------------------------------------------------
\par\pagebreak[1]\par
\paragraph{Exercice 21}%
\hfill{\tiny 7812}%
\begingroup~

Soit $A = \begin{pmatrix} -1 & 1 & -1 \\ 1 & -1 & 0 \\ 1 & 0 & -1 \end{pmatrix}$.
D\'eterminer $A^{100}$.
\endgroup

% -----------------------------------------------------------------------------
\par\pagebreak[1]\par
\paragraph{Exercice 22}%
\hfill{\tiny 1419}%
\begingroup~

Soit $A = \begin{pmatrix} 1 & -1 & 0 & 0 \\ 0 & 1 & 0 & 0 \\ 0 & 0 & -1 & 1 \\ 0 & 0 & 0 & -1 \end{pmatrix}$.
Calculer $A^n$ pour tout $n \in \Z$.
\endgroup

% -----------------------------------------------------------------------------
\par\pagebreak[1]\par
\paragraph{Exercice 23}%
\hfill{\tiny 5902}%
\begingroup~

On d\'efinit la matrice
\[ A = \begin{pmatrix} 0 & 1 & 0 \\ 0 & 0 & 1 \\ 1 & -3 & 3 \end{pmatrix}. \]
Trouver une matrice triangulaire semblable \`a $A$ puis
calculer $A^n$ pour $n\in \Z$.
\endgroup

\subsection{Polyn\^ome caract\'eristique}

% -----------------------------------------------------------------------------
\par\pagebreak[1]\par
\paragraph{Exercice 24}%
\hfill{\tiny 2593}%
\begingroup~

Soit $A\in\MnK$ inversible et $B = A^{-1}$.
Exprimer le polyn\^ome caract\'eristique $\chi_B$ en fonction de $\chi_A$.
\endgroup

% -----------------------------------------------------------------------------
\par\pagebreak[1]\par
\paragraph{Exercice 25}%
\hfill{\tiny 2876}%
\begingroup~

Soit $E$ un $\C$-espace vectoriel de dimension finie. Soit $(u,v)\in\LE^2$.
\begin{enumerate}
\item Montrer que $\Sp(u\circ v) = \Sp(v\circ u)$.
\item Si $u$ est inversible, montrer que $\chi_{u\circ v} =\chi_{v\circ u}$.
\item On rappelle qu'il existe une suite $(\epsilon_k)_{k\in \N}$ telle que $\epsilon_k \To{k\infty} 0$ et $\forall k\in \N$, $u +\epsilon_k \Id_E$ est inversible.
  En d\'eduire que $\chi_{u\circ v} =\chi_{v\circ u}$.
\end{enumerate}
\endgroup

% -----------------------------------------------------------------------------
\par\pagebreak[1]\par
\paragraph{Exercice 26 (magique)}%
\hfill{\tiny 0186}%
\begingroup~

Soit $(A,B)\in\MnK^2$,
$C = \begin{pmatrix} XI_n & A \\ B & I_n \end{pmatrix}$ et $D = \begin{pmatrix} I_n & -A \\ 0 & XI_n \end{pmatrix}$.
En \'ecrivant que $\det(CD) = \det(DC)$, montrer que $\chi_{AB} =\chi_{BA}$.
On retrouve ainsi le r\'esultat de l'exercice pr\'ec\'edent.
\endgroup

% -----------------------------------------------------------------------------
\par\pagebreak[1]\par
\paragraph{Exercice 27}%
\hfill{\tiny 7148}%
\begingroup~

Pour $A\in\MnC$, on note $\chi_A$ le polyn\^ome caract\'eristique
de $A$. Montrer que $A$ et $B$ n'ont pas de valeur propre commune
si et seulement si $\chi_A(B)$ est inversible.
\endgroup

\subsection{Endomorphismes nilpotents}

\Para{Rappel}
Un endormorphisme $u$ est dit \emph{nilpotent} si et seulement si $\exists n\in \N^*$ tel que $u^n = \tilde0$.

% -----------------------------------------------------------------------------
\par\pagebreak[1]\par
\paragraph{Exercice 28}%
\hfill{\tiny 1527}%
\begingroup~

D\'eterminer les matrices $A\in\MnK$ qui sont \`a la fois diagonalisables et nilpotentes.
\endgroup

% -----------------------------------------------------------------------------
\par\pagebreak[1]\par
\paragraph{Exercice 29}%
\hfill{\tiny 7284}%
\begingroup~

Soit $u$ et $v$ deux endomorphismes nilpotents de $E$ qui commutent.
Montrer que $u+v$ et $u\circ v$ sont \'egalement nilpotents.
\endgroup

% -----------------------------------------------------------------------------
\par\pagebreak[1]\par
\paragraph{\href{https://psi.miomio.fr/exo/7774.pdf}{Exercice 30}}%
\hfill{\tiny 7774}%
\begingroup~

Soit $u\in\GLE$ et $v\in\LE$ nilpotent tel que $u$ et $v$ commutent.
Montrer que $u+v$ est inversible et pr\'eciser son inverse;
on commencera par le cas $u = \Id_E$.
\endgroup

% -----------------------------------------------------------------------------
\par\pagebreak[1]\par
\paragraph{Exercice 31}%
\hfill{\tiny 8145}%
\begingroup~

Soit $E$ un $\K$-espace vectoriel de dimension $n$ et $u\in\LE$.
\begin{enumerate}
\item On suppose $\K=\C$. Montrer que $u$ est nilpotent si et seulement si $\Sp u = \Acco{0}$.
\item Donner un contre-exemple dans le cas $\K=\R$.
\item On suppose $\K=\C$. Montrer que $u$ est nilpotent si et seulement si $\chi_u = X^n$.
\item Montrer que cela est encore vrai si $\K=\R$ (on pourra se ramener \`a des matrices).
\end{enumerate}
\endgroup

% -----------------------------------------------------------------------------
\par\pagebreak[1]\par
\paragraph{Exercice 32}%
\hfill{\tiny 7269}%
\begingroup~

Soit $A$ et $B$ deux matrices de $\MnR$
telles que $AB-BA=B$.
\begin{enumerate}
\item
  Montrer que $B$ n'est pas inversible.
\item
  Montrer que $AB^k - B^kA = kB^k$ pour tout $k\in \N$.
\item
  En utilisant judicieusement un endomorphisme de $\MnR$,
  en d\'eduire que $B$ est nilpotente.
\end{enumerate}
\endgroup

% -----------------------------------------------------------------------------
\par\pagebreak[1]\par
\paragraph{Exercice 33}%
\hfill{\tiny 6501}%
\begingroup~

Soit $A\in\MnK$.
Montrer que $A$ est nilpotente si et seulement si $\forall k\in \N^*$, $\Tr(A^k) = 0$.
\endgroup

% -----------------------------------------------------------------------------
\par\pagebreak[1]\par
\paragraph{Exercice 34}%
\hfill{\tiny 8774}%
\begingroup~

Soit $E$ un $\K$-espace vectoriel de dimension finie, $(u,v)\in\LE$ tels que $u\circ v - v\circ u = \lambda v$ o\`u $\lambda\in \K^*$.
\begin{enumerate}
\item Montrer que $\forall P\in \K[X]$, $u\circ P(v) - P(v)\circ u =\lambda v\circ P'(v)$.
\item Soit $\Fonction{\Phi}{\LE}{\LE}{f}{u\circ f - f\circ u.}$
  Interpr\'eter le r\'esultat pr\'ec\'edent en termes de valeurs propres de $\Phi$.
\item En d\'eduire que $v$ est nilpotent.
\end{enumerate}
\endgroup

\subsection{Endomorphismes matriciels}

% -----------------------------------------------------------------------------
\par\pagebreak[1]\par
\paragraph{\href{https://psi.miomio.fr/exo/6744.pdf}{Exercice 35}}%
\hfill{\tiny 6744}%
\begingroup~

Soit $\Fonction{u}{\MnK}{\MnK}{M}{\frac23M - \frac13\T{M}}$
\begin{enumerate}
\item D\'eterminer un polyn\^ome annulateur de $u$. En d\'eduire que $u$ est diagonalisable.
\item Calculer $\Tr u$ et $\det u$.
\end{enumerate}
\endgroup

% -----------------------------------------------------------------------------
\par\pagebreak[1]\par
\paragraph{\href{https://psi.miomio.fr/exo/3807.pdf}{Exercice 36}}%
\hfill{\tiny 3807}%
\begingroup~

Soit $\Fonction{v}{\MnK}{\MnK}{M}{\T{M}}$
\begin{enumerate}
\item D\'eterminer un polyn\^ome annulateur de $v$. En d\'eduire que $v$ est diagonalisable.
\item Diagonaliser $v$; on ne cherchera pas \`a expliciter une base de vecteurs propres.
\item Montrer que l'endomorphisme $u$ de l'exercice pr\'ec\'edent s'\'ecrit $u = P(v)$
  pour un polyn\^ome $P\in \K[X]$.
\item Retrouver les r\'esultats de l'exercice~35.
\end{enumerate}
\endgroup

% -----------------------------------------------------------------------------
\par\pagebreak[1]\par
\paragraph{Exercice 37}%
\hfill{\tiny 3758}%
\begingroup~

Soit $\Fonction{u}{\MnK}{\MnK}{M}{-M+\Tr(M)I_n}$
\begin{enumerate}
\item D\'eterminer un polyn\^ome annulateur de $u$ de degr\'e 2.
\item Quels sont les \'el\'ements propres de $u$? $u$ est-elle diagonalisable? inversible?
  Calculer $\det(u)$.
\item \`A quelle condition, liant $\Tr M$ et $\Sp M$, la matrice $u(M)$ est-elle inversible?
\end{enumerate}
\endgroup

% -----------------------------------------------------------------------------
\par\pagebreak[1]\par
\paragraph{Exercice 38}%
\hfill{\tiny 7508}%
\begingroup~

Soit $A\in\MnC$. On pose
\[ \Fonction{\varphi}{\MnC}{\MnC}{M}{-M + \Tr(M)A.} \]
\begin{enumerate}
\item Montrer que $\varphi$ est un endomorphisme de $\MnC$.
\item \`A quelle condition $\varphi$ est-il un isomorphisme?
\item Soit $B\in\MnC$.
  D\'eterminer l'ensemble des solutions de l'\'equation $\varphi(M)=B$.
\end{enumerate}
\endgroup

\subsection{Autres endomorphismes}

% -----------------------------------------------------------------------------
\par\pagebreak[1]\par
\paragraph{Exercice 39}%
\hfill{\tiny 3056}%
\begingroup~

Donner les \'el\'ements propres de $T$ d\'efini sur $\R[X]$
par $T(P)(X) = (2X+1)P(X) + (1-X^2)P'(X)$.
Donner les sous-espaces stables.
\endgroup

% -----------------------------------------------------------------------------
\par\pagebreak[1]\par
\paragraph{Exercice 40}%
\hfill{\tiny 1694}%
\begingroup~

Soit $n$ un entier sup\'erieur ou \'egal \`a~$4$.
On d\'efinit $\Phi{} \colon P \in{} \R_n[X] \mapsto XP'' + (X-4)P' - 3P$.
\begin{enumerate}
\item
  Montrer que $\Phi$ est un endomorphisme de $\R_n[X]$.
\item
  Est-il diagonalisable?
\item
  D\'eterminer la dimension puis une base du noyau de~$\Phi$.
\end{enumerate}
\endgroup

% -----------------------------------------------------------------------------
\par\pagebreak[1]\par
\paragraph{Exercice 41}%
\hfill{\tiny 2082}%
\begingroup~

Soit $n \geq2$ et $u$ l'endomorphisme de $\R_n[X]$ d\'efini par
$\forall P\in \R_n[X]$, $u(P) = P(-4)X + P(6)$.
\begin{enumerate}
\item D\'eterminer l'image et le noyau de $u$.
\item D\'eterminer les valeurs propres de $u$.
  L'endomorphisme $u$ est-il diagonalisable?
\end{enumerate}
\endgroup

% -----------------------------------------------------------------------------
\par\pagebreak[1]\par
\paragraph{Exercice 42}%
\hfill{\tiny 1336}%
\begingroup~

Soit $E$ un $\K$-espace vectoriel de dimension finie, et $a\in\GLE$.
\begin{enumerate}
\item Montrer que $\Fonction{\Phi}{\LE}{\LE}{u}{aua^{-1}}$
  est un automorphisme de l'alg\`ebre $\LE$.
\item Comparer les \'el\'ements propres de $u$ et de $aua^{-1}$.
\end{enumerate}
\endgroup

% -----------------------------------------------------------------------------
\par\pagebreak[1]\par
\paragraph{Exercice 43}%
\hfill{\tiny 3830}%
\begingroup~

D\'eterminer les \'el\'ements propres de l'endomorphisme
\[ \Fonction{\Phi}{\K[X]}{\K[X]}{P}{P(2-X)} \]
\endgroup

% -----------------------------------------------------------------------------
\par\pagebreak[1]\par
\paragraph{Exercice 44}%
\hfill{\tiny 0237}%
\begingroup~

Soit $E = \K_n[X]$ et $u$ l'endomorphisme de~$E$ d\'efini par $u(P) = (X^2-1)P''+(2X+1)P'$.
\begin{enumerate}
\item D\'eterminer la matrice de $u$ dans la base canonique de $\K_n[X]$.
\item Montrer que $u$ est diagonalisable.
\end{enumerate}
\endgroup

% -----------------------------------------------------------------------------
\par\pagebreak[1]\par
\paragraph{Exercice 45}%
\hfill{\tiny 6091}%
\begingroup~

Soit $E$ l'ensemble des fonctions continues de $\R$ dans $\R$ qui tendent vers~$0$ en $\pm\infty$,
$\Fonction{\varphi}{\R}{\R}{x}{2x}$ et $\Fonction{u}{E}{E}{f}{f\circ \varphi}$
\begin{enumerate}
\item Soit $f\in \E\setminus\acco0$ telle que $u(f) = \lambda f$.
  Montrer que $\forall x\in \R$, $\forall n\in \N$, $f(2^n x) =\lambda^n f(x)$.
\item Montrer que $\Abs{\lambda} < 1$.
\item Montrer que $\forall y\in \R$, $\forall n\in \N$, $f(y) =\lambda^n f(y 2^{-n})$, et en d\'eduire $\Abs{\lambda} > 1$.
\item Que vaut $\Sp u$?
\end{enumerate}
\endgroup

% -----------------------------------------------------------------------------
\par\pagebreak[1]\par
\paragraph{Exercice 46}%
\hfill{\tiny 9807}%
\begingroup~

Soit $E$ un espace vectoriel de dimension finie et $f\in\LE$.
On suppose que $f\circ f$ est un projecteur.
\begin{enumerate}
\item Montrer que le spectre de $f$ est inclus dans $\Acco{-1,0,1}$.
\item Montrer que $f$ est diagonalisable si et seulement si $f^3 = f$.
\end{enumerate}
\endgroup

\subsection{Diagonalisation}

% -----------------------------------------------------------------------------
\par\pagebreak[1]\par
\paragraph{Exercice 47}%
\hfill{\tiny 5962}%
\begingroup~

Soit $A\in\MnR$ telle que $A^3 = A + I_n$.
Montrer que $\det A > 0$.
\endgroup

% -----------------------------------------------------------------------------
\par\pagebreak[1]\par
\paragraph{Exercice 48}%
\hfill{\tiny 7290}%
\begingroup~

Soit $A\in\SnR$ et $k\in\Ns$ tels que $A^k = I_n$.
Montrer que $A^2 = I_n$.
\endgroup

% -----------------------------------------------------------------------------
\par\pagebreak[1]\par
\paragraph{Exercice 49}%
\hfill{\tiny 0573}%
\begingroup~

D\'eterminer une condition n\'ecessaire et suffisante portant sur $(a,b)\in \C^2$ pour que la matrice
$M = \begin{pmatrix} 0 &  & (a) \\  & \ddots \\ (b) &  & 0 \end{pmatrix}$
soit diagonalisable.
\endgroup

% -----------------------------------------------------------------------------
\par\pagebreak[1]\par
\paragraph{Exercice 50}%
\hfill{\tiny 8779}%
\begingroup~

Soit $A \in\M{GL}{6}{\R}$ telle que $A^3-3A^2+2A=0$ et $\Tr A=8$.
D\'eterminer le polyn\^ome caract\'eristique de $A$.
\endgroup

% -----------------------------------------------------------------------------
\par\pagebreak[1]\par
\paragraph{\href{https://psi.miomio.fr/exo/0433.pdf}{Exercice 51}}%
\hfill{\tiny 0433}%
\begingroup~

Soit $E$ un $\K$-espace vectoriel de dimension finie et $u\in\LE$.
On suppose que $u^2 - 2 u - 15 \Id_E = 0$.
\begin{enumerate}
\item Montrer que $E = \Ker(u+3\Id_E)\oplus\Ker(u-5\Id_E)$.
\item Montrer que le r\'esultat reste valable m\^eme si $E$ n'est pas de dimension finie.
\end{enumerate}
\endgroup

% -----------------------------------------------------------------------------
\par\pagebreak[1]\par
\paragraph{\href{https://psi.miomio.fr/exo/6203.pdf}{Exercice 52}}%
\hfill{\tiny 6203}%
\begingroup~

\def\ImplD{$\Cro\Rightarrow$}
\def\ImplR{$\Cro\Leftarrow$}
Soit $A\in\MnK$ une matrice de rang $1$.
Montrer que $A$ est diagonalisable si et seulement si $\Tr A\neq0$.
\endgroup

% -----------------------------------------------------------------------------
\par\pagebreak[1]\par
\paragraph{Exercice 53}%
\hfill{\tiny 2978}%
\begingroup~

Soit $C\in\mathrm{M}_{n,1}(\R)$ et $A = C\T{C}$.
\begin{enumerate}
\item Quel est le rang de $A$?
\item En d\'eduire le polyn\^ome caract\'eristique de $A$.
\item $A$ est-elle diagonalisable?
\end{enumerate}
\endgroup

% -----------------------------------------------------------------------------
\par\pagebreak[1]\par
\paragraph{Exercice 54}%
\hfill{\tiny 8160}%
\begingroup~

D\'eterminer les matrice $M\in\GLnC$ telles que $M^2 + \T{M} = I_n$.
On cherchera, si $M$ v\'erifie cette relation, un polyn\^ome annulateur de $M$.
\endgroup

% -----------------------------------------------------------------------------
\par\pagebreak[1]\par
\paragraph{Exercice 55}%
\hfill{\tiny 7395}%
\begingroup~

Trouver $\Ensemble{A \in\MnR}{A^2 = A \text{ et } \Tr A = 0}$.
\endgroup

% -----------------------------------------------------------------------------
\par\pagebreak[1]\par
\paragraph{Exercice 56}%
\hfill{\tiny 9389}%
\begingroup~

Soit $n\in \N$, $n\geq2$.
D\'eterminer les matrices $M\in\MnR$ telles que $M^5 = M^2$
et $\Tr(M) = n$.
\endgroup

% -----------------------------------------------------------------------------
\par\pagebreak[1]\par
\paragraph{Exercice 57}%
\hfill{\tiny 2523}%
\begingroup~

Soit $A$, $B$ et $C$ des matrices de $\MnR$ telles que
$C = A+B$, $C^2 = 2A+3B$ et $C^3 = 5A+6B$.
Les matrices $A$ et $B$ sont-elles diagonalisables?
\endgroup

% -----------------------------------------------------------------------------
\par\pagebreak[1]\par
\paragraph{Exercice 58}%
\hfill{\tiny 5784}%
\begingroup~

Trouver les matrices $A\in\MnR$ telles que $A^2 = A$ et $\Tr A = 0$.
\endgroup

% -----------------------------------------------------------------------------
\par\pagebreak[1]\par
\paragraph{Exercice 59}%
\hfill{\tiny 9579}%
\begingroup~

Soit $A\in\MnC$ telle que $A = A^{-1}$.
\begin{enumerate}
\item $A$ est-elle diagonalisable?
\item Calculer $\sum_{k=0}^{+\infty} \frac{A^k}{k!}$.
\end{enumerate}
\endgroup

% -----------------------------------------------------------------------------
\par\pagebreak[1]\par
\paragraph{\href{https://psi.miomio.fr/exo/8672.pdf}{Exercice 60}}%
\hfill{\tiny 8672}%
\begingroup~

Soit $n\in\Ns$ et $A\in\MnR$ telle que $A^3+A^2+A = 0$.
Montrer que le rang de $A$ est pair.
\endgroup

\subsection{Racines carr\'ees}

% -----------------------------------------------------------------------------
\par\pagebreak[1]\par
\paragraph{\href{https://psi.miomio.fr/exo/3517.pdf}{Exercice 61}}%
\hfill{\tiny 3517}%
\begingroup~

Soit $A = \begin{pmatrix} 5 & -4 & 1 \\ 8 & -7 & 2 \\ 12 & -12 & 4 \end{pmatrix}$.
\begin{enumerate}
\item Trouver une matrice $B\in\mathrm{M}_3(\R)$ diff\'erente de $A$ et de $-A$ telle que $B^2 = A$.
\item Montrer qu'il existe une infinit\'e de matrices $B$ telles que $B^2 = A$.
\end{enumerate}
\endgroup

% -----------------------------------------------------------------------------
\par\pagebreak[1]\par
\paragraph{Exercice 62}%
\hfill{\tiny 4939}%
\begingroup~

Soit $A = \begin{pmatrix} 9 & 0 & 0 \\ 1 & 4 & 0 \\ 1 & 1 & 1 \end{pmatrix}$.
Le but de cet exercice est de determiner \emph{toutes} les matrices $B\in\mathrm{M}_3(\R)$ telles que $B^2 = A$.
\begin{enumerate}
\item Diagonaliser $A$. On \'ecrira $A = P D P^{-1}$ avec $D$ diagonale.
\item Soit $E$ un $\R$-espace vectoriel de dimension~$3$ et $\B$ une base de $E$.
  Soit $f\in\LE$ tel que $\Mat_\B(f) = D$.
  On suppose que $g\in\LE$ v\'erifie $g^2 = f$.
  Montrer que les sous-espace propre de $f$ sont stables par $g$.
\item En d\'eduire que $\Mat_\B(g)$ est diagonale.
\item Soit $B\in\mathrm{M}_3(\R)$ une matrice telle que $B^2 = A$.
  Montrer que $P^{-1} B P$ est diagonale.
\item D\'eterminer l'ensemble des matrices $B\in\mathrm{M}_3(\R)$
  telles que $B^2 = A$.
\end{enumerate}
\endgroup

% -----------------------------------------------------------------------------
\par\pagebreak[1]\par
\paragraph{\href{https://psi.miomio.fr/exo/1483.pdf}{Exercice 63}}%
\hfill{\tiny 1483}%
\begingroup~

Soit $E$ un $\C$-espace vectoriel de dimension finie, $u\in\LE$.
\begin{enumerate}
\item Si $u$ est diagonalisable, montrer que $u^2$ l'est \'egalement.
\item On suppose $u^2$ diagonalisable et $u$ inversible.

  \begin{enumerate}
  \item Montrer que $u^2$ admet un polyn\^ome annulateur de la forme $P = \prod_{i=1}^d (X-\alpha_i)$ o\`u tous les $\alpha_i$ sont non nuls et deux \`a deux distincts.
  \item Soit $Q =\prod_{i=1}^d (X^2-\alpha_i)$. Montrer que $Q$ est scind\'e \`a racines simples.
  \item En d\'eduire que $u$ est diagonalisable.
  \end{enumerate}
\item Trouver un endomorphisme $u$, non inversible, tel que $u^2$ soit diagonalisable sans que $u$ le soit.
\end{enumerate}
\endgroup

% -----------------------------------------------------------------------------
\par\pagebreak[1]\par
\paragraph{Exercice 64}%
\hfill{\tiny 8865}%
\begingroup~

Soit $A\in\MnC$ inversible diagonalisable et $B\in\MnC$, $p\in\Ns$
tels que $A = B^p$.
\begin{enumerate}
\item Montrer que $B$ est diagonalisable.
  On pourra regarder l'exercice~63.
\item Si $A$ n'est pas inversible, la conclusion est-elle encore valable?
\end{enumerate}
\endgroup

\subsection{Matrices par blocs}

% -----------------------------------------------------------------------------
\par\pagebreak[1]\par
\paragraph{\href{https://psi.miomio.fr/exo/3525.pdf}{Exercice 65}}%
\hfill{\tiny 3525}%
\begingroup~

Soit $A\in\MnK$ non nulle et $B = \begin{pmatrix} 0 & A \\ 0 & 0 \end{pmatrix}\in\mathrm{M}_{2n}(\K)$.
Montrer que $B$ n'est pas diagonalisable.
\endgroup

% -----------------------------------------------------------------------------
\par\pagebreak[1]\par
\paragraph{Exercice 66}%
\hfill{\tiny 3651}%
\begingroup~

Soit $A\in\MnR$ une matrice diagonalisable
et $B = \begin{pmatrix} 2A & A \\ A & 2A \end{pmatrix}\in\mathrm{M}_{2n}(\R)$.
Montrer que $B$ est diagonalisable.
\endgroup

% -----------------------------------------------------------------------------
\par\pagebreak[1]\par
\paragraph{Exercice 67}%
\hfill{\tiny 2500}%
\begingroup~

Soit $A$, $B$, $C$, $D$ des matrices carr\'ees de $\MnK$.
On suppose que $CD = DC$ et que $D$ est inversible.
\begin{enumerate}
\item
  En consid\'erant le produit par blocs
  \[ \begin{pmatrix} I_n & -BD^{-1} \\ 0 & I_n \end{pmatrix} \begin{pmatrix} A & B \\ C & D \end{pmatrix} \]
  montrer que
  \[ \det \begin{pmatrix} A & B \\ C & D \end{pmatrix} = \det(AD-BC). \]
\item Montrer que cette derni\`ere formule est encore vraie si l'on ne suppose plus que $D$ est inversible.
\item Trouver un contre-exemple si l'on ne suppose plus que $C$ et $D$ commutent.
\end{enumerate}
\endgroup

\subsection{Divers}

% -----------------------------------------------------------------------------
\par\pagebreak[1]\par
\paragraph{\href{https://psi.miomio.fr/exo/2977.pdf}{Exercice 68}}%
\hfill{\tiny 2977}%
\begingroup~

Soit $(A,B)\in\MnR^2$ semblables en tant que matrices de $\MnC$.
\begin{enumerate}
\item Montrer qu'il existe $(P,Q)\in\MnR^2$ telles que $P + iQ\in\GLnC$ et $(P+iQ)A = B(P+iQ)$.
\item Montrer que $\forall \lambda\in \R$, $(P+\lambda Q)A = B(P+\lambda Q)$.
\item Montrer que $\exists \lambda\in \R$, $P + \lambda Q\in\GLnR$.
\item En d\'eduire que $A$ et $B$ sont semblables en tant que matrices de $\MnR$.
\end{enumerate}
\endgroup

% -----------------------------------------------------------------------------
\par\pagebreak[1]\par
\paragraph{\href{https://psi.miomio.fr/exo/5361.pdf}{Exercice 69}}%
\hfill{\tiny 5361}%
\begingroup~

Soit $E$ un espace vectoriel de dimension finie, $f$ et $g$ deux endomorphismes de $E$.
On suppose que $f$ est diagonalisable.
Montrer que $f$ et $g$ commutent si et seulement si tout sous-espace propre de $f$ est stable par $g$.
\endgroup

\end{document}
