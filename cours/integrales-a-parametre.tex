% autogenerated by ytex.rs

\documentclass{scrartcl}

\usepackage[francais]{babel}
\usepackage{geometry}
\usepackage{scrpage2}
\usepackage{lastpage}
\usepackage{ragged2e}
\usepackage{multicol}
\usepackage{etoolbox}
\usepackage{xparse}
\usepackage{enumitem}
\usepackage{csquotes}
\usepackage{amsmath}
\usepackage{amsfonts}
\usepackage{amssymb}
\usepackage{mathrsfs}
\usepackage{stmaryrd}
\usepackage{dsfont}
\usepackage{eurosym}
\usepackage{numprint}
\usepackage[most]{tcolorbox}
\usepackage{tikz}
\usepackage{tkz-tab}
\usepackage[unicode]{hyperref}
\usepackage[ocgcolorlinks]{ocgx2}

\let\ifTwoColumns\iftrue
\def\Classe{$\Psi$2019--2020}

% reproducible builds
% LuaTeX: \pdfvariable suppressoptionalinfo 1023 \relax
\pdfinfoomitdate=1
\pdftrailerid{}

\newif\ifDisplaystyle
\everymath\expandafter{\the\everymath\ifDisplaystyle\displaystyle\fi}
\newcommand\DS{\displaystyle}

\clearscrheadfoot
\pagestyle{scrheadings}
\thispagestyle{empty}
\ohead{\Classe}
\ihead{\thepage/\pageref*{LastPage}}

\setlist[itemize,1]{label=\textbullet}
\setlist[itemize,2]{label=\textbullet}

\ifTwoColumns
  \geometry{margin=1cm,top=2cm,bottom=3cm,foot=1cm}
  \setlist[enumerate]{leftmargin=*}
  \setlist[itemize]{leftmargin=*}
\else
  \geometry{margin=3cm}
\fi

\makeatletter
\let\@author=\relax
\let\@date=\relax
\renewcommand\maketitle{%
    \begin{center}%
        {\sffamily\huge\bfseries\@title}%
        \ifx\@author\relax\else\par\medskip{\itshape\Large\@author}\fi
        \ifx\@date\relax\else\par\bigskip{\large\@date}\fi
    \end{center}\bigskip
    \ifTwoColumns
        \par\begin{multicols*}{2}%
        \AtEndDocument{\end{multicols*}}%
        \setlength{\columnsep}{5mm}
    \fi
}
\makeatother

\newcounter{ParaNum}
\NewDocumentCommand\Para{smo}{%
  \IfBooleanF{#1}{\refstepcounter{ParaNum}}%
  \paragraph{\IfBooleanF{#1}{{\tiny\arabic{ParaNum}~}}#2\IfNoValueF{#3}{ (#3)}}}

\newcommand\I{i}
\newcommand\mi{i}
\def\me{e}

\def\do#1{\expandafter\undef\csname #1\endcsname}
\docsvlist{Ker,sec,csc,cot,sinh,cosh,tanh,coth,th}
\undef\do

\DeclareMathOperator\ch{ch}
\DeclareMathOperator\sh{sh}
\DeclareMathOperator\th{th}
\DeclareMathOperator\coth{coth}
\DeclareMathOperator\cotan{cotan}
\DeclareMathOperator\argch{argch}
\DeclareMathOperator\argsh{argsh}
\DeclareMathOperator\argth{argth}

\let\epsilon=\varepsilon
\let\phi=\varphi
\let\leq=\leqslant
\let\geq=\geqslant
\let\subsetneq=\varsubsetneq
\let\emptyset=\varnothing

\newcommand{\+}{,\;}

\undef\C
\newcommand\ninf{{n\infty}}
\newcommand\N{\mathbb{N}}
\newcommand\Z{\mathbb{Z}}
\newcommand\Q{\mathbb{Q}}
\newcommand\R{\mathbb{R}}
\newcommand\C{\mathbb{C}}
\newcommand\K{\mathbb{K}}
\newcommand\Ns{\N^*}
\newcommand\Zs{\Z^*}
\newcommand\Qs{\Q^*}
\newcommand\Rs{\R^*}
\newcommand\Cs{\C^*}
\newcommand\Ks{\K^*}
\newcommand\Rp{\R^+}
\newcommand\Rps{\R^+_*}
\newcommand\Rms{\R^-_*}
\newcommand{\Rpinf}{\Rp\cup\Acco{+\infty}}

\undef\B
\newcommand\B{\mathscr{B}}

\undef\P
\DeclareMathOperator\P{\mathbb{P}}
\DeclareMathOperator\E{\mathbb{E}}
\DeclareMathOperator\Var{\mathbb{V}}

\DeclareMathOperator*\PetitO{o}
\DeclareMathOperator*\GrandO{O}
\DeclareMathOperator*\Sim{\sim}
\DeclareMathOperator\Tr{tr}
\DeclareMathOperator\Ima{Im}
\DeclareMathOperator\Ker{Ker}
\DeclareMathOperator\Sp{Sp}
\DeclareMathOperator\Diag{diag}
\DeclareMathOperator\Rang{rang}
\DeclareMathOperator*\Coords{Coords}
\DeclareMathOperator*\Mat{Mat}
\DeclareMathOperator\Pass{Pass}
\DeclareMathOperator\Com{Com}
\DeclareMathOperator\Card{Card}
\DeclareMathOperator\Racines{Racines}
\DeclareMathOperator\Vect{Vect}
\DeclareMathOperator\Id{Id}

\newcommand\DerPart[2]{\frac{\partial #1}{\partial #2}}

\def\T#1{{#1}^T}

\def\pa#1{({#1})}
\def\Pa#1{\left({#1}\right)}
\def\bigPa#1{\bigl({#1}\bigr)}
\def\BigPa#1{\Bigl({#1}\Bigr)}
\def\biggPa#1{\biggl({#1}\biggr)}
\def\BiggPa#1{\Biggl({#1}\Biggr)}

\def\pafrac#1#2{\pa{\frac{#1}{#2}}}
\def\Pafrac#1#2{\Pa{\frac{#1}{#2}}}
\def\bigPafrac#1#2{\bigPa{\frac{#1}{#2}}}
\def\BigPafrac#1#2{\BigPa{\frac{#1}{#2}}}
\def\biggPafrac#1#2{\biggPa{\frac{#1}{#2}}}
\def\BiggPafrac#1#2{\BiggPa{\frac{#1}{#2}}}

\def\cro#1{[{#1}]}
\def\Cro#1{\left[{#1}\right]}
\def\bigCro#1{\bigl[{#1}\bigr]}
\def\BigCro#1{\Bigl[{#1}\Bigr]}
\def\biggCro#1{\biggl[{#1}\biggr]}
\def\BiggCro#1{\Biggl[{#1}\Biggr]}

\def\abs#1{\mathopen|{#1}\mathclose|}
\def\Abs#1{\left|{#1}\right|}
\def\bigAbs#1{\bigl|{#1}\bigr|}
\def\BigAbs#1{\Bigl|{#1}\Bigr|}
\def\biggAbs#1{\biggl|{#1}\biggr|}
\def\BiggAbs#1{\Biggl|{#1}\Biggr|}

\def\acco#1{\{{#1}\}}
\def\Acco#1{\left\{{#1}\right\}}
\def\bigAcco#1{\bigl\{{#1}\bigr\}}
\def\BigAcco#1{\Bigl\{{#1}\Bigr\}}
\def\biggAcco#1{\biggl\{{#1}\biggr\}}
\def\BiggAcco#1{\Biggl\{{#1}\Biggr\}}

\def\ccro#1{\llbracket{#1}\rrbracket}
\def\Dcro#1{\llbracket{#1}\rrbracket}

\def\floor#1{\lfloor#1\rfloor}
\def\Floor#1{\left\lfloor{#1}\right\rfloor}

\def\sEnsemble#1#2{\mathopen\{#1\mid#2\mathclose\}}
\def\bigEnsemble#1#2{\bigl\{#1\bigm|#2\bigr\}}
\def\BigEnsemble#1#2{\Bigl\{#1\Bigm|#2\Bigr\}}
\def\biggEnsemble#1#2{\biggl\{#1\biggm|#2\biggr\}}
\def\BiggEnsemble#1#2{\Biggl\{#1\Biggm|#2\Biggr\}}
\let\Ensemble=\bigEnsemble

\newcommand\IntO[1]{\left]#1\right[}
\newcommand\IntF[1]{\left[#1\right]}
\newcommand\IntOF[1]{\left]#1\right]}
\newcommand\IntFO[1]{\left[#1\right[}

\newcommand\intO[1]{\mathopen]#1\mathclose[}
\newcommand\intF[1]{\mathopen[#1\mathclose]}
\newcommand\intOF[1]{\mathopen]#1\mathclose]}
\newcommand\intFO[1]{\mathopen[#1\mathclose[}

\newcommand\Fn[3]{#1\colon#2\to#3}
\newcommand\CC[1]{\mathscr{C}^{#1}}
\newcommand\D{\mathop{}\!\mathrm{d}}

\newcommand\longto{\longrightarrow}

\undef\M
\newcommand\M[3]{\mathrm{#1}_{#2}\pa{#3}}
\newcommand\MnR{\M{M}{n}{\R}}
\newcommand\MnC{\M{M}{n}{\C}}
\newcommand\MnK{\M{M}{n}{\K}}
\newcommand\GLnR{\M{GL}{n}{\R}}
\newcommand\GLnC{\M{GL}{n}{\C}}
\newcommand\GLnK{\M{GL}{n}{\K}}
\newcommand\DnR{\M{D}{n}{\R}}
\newcommand\DnC{\M{D}{n}{\C}}
\newcommand\DnK{\M{D}{n}{\K}}
\newcommand\SnR{\M{S}{n}{\R}}
\newcommand\AnR{\M{A}{n}{\R}}
\newcommand\OnR{\M{O}{n}{\R}}
\newcommand\SnRp{\mathrm{S}_n^+(\R)}
\newcommand\SnRpp{\mathrm{S}_n^{++}(\R)}

\newcommand\LE{\mathscr{L}(E)}
\newcommand\GLE{\mathscr{GL}(E)}
\newcommand\SE{\mathscr{S}(E)}
\renewcommand\OE{\mathscr{O}(E)}

\newcommand\ImplD{$\Cro\Rightarrow$}
\newcommand\ImplR{$\Cro\Leftarrow$}
\newcommand\InclD{$\Cro\subset$}
\newcommand\InclR{$\Cro\supset$}
\newcommand\notInclD{$\Cro{\not\subset}$}
\newcommand\notInclR{$\Cro{\not\supset}$}

\newcommand\To[1]{\xrightarrow[#1]{}}
\newcommand\Toninf{\To{\ninf}}

\newcommand\Norm[1]{\|#1\|}
\newcommand\Norme{{\Norm{\cdot}}}

\newcommand\Int[1]{\mathring{#1}}
\newcommand\Adh[1]{\overline{#1}}

\newcommand\Uplet[2]{{#1},\dots,{#2}}
\newcommand\nUplet[3]{(\Uplet{{#1}_{#2}}{{#1}_{#3}})}

\newcommand\Fonction[5]{{#1}\left|\begin{aligned}{#2}&\;\longto\;{#3}\\{#4}&\;\longmapsto\;{#5}\end{aligned}\right.}

\DeclareMathOperator\orth{\bot}
\newcommand\Orth[1]{{#1}^\bot}
\newcommand\PS[2]{\langle#1,#2\rangle}

\newcommand{\Tribu}{\mathscr{T}}
\newcommand{\Part}{\mathcal{P}}
\newcommand{\Pro}{\bigPa{\Omega,\Tribu}}
\newcommand{\Prob}{\bigPa{\Omega,\Tribu,\P}}

\newcommand\DEMO{$\spadesuit$}
\newcommand\DUR{$\spadesuit$}

\newenvironment{psmallmatrix}{\left(\begin{smallmatrix}}{\end{smallmatrix}\right)}


% -----------------------------------------------------------------------------


\begin{document}
\title{Int\'egrales \`a param\`etre}
\maketitle

\Para{Contexte}

Soit $I$ et $J$ deux intervalles (non vides) de $\R$ et
\[ \Fonction{f}{I\times J}\K{(x,t)}{f(x,t).} \]
On s'int\'eresse \`a la fonction $F$ d\'efinie sur $I$ par
\[ F(x) = \int_J f(x,t) \D t. \]

% -----------------------------------------------------------------------------
\section{Continuit\'e}

\Para{Th\'eor\`eme}[continuit\'e sous le signe $\int$]

On suppose:
\begin{enumerate}
\item
  $f$ est continue par rapport \`a $x$, c.-\`a-d.

  \begin{itemize}
  \item
    pour tout $t\in J$ fix\'e, la fonction $x \mapsto f(x,t)$ est continue sur $I$.
  \end{itemize}
\item
  $f$ est continue par morceaux par rapport \`a $t$, c.-\`a-d.

  \begin{itemize}
  \item
    pour tout $x\in I$ fix\'e, la fonction $t \mapsto f(x,t)$ est continue par morceaux sur $J$.
  \end{itemize}
\item
  \emph{Hypoth\`ese de domination:}
  Il existe une fonction $\Fn\varphi{J}{\Rp}$ telle que

  \begin{itemize}
  \item
    pour tout $(x,t)\in I\times J$, on a $\Abs{f(x,t)}\leq \varphi(t)$,
  \item
    $\varphi$ est continue par morceaux et int\'egrable sur $J$.
  \end{itemize}
\end{enumerate}

Alors:
\begin{itemize}
\item
  La fonction $F \colon x \mapsto \int_J f(x,t) \D t$ est d\'efinie et continue sur $I$.
\end{itemize}

\Para{Remarque}[version locale]

On peut remplacer l'hypoth\`ese de domination par:
\begin{itemize}
\item
  \emph{Hypoth\`ese de domination locale:}
  Pour tout segment $K$ inclus dans $I$,
  il existe une fonction $\varphi_K \colon J \to \Rp$ telle que

  \begin{itemize}
  \item
    pour tout $(x,t)\in K\times J$, on a $\Abs{f(x,t)}\leq \varphi_K(t)$,
  \item
    $\varphi_K$ est continue par morceaux et int\'egrable sur $J$.
  \end{itemize}
\end{itemize}

% -----------------------------------------------------------------------------
\section{D\'erivabilit\'e}

\Para{Th\'eor\`eme}[d\'erivation sous le signe $\int$]

On suppose:
\begin{enumerate}
\item
  $f$ est de classe $\CC1$ par rapport \`a $x$, c.-\`a-d.

  \begin{itemize}
  \item
    pour tout $t\in J$ fix\'e, la fonction $x \mapsto f(x,t)$
    est de classe $\CC1$ sur $I$.
  \end{itemize}
\item
  $f$ est continue par morceaux et int\'egrable par rapport \`a $t$, c.-\`a-d.

  \begin{itemize}
  \item
    pour tout $x\in I$ fix\'e, la fonction $t \mapsto f(x,t)$
    est continue par morceaux et int\'egrable sur $J$.
  \end{itemize}
\item
  $\DerPart fx$ est continue par morceaux par rapport \`a $t$, c.-\`a-d.

  \begin{itemize}
  \item
    pour tout $x\in I$ fix\'e, la fonction $t \mapsto \DerPart fx$
    est continues par morceaux sur $J$.
  \end{itemize}
\item
  \emph{Hypoth\`ese de domination:}
  Il existe une fonction $\varphi\colon J \to \Rp$ telle que

  \begin{itemize}
  \item
    pour tout $(x,t)\in I\times J$, on a $\left| \DerPart fx(x,t) \right| \leq{} \varphi(t)$,
  \item
    $\varphi$ est continue par morceaux et int\'egrable sur $J$.
  \end{itemize}
\end{enumerate}

Alors:
\begin{itemize}
\item
  La fonction $F \colon x \mapsto \int_J f(x,t) \D t$ est d\'efinie et de classe $\CC1$ sur $I$.
\item
  Pour tout $x\in I$, on a \[ F'(x) = \int_J \DerPart fx(x,t) \D t. \]
\end{itemize}

\Para{Remarque}[version locale]

On peut remplacer l'hypoth\`ese de domination par:
\begin{itemize}
\item
  \emph{Hypoth\`ese de domination locale:}
  Pour tout segment $K$ inclus dans $I$,
  il existe une fonction $\varphi_K \colon J \to \Rp$ telle que

  \begin{itemize}
  \item
    pour tout $(x,t)\in K\times J$, on a $\left| \DerPart fx(x,t) \right| \leq{} \varphi_K(t)$,
  \item
    $\varphi_K$ est continue par morceaux et int\'egrable sur $J$.
  \end{itemize}
\end{itemize}

\Para{Th\'eor\`eme}[extension aux fonctions de classe $\CC p$]

On suppose:
\begin{enumerate}
\item
  $f$ est de classe $\CC p$ par rapport \`a $x$, c.-\`a-d.

  \begin{itemize}
  \item
    pour tout $t\in J$ fix\'e, la fonction $x \mapsto f(x,t)$ est de classe $\CC p$ sur $I$.
  \end{itemize}
\item
  pour tout $0\leq k<p$,
  $\frac{\partial^k f}{\partial x^k}$ est continue par morceaux
  et int\'egrable par rapport \`a $t$, c.-\`a-d.

  \begin{itemize}
  \item
    pour tout $k\in\Dcro{0,p-1}$ et pour tout $x\in I$ fix\'es,
    la fonction $t \mapsto \frac{\partial^k f}{\partial x^k} (x,t)$
    est continue par morceaux et int\'egrable sur $J$.
  \end{itemize}
\item
  $\frac{\partial^p f}{\partial x^p}$ est continue par morceaux par rapport \`a $t$, c.-\`a-d.

  \begin{itemize}
  \item
    pour tout $x\in I$ fix\'e, la fonction $t \mapsto \frac{\partial^p f}{\partial x^p} (x,t)$
    est continue par morceaux sur $J$.
  \end{itemize}
\item
  \emph{Hypoth\`ese de domination:}
  Il existe une fonction $\varphi\colon J \to \Rp$ telle que

  \begin{itemize}
  \item
    pour tout $(x,t)\in I\times J$, on a $\left| \frac{\partial^p f}{\partial x^p} (x,t) \right| \leq{} \varphi(t)$,
  \item
    $\varphi$ est continue par morceaux et int\'egrable sur $J$.
  \end{itemize}
\end{enumerate}

Alors:
\begin{itemize}
\item
  La fonction $F \colon x \mapsto \int_J f(x,t) \D t$ est d\'efinie et de classe $\CC p$ sur $I$.
\item
  Pour tout $k\in\Dcro{0,p}$, pour tout $x\in I$,
  on a \[ F^{(k)}(x) = \int_J \frac{\partial^k f}{\partial x^k}(x,t) \D t. \]
\end{itemize}

\Para{Remarque}[version locale]

On peut remplacer l'hypoth\`ese de domination par:
\begin{itemize}
\item
  \emph{Hypoth\`ese de domination locale:}
  Pour tout segment $K$ inclus dans $I$,
  il existe une fonction $\varphi_K \colon J \to \Rp$ telle que

  \begin{itemize}
  \item
    pour tout $(x,t)\in K\times J$,
    on a $\left| \frac{\partial^p f}{\partial x^p} (x,t) \right| \leq{} \varphi_K(t)$,
  \item
    $\varphi_K$ est continue par morceaux et int\'egrable sur $J$.
  \end{itemize}
\end{itemize}

% -----------------------------------------------------------------------------
\section{Exercices}

% -----------------------------------------------------------------------------
\par\pagebreak[1]\par
\paragraph{Exercice 1}%
\hfill{\tiny 9570}%
\begingroup~

On consid\`ere la fonction d\'efinie par \[ f(x) = \int_0^1 e^{-x/t} \D t. \]
\begin{enumerate}
\item
  D\'eterminer l'ensemble de d\'efinition de $f$.
\item
  Montrer que $f$ est continue sur son ensemble de d\'efinition.
\item
  Montrer que $f$ est de classe $\CC1$ sur $\Rps$.
\item
  Montrer que $f$ est de classe $\CC\infty$ sur $\Rps$.
\item
  V\'erifier que $\forall x > 0$, \[ f''(x) = \frac{e^{-x}}{x}. \]
\end{enumerate}
\endgroup

% -----------------------------------------------------------------------------
\par\pagebreak[1]\par
\paragraph{Exercice 2}%
\hfill{\tiny 8746}%
\begingroup~

On cherche \`a calculer l'int\'egrale de Gauss
\[ \int_0^{+\infty} e^{-x^2} \D x. \]
Pour cela, on consid\`ere les fonctions d\'efinies par
\[ f(x) = \left(\int_0^x e^{-t^2} \D t \right)^2
  \quad\text{et}\quad
g(x) = \int_0^1 \frac{e^{-x^2(1+t^2)}}{1+t^2} \D t. \]
\begin{enumerate}
\item
  Montrer que $f$ et $g$ sont d\'efinies et de classe $\CC1$ sur $\R$.
\item
  Calculer $f'$ et $g'$.
\item
  En d\'eduire que $\forall x\in \R$, \[ f(x) + g(x) = \frac\pi4. \]
\item
  En d\'eduire la valeur de l'int\'egrale de Gauss.
\end{enumerate}
\endgroup

% -----------------------------------------------------------------------------
\par\pagebreak[1]\par
\paragraph{Exercice 3 (th\'eor\`eme de division des fonctions $\CC\infty$)}%
\hfill{\tiny 0899}%
\begingroup~

Soit $\Fn{f}\R \R$ de classe $\CC\infty$ et
\[ g(x) = \begin{cases}
    \dfrac{f(x)-f(0)}{x} & \text{si } x\neq0 \\
    \hfil f'(0)               & \text{si } x=0.
\end{cases} \]
\begin{enumerate}
\item
  V\'erifier que $g(x) = \int_0^1 f'(tx) \D t$.
\item
  En d\'eduire que $g$ est de classe $\CC\infty$.
\item
  \emph{G\'en\'eralisation.}
  Montrer que la fonction
  \[ g_n \colon x \mapsto \frac{1}{x^n}
  \left( f(x) - \sum_{k=0}^{n-1} \frac{x^k}{k!} f^{(k)}(0) \right) \]
  se prolonge en une fonction de classe $\CC\infty$ en $0$.
\end{enumerate}
\endgroup

% -----------------------------------------------------------------------------
\par\pagebreak[1]\par
\paragraph{Exercice 4}%
\hfill{\tiny 9643}%
\begingroup~

Montrer qu'il existe un unique r\'eel $x\in[0,\pi]$ tel que
\[ \int_0^\pi{} \cos(x\sin\theta) \D\theta{} = 0. \]
et calculer une valeur approch\'ee de $x$ \`a $10^{-2}$ pr\`es.

\emph{Indication:} Montrer que
\[ \int_0^\pi{} \cos(x\sin\theta) \D\theta{} = 2\int_0^1 \frac{\cos(xt)}{\sqrt{1-t^2}} \D t. \]
\endgroup

% -----------------------------------------------------------------------------
\par\pagebreak[1]\par
\paragraph{Exercice 5 (la fonction gamma)}%
\hfill{\tiny 7174}%
\begingroup~

Soit \[ \Gamma(x) = \int_0^{+\infty} t^{x-1} e^{-t} \D t. \]
\begin{enumerate}
\item
  Montrer que $\Gamma$ est de classe $\CC\infty$ sur $\Rps$.
\item
  Montrer que $\Gamma$ est convexe.
\item
  Montrer que $\ln\circ \Gamma$ est convexe.
\item
  Montrer que $\forall x\in\Rps$, $\Gamma(x+1) = x\Gamma(x)$.
\item
  Montrer que $\forall n\in \N$, $\Gamma(n+1) = n!$.
\item
  Effectuer le changement de variables $t = n + y\sqrt n$
  dans l'int\'egrale $\Gamma(n+1)$,
  pour obtenir une expression de la forme
  \[ \Gamma(n+1) = n^n e^{-n} \sqrt n \int_{-\infty}^{+\infty} f_n(y) \D y. \]
\item
  Soit $\varphi(y) = \begin{cases}
      e^{-y^2/2}  & \text{si } y < 0 \\
      (1+y)e^{-y} & \text{si } y\geq0.
  \end{cases}$

  V\'erifier que $\forall n\in\Ns$, $\forall y\in \R$, $\Abs{f_n(y)}\leq \varphi(y)$.
\item
  En d\'eduire la \emph{formule de Stirling}
  \[ n! \Sim_{n\to+\infty} \left( \frac{n}{e} \right)^n \sqrt{2\pi n}. \]
\end{enumerate}
\endgroup

% -----------------------------------------------------------------------------
\par\pagebreak[1]\par
\paragraph{Exercice 6}%
\hfill{\tiny 8743}%
\begingroup~

Soit $\Fn{f}{\Rp}\R$ continue telle que
$\int_0^{+\infty} f$ converge absolument.
On pose \[ \varphi(a) = \int_0^{+\infty} e^{-at}f(t) \D t. \]
\begin{enumerate}
\item
  Montrer que $\varphi$ est d\'efinie sur $\Rp$.
\item
  Montrer que $\varphi$ est continue sur $\Rp$.
\item
  Montrer que $\varphi$ est de classe $\CC\infty$ sur $\Rps$.
\end{enumerate}
\endgroup

% -----------------------------------------------------------------------------
\par\pagebreak[1]\par
\paragraph{\href{https://psi.miomio.fr/exo/5290.pdf}{Exercice 7} (le m\^eme en nettement plus dur)}%
\hfill{\tiny 5290}%
\begingroup~

Soit $\Fn{f}{\Rp}\R$ continue telle que
$\int_0^{+\infty} f$ converge (pas n\'ecessairement absolument).
On pose \[ \varphi(a) = \int_0^{+\infty} e^{-at}f(t) \D t \quad\text{et}\quad
F(x) = \int_x^{+\infty} f(t) \D t. \]
\begin{enumerate}
\item
  Montrer que $F$ est d\'efinie, continue et born\'ee sur $\Rp$.
\item
  Montrer que $\varphi$ est d\'efinie sur $\Rp$ et que $\forall a\in\Rps$,
  \[ \varphi(a) = F(0) - a\int_0^{+\infty} e^{-at} F(t) \D t. \]
\item
  Montrer que $\varphi$ est de classe $\CC\infty$ sur $\Rps$.
\item
  Montrer que $\varphi$ est continue en $0$.
\end{enumerate}
\endgroup

% -----------------------------------------------------------------------------
\par\pagebreak[1]\par
\paragraph{Exercice 8}%
\hfill{\tiny 4901}%
\begingroup~

On consid\`ere $\DS f(x) = \int_0^{+\infty} \frac{\D t}{t^x (1+t)}$.
\begin{enumerate}
\item
  Domaine de d\'efinition, monotonie, convexit\'e de $f$ (sans d\'eriver $f$).
\item
  Continuit\'e, d\'erivabilit\'e, calcul de $f^{(k)}$.
\item
  Donner un \'equivalent de $f(x)$ en $0$ et en $1$.
\item
  Calculer $f(\frac1n)$ pour $n\in \N$, $n\geq2$.
\end{enumerate}
\endgroup

% -----------------------------------------------------------------------------
\par\pagebreak[1]\par
\paragraph{Exercice 9}%
\hfill{\tiny 9011}%
\begingroup~

Soit $\DS f(x) = \int_0^{+\infty} \frac{e^{-tx}}{1+t^2} \D t$.
\begin{enumerate}
\item
  Montrer que $f$ est d\'efinie et continue sur $\R$.
\item
  Montrer que $f$ est de classe $\CC\infty$ sur $\Rps$.
\item
  Montrer que $\forall x > 0$, $f(x) + f''(x) = \frac{1}{x}$.
\end{enumerate}
\endgroup

% -----------------------------------------------------------------------------
\par\pagebreak[1]\par
\paragraph{Exercice 10}%
\hfill{\tiny 3916}%
\begingroup~

Soit $\DS f(x) = \int_0^{+\infty} \frac{\D t}{1+x^3+t^3}$.
\begin{enumerate}
\item
  Montrer que $f$ est d\'efinie sur $\Rp$.
\item
  \`A l'aide du changement de variable $u = \frac1t$, calculer $f(0)$.
\item
  Montrer que $f$ est continue et d\'ecroissante.
\item
  D\'eterminer $\lim\limits_{+\infty} f$.
\end{enumerate}
\endgroup

% -----------------------------------------------------------------------------
\par\pagebreak[1]\par
\paragraph{Exercice 11}%
\hfill{\tiny 1793}%
\begingroup~

Soit $\DS f(x) = \int_0^{\pi/2} \sin^x(t) \D t$.
\begin{enumerate}
\item
  Montrer que $f$ est d\'efinie et positive sur $\intO{-1,+\infty}$.
\item
  Montrer que $f$ est de classe $\CC1$ et pr\'eciser sa monotonie.
\item
  Former une relation entre $f(x)$ et $f(x+2)$ pour tout $x > -1$.
\item
  On pose pour $x > 0$, $\varphi(x) = xf(x)f(x-1)$.
  Montrer que
  \[ \forall x > 0, \quad\varphi(x+1) =\varphi(x). \]
\item
  D\'eterminer un \'equivalent de $f$ en $-1^+$.
\end{enumerate}
\endgroup

% -----------------------------------------------------------------------------
\par\pagebreak[1]\par
\paragraph{Exercice 12}%
\hfill{\tiny 3248}%
\begingroup~

On consid\`ere les fonctions $f$ et $g$ d\'efinies sur $\Rp$ par
\[ f(x) = \int_0^{+\infty} \frac{e^{-xt}}{1+t^2} \D t
  \quad\text{et}\quad
g(x) = \int_0^{+\infty} \frac{\sin t}{x+t} \D t. \]
\begin{enumerate}
\item
  Monter que $f$ et $g$ sont de classe $\CC2$ sur $\Rps$ et
  qu'elles sont solutions de l'\'equation diff\'erentielle
  $y'' + y = \frac1x$.
\item
  Montrer que $f$ et $g$ sont continues en $0$.
\item
  En d\'eduire que \[ \int_0^{+\infty} \frac{\sin t}{t} \D t = \frac\pi2. \]
\end{enumerate}
\endgroup

% -----------------------------------------------------------------------------
\par\pagebreak[1]\par
\paragraph{Exercice 13}%
\hfill{\tiny 3460}%
\begingroup~

Existence et calcul de
\[ \varphi(x) = \int_0^{+\infty} e^{-t^2} \cos(xt) \D t. \]
On pourra montrer que $\varphi$ est solution d'une \'equation diff\'erentielle
que l'on r\'esoudra.
\endgroup

% -----------------------------------------------------------------------------
\par\pagebreak[1]\par
\paragraph{Exercice 14}%
\hfill{\tiny 7817}%
\begingroup~

Soit $f$ d\'efinie sur $\Rp$ par \[ f(x) = \int_0^{+\infty} \frac{\arctan(xt)}{1+t^2} \D t. \]
\begin{enumerate}
\item
  \'Etudier la continuit\'e et la d\'erivabilit\'e de $f$.
\item
  Pour $x > 0$, calculer $f'(x)$.
\item
  En d\'eduire que \[ \int_0^1 \frac{\ln t}{t^2-1} \D t = \frac{\pi^2}{8}. \]
\item
  Bonus: montrer que $f$ n'est pas d\'erivable en $0$.
  Pour cela, on pourra minorer $f(x)/x$.
\end{enumerate}
\endgroup

% -----------------------------------------------------------------------------
\par\pagebreak[1]\par
\paragraph{\href{https://psi.miomio.fr/exo/3286.pdf}{Exercice 15}}%
\hfill{\tiny 3286}%
\begingroup~

Soit $\DS f(x) = \int_0^{+\infty} \frac{\cos(xt)}{1+t^2} \D t$.
\begin{enumerate}
\item
  Montrer que $f$ est d\'efinie et continue sur $\R$.
\item
  Montrer que $f$ est de classe $\CC1$ sur $\Rps$
  et que pour tout $x > 0$, on a
  \[ f'(x) = \frac{1}{x}\int_0^{+\infty} \cos(xt)\cdot\frac{t^2 - 1}{(1+t^2)^2} \D t. \]
\item
  Montrer que $f$ est de classe $\CC2$ sur $\Rps$
  et que pour $x > 0$, on a
  \[ f''(x) = \frac{2}{x^2}\int_0^{+\infty} \cos(xt)\cdot\frac{1-3t^2}{(1+t^2)^3} \D t. \]
\item
  En d\'eduire que
  \[ \forall{} x > 0 \+ f''(x) = f(x). \]
\item
  En d\'eduire une expression simple de $f$;
  on pourra calculer $f(0)$ et $\lim\limits_{+\infty} f$.
\item
  Calculer \[ \int_0^{+\infty} \frac{t\sin(xt)}{1+t^2}\D t. \]
\end{enumerate}
\endgroup

% -----------------------------------------------------------------------------
\par\pagebreak[1]\par
\paragraph{Exercice 16}%
\hfill{\tiny 2745}%
\begingroup~

En utilisant une \'equation diff\'erentielle lin\'eaire du premier ordre, calculer
\[ f(x) =\int_0^{+\infty} \frac{e^{-t}}{\sqrt t}{e^{\I x t}} \D t. \]
\endgroup

\end{document}
