% autogenerated by ytex.rs

\documentclass{scrartcl}

\usepackage[francais]{babel}
\usepackage{geometry}
\usepackage{scrpage2}
\usepackage{lastpage}
\usepackage{multicol}
\usepackage{etoolbox}
\usepackage{xparse}
\usepackage{enumitem}
% \usepackage{csquotes}
\usepackage{amsmath}
\usepackage{amsfonts}
\usepackage{amssymb}
\usepackage{mathrsfs}
\usepackage{stmaryrd}
\usepackage{dsfont}
% \usepackage{eurosym}
% \usepackage{numprint}
% \usepackage[most]{tcolorbox}
\usepackage{tikz}
% \usepackage{tkz-tab}
\usepackage[unicode]{hyperref}
\usepackage[ocgcolorlinks]{ocgx2}

\let\ifTwoColumns\iftrue
\def\Classe{$\Psi$2019--2020}

% reproducible builds
% LuaTeX: \pdfvariable suppressoptionalinfo 1023 \relax
\pdfinfoomitdate=1
\pdftrailerid{}

\newif\ifDisplaystyle
\everymath\expandafter{\the\everymath\ifDisplaystyle\displaystyle\fi}
\newcommand\DS{\displaystyle}

\clearscrheadfoot
\pagestyle{scrheadings}
\thispagestyle{empty}
\ohead{\Classe}
\ihead{\thepage/\pageref*{LastPage}}

\setlist[itemize,1]{label=\textbullet}
\setlist[itemize,2]{label=\textbullet}

\ifTwoColumns
  \geometry{margin=1cm,top=2cm,bottom=3cm,foot=1cm}
  \setlist[enumerate]{leftmargin=*}
  \setlist[itemize]{leftmargin=*}
\else
  \geometry{margin=3cm}
\fi

\makeatletter
\let\@author=\relax
\let\@date=\relax
\renewcommand\maketitle{%
    \begin{center}%
        {\sffamily\huge\bfseries\@title}%
        \ifx\@author\relax\else\par\medskip{\itshape\Large\@author}\fi
        \ifx\@date\relax\else\par\bigskip{\large\@date}\fi
    \end{center}\bigskip
    \ifTwoColumns
        \par\begin{multicols*}{2}%
        \AtEndDocument{\end{multicols*}}%
        \setlength{\columnsep}{5mm}
    \fi
}
\makeatother

\newcounter{ParaNum}
\NewDocumentCommand\Para{smo}{%
  \IfBooleanF{#1}{\refstepcounter{ParaNum}}%
  \paragraph{\IfBooleanF{#1}{{\tiny\arabic{ParaNum}~}}#2\IfNoValueF{#3}{ (#3)}}}

\newcommand\I{i}
\newcommand\mi{i}
\def\me{e}

\def\do#1{\expandafter\undef\csname #1\endcsname}
\docsvlist{Ker,sec,csc,cot,sinh,cosh,tanh,coth,th}
\undef\do

\DeclareMathOperator\ch{ch}
\DeclareMathOperator\sh{sh}
\DeclareMathOperator\th{th}
\DeclareMathOperator\coth{coth}
\DeclareMathOperator\cotan{cotan}
\DeclareMathOperator\argch{argch}
\DeclareMathOperator\argsh{argsh}
\DeclareMathOperator\argth{argth}

\let\epsilon=\varepsilon
\let\phi=\varphi
\let\leq=\leqslant
\let\geq=\geqslant
\let\subsetneq=\varsubsetneq
\let\emptyset=\varnothing

\newcommand{\+}{,\;}

\undef\C
\newcommand\ninf{{n\infty}}
\newcommand\N{\mathbb{N}}
\newcommand\Z{\mathbb{Z}}
\newcommand\Q{\mathbb{Q}}
\newcommand\R{\mathbb{R}}
\newcommand\C{\mathbb{C}}
\newcommand\K{\mathbb{K}}
\newcommand\Ns{\N^*}
\newcommand\Zs{\Z^*}
\newcommand\Qs{\Q^*}
\newcommand\Rs{\R^*}
\newcommand\Cs{\C^*}
\newcommand\Ks{\K^*}
\newcommand\Rp{\R^+}
\newcommand\Rps{\R^+_*}
\newcommand\Rms{\R^-_*}
\newcommand{\Rpinf}{\Rp\cup\Acco{+\infty}}

\undef\B
\newcommand\B{\mathscr{B}}

\undef\P
\DeclareMathOperator\P{\mathbb{P}}
\DeclareMathOperator\E{\mathbb{E}}
\DeclareMathOperator\Var{\mathbb{V}}

\DeclareMathOperator*\PetitO{o}
\DeclareMathOperator*\GrandO{O}
\DeclareMathOperator*\Sim{\sim}
\DeclareMathOperator\Tr{tr}
\DeclareMathOperator\Ima{Im}
\DeclareMathOperator\Ker{Ker}
\DeclareMathOperator\Sp{Sp}
\DeclareMathOperator\Diag{diag}
\DeclareMathOperator\Rang{rang}
\DeclareMathOperator*\Coords{Coords}
\DeclareMathOperator*\Mat{Mat}
\DeclareMathOperator\Pass{Pass}
\DeclareMathOperator\Com{Com}
\DeclareMathOperator\Card{Card}
\DeclareMathOperator\Racines{Racines}
\DeclareMathOperator\Vect{Vect}
\DeclareMathOperator\Id{Id}

\newcommand\DerPart[2]{\frac{\partial #1}{\partial #2}}

\def\T#1{{#1}^T}

\def\pa#1{({#1})}
\def\Pa#1{\left({#1}\right)}
\def\bigPa#1{\bigl({#1}\bigr)}
\def\BigPa#1{\Bigl({#1}\Bigr)}
\def\biggPa#1{\biggl({#1}\biggr)}
\def\BiggPa#1{\Biggl({#1}\Biggr)}

\def\pafrac#1#2{\pa{\frac{#1}{#2}}}
\def\Pafrac#1#2{\Pa{\frac{#1}{#2}}}
\def\bigPafrac#1#2{\bigPa{\frac{#1}{#2}}}
\def\BigPafrac#1#2{\BigPa{\frac{#1}{#2}}}
\def\biggPafrac#1#2{\biggPa{\frac{#1}{#2}}}
\def\BiggPafrac#1#2{\BiggPa{\frac{#1}{#2}}}

\def\cro#1{[{#1}]}
\def\Cro#1{\left[{#1}\right]}
\def\bigCro#1{\bigl[{#1}\bigr]}
\def\BigCro#1{\Bigl[{#1}\Bigr]}
\def\biggCro#1{\biggl[{#1}\biggr]}
\def\BiggCro#1{\Biggl[{#1}\Biggr]}

\def\abs#1{\mathopen|{#1}\mathclose|}
\def\Abs#1{\left|{#1}\right|}
\def\bigAbs#1{\bigl|{#1}\bigr|}
\def\BigAbs#1{\Bigl|{#1}\Bigr|}
\def\biggAbs#1{\biggl|{#1}\biggr|}
\def\BiggAbs#1{\Biggl|{#1}\Biggr|}

\def\acco#1{\{{#1}\}}
\def\Acco#1{\left\{{#1}\right\}}
\def\bigAcco#1{\bigl\{{#1}\bigr\}}
\def\BigAcco#1{\Bigl\{{#1}\Bigr\}}
\def\biggAcco#1{\biggl\{{#1}\biggr\}}
\def\BiggAcco#1{\Biggl\{{#1}\Biggr\}}

\def\ccro#1{\llbracket{#1}\rrbracket}
\def\Dcro#1{\llbracket{#1}\rrbracket}

\def\floor#1{\lfloor#1\rfloor}
\def\Floor#1{\left\lfloor{#1}\right\rfloor}

\def\sEnsemble#1#2{\mathopen\{#1\mid#2\mathclose\}}
\def\bigEnsemble#1#2{\bigl\{#1\bigm|#2\bigr\}}
\def\BigEnsemble#1#2{\Bigl\{#1\Bigm|#2\Bigr\}}
\def\biggEnsemble#1#2{\biggl\{#1\biggm|#2\biggr\}}
\def\BiggEnsemble#1#2{\Biggl\{#1\Biggm|#2\Biggr\}}
\let\Ensemble=\bigEnsemble

\newcommand\IntO[1]{\left]#1\right[}
\newcommand\IntF[1]{\left[#1\right]}
\newcommand\IntOF[1]{\left]#1\right]}
\newcommand\IntFO[1]{\left[#1\right[}

\newcommand\intO[1]{\mathopen]#1\mathclose[}
\newcommand\intF[1]{\mathopen[#1\mathclose]}
\newcommand\intOF[1]{\mathopen]#1\mathclose]}
\newcommand\intFO[1]{\mathopen[#1\mathclose[}

\newcommand\Fn[3]{#1\colon#2\to#3}
\newcommand\CC[1]{\mathscr{C}^{#1}}
\newcommand\D{\mathop{}\!\mathrm{d}}

\newcommand\longto{\longrightarrow}

\undef\M
\newcommand\M[3]{\mathrm{#1}_{#2}\pa{#3}}
\newcommand\MnR{\M{M}{n}{\R}}
\newcommand\MnC{\M{M}{n}{\C}}
\newcommand\MnK{\M{M}{n}{\K}}
\newcommand\GLnR{\M{GL}{n}{\R}}
\newcommand\GLnC{\M{GL}{n}{\C}}
\newcommand\GLnK{\M{GL}{n}{\K}}
\newcommand\DnR{\M{D}{n}{\R}}
\newcommand\DnC{\M{D}{n}{\C}}
\newcommand\DnK{\M{D}{n}{\K}}
\newcommand\SnR{\M{S}{n}{\R}}
\newcommand\AnR{\M{A}{n}{\R}}
\newcommand\OnR{\M{O}{n}{\R}}
\newcommand\SnRp{\mathrm{S}_n^+(\R)}
\newcommand\SnRpp{\mathrm{S}_n^{++}(\R)}

\newcommand\LE{\mathscr{L}(E)}
\newcommand\GLE{\mathscr{GL}(E)}
\newcommand\SE{\mathscr{S}(E)}
\renewcommand\OE{\mathscr{O}(E)}

\newcommand\ImplD{$\Cro\Rightarrow$}
\newcommand\ImplR{$\Cro\Leftarrow$}
\newcommand\InclD{$\Cro\subset$}
\newcommand\InclR{$\Cro\supset$}
\newcommand\notInclD{$\Cro{\not\subset}$}
\newcommand\notInclR{$\Cro{\not\supset}$}

\newcommand\To[1]{\xrightarrow[#1]{}}
\newcommand\Toninf{\To{\ninf}}

\newcommand\Norm[1]{\|#1\|}
\newcommand\Norme{{\Norm{\cdot}}}

\newcommand\Int[1]{\mathring{#1}}
\newcommand\Adh[1]{\overline{#1}}

\newcommand\Uplet[2]{{#1},\dots,{#2}}
\newcommand\nUplet[3]{(\Uplet{{#1}_{#2}}{{#1}_{#3}})}

\newcommand\Fonction[5]{{#1}\left|\begin{aligned}{#2}&\;\longto\;{#3}\\{#4}&\;\longmapsto\;{#5}\end{aligned}\right.}

\DeclareMathOperator\orth{\bot}
\newcommand\Orth[1]{{#1}^\bot}
\newcommand\PS[2]{\langle#1,#2\rangle}

\newcommand{\Tribu}{\mathscr{T}}
\newcommand{\Part}{\mathcal{P}}
\newcommand{\Pro}{\bigPa{\Omega,\Tribu}}
\newcommand{\Prob}{\bigPa{\Omega,\Tribu,\P}}

\newcommand\DEMO{$\spadesuit$}
\newcommand\DUR{$\spadesuit$}

\newenvironment{psmallmatrix}{\left(\begin{smallmatrix}}{\end{smallmatrix}\right)}

% -----------------------------------------------------------------------------

\usetikzlibrary{automata, arrows.meta, chains, quotes}

\begin{document}
\title{Cha\^ines de Markov}
\maketitle

Ce chapitre n'est pas vraiment au programme,
mais intervient dans de nombreux sujets d'\'ecrit de concours.

% -----------------------------------------------------------------------------
\section{Un exemple}

Un automate poss\`ede quatre \'etats, not\'es $A$, $B$, $C$ et $D$.
\`A l'instant $t = 0$, l'automate est dans l'\'etat $A$.
\`A chaque instant, l'automate change d'\'etat selon le diagramme ci-dessous.
On cherche \`a \'etudier le comportement de l'automate au bout d'un grand nombre d'\'etapes.

\begin{center}
  \begin{tikzpicture}[-Triangle]
    \node[state] (A) at (0,0) {$A$};
    \node[state] (B) at (5,0) {$B$};
    \node[state] (C) at (5,-5) {$C$};
    \node[state] (D) at (0,-5) {$D$};

    \path
    (A) edge[loop left,  "$1/6$"] node {} (A)
    (A) edge[bend left,  "$1/6$"] node {} (B)
    (A) edge[right, "$1/3$"] node {} (C)
    (A) edge[right, "$1/3$"] node {} (D)

    (B) edge[bend left,  "$1/3$"] node {} (A)
    (B) edge[left, "$2/3$"] node {} (C)

    (C) edge[loop right, "$1/3$"] node {} (C)
    (C) edge[bend left,  "$2/3$"] node {} (D)

    (D) edge[bend left,  "$1$"] node {} (C);
  \end{tikzpicture}
\end{center}

Pour cela, on introduit la \emph{matrice de transition} du graphe:
\[
  A = \frac16 \begin{pmatrix}
    1 & 1 & 2 & 2 \\
    2 & 0 & 4 & 0 \\
    0 & 0 & 2 & 4 \\
  0 & 0 & 6 & 0 \end{pmatrix}
\]

Ici, elle est diagonalisable, mais ce n'est pas toujours le cas.
\[ \chi_A = (X+2/3)(X+1/6)(X-1/3)(X-1) \]
On a $A = PDP^{-1}$, o\`u
\[ D = \Diag\Pa{-\frac23, -\frac16, \frac13, 1}, \]
\[ P = \begin{pmatrix} -8 & -1 & 1 & 1 \\ 22 & 2 & 1 & 1 \\ -18 & 0 & 0 & 1 \\ 27 & 0 & 0 & 1 \end{pmatrix}, \]
\[ P^{-1} = \frac{1}{45} \begin{pmatrix} 0 & 0 & -1 & 1 \\ -15 & 15 & -10 & -10 \\ 30 & 15 & -25 & -20 \\ 0 & 0 & 27 & 18 \end{pmatrix}.\]

Notons \[ U_n = \begin{pmatrix} \P(X_n=A) \\ \P(X_n=B) \\ \P(X_n=C) \\ \P(X_n=D) \end{pmatrix} \]

On a \[ U_{n+1} = \T{A} U_n, \]
de sorte que, par une r\'ecurrence triviale
\[ U_n = \pa{\T{A}}^n U_0 \]

On trouve,
\[ U_n \Toninf \begin{pmatrix} 0 \\ 0 \\ 3/5 \\ 2/5 \end{pmatrix} \]

Les \'etats $A$ et $B$ sont dits \emph{transients}:
on peut montrer que (presque s\^urement) l'automate n'y passera qu'un nombre fini de fois.
Les \'etats $C$ et $D$ sont dits \emph{r\'ecurrents}:
l'automate y passera (presque s\^urement) une infinit\'e de fois.

% -----------------------------------------------------------------------------
\section{Matrices stochastiques}

\Para{D\'efinition}

Une matrice $M \in{} \MnR$ est \emph{stochastique}
si et seulement si elle est \`a coefficients positifs et si la somme de chaque ligne vaut~1.

\Para{Exemple}

La matrice de transition d'une cha\^ine de Markov est toujours stochastique.

\Para{Proposition}

Le produit de deux matrices stochastique l'est \'egalement.

\Para{Proposition}
Soit $M \in{} \MnK$.
Alors $M$ et $M^T$ ont
\begin{itemize}
\item
  le m\^eme polyn\^ome caract\'eristique;
\item
  les m\^emes valeurs propres, le m\^eme spectre;
\item
  mais pas les m\^emes vecteurs propres, ni les m\^emes sous-espaces propres;
\item
  les m\^emes multiplicit\'es pour la valeur propre $\lambda$;
\item
  les m\^emes dimensions des sous-espaces propres associ\'es \`a la valeur propre $\lambda$.
\end{itemize}

\Para{Th\'eor\`eme (facile)}

Soit $M$ une matrice stochastique.
Alors
\begin{itemize}
\item
  1 est une valeur propre de $M$;
\item
  toute valeur propre $\lambda$ de $M$ v\'erifie $\abs{\lambda} \leq{} 1$.
\end{itemize}

\Para{Th\'eor\`eme de Perron-Frobenius (dur)}

Soit $M$ une matrice stochastique.
On suppose qu'il existe un $p\in\Ns$ tel que tous les coefficients de $M^p$ soient strictement positifs.
Alors
\begin{itemize}
\item
  $1$ est une valeur propre simple de $M$;
\item
  toute valeur propre $\lambda$ de $M$ diff\'erente de~1 v\'erifie $\abs{\lambda} < 1$.
\end{itemize}

% -----------------------------------------------------------------------------
\section{Cha\^ines de Markov}

On consid\`ere un espace probabilis\'e $\pa{\Omega,\mathcal{T},\P}$.
Pour simplifier les notations, on notera les \'etats $1,2,\dots,n$.

\Para{D\'efinition}

Une \emph{cha\^ine de Markov}
\`a valeurs dans $\ccro{1,n}$
est une suite de variables al\'eatoires variables al\'eatoires $(X_n)_{n\in \N}$ \`a valeurs dans $\ccro{1,n}$
telles que les quantit\'es
\[ \P(X_{k+1}=j \mid X_k = i) \]
soient ind\'ependantes de $k$.
Autrement dit,
\[ \forall k\in \N{} \+ \forall(i,j)\in\ccro{1,n}^2 \+ \P(X_{k+1}=j \mid X_k = i) = p_{i,j}. \]
o\`u $A = \bigPa{p_{i,j}}_{1\leq i,j\leq n} \in{} \MnR$ est une matrice stochastique fix\'ee.

\Para{Proposition}

Le coefficient $(i,j)$ de la matrice $P^k$
est la probabilit\'e de passer de l'\'etat $i$ \`a l'\'etat $j$ en $k$ \'etapes,
c.-\`a-d.
\[ \P(X_{m+k} = j \mid X_m = j) = \Cro{A^k}_{i,j} \]

\Para{Proposition}
Notons
\[ U_m = \begin{pmatrix} \P(X_m = 1) \\ \P(X_m = 2) \\ \vdots \\ \P(X_m = n) \end{pmatrix} \in{} \M{M}{n,1}{\R}. \]
On a pour tout $m\in \N$,
\[ U_{m+1} = \T{A} U_m, \]
d'o\`u
\[ U_m = \Pa{\T{A}}^m U_0. \]

\Para{Proposition}
Si $U_m \to \pi{} \in{} \M{M}{n,1}{\R}$,
alors $\pi$ est un vecteur propre de $\T{A}$ associ\'e \`a la valeur propre~1.
Tous les coefficients de $\pi$ sont positifs et leur somme vaut~1.

\Para{Th\'eor\`eme (dur)}
Si $A$ v\'erifie les hypoth\`eses du th\'eor\`eme de Perron-Frobenius,
c.-\`a-d. s'il existe un $p\in\Ns$ tel que $A^p$ soit \`a coefficients strictement positifs,
alors:
\begin{itemize}
\item
  il existe un unique vecteur $\pi$
  \`a coefficients positifs et dont la somme des coefficients vaut~1
  tel que $\T{A} \pi{} = \pi$;
\item
  la suite $(U_m)_{m\in \N}$ tend vers $\pi$
  ind\'ependamment des conditions initiales $U_0$.
\end{itemize}

% -----------------------------------------------------------------------------
\section{Exercices}

% -----------------------------------------------------------------------------
\par\pagebreak[1]\par
\paragraph{\href{https://psi.miomio.fr/exo/4098.pdf}{Exercice 1} (d'apr\`es Wikipedia)}%
\hfill{\tiny 4098}%
\begingroup~

\usetikzlibrary{automata, arrows.meta, chains, quotes}
Doudou le hamster ne conna\^it que trois endroits dans sa cage:
les copeaux o\`u il dort, la mangeoire o\`u il mange et la roue o\`u il fait de l'exercice.
Ses journ\'ees sont assez semblables les unes aux autres, et son activit\'e se repr\'esente ais\'ement par une cha\^ine de Markov.
Toutes les minutes, il peut soit changer d'activit\'e, soit continuer celle qu'il \'etait en train de faire.
L'appellation processus sans m\'emoire n'est pas du tout exag\'er\'ee pour parler de Doudou.

Quand il dort, il a 9 chances sur 10 de ne pas se r\'eveiller la minute suivante.
Quand il se r\'eveille, il y a 1 chance sur 2 qu'il aille manger et 1 chance sur 2 qu'il parte faire de l'exercice.
Le repas ne dure qu'une minute, apr\`es il fait autre chose.
Apr\`es avoir mang\'e, il y a 3 chances sur 10 qu'il parte courir dans sa roue, mais surtout 7 chances sur 10 qu'il retourne dormir.
Courir est fatigant pour Doudou; il y a 8 chances sur 10 qu'il retourne dormir au bout d'une minute.
Sinon il continue en oubliant qu'il est d\'ej\`a un peu fatigu\'e.

Combien de temps Doudou passe-t-il \`a dormir, manger et faire de l'exercice?
\endgroup

\end{document}
