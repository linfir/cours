% autogenerated by ytex.rs

\documentclass{scrartcl}

\usepackage[francais]{babel}
\usepackage{geometry}
\usepackage{scrpage2}
\usepackage{lastpage}
\usepackage{multicol}
\usepackage{etoolbox}
\usepackage{xparse}
\usepackage{enumitem}
% \usepackage{csquotes}
\usepackage{amsmath}
\usepackage{amsfonts}
\usepackage{amssymb}
\usepackage{mathrsfs}
\usepackage{stmaryrd}
\usepackage{dsfont}
% \usepackage{eurosym}
% \usepackage{numprint}
% \usepackage[most]{tcolorbox}
% \usepackage{tikz}
% \usepackage{tkz-tab}
\usepackage[unicode]{hyperref}
\usepackage[ocgcolorlinks]{ocgx2}

\let\ifTwoColumns\iftrue
\def\Classe{$\Psi$2019--2020}

% reproducible builds
% LuaTeX: \pdfvariable suppressoptionalinfo 1023 \relax
\pdfinfoomitdate=1
\pdftrailerid{}

\newif\ifDisplaystyle
\everymath\expandafter{\the\everymath\ifDisplaystyle\displaystyle\fi}
\newcommand\DS{\displaystyle}

\clearscrheadfoot
\pagestyle{scrheadings}
\thispagestyle{empty}
\ohead{\Classe}
\ihead{\thepage/\pageref*{LastPage}}

\setlist[itemize,1]{label=\textbullet}
\setlist[itemize,2]{label=\textbullet}

\ifTwoColumns
  \geometry{margin=1cm,top=2cm,bottom=3cm,foot=1cm}
  \setlist[enumerate]{leftmargin=*}
  \setlist[itemize]{leftmargin=*}
\else
  \geometry{margin=3cm}
\fi

\makeatletter
\let\@author=\relax
\let\@date=\relax
\renewcommand\maketitle{%
    \begin{center}%
        {\sffamily\huge\bfseries\@title}%
        \ifx\@author\relax\else\par\medskip{\itshape\Large\@author}\fi
        \ifx\@date\relax\else\par\bigskip{\large\@date}\fi
    \end{center}\bigskip
    \ifTwoColumns
        \par\begin{multicols*}{2}%
        \AtEndDocument{\end{multicols*}}%
        \setlength{\columnsep}{5mm}
    \fi
}
\makeatother

\newcounter{ParaNum}
\NewDocumentCommand\Para{smo}{%
  \IfBooleanF{#1}{\refstepcounter{ParaNum}}%
  \paragraph{\IfBooleanF{#1}{{\tiny\arabic{ParaNum}~}}#2\IfNoValueF{#3}{ (#3)}}}

\newcommand\I{i}
\newcommand\mi{i}
\def\me{e}

\def\do#1{\expandafter\undef\csname #1\endcsname}
\docsvlist{Ker,sec,csc,cot,sinh,cosh,tanh,coth,th}
\undef\do

\DeclareMathOperator\ch{ch}
\DeclareMathOperator\sh{sh}
\DeclareMathOperator\th{th}
\DeclareMathOperator\coth{coth}
\DeclareMathOperator\cotan{cotan}
\DeclareMathOperator\argch{argch}
\DeclareMathOperator\argsh{argsh}
\DeclareMathOperator\argth{argth}

\let\epsilon=\varepsilon
\let\phi=\varphi
\let\leq=\leqslant
\let\geq=\geqslant
\let\subsetneq=\varsubsetneq
\let\emptyset=\varnothing

\newcommand{\+}{,\;}

\undef\C
\newcommand\ninf{{n\infty}}
\newcommand\N{\mathbb{N}}
\newcommand\Z{\mathbb{Z}}
\newcommand\Q{\mathbb{Q}}
\newcommand\R{\mathbb{R}}
\newcommand\C{\mathbb{C}}
\newcommand\K{\mathbb{K}}
\newcommand\Ns{\N^*}
\newcommand\Zs{\Z^*}
\newcommand\Qs{\Q^*}
\newcommand\Rs{\R^*}
\newcommand\Cs{\C^*}
\newcommand\Ks{\K^*}
\newcommand\Rp{\R^+}
\newcommand\Rps{\R^+_*}
\newcommand\Rms{\R^-_*}
\newcommand{\Rpinf}{\Rp\cup\Acco{+\infty}}

\undef\B
\newcommand\B{\mathscr{B}}

\undef\P
\DeclareMathOperator\P{\mathbb{P}}
\DeclareMathOperator\E{\mathbb{E}}
\DeclareMathOperator\Var{\mathbb{V}}

\DeclareMathOperator*\PetitO{o}
\DeclareMathOperator*\GrandO{O}
\DeclareMathOperator*\Sim{\sim}
\DeclareMathOperator\Tr{tr}
\DeclareMathOperator\Ima{Im}
\DeclareMathOperator\Ker{Ker}
\DeclareMathOperator\Sp{Sp}
\DeclareMathOperator\Diag{diag}
\DeclareMathOperator\Rang{rang}
\DeclareMathOperator*\Coords{Coords}
\DeclareMathOperator*\Mat{Mat}
\DeclareMathOperator\Pass{Pass}
\DeclareMathOperator\Com{Com}
\DeclareMathOperator\Card{Card}
\DeclareMathOperator\Racines{Racines}
\DeclareMathOperator\Vect{Vect}
\DeclareMathOperator\Id{Id}

\newcommand\DerPart[2]{\frac{\partial #1}{\partial #2}}

\def\T#1{{#1}^T}

\def\pa#1{({#1})}
\def\Pa#1{\left({#1}\right)}
\def\bigPa#1{\bigl({#1}\bigr)}
\def\BigPa#1{\Bigl({#1}\Bigr)}
\def\biggPa#1{\biggl({#1}\biggr)}
\def\BiggPa#1{\Biggl({#1}\Biggr)}

\def\pafrac#1#2{\pa{\frac{#1}{#2}}}
\def\Pafrac#1#2{\Pa{\frac{#1}{#2}}}
\def\bigPafrac#1#2{\bigPa{\frac{#1}{#2}}}
\def\BigPafrac#1#2{\BigPa{\frac{#1}{#2}}}
\def\biggPafrac#1#2{\biggPa{\frac{#1}{#2}}}
\def\BiggPafrac#1#2{\BiggPa{\frac{#1}{#2}}}

\def\cro#1{[{#1}]}
\def\Cro#1{\left[{#1}\right]}
\def\bigCro#1{\bigl[{#1}\bigr]}
\def\BigCro#1{\Bigl[{#1}\Bigr]}
\def\biggCro#1{\biggl[{#1}\biggr]}
\def\BiggCro#1{\Biggl[{#1}\Biggr]}

\def\abs#1{\mathopen|{#1}\mathclose|}
\def\Abs#1{\left|{#1}\right|}
\def\bigAbs#1{\bigl|{#1}\bigr|}
\def\BigAbs#1{\Bigl|{#1}\Bigr|}
\def\biggAbs#1{\biggl|{#1}\biggr|}
\def\BiggAbs#1{\Biggl|{#1}\Biggr|}

\def\acco#1{\{{#1}\}}
\def\Acco#1{\left\{{#1}\right\}}
\def\bigAcco#1{\bigl\{{#1}\bigr\}}
\def\BigAcco#1{\Bigl\{{#1}\Bigr\}}
\def\biggAcco#1{\biggl\{{#1}\biggr\}}
\def\BiggAcco#1{\Biggl\{{#1}\Biggr\}}

\def\ccro#1{\llbracket{#1}\rrbracket}
\def\Dcro#1{\llbracket{#1}\rrbracket}

\def\floor#1{\lfloor#1\rfloor}
\def\Floor#1{\left\lfloor{#1}\right\rfloor}

\def\sEnsemble#1#2{\mathopen\{#1\mid#2\mathclose\}}
\def\bigEnsemble#1#2{\bigl\{#1\bigm|#2\bigr\}}
\def\BigEnsemble#1#2{\Bigl\{#1\Bigm|#2\Bigr\}}
\def\biggEnsemble#1#2{\biggl\{#1\biggm|#2\biggr\}}
\def\BiggEnsemble#1#2{\Biggl\{#1\Biggm|#2\Biggr\}}
\let\Ensemble=\bigEnsemble

\newcommand\IntO[1]{\left]#1\right[}
\newcommand\IntF[1]{\left[#1\right]}
\newcommand\IntOF[1]{\left]#1\right]}
\newcommand\IntFO[1]{\left[#1\right[}

\newcommand\intO[1]{\mathopen]#1\mathclose[}
\newcommand\intF[1]{\mathopen[#1\mathclose]}
\newcommand\intOF[1]{\mathopen]#1\mathclose]}
\newcommand\intFO[1]{\mathopen[#1\mathclose[}

\newcommand\Fn[3]{#1\colon#2\to#3}
\newcommand\CC[1]{\mathscr{C}^{#1}}
\newcommand\D{\mathop{}\!\mathrm{d}}

\newcommand\longto{\longrightarrow}

\undef\M
\newcommand\M[3]{\mathrm{#1}_{#2}\pa{#3}}
\newcommand\MnR{\M{M}{n}{\R}}
\newcommand\MnC{\M{M}{n}{\C}}
\newcommand\MnK{\M{M}{n}{\K}}
\newcommand\GLnR{\M{GL}{n}{\R}}
\newcommand\GLnC{\M{GL}{n}{\C}}
\newcommand\GLnK{\M{GL}{n}{\K}}
\newcommand\DnR{\M{D}{n}{\R}}
\newcommand\DnC{\M{D}{n}{\C}}
\newcommand\DnK{\M{D}{n}{\K}}
\newcommand\SnR{\M{S}{n}{\R}}
\newcommand\AnR{\M{A}{n}{\R}}
\newcommand\OnR{\M{O}{n}{\R}}
\newcommand\SnRp{\mathrm{S}_n^+(\R)}
\newcommand\SnRpp{\mathrm{S}_n^{++}(\R)}

\newcommand\LE{\mathscr{L}(E)}
\newcommand\GLE{\mathscr{GL}(E)}
\newcommand\SE{\mathscr{S}(E)}
\renewcommand\OE{\mathscr{O}(E)}

\newcommand\ImplD{$\Cro\Rightarrow$}
\newcommand\ImplR{$\Cro\Leftarrow$}
\newcommand\InclD{$\Cro\subset$}
\newcommand\InclR{$\Cro\supset$}
\newcommand\notInclD{$\Cro{\not\subset}$}
\newcommand\notInclR{$\Cro{\not\supset}$}

\newcommand\To[1]{\xrightarrow[#1]{}}
\newcommand\Toninf{\To{\ninf}}

\newcommand\Norm[1]{\|#1\|}
\newcommand\Norme{{\Norm{\cdot}}}

\newcommand\Int[1]{\mathring{#1}}
\newcommand\Adh[1]{\overline{#1}}

\newcommand\Uplet[2]{{#1},\dots,{#2}}
\newcommand\nUplet[3]{(\Uplet{{#1}_{#2}}{{#1}_{#3}})}

\newcommand\Fonction[5]{{#1}\left|\begin{aligned}{#2}&\;\longto\;{#3}\\{#4}&\;\longmapsto\;{#5}\end{aligned}\right.}

\DeclareMathOperator\orth{\bot}
\newcommand\Orth[1]{{#1}^\bot}
\newcommand\PS[2]{\langle#1,#2\rangle}

\newcommand{\Tribu}{\mathscr{T}}
\newcommand{\Part}{\mathcal{P}}
\newcommand{\Pro}{\bigPa{\Omega,\Tribu}}
\newcommand{\Prob}{\bigPa{\Omega,\Tribu,\P}}

\newcommand\DEMO{$\spadesuit$}
\newcommand\DUR{$\spadesuit$}

\newenvironment{psmallmatrix}{\left(\begin{smallmatrix}}{\end{smallmatrix}\right)}

% -----------------------------------------------------------------------------

\undef\U
\undef\Var{}
\newcommand{\En}{(E,\Norme)}
\newcommand{\U}{(u_n)_{n\in \N}}
\newcommand\Var{(v_n)_{n\in \N}}

\begin{document}
\title{Espaces vectoriels norm\'es en dimension finie}
\maketitle

% -----------------------------------------------------------------------------
\section{Normes}

\subsection{G\'en\'eralit\'es}

\Para{D\'efinition}

Soit $E$ un $\K$-espace vectoriel.
Une \emph{norme} sur $E$ est une application $\Fn{N}{E}{\R}$ v\'erifiant les axiomes suivants.
Pour tous $(x,y)\in E^2$ et pour tout $\lambda\in \K$:\begin{enumerate}
\item \emph{positivit\'e:} $N(x)\geq0$
\item \emph{s\'eparation:} $N(x) = 0 \implies x = 0_E$
\item \emph{homog\'en\'eit\'e:} $N(\lambda x) = \Abs{\lambda} N(x)$
\item \emph{in\'egalit\'e triangulaire:} $N(x+y) \leq{} N(x) + N(y)$
\end{enumerate}

\Para{Remarque}

Le nom d'in\'egalit\'e triangulaire vient du fait que dans un triangle $ABC$ on a n\'ecessairement $AC \leq AB + BC$.

\Para{Remarque}

On note fr\'equemment $\Norm{x}$ au lieu de $N(x)$.

\Para{Exemples}

Sur $E = \K^n$.

Pour $x = \nUplet x1n\in E$:
\begin{enumerate}
\item $\DS \Norm{x}_1 = \sum_{k=1}^n \Abs{x_k}$
\item $\DS \Norm{x}_2 = \sqrt{\sum_{k=1}^n \Abs{x_k}^2}$
\item $\DS \Norm{x}_\infty{} = \max\BigPa{ \Uplet{\Abs{x_1}}{\Abs{x_n}} }$
\end{enumerate}

\Para{Proposition}[in\'egalit\'e triangulaire invers\'ee]

Soit $E$ un espace vectoriel, $N$ une norme sur $E$ et $(x,y)\in E^2$.
On a $\Abs{N(x) - N(y)}\leq N(x+y)$.

\Para{D\'efinition}

Un \emph{espace vectoriel norm\'e} est un couple $(E,N)$ o\`u $E$ est un espace vectoriel et $N$ un norme sur $E$.
On note parfois $E$ au lieu de $(E,N)$ si le choix de la norme est clair d'apr\`es le contexte.

\subsection{Distance}

\Para{D\'efinition}

Soit $\En$ un espace vectoriel norm\'e.
La \emph{distance associ\'ee} \`a $\Norme$ est l'application $\Fn{d}{E\times E}{\R}$ d\'efinie par:
$\forall(x,y)\in E^2$, $d(x,y) = \Norm{x-y}$.

\Para{Remarque}[HP]

De fa\c con g\'en\'erale, on appelle \emph{distance} sur un ensemble $E$
une application $\Fn{d}{E^2}{\R}$ v\'erifiant les axiomes suivants:
pour tous $(x,y,z)\in E^3$:
\begin{enumerate}
\item \emph{positivit\'e:} $d(x,y) \geq{} 0$
\item \emph{sym\'etrie:} $d(x,y) = d(y,x)$
\item \emph{s\'eparation:} $d(x,y) = 0 \implies x = y$
\item \emph{in\'egalit\'e triangulaire:} $d(x,y) \leq{} d(x,z) + d(z,y)$
\end{enumerate}

Si $d$ est une distance sur $E$, on dit que $(E,d)$ est un \emph{espace m\'etrique}.

\subsection{Boules, sph\`eres}

\Para{D\'efinitions}

Soit $\En$ un espace vectoriel norm\'e.\begin{itemize}
\item La \emph{boule ouverte} de centre $a\in E$ et de rayon $r\in \R$ est $B(a,r) = \Ensemble{x\in E}{d(x,a)<r}$.
\item La \emph{boule ferm\'ee} de centre $a\in E$ et de rayon $r\in \R$ est $B_f(a,r) = \Ensemble{x\in E}{d(x,a)\leq r}$.
\item La \emph{sph\`ere} de centre $a\in E$ et de rayon $r\in \R$ est $S(a,r) = \Ensemble{x\in E}{d(x,a)=r} = B_f(a,r) \setminus B(a,r)$.
\end{itemize}

\Para{Exemple}

On prend $E = \R^2$.

Tracer les sph\`eres unit\'e pour les normes $\Norme_1$, $\Norme_2$ et $\Norme_\infty$.

\subsection{Parties convexes}

\Para{D\'efinition}

Soit $\En$ un espace vectoriel et $A$ une partie de $E$.
On dit que $A$ est une \emph{partie convexe}, ou plus simplement que $A$ est \emph{convexe}, si et seulement si $\forall(x,y)\in A^2$, $\forall \lambda\in[0,1]$, $\lambda x + (1-\lambda)y\in A$.

\Para{Proposition}

Soit $\En$ un espace vectoriel norm\'e, $a\in E$ et $r\in \R$.
Les boules $B(a,r)$ et $B_f(a,r)$ sont convexes.

\subsection{Parties born\'ees, applications born\'ees}

\Para{D\'efinition}

Soit $\En$ un espace vectoriel norm\'e et $A$ une partie de $E$.
On dit que $A$ est une partie \emph{born\'ee} de $E$ si et seulement si $\exists M\geq0$, $\forall x\in A$, $\Norm{x}\leq M$.

\Para{D\'efinition}

Soit $(E,\Norme_E)$ et $(F,\Norme_F)$ deux espaces vectoriels norm\'es et $A\subset E$.
Soit $\Fn{f}{A}{F}$.
On dit que $f$ est une application \emph{born\'ee} si et seulement si la partie $f(A)$ est born\'ee, c.-\`a-d. si et seulement si $\exists M\geq0$, $\forall x\in A$, $\Norm{f(x)}_F\leq M$.

% -----------------------------------------------------------------------------
\section{Suites}

\Para{D\'efinitions}

Soit $\En$ un espace vectoriel norm\'e.\begin{itemize}
\item Une \emph{suite \`a valeurs dans $E$} est une application $\Fn{u}{\N}{E}$.
  On note fr\'equemment $u_n$ au lieu de $u(n)$ et $\U$ au lieu de $u$.
\item Soit $\U$ une suite \`a valeurs dans $E$ et $\ell\in E$.
  On dit que la suite $\U$ \emph{converge} vers $\ell$ si et seulement si
  \[\forall \epsilon> 0\+\exists n_0\in \N\+\forall n\geq n_0\+ \Norm{u_n -\ell}\leq \epsilon\]
  $\ell$ est n\'ecessairement unique et s'appelle \emph{la limite} de la suite $\U$.
  On note $\lim\limits_{\ninf}^{\Norme} u_n = \ell$, ou encore $u_n \xrightarrow[\ninf]{\Norme}\ell$.
\end{itemize}

\Para{Remarque}

Soit $\En$ un espace vectoriel norm\'e, $\U$ une suite \`a valeurs dans $E$ et $\ell\in E$.
Alors $u_n \xrightarrow{\Norme}\ell$ si et seulement si $\Norm{u_n - \ell} \to 0$.

\Para{D\'efinitions}
\begin{enumerate}
\item Une suite $\U$ \`a valeurs dans un espace vectoriel norm\'e $\En$ est dite \emph{convergente} si et seulement si il existe $\ell\in E$
  tel que $\U$ converge vers $\ell$.
\item Elle est dite \emph{divergente} dans le cas contraire.
\end{enumerate}

\Para{D\'efinition (HP)}

Soit $E$ un espace vectoriel, $N$ et $N'$ deux normes sur $E$.
On dit que $N$ et $N'$ sont \emph{\'equivalentes} si et seulement s'il existe deux constantes strictement positives $\alpha$ et $\beta$
telles que \[ \forall x\in E \+ \alpha N(x) \leq{} N'(x) \leq{} \beta N(x). \]

\Para{Propri\'et\'es (HP)}

\begin{itemize}
\item Il s'agit d'une relation d'\'equivalence, c.-\`a-d.
  \begin{itemize}
  \item toute norme est \'equivalente \`a elle-m\^eme;
  \item si $N$ est \'equivalente \`a $N'$, alors $N'$ est \'equivalente \`a $N$;
  \item deux normes \'equivalentes \`a une m\^eme troisi\`eme sont \'equivalentes entre elles.
  \end{itemize}
\item Si $N$ et $N'$ sont deux normes \'equivalentes,
  alors pour toute suite $(u_n)$ \`a valeurs dans $E$ et pour tout vecteur $\ell\in E$, on a
  \[ u_n \xrightarrow{N} \ell{}  \iff u_n \xrightarrow{N'} \ell. \]
\end{itemize}

\Para{Th\'eor\`eme (HP)}

Si $E$ est de dimension finie, toutes les normes sur $E$ sont \'equivalentes.

\Para{Corollaire fondamental}

Si $E$ est de dimension finie, le choix de la norme n'influe pas sur la convergence d'une suite, ni sur sa limite.
On pourra donc \'ecrire $u_n \to \ell$, sans pr\'eciser la norme choisie.

\Para{Proposition}

En dimension finie, la convergence d'une suite de vecteurs est \'equivalente \`a la convergence des suites coordonn\'ees.
Plus pr\'ecis\'ement, soit:
\begin{enumerate}
\item $\En$ un $\K$-espace vectoriel norm\'e de dimension finie
\item $\B = \nUplet e1p$ une base de $E$
\item $(u_n)_{n\in \N}$ une suite \`a valeurs dans $E$
\item $\Uplet{(u^1_n)_{n\in \N}}{(u^p_n)_{n\in \N}}$ des suites num\'eriques telles que:
  $\forall n\in \N$, $u_n = u^1_n e_1 + u^2_n e_2 + \cdots + u^p_n e_p$
\item $\ell{} = \ell^1 e_1 + \ell^2 e_2 + \cdots + \ell^p e_p$ o\`u $\ell\in E$ et $(\Uplet{\ell^1}{\ell^p})\in \K^p$.
\end{enumerate}
Alors \[ u_n \to \ell{} \iff \forall k\in\Dcro{1,p} \+ u^k_n \to\ell^k. \]

\Para{Proposition}[op\'erations alg\'ebriques sur les limites]

Soit $\En$ un $\K$-espace vectoriel norm\'e.
Soit $\U$ et $\Var$ deux suites \`a valeurs dans $E$
et $(\lambda_n)_{n\in \N}$ une suite \`a valeurs dans $\K$.\begin{itemize}
\item Si $u_n \to \ell$ et $v_n \to \ell'$, alors $u_n + v_n \to \ell+\ell'$.
\item Si $\lambda_n \to \alpha$ et $u_n \to \ell$, alors $\lambda_n u_n \to \alpha \ell$.
\end{itemize}

\Para{D\'efinitions}[relations de comparaison entre suites]

Soit $\En$ un $\K$-espace vectoriel norm\'e.
Soit $\U$ une suite \`a valeurs dans $E$ et $(\alpha_n)_{n\in \N}$ une suite num\'erique \`a valeurs dans $\K$.\begin{enumerate}
\item On dit que $u_n = O(\alpha_n)$ si et seulement si $\Norm{u_n} = O(\alpha_n)$ si et seulement si $\exists N\in \N$, $\exists K\in\Rp$, $\forall n\geq N$, $\Norm{u_n} \leq{} K\Abs{\alpha_n}$.
  On dit alors que $\U$ est \emph{domin\'ee} par $(\alpha_n)_{n\in \N}$.
\item On dit que $u_n = o(\alpha_n)$ si et seulement si $\Norm{u_n} = o(\alpha_n)$ si et seulement si $\forall \epsilon>0$, $\exists N\in \N$, $\forall n\geq N$, $\Norm{u_n} \leq{} \epsilon\Abs{\alpha_n}$.
  On dit alors que $\U$ est \emph{n\'egligeable} devant $(\alpha_n)_{n\in \N}$.
\end{enumerate}

% -----------------------------------------------------------------------------
\section{Topologie}

\subsection{Ouverts et ferm\'es}

\Para{D\'efinitions}

Soit $\En$ un espace vectoriel norm\'e.\begin{itemize}
\item Une \emph{partie ouverte} de $E$, ou plus simplement un \emph{ouvert} de $E$, est une partie $U$ de $E$ telle que $\forall x\in U$, $\exists r>0$, $B(x,r)\subset U$.
\item Une \emph{partie ferm\'ee} de $E$, ou plus simplement un \emph{ferm\'e} de $E$, est une partie $F$ de $E$ telle que $E\setminus F$ est un ouvert de $E$.
\end{itemize}

\Para{Proposition}

Soit $\En$ un espace vectoriel norm\'e. Soit $a\in E$ et $r\in \R$. Alors:\begin{itemize}
\item la boule ouverte $B(a,r)$ est une partie ouverte de $E$;
\item la boule ferm\'ee $B_f(a,r)$ est une partie ferm\'ee de $E$.
\end{itemize}

\Para{Proposition}[propri\'et\'es des ouverts]

Soit $\En$ un espace vectoriel norm\'e.\begin{itemize}
\item $\emptyset$ et $E$ sont des ouverts.
\item \emph{stabilit\'e par union quelconque}:
  Si $\forall i\in I$, $U_i$ est un ouvert de $E$, alors $\bigcup_{i\in I} U_i$ est un ouvert de $E$.
\item \emph{stabilit\'e par intersection finie}:
  Si $U_1, \dots, U_n$ sont des ouverts de $E$, alors $\bigcap_{i=1}^n U_i$ est un ouvert de $E$.
\end{itemize}

\Para{Proposition}[propri\'et\'es des ferm\'es]

Soit $\En$ un espace vectoriel norm\'e.\begin{itemize}
\item $\emptyset$ et $E$ sont des ferm\'es.
\item \emph{stabilit\'e par intersection quelconque}:
  Si $\forall i\in I$, $F_i$ est un ferm\'e de $E$, alors $\bigcap_{i\in I} F_i$ est un ferm\'e de $E$.
\item \emph{stabilit\'e par union finie}:
  Si $F_1, \cdots, F_n$ sont des ferm\'es de $E$, alors $\bigcup_{i=1}^n F_i$ est un ferm\'e de $E$.
\end{itemize}

\Para{Proposition}

Soit $E$ un espace vectoriel \emph{de dimension finie}.
Alors la topologie de $E$ ne d\'epend pas du choix de la norme sur $E$.

Autrement dit, si $A\subset E$, le caract\`ere ouvert (ou ferm\'e) de $A$ ne d\'epend pas du choix de la norme sur $E$.

\subsection{Point int\'erieur, point adh\'erent}

\Para{Proposition-D\'efinition}

Soit $\En$ un espace vectoriel norm\'e, $A$ une partie de $E$ et $x\in E$.
Les conditions suivantes sont \'equivalentes:\begin{enumerate}
\item il existe un ouvert $U$ de $E$ tel que $x\in U$ et $U\subset A$;
\item $\exists r > 0$, $B(x,r)\subset A$;
\item pour toute suite $\U$ \`a valeurs dans $E$ convergeant vers $x$, on a $u_n\in A$ \`a partir d'un certain rang.
\end{enumerate}

On dit que $x$ est un \emph{point int\'erieur} \`a $A$ lorsque ces conditions sont v\'erifi\'ees.

\Para{Proposition-D\'efinition}

Soit $\En$ un espace vectoriel norm\'e, $A$ une partie de $E$ et $x\in E$.
Les conditions suivantes sont \'equivalentes:\begin{enumerate}
\item pour tout ouvert $U$ de $E$ tel que $x\in U$, on a $U\cap A\neq\emptyset$;
\item $\forall r>0$, $B(x,r)\cap A \neq\emptyset$;
\item $\forall \epsilon>0$, $\exists a\in A$, $\Norm{x-a}\leq \epsilon$;
\item il existe une suite $\U$ \`a valeurs \emph{dans $A$} telle que $u_n \to x$.
\end{enumerate}

On dit que $x$ est un \emph{point adh\'erent} \`a $A$ lorsque ces conditions sont v\'erifi\'ees.

\subsection{Int\'erieur, adh\'erence, fronti\`ere}

\Para{D\'efinitions}

Soit $\En$ un espace vectoriel norm\'e, $A$ une partie de $E$.\begin{itemize}
\item L'ensemble des points int\'erieurs de $A$ s'appelle l'\emph{int\'erieur} de $A$,
  et on le note $\Int{A}$
\item L'ensemble des points adh\'erents \`a $A$ s'appelle l'\emph{adh\'erence} de $A$,
  et on le note $\Adh{A}$.
\item L'adh\'erence de $A$ priv\'ee de l'int\'erieur de $A$ s'appelle la \emph{fronti\`ere} de $A$ et on le note $\mathrm{Fr}(A) = \partial A = \Adh{A}\setminus\Int{A}$.
\end{itemize}

\Para{Proposition}
\begin{itemize}
\item si $A \subset{} B$, alors $\Int A \subset{} \Int B$ et $\Adh A \subset{} \Adh B$;
\item on a toujours $\Int{A} \subset{} A \subset{} \Adh{A}$;
\item $\Int{A}$ est un ouvert, $\Adh{A}$ est un ferm\'e;
\item $A = \Int{A}$ si et seulement si $A$ est un ouvert;
\item $A = \Adh{A}$ si et seulement si $A$ est un ferm\'e.
\end{itemize}

\subsection{Densit\'e}

\Para{Proposition-D\'efinition}

Soit $\En$ un espace vectoriel norm\'e et $A$ une partie de $E$.
Les conditions suivantes sont \'equivalentes:\begin{enumerate}
\item pour tout ouvert $U$ non vide de $E$, on a $U\cap A\neq\emptyset$;
\item $\forall x\in E$, $\forall r > 0$, $B(x,r)\cap A\neq\emptyset$;
\item tout point de $E$ est adh\'erent \`a $A$;
\item $\Adh{A} = E$;
\item pour tout $x\in E$, il existe une suite $\U$ \`a valeurs \emph{dans $A$} telle que $u_n \to x$.
\end{enumerate}

On dit que $A$ est une partie \emph{dense} de $E$ lorsque ces conditions sont v\'erifi\'ees.

% -----------------------------------------------------------------------------
\section{Applications dans un espace vectoriel norm\'e}

\subsection{Limite d'une application}

\Para{D\'efinition}

Soit $(E,\Norme_E)$ et $(F,\Norme_F)$ deux espaces vectoriels norm\'es, $A$ une partie de $E$, $\Fn{f}{A}{F}$, $a\in E$ un point adh\'erent \`a $A$ et $\ell\in F$.
On dit que $f$ admet $\ell$ pour \emph{limite} en $a$ si et seulement si
$\forall \epsilon> 0$, $\exists \delta>0$, $\forall x\in A$, $\Norm{x-a}_E\leq \delta\implies \Norm{f(x)-\ell}_F\leq \epsilon$.
On \'ecrit alors $\lim_{x \to a} f(x) = \lim_a f = \ell$.

\Para{Remarque}

On peut comme d'habitude \'etendre la d\'efinition de la limite dans les cas suivants:\begin{itemize}
\item si $E = \R$ et $a = \pm\infty$;
\item si $F = \R$ et $\ell= \pm\infty$.
\end{itemize}

\Para{Th\'eor\`eme}[crit\`ere s\'equentiel des limites]

Soit $E$ et $F$ deux espaces vectoriels norm\'es et $A\subset E$.
Soit $\Fn{f}{A}{F}$, $a$ un point adh\'erent \`a $A$ et $b\in F$.
Les conditions suivantes sont \'equivalentes:\begin{enumerate}
\item $\lim_a f = b$
\item Toute suite \`a valeur dans $A$ telle que $u_n \to a$ v\'erifie $f(u_n) \to b$
\end{enumerate}

\Para{Proposition}[limite et coordonn\'ees]

Soit $(E,\Norme_E)$ et $(F,\Norme_F)$ deux espaces vectoriels norm\'es, $A$ une partie de $E$, $\Fn{f}{A}{F}$, $a\in E$ un point adh\'erent \`a $A$ et $\ell\in F$.
On suppose que:\begin{itemize}
\item \emph{$F$ est de dimension finie} et $\B = \nUplet e1p$ une base de $F$.
\item $\forall x\in A$, $f(x) = \sum_{k=1}^p f_k(x) e_k$ o\`u $\Fn{f_k}{A}{\K}$.
\item $\ell= \sum_{k=1}^p \ell_k e_k$ o\`u $\nUplet\ell1p \in \K^p$.
\end{itemize}

Alors $\lim_{x \to a} f(x) = \ell$ si et seulement si $\forall k\in\Dcro{1,p}\+ \lim_{x \to a} f_k(x) =\ell_k$.

\Para{Proposition}[limite d'une compos\'ee]

Soit $E$, $F$, $G$ trois espaces vectoriels norm\'es, $A\subset E$ et $B\subset F$.
Soit $\Fn{f}{A}{F}$ et $\Fn{g}{B}{G}$.
On suppose:\begin{itemize}
\item $f(A)\subset B$ de sorte que la compos\'ee $\Fn{g\circ f}{A}{G}$ existe;
\item $a$ est un point adh\'erent \`a $A$ et $\lim_a f = b$;
\item $b$ est un point adh\'erent \`a $B$ et $\lim_b g = \ell$.
\end{itemize}

Alors $\lim_a g\circ f = \ell$.

\Para{Proposition}[op\'erations alg\'ebriques sur les limites]

Soit $E$ et $F$ deux espaces vectoriels norm\'es et $A\subset E$.
Soit $\Fn{f}{A}{F}$, $g \colon A \to F$ et $k \colon A \to\K$.
Soit $a$ un point adh\'erent \`a $A$.\begin{itemize}
\item Si $\lim_a f =\ell$ et $\lim_a g =\ell'$, alors $\lim_a (f+g) = \ell+\ell'$.
\item Si $\lim_a k =\alpha$ et $\lim_a f =\ell$, alors $\lim_a (kf) = \alpha \ell$.
\end{itemize}

\subsection{Continuit\'e}

\Para{D\'efinition}

Soit $E$ et $F$ deux espaces vectoriels norm\'es et $A\subset E$.
Soit $\Fn{f}{A}{F}$.
On dit que $f$ est \emph{continue en $a\in A$} si et seulement si $\lim_a f = f(a)$, c.-\`a-d. si et seulement si
\begin{multline*}
  \forall \epsilon>0 \+ \exists \delta>0 \+ \forall x\in A \+ \\
  \Norm{x-a}_E\leq \delta\implies \Norm{f(x)-f(a)}_F\leq \epsilon.
\end{multline*}
On dit que $f$ est \emph{continue} si et seulement si elle est continue en tout point de $A$.

\Para{Proposition}[continuit\'e et coordonn\'ees]

Soit $E$ et $F$ deux espaces vectoriels norm\'es et $A\subset E$.
Soit $\Fn{f}{A}{F}$ et $a\in A$.
On suppose que:\begin{itemize}
\item $F$ est de dimension finie et $\mathcal{B} = \nUplet e1p$ une base de $F$;
\item $\forall x\in A$, $f(x) =\sum_{k=1}^p f_k(x) e_k$ o\`u $f_k \colon A \to\K$.
\end{itemize}

Alors $f$ est continue en $a$ si et seulement si les fonctions $\Uplet{f_1}{f_p}$ sont toutes continues en $a$.

\Para{Proposition}[propri\'et\'es des fonctions continues]
\begin{itemize}
\item La compos\'ee de deux fonctions continues est continue.
\item La restriction d'une fonction continue est continue.
\end{itemize}

\Para{Th\'eor\`eme}

Soit $E$ et $F$ deux espaces vectoriels norm\'es et $\Fn{f}{E}{F}$ une fonction continue.
Soit $A\subset F$ une partie de $F$\begin{itemize}
\item Si $A$ est un ouvert de $F$, alors $f^{-1}(A)$ est un ouvert de $E$.
\item Si $A$ est un ferm\'e de $F$, alors $f^{-1}(A)$ est un ferm\'e de $E$.
\end{itemize}

\Para{Corollaire}

Soit $E$ un espaces vectoriels norm\'es et $\Fn{f}{E}{\R}$ une fonction continue.\begin{itemize}
\item $\Ensemble{x\in E}{f(x)>0}$ est un ouvert de $E$.
\item $\Ensemble{x\in E}{f(x)=0}$ est un ferm\'e de $E$.
\item $\Ensemble{x\in E}{f(x)\geq0}$ est un ferm\'e de $E$.
\end{itemize}

\subsection{Applications lipschitziennes}

\Para{D\'efinition}

Soit $(E, \Norme_E)$ et $(F, \Norme_F)$ deux $\K$-espaces vectoriels norm\'es et $A\subset E$.
Soit $\Fn{f}{A}{F}$.\begin{itemize}
\item Soit $k\in\Rp$. L'application $f$ est dite \emph{$k$-lipschitzienne} si et seulement si $\forall(x,y)\in A^2$, $\Norm{f(x)-f(y)}_F\leq k \Norm{x-y}_E$.
\item L'application $f$ est dite \emph{lipschitzienne}, ou v\'erifie la \emph{propri\'et\'e de Lipschitz}, si et seulement si il existe $k\in\Rp$ telle que $f$ soit $k$-lipschitzienne.
\end{itemize}

\Para{Proposition}

La compos\'ee de deux applications lipschitzienne l'est \'egalement.

\Para{Proposition}

Toute application lipschitzienne est continue.
La r\'eciproque est fausse.

\subsection{Applications lin\'eaires}

\Para{Proposition}[continuit\'e des applications lin\'eaires]

Soit $E$ un espace vectoriel norm\'e de dimension finie, $F$ un espace vectoriel norm\'e et $\Fn{f}{E}{F}$ \emph{lin\'eaire}.
Alors $f$ est lipschitzienne (et donc continue).

\Para{Proposition}[continuit\'e des applications multilin\'eaire]

Toute application multilin\'eaire dont l'ensemble de d\'epart est un espace vectoriel norm\'e de dimension finie est continue.

\subsection{Compacit\'e}

\Para{D\'efinition}

Soit $E$ un espace vectoriel norm\'e de dimension finie.
On dit qu'une partie $A$ de $E$ est \emph{compacte} si et seulement si elle est ferm\'ee et born\'ee

\Para{Th\'eor\`eme}

Soit $E$ et $F$ deux espaces vectoriels norm\'es de dimensions finies, $A\subset E$ et $\Fn{f}{A}{F}$ une fonction \emph{continue}.
Si $K\subset A$ est une partie compacte de $E$,
alors $f(K)$ est une partie compacte de $F$.

\Para{Corollaire important}

Soit $E$ un espace vectoriel norm\'e de dimension finie, $A$ une partie ferm\'ee et born\'ee de $E$ et $\Fn{f}{A}{\R}$ une fonction continue.
Alors $f$ est born\'ee et ses bornes sont atteintes.

% -----------------------------------------------------------------------------
\section{Exercices}

\subsection{Normes}

% -----------------------------------------------------------------------------
\par\pagebreak[1]\par
\paragraph{Exercice 1}%
\hfill{\tiny 7918}%
\begingroup~

Soit $E=\K^n$ et $\Norme_1$, $\Norme_2$, $\Norme_\infty$ les normes usuelles.
\begin{enumerate}
\item Montrer que pour tout $x\in \K^n$, on a
  \begin{align*}
    \Norm{x}_1 & \leq{} \sqrt n \, \Norm{x}_2 \\
    \Norm{x}_2 & \leq{} \sqrt n \, \Norm{x}_\infty{} \\
    \Norm{x}_\infty{} & \leq{} \Norm{x}_1
  \end{align*}
\item En d\'eduire que les normes $\Norme_1$, $\Norme_2$ et $\Norme_\infty$ sont \'equivalentes.
\item Montrer que les constantes sont optimales.
\end{enumerate}
\endgroup

% -----------------------------------------------------------------------------
\par\pagebreak[1]\par
\paragraph{Exercice 2 (normes matricielles)}%
\hfill{\tiny 2086}%
\begingroup~

Soit $E = \MnK$.
Pour $A\in E$, $A = \BigPa{a_{i,j}}_{1\leq i,j\leq n}$, on pose:
\begin{itemize}
\item $\DS N_1(A) = \sum_{i=1}^n\sum_{j=1}^n \Abs{a_{i,j}}$;
\item $\DS N_2(A) = \sqrt{\Tr(\T{\bar A}A)} = \sqrt{\sum_{i=1}^n\sum_{j=1}^n \Abs{a_{i,j}}^2}$;
\item $\DS N_\infty(A) = \max_{1\leq i,j\leq n} \Abs{a_{i,j}}$;
\item $\DS N(A) = \max_{1\leq i\leq n} \sum_{j=1}^n \Abs{a_{i,j}}$.
\end{itemize}
\begin{enumerate}
\item Montrer qu'il s'agit de normes sur $E$.
\item Montrer que la norme $N$ est une \emph{norme d'alg\`ebre}, c.-\`a-d. que
  \[ \forall(A,B)\in E^2 \+ N(AB) \leq{} N(A) N(B). \]
\end{enumerate}
\endgroup

% -----------------------------------------------------------------------------
\par\pagebreak[1]\par
\paragraph{Exercice 3 (normes de polyn\^omes)}%
\hfill{\tiny 8972}%
\begingroup~

Soit $E = \K[X]$.
Pour $P\in E$, $\DS P =\sum_{k=0}^d a_k X^k$, on pose:
\begin{itemize}
\item $\DS N_1(P) = \sum_{k=0}^d \Abs{a_k}$;
\item $\DS N_2(P) = \sqrt{\sum_{k=0}^d \Abs{a_k}^2}$;
\item $\DS N_\infty(P) = \max_{0\leq k\leq d} \Abs{a_k}$;
\item $\DS N(P) = \sup_{x\in[0,1]} \Abs{P(x)}$.
\end{itemize}
\begin{enumerate}
\item Montrer qu'il s'agit de normes sur $E$.
\item On pose $P_n = n^{-3/4} \sum_{k=0}^{n-1} X^k$.
  D\'eterminer $\lim_\ninf N_1(P_n)$ et $\lim_\ninf N_2(P_n)$.
  Que remarquez-vous?
\end{enumerate}
\endgroup

% -----------------------------------------------------------------------------
\par\pagebreak[1]\par
\paragraph{Exercice 4 (normes de fonctions)}%
\hfill{\tiny 7922}%
\begingroup~

Soit $I$ un intervalle de $\R$.
\begin{enumerate}
\item
  On note $\mathscr{L}^1$ l'ensemble des fonctions $f$ continues et int\'egrables de $I$ dans $\K$.
  Pour $f \in{} \mathscr{L}^1$, on note
  \[ \Norm{f}_1 = \int_I \abs{f(x)} \D x. \]
  Montrer qu'il s'agit d'une norme sur $\mathscr{L}^1$.

\item
  On note $\mathscr{L}^2$ l'ensemble des fonctions $f$ continues de $I$ dans $\K$ telles que $f^2$ est int\'egrable sur $I$.
  Pour $f \in{} \mathscr{L}^2$, on note
  \[ \Norm{f}_2 = \sqrt{\int_I \abs{f(x)}^2 \D x}. \]
  Montrer que $\mathscr{L}^2$ est un espace vectoriel et que $\Norme_2$ est une norme sur $\mathscr{L}^2$.

\item
  On note $\mathscr{L}^\infty$ l'ensemble des fonctions $f$ continues et born\'ees de $I$ dans $\K$.
  Pour $f \in{} \mathscr{L}^\infty$, on note
  \[ \Norm{f}_\infty{} = \sup_{x\in I} \abs{f(x)}. \]
  Montrer qu'il s'agit d'une norme sur $\mathscr{L}^\infty$.

\end{enumerate}
\endgroup

% -----------------------------------------------------------------------------
\par\pagebreak[1]\par
\paragraph{Exercice 5}%
\hfill{\tiny 4934}%
\begingroup~

Sur $\K^n$, on pose, pour $p>1$,
\[ \Norm{ \nUplet x1n }_p = \Pa{ \sum_{k=1}^n \Abs{x_k}^p }^{\frac1p}. \]
\begin{enumerate}
\item Montrer que $\Norme_p$ est une norme;
  (on admettra l'in\'egalit\'e triangulaire, un peu d\'elicate (voir l'exercice suivant).
\item D\'eterminer pour $x\in \K^n$, $\lim\limits_{p\to+\infty} \Norm{x}_p$.
\end{enumerate}
\endgroup

% -----------------------------------------------------------------------------
\par\pagebreak[1]\par
\paragraph{\href{https://psi.miomio.fr/exo/4948.pdf}{Exercice 6} (suite: preuve de l'in\'egalit\'e triangulaire)}%
\hfill{\tiny 4948}%
\begingroup~

Soit $p > 1$.
On note $q = \frac{p}{p-1}$.
\begin{enumerate}
\item
  Montrer que $q > 1$ et que $\frac{1}{p} + \frac{1}{q} = 1$.
\item
  Montrer que $\forall(x,y)\in \R_+^2$, $xy \leq{} \frac{x^p}{p} + \frac{y^q}{q}$.
\item
  Soit $a = \nUplet a1n \in{} \R_+^n$ et $b = \nUplet b1n \in{} \R_+^n$.
  \begin{enumerate}
  \item
    Montrer que pour tout $\lambda\in\Rps$,
    \[ \sum_{k=1}^n a_k b_k \leq\frac{\lambda^p}{p} \sum_{k=1}^n a_k^p + \frac{1}{q\lambda^q} \sum_{k=1}^n b_k^q. \]
  \item En d\'eduire l'\emph{in\'egalit\'e de H\"older} :
    \[ \sum_{k=1}^n a_k b_k \leq\left( \sum_{k=1}^n a_k^p \right)^{\frac1p} \left( \sum_{k=1}^n b_k^q \right)^{\frac1q}. \]
  \end{enumerate}
\item
  Soit $u = \nUplet u1n\in \R_+^n$ et $v = \nUplet v1n\in \R_+^n$.
  On note $\DS S = \sum_{k=1}^n (u_k+v_k)^p$.
  \begin{enumerate}
  \item
    Montrer que, pour $k\in\Dcro{1,n}$,
    \[ (u_k+v_k)^p = u_k (u_k+v_k)^{\frac{p}{q}} + v_k (u_k+v_k)^{\frac{p}{q}}. \]
  \item
    En d\'eduire que
    \[ S \leq{} \Pa{ \sum_{k=1}^n u_k^p }^{\frac1p} S^{\frac1q} + \Pa{ \sum_{k=1}^n v_k^p }^{\frac1p} S^{\frac1q} \]
    puis que
    \[ \Pa{ \sum_{k=1}^n (u_k+v_k)^p }^{\frac1p} \leq{} \Pa{ \sum_{k=1}^n u_k^p }^{\frac1p} + \Pa{ \sum_{k=1}^n v_k^p }^{\frac1p}. \]
  \end{enumerate}
\item
  Soit $x = \nUplet x1n\in \K^n$ et $y = \nUplet y1n\in \K^n$.
  On note $u_k = \Abs{x_k}$ et $v_k = \Abs{y_k}$ pour $k\in\Dcro{1,n}$.
  Montrer que $\Norm{x+y}_p \leq{} \Norm{x}_p + \Norm{y}_p$.
\end{enumerate}
\endgroup

% -----------------------------------------------------------------------------
\par\pagebreak[1]\par
\paragraph{Exercice 7}%
\hfill{\tiny 2221}%
\begingroup~

Soit $E = \R^2$.
On pose, pour $(x,y)\in E$,
\[ N(x,y) = \sup_{t\in \R} \Abs{ \frac{x+ty}{1+t+t^2} }. \]
\begin{enumerate}
\item Montrer que $N$ est une norme.
\item Calculer $N(x,1)$ et $N(1,0)$. En d\'eduire que, pour tous $(x,y)\in E^2$,
  \[ N(x,y) = \frac23\sqrt{x^2-xy+y^2} + \frac13 \Abs{2x-y}. \]
\item Dessiner la sph\`ere unit\'e pour la norme $N$.
\end{enumerate}
\endgroup

% -----------------------------------------------------------------------------
\par\pagebreak[1]\par
\paragraph{Exercice 8}%
\hfill{\tiny 1303}%
\begingroup~

Soit $E$ et $F$ deux espaces vectoriels norm\'es et $A\subset E$.
On note $\B(A,F)$ l'ensemble des applications born\'ees de $A$ dans $F$.
Pour $f\in\B(A,F)$, on pose $\Norm{f}_\infty= \sup_{x\in A} \Norm{f(x)}_F$.\begin{enumerate}
\item Montrer qu'il s'agit d'un espace vectoriel.
\item Montrer que $\Norm{f}_\infty$ existe.
\item Montrer que $\Norme_\infty$ est une norme sur $\B(A,F)$.
\end{enumerate}
\endgroup

% -----------------------------------------------------------------------------
\par\pagebreak[1]\par
\paragraph{\href{https://psi.miomio.fr/exo/6918.pdf}{Exercice 9}}%
\hfill{\tiny 6918}%
\begingroup~

Soit $E = \MnK$ et $\Norme$ une norme sur $E$.
Montrer qu'il existe $k > 0$ telle que
\[ \forall(A,B) \in{} E^2 \+ \Norm{AB} \leq{} k \Norm A \, \Norm B. \]
\endgroup

% -----------------------------------------------------------------------------
\par\pagebreak[1]\par
\paragraph{\href{https://psi.miomio.fr/exo/8573.pdf}{Exercice 10} (convergence normale d'une s\'erie)}%
\hfill{\tiny 8573}%
\begingroup~

Soit $(E, \Norme)$ un espace vectoriel norm\'e de dimension finie et $(u_n)_{n\in \N}$ une suite \`a valeurs dans $E$.
On suppose que la s\'erie $\sum_n \Norm{u_n}$ converge; on dit que la s\'erie $\sum_n u_n$ \emph{converge normalement}.

Montrer que la s\'erie $\sum_n u_n$ converge.
\endgroup

% -----------------------------------------------------------------------------
\par\pagebreak[1]\par
\paragraph{Exercice 11 (th\'eor\`eme du point fixe)}%
\hfill{\tiny 4936}%
\begingroup~

Soit $(E, \Norme)$ un espace vectoriel norm\'e de dimension finie et $A$ un ferm\'e non vide de $E$.
Soit $\Fn{f}{A}{A}$ une application $k$-lipschitzienne o\`u $k < 1$; une telle application est dite \emph{contractante}.

On se propose de montrer que $f$ admet un unique point fixe, c.-\`a-d. que l'\'equation $f(x)=x$ admet une unique solution.
Ce r\'esultat porte le nom de \emph{th\'eor\`eme du point fixe}.

Pour cela, on d\'efinit une suite r\'ecurrente par $u_0\in A$ et $u_{n+1} = f(u_n)$.
On pose \'egalement $v_n = u_{n+1} - u_{n}$.
\begin{enumerate}
\item Montrer que $\Norm{v_{n+1}} \leq{} k \Norm{v_n}$, puis que $\Norm{v_n}\leq k^n \Norm{v_0}$.
\item En d\'eduire que la s\'erie $\sum_n v_n$ converge;
  on pourra utiliser le r\'esultat de l'exercice pr\'ec\'edent.
\item Montrer que la suite $(u_n)$ converge, que sa limite est dans $A$,
  et qu'il s'agit d'un point fixe de $f$.
\item Montrer par ailleurs que $f$ admet au plus un point fixe.
\end{enumerate}
\endgroup

% -----------------------------------------------------------------------------
\par\pagebreak[1]\par
\paragraph{Exercice 12}%
\hfill{\tiny 5315}%
\begingroup~

Soit $E$ un espace vectoriel de dimension finie.
On consid\`ere l'espace vectoriel $\LE$ muni d'une norme quelconque.
V\'erifier que l'application $\Fonction{\circ}{\LE^2}{\LE}{(u,v)}{u\circ v}$
est continue.
\endgroup

% -----------------------------------------------------------------------------
\par\pagebreak[1]\par
\paragraph{Exercice 13}%
\hfill{\tiny 8571}%
\begingroup~

Soit $I$ un intervalle de $\R$ et $\Fn fI\K$ une fonction de classe $\CC1$.
Montrer que $f$ est lipschitzienne si et seulement si $f'$ est born\'ee.
\endgroup

\subsection{Topologie}

% -----------------------------------------------------------------------------
\par\pagebreak[1]\par
\paragraph{Exercice 14 (exemples de parties de $\R$)}%
\hfill{\tiny 5384}%
\begingroup~

Soit $E = \R$.
Parmi les parties suivantes, lesquelles sont ouvertes? lesquelles sont ferm\'ees?

$\intO{0,1}$,
$\intF{0,1}$,
$\intFO{0,1}$,
$\Z$,
$\Rp$,
$\Q$,
$\R\setminus \Z$,
$\R\setminus \Q$,
$\emptyset$,
$\R$
$\Ensemble{\frac1n}{n\in\Ns}$,
$\Ensemble{\frac1n}{n\in\Ns}\cup\Acco{0}$
\endgroup

% -----------------------------------------------------------------------------
\par\pagebreak[1]\par
\paragraph{Exercice 15}%
\hfill{\tiny 4793}%
\begingroup~

Soit $E$ un espace vectoriel norm\'e de dimension finie et $F$ un sous-espace vectoriel de $E$.
Montrer que $F$ est un ferm\'e.

Le r\'esultat est-il encore vrai si $E$ n'est pas de dimension finie?
\endgroup

% -----------------------------------------------------------------------------
\par\pagebreak[1]\par
\paragraph{\href{https://psi.miomio.fr/exo/7364.pdf}{Exercice 16}}%
\hfill{\tiny 7364}%
\begingroup~

\begin{enumerate}
\item
  Montrer que le groupe lin\'eaire $\GLnK$ est un ouvert de $\MnK$.
\item
  Montrer que le groupe sp\'ecial lin\'eaire $\M{SL}{n}{\K} = \Ensemble{M\in\MnK}{\det M=1}$ est un ferm\'e de $\MnK$.
\end{enumerate}
\endgroup

% -----------------------------------------------------------------------------
\par\pagebreak[1]\par
\paragraph{Exercice 17}%
\hfill{\tiny 1445}%
\begingroup~

Soit $A$ et $B$ deux parties d'un espace vectoriel norm\'e $E$.
On note $A+B = \Ensemble{a+b}{(a,b) \in A\times B}$.\begin{enumerate}
\item Montrer que si $A$ est un ouvert, $A+B$ l'est \'egalement.
\item Montrer \`a l'aide d'un contre-exemple que $A$ et $B$ ferm\'es n'entra\^ine pas n\'ec\'essairement $A+B$ ferm\'e.
  On pourra prendre $E=\R$, $A=\Ns$ et $B=\Ensemble{e^{-n}-n}{n\in\Ns}$.
\end{enumerate}
\endgroup

% -----------------------------------------------------------------------------
\par\pagebreak[1]\par
\paragraph{Exercice 18 (utilisation de la densit\'e)}%
\hfill{\tiny 0932}%
\begingroup~

Soit $\Fn{f}{\R}{\R}$ telle que $\forall(x,y)\in \R^2$, $f(x+y) = f(x) + f(y)$.\begin{enumerate}
\item Montrer que $\forall x\in \R$, $\forall n\in \N$, $f(n x) = nf(x)$.
\item Montrer que $\forall x\in \R$, $\forall n\in \Z$, $f(n x) = nf(x)$.
\item Montrer que $\forall x\in \R$, $\forall r\in \Q$, $f(r x) = rf(x)$.
\item On suppose de plus qu'il existe un intervalle $[a,b]$ avec $a < b$ telle que $f$ soit born\'ee sur $[a,b]$.

  Une fonction qui ne v\'erifierait pas cela serait tr\`es bizarre (mais cela existe!)
  \begin{enumerate}
  \item Montrer que $f$ est born\'ee au voisinage de $0$, c.-\`a-d. qu'il existe $\alpha>0$ tel que $f$ soit born\'ee sur l'intervalle $[-\alpha,\alpha]$.
  \item Montrer que $f$ est continue en $0$.
  \item Montrer que $f$ est continue sur $\R$.
  \item Montrer que $\exists \lambda\in \R$, $\forall x\in \R$, $f(x) = \lambda x$.
  \end{enumerate}
\end{enumerate}
\endgroup

% -----------------------------------------------------------------------------
\par\pagebreak[1]\par
\paragraph{Exercice 19 (fronti\`ere)}%
\hfill{\tiny 4438}%
\begingroup~

Soit $E$ un espace vectoriel norm\'e et $A\subset E$.

On appelle \emph{fronti\`ere} de $A$
l'ensemble $\partial A = \Adh A\setminus\Int A$.\begin{enumerate}
\item Montrer que $A$ est ferm\'e si et seulement si $\partial A\subset A$.
\item Montrer que $A$ est ouvert si et seulement si $\partial A\cap A=\emptyset$.
\end{enumerate}
\endgroup

% -----------------------------------------------------------------------------
\par\pagebreak[1]\par
\paragraph{Exercice 20 (distance \`a une partie)}%
\hfill{\tiny 9661}%
\begingroup~

Soit $E$ un espace vectoriel norm\'e.

\'Etant donn\'e une partie non vide $A$ de $E$ et un point $x$ de $E$,
on pose $d(x,A) = \inf_{a\in A} d(x,a) = \inf_{a\in A} \Norm{x-a}$
\begin{enumerate}
\item Montrer que cet $\inf$ existe.
\item Montrer que l'application $d_A \colon x \mapsto d(x,A)$ est $1$-lipschitzienne et en d\'eduire qu'elle est continue.
\item Montrer que $d(x,A) = 0$ si et seulement si $x\in\Adh{A}$.
\item Montrer que $d(x,A) = d(x,\Adh{A})$.
\item Montrer que si $A$ est convexe, l'application $d_A \colon x \mapsto d(x,A)$ est \'egalement convexe.
\end{enumerate}
\endgroup

% -----------------------------------------------------------------------------
\par\pagebreak[1]\par
\paragraph{\href{https://psi.miomio.fr/exo/5417.pdf}{Exercice 21} (crit\`ere s\'equentiel des ouverts)}%
\hfill{\tiny 5417}%
\begingroup~

Soit $E$ un espace vectoriel norm\'e et $A\subset E$.

Montrer que $A$ est un ouvert si et seulement si
pour tout $a\in A$,
pour toute suite $(u_n)$ \`a valeurs dans $E$ telle que $u_n \to a$,
il existe un rang $N\in \N$ tel que pour tout $n\geq N$ on ait $u_n\in A$.
\endgroup

% -----------------------------------------------------------------------------
\par\pagebreak[1]\par
\paragraph{\href{https://psi.miomio.fr/exo/3118.pdf}{Exercice 22} (crit\`ere s\'equentiel des ferm\'es)}%
\hfill{\tiny 3118}%
\begingroup~

Soit $E$ un espace vectoriel norm\'e et $A\subset E$.

Montrer que $A$ est un ferm\'e si et seulement si
pour toute suite convergente $(u_n)$ \`a valeurs dans $A$,
on a $\DS \lim_\ninf u_n \in{} A$.
\endgroup

% -----------------------------------------------------------------------------
\par\pagebreak[1]\par
\paragraph{\href{https://psi.miomio.fr/exo/7725.pdf}{Exercice 23}}%
\hfill{\tiny 7725}%
\begingroup~

Soit $E$ et $F$ deux espaces vectoriels norm\'es et $\Fn{f}{E}{F}$.

Montrer que $f$ est continue si et seulement si l'image r\'eciproque par $f$ de tout ouvert de $F$ est un ouvert de $E$.
\endgroup

% -----------------------------------------------------------------------------
\par\pagebreak[1]\par
\paragraph{Exercice 24 (voisinages)}%
\hfill{\tiny 8266}%
\begingroup~

Soit $E$ un espace vectoriel norm\'e et $a\in E$.
Un \emph{voisinage} de $a$ est une partie $V \subset{} E$ telle qu'il existe un ouvert $U$ tel que $a \in{} U$ et $U \subset{} V$.
On note $\mathscr{V}(a)$ l'ensemble des voisinages de $a$.
\begin{enumerate}
\item Montrer que $V \in{} \mathscr{V}(a)$ si et seulement si il existe $r>0$ tel que $B(a,r) \subset{} V$.
\item Soit $A$ une partie de $E$.
  Montrer que $A$ est un ouvert si et seulement si $A$ est un voisinage de chacun de ses points.
\item Soit $(u_n)$ une suite \`a valeurs dans $E$.
  Montrer que $u_n \to \ell$ si et seulement si pour tout voisinage $V$ de $\ell$, on a $u_n \in{} V$ \`a partir d'un certain rang.
\item Soit $A$ une partie de $E$, $a\in A$, $F$ un espace vectoriel norm\'e et $\Fn f AF$ une fonction.
  Montrer que $f$ est continue en $a$ si et seulement si l'image r\'eciproque par $f$ de tout voisinage de $f(a)$ est un voisinage de $a$.
\end{enumerate}
\endgroup

\end{document}
