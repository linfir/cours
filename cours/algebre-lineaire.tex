% autogenerated by ytex.rs

\documentclass{scrartcl}

\usepackage[francais]{babel}
\usepackage{geometry}
\usepackage{scrpage2}
\usepackage{lastpage}
\usepackage{ragged2e}
\usepackage{multicol}
\usepackage{etoolbox}
\usepackage{xparse}
\usepackage{enumitem}
\usepackage{csquotes}
\usepackage{amsmath}
\usepackage{amsfonts}
\usepackage{amssymb}
\usepackage{mathrsfs}
\usepackage{stmaryrd}
\usepackage{dsfont}
\usepackage{eurosym}
% \usepackage{numprint}
% \usepackage[most]{tcolorbox}
\usepackage{tikz}
% \usepackage{tkz-tab}
\usepackage[unicode]{hyperref}
\usepackage[ocgcolorlinks]{ocgx2}

\let\ifTwoColumns\iftrue
\def\Classe{$\Psi$2019--2020}

% reproducible builds
% LuaTeX: \pdfvariable suppressoptionalinfo 1023 \relax
\pdfinfoomitdate=1
\pdftrailerid{}

\newif\ifDisplaystyle
\everymath\expandafter{\the\everymath\ifDisplaystyle\displaystyle\fi}
\newcommand\DS{\displaystyle}

\clearscrheadfoot
\pagestyle{scrheadings}
\thispagestyle{empty}
\ohead{\Classe}
\ihead{\thepage/\pageref*{LastPage}}

\setlist[itemize,1]{label=\textbullet}
\setlist[itemize,2]{label=\textbullet}

\ifTwoColumns
  \geometry{margin=1cm,top=2cm,bottom=3cm,foot=1cm}
  \setlist[enumerate]{leftmargin=*}
  \setlist[itemize]{leftmargin=*}
\else
  \geometry{margin=3cm}
\fi

\makeatletter
\let\@author=\relax
\let\@date=\relax
\renewcommand\maketitle{%
    \begin{center}%
        {\sffamily\huge\bfseries\@title}%
        \ifx\@author\relax\else\par\medskip{\itshape\Large\@author}\fi
        \ifx\@date\relax\else\par\bigskip{\large\@date}\fi
    \end{center}\bigskip
    \ifTwoColumns
        \par\begin{multicols*}{2}%
        \AtEndDocument{\end{multicols*}}%
        \setlength{\columnsep}{5mm}
    \fi
}
\makeatother

\newcounter{ParaNum}
\NewDocumentCommand\Para{smo}{%
  \IfBooleanF{#1}{\refstepcounter{ParaNum}}%
  \paragraph{\IfBooleanF{#1}{{\tiny\arabic{ParaNum}~}}#2\IfNoValueF{#3}{ (#3)}}}

\newcommand\I{i}
\newcommand\mi{i}
\def\me{e}

\def\do#1{\expandafter\undef\csname #1\endcsname}
\docsvlist{Ker,sec,csc,cot,sinh,cosh,tanh,coth,th}
\undef\do

\DeclareMathOperator\ch{ch}
\DeclareMathOperator\sh{sh}
\DeclareMathOperator\th{th}
\DeclareMathOperator\coth{coth}
\DeclareMathOperator\cotan{cotan}
\DeclareMathOperator\argch{argch}
\DeclareMathOperator\argsh{argsh}
\DeclareMathOperator\argth{argth}

\let\epsilon=\varepsilon
\let\phi=\varphi
\let\leq=\leqslant
\let\geq=\geqslant
\let\subsetneq=\varsubsetneq
\let\emptyset=\varnothing

\newcommand{\+}{,\;}

\undef\C
\newcommand\ninf{{n\infty}}
\newcommand\N{\mathbb{N}}
\newcommand\Z{\mathbb{Z}}
\newcommand\Q{\mathbb{Q}}
\newcommand\R{\mathbb{R}}
\newcommand\C{\mathbb{C}}
\newcommand\K{\mathbb{K}}
\newcommand\Ns{\N^*}
\newcommand\Zs{\Z^*}
\newcommand\Qs{\Q^*}
\newcommand\Rs{\R^*}
\newcommand\Cs{\C^*}
\newcommand\Ks{\K^*}
\newcommand\Rp{\R^+}
\newcommand\Rps{\R^+_*}
\newcommand\Rms{\R^-_*}
\newcommand{\Rpinf}{\Rp\cup\Acco{+\infty}}

\undef\B
\newcommand\B{\mathscr{B}}

\undef\P
\DeclareMathOperator\P{\mathbb{P}}
\DeclareMathOperator\E{\mathbb{E}}
\DeclareMathOperator\Var{\mathbb{V}}

\DeclareMathOperator*\PetitO{o}
\DeclareMathOperator*\GrandO{O}
\DeclareMathOperator*\Sim{\sim}
\DeclareMathOperator\Tr{tr}
\DeclareMathOperator\Ima{Im}
\DeclareMathOperator\Ker{Ker}
\DeclareMathOperator\Sp{Sp}
\DeclareMathOperator\Diag{diag}
\DeclareMathOperator\Rang{rang}
\DeclareMathOperator*\Coords{Coords}
\DeclareMathOperator*\Mat{Mat}
\DeclareMathOperator\Pass{Pass}
\DeclareMathOperator\Com{Com}
\DeclareMathOperator\Card{Card}
\DeclareMathOperator\Racines{Racines}
\DeclareMathOperator\Vect{Vect}
\DeclareMathOperator\Id{Id}

\newcommand\DerPart[2]{\frac{\partial #1}{\partial #2}}

\def\T#1{{#1}^T}

\def\pa#1{({#1})}
\def\Pa#1{\left({#1}\right)}
\def\bigPa#1{\bigl({#1}\bigr)}
\def\BigPa#1{\Bigl({#1}\Bigr)}
\def\biggPa#1{\biggl({#1}\biggr)}
\def\BiggPa#1{\Biggl({#1}\Biggr)}

\def\pafrac#1#2{\pa{\frac{#1}{#2}}}
\def\Pafrac#1#2{\Pa{\frac{#1}{#2}}}
\def\bigPafrac#1#2{\bigPa{\frac{#1}{#2}}}
\def\BigPafrac#1#2{\BigPa{\frac{#1}{#2}}}
\def\biggPafrac#1#2{\biggPa{\frac{#1}{#2}}}
\def\BiggPafrac#1#2{\BiggPa{\frac{#1}{#2}}}

\def\cro#1{[{#1}]}
\def\Cro#1{\left[{#1}\right]}
\def\bigCro#1{\bigl[{#1}\bigr]}
\def\BigCro#1{\Bigl[{#1}\Bigr]}
\def\biggCro#1{\biggl[{#1}\biggr]}
\def\BiggCro#1{\Biggl[{#1}\Biggr]}

\def\abs#1{\mathopen|{#1}\mathclose|}
\def\Abs#1{\left|{#1}\right|}
\def\bigAbs#1{\bigl|{#1}\bigr|}
\def\BigAbs#1{\Bigl|{#1}\Bigr|}
\def\biggAbs#1{\biggl|{#1}\biggr|}
\def\BiggAbs#1{\Biggl|{#1}\Biggr|}

\def\acco#1{\{{#1}\}}
\def\Acco#1{\left\{{#1}\right\}}
\def\bigAcco#1{\bigl\{{#1}\bigr\}}
\def\BigAcco#1{\Bigl\{{#1}\Bigr\}}
\def\biggAcco#1{\biggl\{{#1}\biggr\}}
\def\BiggAcco#1{\Biggl\{{#1}\Biggr\}}

\def\ccro#1{\llbracket{#1}\rrbracket}
\def\Dcro#1{\llbracket{#1}\rrbracket}

\def\floor#1{\lfloor#1\rfloor}
\def\Floor#1{\left\lfloor{#1}\right\rfloor}

\def\sEnsemble#1#2{\mathopen\{#1\mid#2\mathclose\}}
\def\bigEnsemble#1#2{\bigl\{#1\bigm|#2\bigr\}}
\def\BigEnsemble#1#2{\Bigl\{#1\Bigm|#2\Bigr\}}
\def\biggEnsemble#1#2{\biggl\{#1\biggm|#2\biggr\}}
\def\BiggEnsemble#1#2{\Biggl\{#1\Biggm|#2\Biggr\}}
\let\Ensemble=\bigEnsemble

\newcommand\IntO[1]{\left]#1\right[}
\newcommand\IntF[1]{\left[#1\right]}
\newcommand\IntOF[1]{\left]#1\right]}
\newcommand\IntFO[1]{\left[#1\right[}

\newcommand\intO[1]{\mathopen]#1\mathclose[}
\newcommand\intF[1]{\mathopen[#1\mathclose]}
\newcommand\intOF[1]{\mathopen]#1\mathclose]}
\newcommand\intFO[1]{\mathopen[#1\mathclose[}

\newcommand\Fn[3]{#1\colon#2\to#3}
\newcommand\CC[1]{\mathscr{C}^{#1}}
\newcommand\D{\mathop{}\!\mathrm{d}}

\newcommand\longto{\longrightarrow}

\undef\M
\newcommand\M[3]{\mathrm{#1}_{#2}\pa{#3}}
\newcommand\MnR{\M{M}{n}{\R}}
\newcommand\MnC{\M{M}{n}{\C}}
\newcommand\MnK{\M{M}{n}{\K}}
\newcommand\GLnR{\M{GL}{n}{\R}}
\newcommand\GLnC{\M{GL}{n}{\C}}
\newcommand\GLnK{\M{GL}{n}{\K}}
\newcommand\DnR{\M{D}{n}{\R}}
\newcommand\DnC{\M{D}{n}{\C}}
\newcommand\DnK{\M{D}{n}{\K}}
\newcommand\SnR{\M{S}{n}{\R}}
\newcommand\AnR{\M{A}{n}{\R}}
\newcommand\OnR{\M{O}{n}{\R}}
\newcommand\SnRp{\mathrm{S}_n^+(\R)}
\newcommand\SnRpp{\mathrm{S}_n^{++}(\R)}

\newcommand\LE{\mathscr{L}(E)}
\newcommand\GLE{\mathscr{GL}(E)}
\newcommand\SE{\mathscr{S}(E)}
\renewcommand\OE{\mathscr{O}(E)}

\newcommand\ImplD{$\Cro\Rightarrow$}
\newcommand\ImplR{$\Cro\Leftarrow$}
\newcommand\InclD{$\Cro\subset$}
\newcommand\InclR{$\Cro\supset$}
\newcommand\notInclD{$\Cro{\not\subset}$}
\newcommand\notInclR{$\Cro{\not\supset}$}

\newcommand\To[1]{\xrightarrow[#1]{}}
\newcommand\Toninf{\To{\ninf}}

\newcommand\Norm[1]{\|#1\|}
\newcommand\Norme{{\Norm{\cdot}}}

\newcommand\Int[1]{\mathring{#1}}
\newcommand\Adh[1]{\overline{#1}}

\newcommand\Uplet[2]{{#1},\dots,{#2}}
\newcommand\nUplet[3]{(\Uplet{{#1}_{#2}}{{#1}_{#3}})}

\newcommand\Fonction[5]{{#1}\left|\begin{aligned}{#2}&\;\longto\;{#3}\\{#4}&\;\longmapsto\;{#5}\end{aligned}\right.}

\DeclareMathOperator\orth{\bot}
\newcommand\Orth[1]{{#1}^\bot}
\newcommand\PS[2]{\langle#1,#2\rangle}

\newcommand{\Tribu}{\mathscr{T}}
\newcommand{\Part}{\mathcal{P}}
\newcommand{\Pro}{\bigPa{\Omega,\Tribu}}
\newcommand{\Prob}{\bigPa{\Omega,\Tribu,\P}}

\newcommand\DEMO{$\spadesuit$}
\newcommand\DUR{$\spadesuit$}

\newenvironment{psmallmatrix}{\left(\begin{smallmatrix}}{\end{smallmatrix}\right)}

% -----------------------------------------------------------------------------

\usepackage{tikz}
\usetikzlibrary{matrix,calc}

\newcommand\myop{\mathop\square}
\newcommand\sqn[2]{\ccro{1,#1}\times\ccro{1,#2}}
\newcommand\MM[1]{\M{M}{#1}{\K}}
\newcommand\BC{\mathscr{C}}

\begin{document}
\title{Alg\`ebre lin\'eaire}
\maketitle

% \tableofcontents

\section{Espace vectoriel}

\subsection{Loi de composition interne}

\Para{D\'efinitions}

Soit $E$ un ensemble.
Une \emph{loi de composition interne} $\myop$ sur $E$ est une application
\[ \Fonction{\myop}{E\times E}{E}{(x,y)}{x \myop y} \]
\begin{enumerate}
\item
  On dit que $\myop$ est \emph{associative} si $\forall(x,y,z)\in E^3$, $x \myop (y \myop z) = (x \myop y) \myop z$.
  On ne consid\`erera que des loi associatives.
\item
  On dit que $\myop$ est \emph{commutative} si $\forall(x,y)\in E^2$, $x \myop y = y \myop x$;
\item
  On dit que $\myop$ \emph{admet un neutre} si $\exists e\in E$ tel que $\forall x\in E$, $x \myop e = e \myop x = x$.
  Il existe au plus un \'el\'ement $e$ v\'erifiant cette propri\'et\'e, et on l'appelle \emph{le} neutre de la loi $\myop$.
\item
  Si $\myop$ est une loi associative qui admet un neutre $e$, et si $x\in E$, on appelle \emph{sym\'etrique} (ou \emph{inverse}) de $x$ pour la loi $\myop$ tout \'el\'ement $x'\in E$ tel que $x \myop x' = x' \myop x = e$.
  Si $\myop$ est \'egalement associative, il existe au plus un \'el\'ement $x'$ v\'erifiant cette propri\'et\'e, et on l'appelle \emph{le} sym\'etrique de $x$ pour la loi $\myop$.
\end{enumerate}

\Para{Proposition}

Soit $E$ un ensemble muni d'une loi de composition interne associative $\myop$.
Soit $(x,y)\in E^2$.
Si $x$ et $y$ sont sym\'etrisables pour $\myop$, de sym\'etriques $x'$ et $y'$ respectivement,
alors $x \myop y$ est \'egalement sym\'etrisable et son sym\'etrique est $y' \myop x'$.

\subsection{Groupe}

\Para{D\'efinitions}

Un \emph{groupe} est un couple $(G,\myop)$ o\`u $G$ est un ensemble et $\myop$ une loi de composition interne sur $G$ associative, admettant un neutre et pour laquelle tout \'el\'ement de $G$ admet un sym\'etrique pour la loi $\myop$.
Un groupe est dit \emph{ab\'elien} ou \emph{commutatif} si la loi $\myop$ est de plus commutative.

\subsection{Corps}

\Para{D\'efinition}

Un \emph{corps} est un triplet $(\K,+,\times)$ o\`u
\begin{enumerate}
\item
  $\K$ est un ensemble;
\item
  $+$ et $\times$ sont des lois de composition internes sur $\K$;
\item
  $(\K,+)$ est un groupe ab\'elien, c.-\`a-d.
  $+$ est associative, commutative, admet un neutre et tout \'el\'ement de $\K$ admet un sym\'etrique pour $+$.
  On note $0_\K$ le neutre de $+$;
\item
  $\times$ est associative, commutative, admet un neutre et tout \'el\'ement de $\K\setminus\acco{0_\K}$ admet un sym\'etrique pour $\times$.
  On note $1_\K$ le neutre de $\times$;
\item
  La loi $\times$ est distributive sur la loi $+$, c.-\`a-d.
  $\forall(x,y,z)\in \K^3$, $x\times(y+z) = (x\times y)+(x\times z)$;
\item
  $0_\K{} \neq{} 1_\K$.
\end{enumerate}

\Para{Remarque}

Lorsque le contexte est clair, on \'ecrit souvent $\K$ au lieu de $(\K,+,\times)$.

\Para{Exemples}
\begin{itemize}
\item le corps des r\'eels $\R$, des complexes $\C$, des rationnels $\Q$;
\item si $\K$ est un corps, le corps des fractions rationnelles \`a coefficients dans $\K$, not\'e $\K(X)$;
\item $\Q[i] = \Ensemble{a+ib}{(a,b)\in \Q^2}$;
\item le corps des entiers modulo un nombre premier $p$, not\'e $\Z/p\Z$.
\end{itemize}
Tous ces corps sont munis des \enquote{lois usuelles}.

\subsection{Espace vectoriel}

\Para{D\'efinition}

Soit $\K$ un corps.
Un $\K$-espace vectoriel est un triplet $(E,+,\bullet)$ o\`u
$+$ est une loi de composition interne sur $E$ et
$\bullet$ est une loi de composition externe sur $E$, c.-\`a-d. une application
\[ \Fonction{\bullet}{\K\times E}{E}{(\lambda,x)}{\lambda\bullet x} \]
v\'erifiant les axiomes suivants:
\begin{enumerate}
\item $(E, +)$ est un groupe ab\'elien;
\item la loi $\bullet$ est compatible avec la structure de groupe $(E, +)$, c.-\`a-d.
  \begin{enumerate}
  \item $\forall(\lambda,\mu)\in \K^2$, $\forall x\in E$, $(\lambda+\mu) \bullet x = (\lambda\bullet x) + (\mu\bullet x)$;
  \item $\forall \lambda\in \K$, $\forall(x,y)\in E^2$, $\lambda\bullet (x+y) = (\lambda\bullet x) + (\lambda\bullet y)$;
  \item $\forall x\in E$, $1_\K\bullet x = x$;
  \item $\forall(\lambda,\mu)\in \K^2$, $\forall x\in E$, $\lambda\bullet (\mu\bullet x) = (\lambda \mu) \bullet x$.
  \end{enumerate}
\end{enumerate}

\Para{Exemples}

\begin{itemize}
\item
  $(\K^n, \tilde +, \bullet)$
  o\`u $\nUplet x1n \tilde + \nUplet y1n = \pa{x_1+y_1, x_2+y_2, \dots, x_n+y_n}$
  et $\lambda\bullet \nUplet x1n = \pa{\lambda x_1, \lambda x_2, \dots, \lambda x_n}$.
\item
  les matrices $\M{M}{n,p}{\K}$ muni des lois usuelles (les expliciter!),
\item
  si $X$ est un ensemble et $E$ un $\K$-espace vectoriel, l'ensemble des fonctions $\mathcal{F}(X,E)$ muni des lois usuelles.
\end{itemize}

\subsection{Espace vectoriel produit}

\Para{D\'efinition}

Soit $(E_1, +_1, \bullet_1)$, $(E_2, +_2, \bullet_2)$, ..., $(E_p, +_p, \bullet_p)$ des $\K$-espaces vectoriels.
On pose \[ E = \prod_{k=1}^p E_k = \Ensemble{\nUplet x1p}{ x_1\in E_1, \dots, x_p\in E_p }. \]
$E$ est le produit cart\'esien des ensembles $E_1, \dots, E_n$.
On d\'efinit la loi de composition interne $+$ sur $E$ par:
$\forall\nUplet x1p\in E$, $\forall\nUplet y1p\in E$,
\[ \nUplet x1p + \nUplet y1p = \nUplet z1p \]
o\`u $z_k = x_k +_k y_k$ pour $k\in\Dcro{1,p}$.
De m\^eme, on d\'efinit la loi de composition externe $\bullet$ sur $E$ par:
$\forall \lambda\in \K\+\forall\nUplet x1p\in E$
\[ \lambda\bullet \nUplet x1p = \nUplet y1p \]
o\`u $y_k = \lambda\bullet_k x_k$ pour $k\in\Dcro{1,p}$.

Alors $(E, +, \bullet)$ est un espace vectoriel, appel\'e espace vectoriel produit des espaces vectoriels $(E_1, +_1, \bullet_1)$, ..., $(E_p, +_p, \bullet_p)$ et not\'e \[ E = \prod_{k=1}^p E_k. \]

\Para{Remarque}

Pour $p\in\Ns$ et $E$ un $\K$-espace vectoriel, on d\'efinit $E^p$ par \[ E^p = \prod_{k=1}^p E. \]
Notez le cas particulier $E = \K$.

\subsection{Famille finie}

\Para{D\'efinition}

Soit $E$ un $\K$-espace vectoriel.
Une \emph{famille finie} de vecteurs de $E$ est un $p$-uplet $\mathcal{F} = \nUplet x1p$ form\'ee d'\'el\'ements de $E$, o\`u $p\in \N$.
Dans le cas $p=0$, on notera $\emptyset$ l'unique famille de $0$~\'el\'ement.

\Para{D\'efinitions}
Avec les m\^emes notations, une \emph{combinaison lin\'eaire} de la famille $\mathcal{F}$ est un vecteur $x \in E$ de la forme $x = \sum_{i=1}^p \alpha_i x_i$ o\`u $\nUplet \alpha1p \in{} \K^p$.
On note $\Vect \mathcal{F}$ l'ensemble des combinaison lin\'eaires de la famille $\nUplet x1p$.
Par convention, $\Vect \emptyset= \acco{0_E}$.

\Para{D\'efinitions}

Avec les m\^emes notations, on dit que la famille $\mathcal{F}$ est \emph{libre} si et seulement si la seule combinaison lin\'eaire de $\mathcal{F}$ nulle est triviale, c.-\`a-d. si et seulement si
\begin{multline*}
  \forall\nUplet \lambda1p \in \K^p, \\
  \Pa{ \sum_{i=1}^p \lambda_i x_i = 0_E \implies \forall i \in\ccro{1,p} \+ \lambda_i = 0_\K{} }.
\end{multline*}

On dit que la famille $\mathcal{F}$ est \emph{g\'en\'eratrice} si et seulement si tout vecteur de $E$ est combinaison lin\'eaire de $\mathcal{F}$, c.-\`a-d. si et seulement si $\Vect \mathcal{F} = E$, c.-\`a-d. si et seulement si
\[ \forall x \in \E\+ \exists\nUplet \alpha1p \in \K^p \+ x = \sum_{i=1}^p \alpha_i x_i. \]

On dit que la famille $\mathcal{F}$ est une \emph{base} de $E$ si et seulement si elle est libre et g\'en\'eratrice.
De fa\c con \'equivalente, la famille $\nUplet x1n$ est une base de $E$ si et seulement si
\[ \forall x\in E \+ \exists!\nUplet\alpha1p \+ x = \sum_{i=1}^p \alpha_i x_i. \]

\Para{Proposition}
Avec les m\^emes notations, les conditions suivantes sont \'equivalentes:
\begin{enumerate}
\item
  $\mathcal{F}$ est une base de $E$;
\item
  $\mathcal{F}$ est libre et maximale:
  $\mathcal{F}$ est libre et pour tout $x \in E$, la famille $\mathcal{F}$ augment\'ee du vecteur $x$ n'est pas libre;
\item
  $\mathcal{F}$ est g\'en\'eratrice et minimale:
  $\mathcal{F}$ est g\'en\'eratrice, mais si l'on enl\`eve un vecteur quelconque de $\mathcal{F}$, la famille r\'esultante n'est pas g\'en\'eratrice.
\end{enumerate}

\subsection{Sous-espace vectoriel}

\Para{D\'efinition}

Soit $(E,+,\cdot)$ un $\K$-espace vectoriel et $F$ une partie de $E$.
On dit que $F$ est un \emph{sous-espace vectoriel} de $E$ si et seulement si
\begin{enumerate}
\item
  $F$ est non vide;
\item
  $F$ est stable par $+$, c.-\`a-d. $\forall(x,y) \in F^2$, $x+y\in F$;
\item
  $F$ est stable par $\cdot$, c.-\`a-d. $\forall \lambda\in \K$, $\forall x \in F$, $\lambda x\in F$.
\end{enumerate}

\Para{Crit\`ere}

Soit $E$ un $\K$-espace vectoriel et $F\subset E$.
$F$ est un sous-espace vectoriel de $E$ si et seulement si
\begin{enumerate}
\item
  $0_E\in F$;
\item
  $\forall(\lambda,\mu)\in \K^2$, $\forall(x,y) \in F^2$, $\lambda x + \mu y \in{} F$.
\end{enumerate}

\Para{Proposition}
Soit $E$ un $\K$-espace vectoriel, $F_1, \dots, F_p$ des sous-espaces vectoriels.
Alors l'intersection $\bigcap_{i=1}^p F_i$ est \'egalement un sous-espace vectoriel.

\Para{Attention}

En revanche, l'union $\bigcup_{i=1}^p F_i$ n'est presque jamais un sous-espace vectoriel.

\Para{Proposition-D\'efinition}

Soit $E$ un $\K$-espace vectoriel et $\nUplet x1p$ une famille de vecteurs de $E$.
L'ensemble $\Vect\nUplet x1p$ est un sous-espace vectoriel de $E$, appel\'e \emph{espace engendr\'e} par la famille $\nUplet x1p$.
Il s'agit du plus petit (pour l'inclusion) sous-espace vectoriel de $E$ contenant $\Uplet{x_1}{x_p}$.

\subsection{Somme de sous-espaces vectoriels}

\Para{Proposition-D\'efinition}

Soit $E$ un $\K$-espace vectoriel et $\Uplet{F_1}{F_p}$ des sous-espaces vectoriels de $E$.
On appelle \emph{somme} de $\Uplet{F_1}{F_p}$ l'ensemble $S$ des vecteurs de la forme $f_1 + \dots + f_p$
o\`u $f_1\in F_1$, $f_2\in F_2$, ..., $F_p\in F_p$;
autrement dit,
\[ S = \BiggEnsemble{x \in E}{ \exists\nUplet f1p \in{} \prod_{i=1}^p F_i \+ x = \sum_{i=1}^p f_i }. \]
On le note $S = \sum_{k=1}^p F_k$.
Il s'agit d'un sous-espace vectoriel de $E$;
plus pr\'ecis\'ement $S$ est le plus petit (au sens de l'inclusion) sous-espace vectoriel de $E$ contenant $\Uplet{F_1}{F_p}$.

\Para{D\'efinition}
Avec les m\^emes notations, on dit que la somme $\sum_{i=1}^p F_i$ est \emph{directe} si et seulement si tout vecteur de la somme se d\'ecompose \emph{de fa\c con unique} sous la forme $f_1 + \dots + f_p$ o\`u $f_1\in F_1$, $f_2\in F_2$, ..., $F_p\in F_p$.
On note alors la somme $\bigoplus_{k=1}^p F_k = \sum_{k=1}^p F_k$.

\Para{Cas de deux sous-espaces vectoriels}
Soit $E$ un $\K$-espace vectoriel, $F$ et $G$ deux sous-espaces vectoriels.
Alors la somme $F+G$ est directe si et seulement si $F\cap G = \acco{0_E}$.

\Para{Attention}

Ce crit\`ere ne se g\'en\'eralise pas (simplement) pour $p > 2$.

Contre-exemple: $E = \R^2$, $e_1 = (1,0)$, $e_2 = (0,1)$ et $e_3 = (1,1)$.
V\'erifiez que $\R e_1 \cap{} \R e_2 = \R e_1 \cap{} \R e_3 = \R e_2 \cap{} \R e_3 = \Acco{(0,0)}$ mais que la somme $\R e_1 + \R e_2 + \R e_3$ n'est pas directe.

\Para{Crit\`ere}

Soit $E$ un $\K$-espace vectoriel et $\Uplet{F_1}{F_p}$ des sous-espaces vectoriels de $E$. La somme $\sum_{i=1}^p F_i$ est directe si et seulement si
\begin{multline*}
  \forall x_1 \in F_1 \+ \forall x_2 \in F_2 \+ \dots \+ \forall x_p \in F_p, \\
  \sum_{k=1}^p x_k = 0 \implies x_1 = \dots = x_p = 0.
\end{multline*}

\Para{D\'efinition}

Soit $E$ un $\K$-espace vectoriel et $\Uplet{F_1}{F_p}$ des sous-espaces vectoriels de $E$.
Si la somme $\sum_{k=1}^p F_k$ est directe et \'egale \`a $E$,
on dit que $\Uplet{F_1}{F_p}$ sont \emph{suppl\'ementaires}, et on note
\[ E = \bigoplus_{k=1}^p F_k. \]

Autrement dit, $\Uplet{F_1}{F_p}$ sont suppl\'ementaires
si et seulement si tout vecteur $x \in{} E$ se d\'ecompose de fa\c con unique
$x = x_1 + \dots + x_p$
o\`u $x_i \in{} F_i$ pour tout $i \in{} \ccro{1,p}$.

\subsection{Application lin\'eaire}

\Para{D\'efinition}

Soit $E$ et $F$ deux $\K$-espaces vectoriels.
Une application $\Fn{f}{E}{F}$ est dite \emph{lin\'eaire} si et seulement si
\begin{enumerate}
\item
  $\forall(x,y)\in E^2$, $f(x+y) = f(x) + f(y)$;
\item
  $\forall \lambda\in \K$, $\forall x\in E$, $f(\lambda x) = \lambda f(x)$.
\end{enumerate}
Ces deux conditions sont \'equivalentes \`a
\begin{enumerate}[resume]
\item $\forall(\lambda,\mu)\in \K^2$, $\forall(x,y)\in E^2$, $f(\lambda x + \mu y) = \lambda f(x) + \mu f(y)$.
\end{enumerate}

\Para{D\'efinitions}

\begin{itemize}
\item
  Une application lin\'eaire est appel\'ee \'egalement \emph{morphisme} d'espaces vectoriels.
\item
  Une application lin\'eaire dont l'espace de d\'epart est le m\^eme que celui d'arriv\'ee est un \emph{endomorphisme}.
\item
  Un \emph{isomorphisme} est une application lin\'eaire bijective.
\item
  Un \emph{automorphisme} est un endomorphisme bijectif.
\end{itemize}

\Para{Proposition-D\'efinition}

Soit $E$ et $F$ deux $\K$-espaces vectoriels, $E'$ un sous-espace vectoriel de $E$, $F'$ un sous-espace vectoriel de $F$ et $\Fn uEF$ une application lin\'eaire.
\begin{itemize}
\item
  L'ensemble $\Ensemble{u(x)}{x \in E'}$ est un sous-espace vectoriel de $F$, appel\'e \emph{image directe} de $E'$ par $u$ et not\'e $u(E')$.
\item
  L'ensemble $\Ensemble{x \in E}{u(x) \in F'}$ est un sous-espace vectoriel de $E$, appel\'e \emph{image r\'eciproque} de $F'$ par $u$ et not\'e $u^{-1}(F')$. Notez que ceci est bien d\'efini que $u$ soit bijectif ou non.
\end{itemize}

\Para{D\'efinitions}

Soit $E$ et $F$ deux $\K$-espaces vectoriels, et $\Fn uEF$ une application lin\'eaire.
\begin{itemize}
\item
  Le \emph{noyau} de $u$ est le sous-espace vectoriel $u^{-1}\pa{\acco{0_F}} = \Ensemble{x\in E}{u(x) = 0_F}$;
  on le note $\Ker u$.
\item
  L'\emph{image} de $u$ est le sous-espace vectoriel $u(E) = \Ensemble{u(x)}{x\in E}$;
  on le note $\Ima u$.
\end{itemize}

\Para{Proposition-D\'efinition}

Soit $E$ et $F$ deux $\K$-espaces vectoriels.
L'ensemble des applications lin\'eaires de $E$ dans $F$ est not\'e $\mathscr{L}(E,F)$;
il s'agit d'un sous-espace vectoriel de l'espace des fonctions de $E$ dans $F$ muni des lois usuelles.
L'espace vectoriel des endomorphismes de $E$ se note $\LE = \mathscr{L}\pa{E,E}$.

\Para{Proposition}
Soit $E$ et $F$ deux $\K$-espaces vectoriels et $\Fn uEF$ une application lin\'eaire.
\begin{itemize}
\item
  $u$ est injective si et seulement si $\Ker u = \acco{0_E}$;
\item
  $u$ est surjective si et seulement si $\Ima u = F$;
\item
  si $u$ est bijective, alors $u^{-1}$ est \'egalement lin\'eaire.
\end{itemize}

\Para{Proposition}

Soit $E$ un $\K$-espace vectoriel de dimension finie, $\nUplet e1n$ une base de $E$.
Soit $F$ un $\K$-espace vectoriel et $\Fn uEF$ une application lin\'eaire.
Alors $u$ est bijective si et seulement si la famille $\bigPa{u(e_1),\dots,u(e_n)}$ est une base de $F$.

% -----------------------------------------------------------------------------
\section{Dimension}

\subsection{Espace vectoriel de dimension finie}

\Para{D\'efinition}

Soit $E$ un $\K$-espace vectoriel.
On dit que $E$ est \emph{de dimension finie} s'il existe une famille finie g\'en\'eratrice de $E$,
et \emph{de dimension infinie} sinon.

\Para{Th\'eor\`eme de la base extraite}

Soit $E$ un espace vectoriel et $\nUplet e1q$ une famille g\'en\'eratrice.
Alors il existe $I \subset\ccro{1,q}$ tel que la famille $(e_i)_{i\in I}$ soit une base de $E$.

\Para{Corollaire}

Tout espace vectoriel de dimension finie admet une base.

\Para{Lemme de Steinitz}

Soit $E$ un espace vectoriel, $\nUplet e1n$ une base de $E$ et $\nUplet v1p$ une famille libre de $E$.
Alors $p \leq n$ et il existe une permutation $\pa{\Uplet{e'_1}{e'_n}}$ de $\nUplet e1n$ telle que
la famille $(v_1,\dots,v_p,e'_{p+1},\dots,e'_n)$ soit une base de $E$.

\Para{Corollaire}

Soit $E$ un espace vectoriel de dimension finie et $\nUplet e1p$ et $\nUplet f1q$ deux bases de $E$.
Alors $p=q$.

\Para{D\'efinition}

La \emph{dimension} d'un espace vectoriel de dimension finie est \'egale au cardinal d'une base quelconque de $E$.

\Para{Corollaire}

Si $E$ est un espace vectoriel de dimension finie de dimension $n$, alors toute famille de $n+1$ vecteurs est li\'ee.

\Para{Th\'eor\`eme de la base incompl\`ete}

Soit $E$ un espace vectoriel de dimension finie, et $\nUplet e1p$ une famille libre de $E$.
Alors il existe un entier $n\geq p$ et des vecteurs $\Uplet{e_{p+1}}{e_n}$
tels que $\nUplet e1n$ soit une base de $E$.

\medskip

On a en fait le r\'esultat plus pr\'ecis suivant:

\medskip

Soit $E$ un espace vectoriel,
$\mathcal{L} = \nUplet e1p$ une famille libre de $E$
et $\mathcal{F} = \nUplet f1q$ une famille de vecteurs de $E$
telle que $\mathcal{L} \cup{} \mathcal{F}$ soit g\'en\'eratrice.
Alors il existe un entier $n\geq p$ et des vecteurs $\Uplet{e_{p+1}}{e_n}$ de $\mathcal{F}$
tels que $\nUplet e1n$ soit une base de $E$.

\subsection{Applications}

\Para{Proposition}

Soit $E$ un espace vectoriel de dimension finie et $F$ un sous-espace vectoriel de $E$.
Alors $F$ est \'egalement de dimension finie et $\dim F \leq{} \dim E$, avec \'egalit\'e si et seulement si $F = E$.

\Para{Proposition}
Dans un espace vectoriel de dimension finie, tout sous-espace vectoriel admet un suppl\'ementaire.

Autrement dit, si $E$ est un espace vectoriel de dimension finie et $F$ un sous-espace vectoriel de $E$,
alors il existe un sous-espace vectoriel $G$ de $E$ tel que $E = F \oplus G$.

\Para{Proposition}

Soit $\Uplet{E_1}{E_p}$ des $\K$-espaces vectoriels.
L'espace vectoriel produit $\prod_{k=1}^p E_k$ est de dimension finie si et seulement si $\forall k\in\Dcro{1,p}$, $E_k$ est de dimension finie.
De plus, dans ce cas, \[ \dim \Pa{ \prod_{k=1}^p E_k } = \sum_{k=1}^p \dim E_k. \]

\Para{Corollaire}

Pour $p\in\Ns$ et $E$ un $\K$-espace vectoriel.
L'espace vectoriel $E^p$ est de dimension finie si et seulement si $E$ l'est;
dans ce cas, on a $\dim(E^p) = p\dim E$.

\Para{Proposition}

Soit $E$ et $F$ deux $\K$-espace vectoriel de dimension finie.
Alors $\mathscr{L}(E,F)$ est \'egalement de dimension finie
et $\dim \mathscr{L}(E,F) = \dim(E) \times\dim(F)$.

\subsection{Th\'eor\`eme du rang}

\Para{Th\'eor\`eme}

Soit $E$ et $F$ deux $\K$-espaces vectoriels et $\Fn uEF$ une application lin\'eaire.
Si $S$ est un suppl\'ementaire de $\Ker u$, alors $u$ induit un isomorphisme de $S$ sur $\Ima u$,
c.-\`a-d. que l'application lin\'eaire
\[ \Fonction{v}{S}{\Ima u}{x}{u(x)} \]
est bijective.

\Para{Corollaire}[th\'eor\`eme du rang]

Soit $E$ et $F$ deux $\K$-espaces vectoriels et $\Fn uEF$ une application lin\'eaire.
On suppose que $E$ est de dimension finie.
Alors $\Ima u$ est de dimension finie et
\[ \dim E = \Rang u + \dim \Ker u. \]

\Para{Corollaire important}

Soit $E$ et $F$ deux $\K$-espaces vectoriels de dimensions finies et $\Fn uEF$ une application lin\'eaire.
On suppose que $\dim E = \dim F$.
Les conditions suivantes sont \'equivalentes:
\begin{enumerate}
\item
  $u$ est injective;
\item
  $u$ est surjective;
\item
  $u$ est bijective.
\end{enumerate}
Notez que ceci s'applique en particulier au cas des endomorphismes.

\subsection{Somme directe}

\Para{Proposition}

Soit $E$ un $\K$-espace vectoriel et $\Uplet{F_1}{F_p}$ des sous-espaces vectoriels de dimension finie de $E$.
Alors la somme $\sum_{k=1}^p F_k$ est \'egalement de dimension finie, et
\[ \dim\Pa{ \sum_{k=1}^p F_k } \leq{} \sum_{k=1}^p \dim(F_k). \]
De plus, il y a \'egalit\'e si et seulement si la somme $\sum_{k=1}^p F_k$ est directe.

\Para{Formule de Grassmann}

Soit $E$ un $\K$-espace vectoriel, $F$ et $G$ deux sous-espaces vectoriels de dimensions finies de $E$.
Alors
\[ \dim(F+G) + \dim(F\cap G) = \dim(F) + \dim(G). \]

\Para{Proposition}

Soit $E$ un $\K$-espace vectoriel de dimension finie et $F_1, \dots, F_p$ des sous-espaces vectoriels de $E$.
On suppose que \[ \sum_{k=1}^p \dim F_k = \dim E. \]
Les conditions suivantes sont alors \'equivalentes:
\begin{enumerate}
\item
  les sous-espaces vectoriels $F_1, \dots, F_p$ sont suppl\'ementaires;
\item
  les sous-espaces vectoriels $F_1, \dots, F_p$ sont en somme directe;
\item
  $\sum_{k=1}^p F_k = E$;
\item
  tout vecteur $x\in E$ se d\'ecompose sous la forme
  $x = \sum_{k=1}^p f_k$
  o\`u $f_1\in F_1$, $\dots$, $f_p\in F_p$,
\item pour tous $f_1\in F_1$, $f_2\in F_2$, $\dots$, $f_p\in F_p$,
  si $\sum_{k=1}^p f_k = 0$, alors $f_1 = \dots = f_p = 0$.
\end{enumerate}

% -----------------------------------------------------------------------------
\section{Forme lin\'eaire et hyperplan}

\Para{D\'efinitions}

Soit $E$ un $\K$-espace vectoriel.
\begin{itemize}
\item
  Une \emph{forme lin\'eaire} sur $E$ est une application lin\'eaire $\Fn\varphi E\K$.
\item
  L'ensemble des formes lin\'eaires sur $E$ s'appelle le \emph{dual} de $E$
  et se note $E^* = \mathscr{L}(E,\K)$.
\item
  Une \emph{droite} de $E$ est un sous-espace vectoriel de dimension $1$.
  Autrement dit, $D$ est une droite (vectorielle) si et seulement si $\exists x\in E\setminus\Acco{0_E}$, $D =\K x$.
\item
  Un \emph{hyperplan} de $E$ est un sous-espace vectoriel $H$
  qui admet une droite comme suppl\'ementaire.
  Autrement dit, si $H$ est un sous-espace vectoriel de $E$,
  $H$ est un hyperplan si et seulement si $\exists x\in E\setminus\Acco{0_E}$, $E = H\oplus \K x$.
\end{itemize}

\Para{Proposition}

Si $E$ est un $\K$-espace vectoriel de dimension finie, alors $\dim E^* = \dim E$.

\Para{Th\'eor\`eme}

Soit $E$ un $\K$-espace vectoriel de dimension $n\in\Ns$ et $H$ un sous-espace vectoriel de $E$.
Les conditions suivantes sont \'equivalentes:
\begin{enumerate}
\item $H$ est un hyperplan, c.-\`a-d. qu'il existe une droite $D$ telle que $E = H \oplus{} D$;
\item $\dim H = n - 1$;
\item il existe une forme lin\'eaire non nulle $\varphi$ telle que $H = \Ker \varphi$.
\end{enumerate}

\Para{Remarque}

En dimension quelconque, on a encore l'\'equivalence entre~1 et~3.

\Para{Exemple}

On se place dans $E = \R^3$.
\begin{itemize}
\item
  Soit $P = \Ensemble{(x,y,z)\in \R^3}{x+y+z=0}$.
  $P$ est un sous-espace vectoriel de $E$ de dimension $2$, donc un hyperplan.
\item
  Soit $x = (1,0,0)$ et $D = \K x$. On a $E = P \oplus D$.
\item Soit \[ \Fonction{\varphi}{E}{\R}{(x,y,z)}{x+y+z.} \]
  $\varphi$ est une forme lin\'eaire.
\item On a $P = \Ker \varphi$.
\end{itemize}

\Para{Proposition}

Soit $E$ un $\K$-espace vectoriel, $\varphi$ et $\psi$ deux formes lin\'eaires non nulles sur $E$.
Alors \[ \Ker \varphi{} = \Ker \psi{} \iff \exists \lambda\in\Ks \+ \varphi{} = \lambda \psi. \]
Autrement dit, deux formes lin\'eaires non nulles d\'efinissent le m\^eme hyperplan si et seulement si elles sont proportionnelles.

\Para{Remarque}

Soit $E$ un $\K$-espace vectoriel de dimension finie.  On a une correspondance bijective entre:
\begin{itemize}
\item les hyperplans sur $E$, et
\item les formes lin\'eaires non nulles sur $E$, \`a multiplication par un scalaire non nul pr\`es.
\end{itemize}

% -----------------------------------------------------------------------------
\section{Matrice}

% -----------------------------------------------------------------------------
\subsection{D\'efinition alg\'ebrique}

\Para{D\'efinition}

Soit $(n,p)\in \N_*^2$.
Une \emph{matrice} de type $(n,p)$ est une application de $\sqn np$ dans $\K$.
L'ensemble des matrices de type $(n,p)$ est not\'e $\MM{n,p}$.
La matrice $M \colon (i,j) \mapsto m_{i,j}$ est not\'ee $M = \pa{ m_{i,j} }_{\substack{1\leq i\leq n\\1\leq j\leq p}}$.

\Para{Op\'erations}

\begin{enumerate}
\item
  Si $A$ et $B$ sont deux matrices de type $(n,p)$,
  on d\'efinit la matrice $C = A+B$ de type $(n,p)$
  par $c_{i,j} = a_{i,j} + b_{i,j}$
  pour tout $(i,j)\in\sqn np$.
\item
  Si $\lambda$ est un scalaire et $A$ une matrice de type $(n,p)$,
  on d\'efinit la matrice $B = \lambda A$ de type $(n,p)$
  par $b_{i,j} = \lambda a_{i,j}$ pour tout $(i,j)\in\sqn np$.
\item
  Si $A$ est une matrice de type $(n,p)$ et $B$ est une matrice de type $(p,q)$,
  on d\'efinit la matrice $C = A \times{} B$ de type $(n,q)$
  par \[ c_{i,j} = \sum_{k=1}^p a_{i,k} b_{k,j} \] pour tout $(i,j)\in\sqn nq$.
\end{enumerate}

\Para{Propri\'et\'es}

\begin{enumerate}
\item
  $(\MM{n,p},+,\cdot)$ est un espace vectoriel.
\item
  Si $A \in{} \MM{n,p}$, $B \in{} \MM{p,q}$ et $C \in{} \MM{q,r}$, alors
  les produits suivants sont bien d\'efinis
  et $(A \times{} B) \times{} C = A \times{} (B \times{} C)$.
\item
  Si $\lambda\in{} \K$, $A \in{} \MM{n,p}$ et $B \in{} \MM{p,q}$,
  alors $(\lambda A) \times{} B = A \times{} (\lambda B) = \lambda(A \times{} B)$.
\item
  Si $A \in{} \MM{n,p}$, $B \in{} \MM{n,p}$ et $C \in{} \MM{p,q}$,
  alors les sommes et les produits suivants sont bien d\'efinis
  et $(A + B) \times{} C = A\times C + B\times C$.
\item
  Si $A \in{} \MM{n,p}$, $B \in{} \MM{p,q}$ et $C \in{} \MM{p,q}$,
  alors les sommes et les produits suivants sont bien d\'efinis
  et $A \times{} (B + C) = A\times B + A\times C$.
\end{enumerate}

\Para{Remarque}

En g\'en\'eral $A \times{} B \neq{} B \times{} A$, m\^eme dans le cas de deux matrices $(n,n)$.

% -----------------------------------------------------------------------------
\subsection{Lien avec les espaces vectoriels}

\Para{D\'efinition: coordonn\'ees d'un vecteur}

Soit $E$ un $\K$-espace vectoriel de dimension $n$ et $\B = \nUplet e1n$ une base de $E$.
On sait que tout vecteur $x\in E$ se d\'ecompose de fa\c con unique sous la forme
$x = \xi_1 e_1 + \dots + \xi_n e_n$ o\`u $\nUplet \xi1n \in{} \K^n$
On appelle coordonn\'ees de $x$ dans la base $E$ le $n$-uplet
\[ \Coords_\B(x) = \nUplet \xi1n \in \K^n. \]

\Para{Proposition}
Soit $E$ un espace vectoriel de dimension $n$ muni d'une base $\B$.
L'application
\[ \Fonction{}{E}{\K^n}{x}{\Coords_\B(x)} \]
est un isomorphisme.

\Para{Remarque}

On identifiera fr\'equemment les $n$-uplets avec les matrices colonnes;
cela est l\'egitime car l'application
\[ \Fonction{\phi}{\K^n}{\MM{n,1}}{\nUplet\xi1n}{\begin{pmatrix} \xi_1 \\ \vdots \\ \xi_n \end{pmatrix}} \]
est un isomorphisme.

\Para{D\'efinition: matrice d'un endomorphisme}

Soit $E$ et $F$ deux espaces vectoriels de dimensions finies, $\B = \nUplet e1p$ une base de $E$ et $\B' = \nUplet f1n$ une base de $F$.
Pour tout $j\in\ccro{1,p}$, le vecteur $u(e_j)$ se d\'ecompose dans la base $\B'$:
\[ u(e_j) = \sum_{i=1}^n a_{i,j} f_i. \]
On appelle \emph{matrice de l'endomorphisme $u$} la matrice suivante, de type $(n,p)$:
\[ \Mat(u,\B \to \B') = \Mat_{\B\to\B'}(u) = \bigPa{ a_{i,j} }_{\substack{1\leq i\leq n\\1\leq j\leq p}}. \]
\begin{center}
  \begin{tikzpicture}[
    every left delimiter/.style={xshift=0.75em},
    every right delimiter/.style={xshift=-0.75em},
    dots/.style={
      line width=1pt,
      line cap=round,
      dash pattern=on 0pt off 5pt,
      shorten >=.1cm,
    shorten <=.1cm}]
    \matrix (M) [
    matrix of nodes,
    left delimiter=(,
    right delimiter=),
    ]{
      \node (A) {$a_{1,1}$}; &[1.1cm] \node (B) {$a_{1,p}$}; \\[1.1cm]
      \node (C) {$a_{n,1}$}; &        \node (D) {$a_{n,p}$}; \\
    };

    \draw (M.west) node[left] {$\Mat\limits_{\B\to\B'}(u)=$};
    \draw [dots] (A.east)  -- (B.west);
    \draw [dots] (C.east)  -- (D.west);
    \draw [dots] (A.south) -- (C.north);
    \draw [dots] (B.south) -- (D.north);
    \draw (A) [yshift=0.7cm] node (E) {$u(e_1)$};
    \draw (B) [yshift=0.7cm] node (F) {$u(e_p)$};
    \draw (B) [xshift=1cm]   node (G) {$f_1$};
    \draw (D) [xshift=1cm]   node (H) {$f_n$};
    \draw [dots] (E.east)  -- (F.west);
    \draw [dots] (G.south) -- (H.north);
  \end{tikzpicture}
\end{center}
Dans le cas o\`u $E=F$ et $\B=\B'$, on la note $\Mat(u,\B)$ ou $\Mat_\B(u)$.

\Para{D\'efinition}

Inversement, si l'on se donne une matrice $M \in{} \MM{n,p}$,
on peut d\'efinir une application lin\'eaire, dite \emph{canoniquement associ\'ee \`a $M$}, par
\[ \Fonction{f_M}{\K^p}{\K^n}{x}{Mx} \]
o\`u l'on identifie $\K^n$ et $\MM{n,1}$ ainsi que $\K^p$ et $\MM{p,1}$.

Si l'on note $\B$ la base canonique de $\K^p$ et $\B'$ la base canonique de $\K^n$,
on v\'erifie facilement que
\[ \Mat(f_M, \B \to \B') = M. \]

\Para{Th\'eor\`eme}

Soit $E$ et $F$ deux espaces vectoriels de dimensions finies munis de bases $\B$ et $\B'$.
Soit $\Fn uEF$ une application lin\'eaire et $x$ un vecteur de $E$.
Alors \[ \Coords_{\B'}\bigPa{u(x)} = \Mat_{\B\to\B'}(u) \times{} \Coords_\B(x). \]

\Para{Th\'eor\`eme}

Soit $E$ et $F$ deux espaces vectoriels de dimensions finies munis de bases $\B$ et $\B'$.
L'application
\[ \Fonction{}{\mathscr{L}(E,F)}{\MM{n,p}}{u}{\Mat_{\B\to\B'}(u)} \]
est un isomorphisme.

\Para{Th\'eor\`eme}

Soit $E$, $F$, $G$ trois espaces vectoriels de dimensions finies munis des bases respectives $\B$, $\B'$ et $\B''$.
Soit $\Fn uEF$ et $\Fn vFG$ deux applications lin\'eaires.
Alors
\[ \Mat_{\B\to\B''}(v\circ u) = \Mat_{\B'\to\B''}(v) \times\Mat_{\B\to\B'}(u). \]

% -----------------------------------------------------------------------------
\subsection{Formules de changement de base}

\Para{D\'efinition}

Soit $E$ un espace vectoriel de dimension finie, $\B$ et $\B'$ deux bases de $E$.
On appelle \emph{matrice de passage} de $\B$ \`a $\B'$ la matrice
\[ \Pass(\B\to\B') = \Mat_{\B'\to\B}(\Id_E). \]

De fa\c con plus explicite, notons $\B = \nUplet e1n$,
$\B' = \nUplet f1n$ et $\Pass(\B\to\B') = \bigPa{ a_{i,j} }$.
Dans ce cas, on a
\[ \forall j \in{} \ccro{1,n} \+ f_j = \sum_{i=1}^n a_{i,j} e_i. \]
Ainsi,
\begin{center}
  \begin{tikzpicture}[
    every left delimiter/.style={xshift=0.75em},
    every right delimiter/.style={xshift=-0.75em},
    dots/.style={
      line width=1pt,
      line cap=round,
      dash pattern=on 0pt off 5pt,
      shorten >=.1cm,
    shorten <=.1cm}]
    \matrix (M) [
    matrix of nodes,
    left delimiter=(,
    right delimiter=),
    ]{
      \node (A) {$a_{1,1}$}; &[1.1cm] \node (B) {$a_{1,n}$}; \\[1.1cm]
      \node (C) {$a_{n,1}$}; &        \node (D) {$a_{n,n}$}; \\
    };

    \draw (M.west) node[left] {$\Pass(\B\to\B')=$};
    \draw [dots] (A.east)  -- (B.west);
    \draw [dots] (C.east)  -- (D.west);
    \draw [dots] (A.south) -- (C.north);
    \draw [dots] (B.south) -- (D.north);
    \draw (A) [yshift=0.7cm] node (E) {$f_1$};
    \draw (B) [yshift=0.7cm] node (F) {$f_n$};
    \draw (B) [xshift=1cm]   node (G) {$e_1$};
    \draw (D) [xshift=1cm]   node (H) {$e_n$};
    \draw [dots] (E.east)  -- (F.west);
    \draw [dots] (G.south) -- (H.north);
  \end{tikzpicture}
\end{center}

\Para{Proposition}

Soit $E$ un espace vectoriel de dimension finie et $\B$, $\B'$ et $\B''$ trois bases de $E$.
Alors
\begin{enumerate}
\item
  $\Pass(\B \to \B') \times{} \Pass(\B' \to \B'') = \Pass(\B \to \B'')$;
\item
  $\Pass(\B \to \B')$ est inversible, et son inverse est $\Pass(\B' \to \B)$.
\end{enumerate}

\Para{Proposition}

Soit $E$ un espace vectoriel de dimension finie, $\B$ et $\B'$ deux bases de $E$.
Soit $x$ un vecteur de $E$.
Alors \[ X = PX'\]
o\`u
\begin{itemize}
\item $P = \Pass(\B \to \B')$,
\item $X = \Coords_\B(x)$,
\item $X' = \Coords_{\B'}(x)$.
\end{itemize}

\Para{Proposition}

Soit $E$ un espace vectoriel de dimension finie muni de deux bases $\B$ et $\B'$.
Soit $F$ un espace vectoriel de dimension finie muni de deux bases $\BC$ et $\BC'$.
Soit $\Fn uEF$ une application lin\'eaire.
Alors on a \[ A' = Q^{-1} A P \]
o\`u
\begin{itemize}
\item $P = \Pass(\B \to \B')$,
\item $Q = \Pass(\BC \to \BC')$,
\item $A = \Mat_{\B \to \BC}(u)$,
\item $A' = \Mat_{\B' \to \BC'}(u)$.
\end{itemize}

\Para{Corollaire}

Soit $E$ un espace vectoriel de dimension finie muni de deux bases $\B$ et $\B'$.
Soit $u$ un endomorphisme de $E$.
Alors on a \[ A' = P^{-1} A P \]
o\`u
\begin{itemize}
\item $P = \Pass(\B \to \B')$,
\item $A = \Mat_{\B}(u)$,
\item $A' = \Mat_{\B'}(u)$.
\end{itemize}

% -----------------------------------------------------------------------------
\section{Matrice carr\'ee}

\subsection{Matrice inversible}

%\Para{D\'efinition}

La \emph{matrice identit\'e} de taille $n$ est la matrice
$I_n = \bigPa{ \delta_{i,j} } \in{} \MnK$.

\Para{Th\'eor\`eme}

Soit $A$ et $B$ deux matrices carr\'ees de taille $n$.
On suppose que $AB = I_n$.
Alors $BA = I_n$.

% -----------------------------------------------------------------------------
\subsection{Similitude}

\Para{D\'efinition}

Soit $A$ et $B$ deux matrices carr\'ees de $\MnK$.
On dit que $A$ et $B$ sont \emph{semblables}
si et seulement s'il existe une matrice inversible $P\in\GLnK$
telle que \[ B = P^{-1} A P. \]

\Para{Propri\'et\'es}

Il s'agit d'une relation d'\'equivalence, c.-\`a-d.:
\begin{enumerate}
\item $A \sim A$;
\item si $A \sim B$, alors $B \sim A$;
\item si $A \sim B$ et $B \sim C$, alors $A \sim C$.
\end{enumerate}

\Para{Proposition}

Soit $u$ un endomorphisme de $E$.
Les matrices $\Mat_\B(u)$ et $\Mat_{\B'}(u)$ sont semblables.

\Para{Proposition}

Soit $A$ et $B$ deux matrices.
Soit $E$ un espace vectoriel de dimension $n$ et $\B$ une base de $E$.
Notons $u$ l'unique endomorphisme de $E$ tel que $\Mat_\B(u) = A$.
Alors $A$ et $B$ sont semblables si et seulement si il existe une base $\B'$ de $E$ telle que $\Mat_{\B'}(u) = B$.

% -----------------------------------------------------------------------------
\subsection{Trace}

\Para{D\'efinition}

Soit $A = \bigl(a_{i,j}\bigr)_{1\leq i,j\leq n}\in\MnK$ une matrice carr\'ee.
La \emph{trace} de la $A$ est le scalaire
\[ \Tr(A) = \sum_{k=1}^n a_{k,k}. \]

\Para{Propri\'et\'es}

\begin{enumerate}
\item La trace est une application lin\'eaire de $\MnK$ dans $\K$.
  Autrement dit, pour toutes matrices $(A,B)\in\MnK^2$ et pour tous $(\lambda,\mu)\in \K^2$, on a
  \[ \Tr(\lambda A + \mu B) = \lambda\Tr(A) + \mu\Tr(B). \]
\item Pour toutes matrices $(A,B) \in\MnK^2$, on a
  \[ \Tr(AB) = \Tr(BA). \]
\item Si $A$ et $B$ sont deux matrices semblables de $\MnK$, alors elles ont la m\^eme trace.
\end{enumerate}

\Para{Attention}

En g\'en\'eral, $\Tr(ABC)\neq\Tr(ACB)$.

\Para{D\'efinition}[trace d'un endomorphisme]

Soit $E$ un $\K$-espace vectoriel de dimension finie et $u$ un endomorphisme de $E$.
Soit $\B$ une base de $E$ et $M$ la matrice de $u$ dans la base $\B$.
La quantit\'e $\Tr(M)$ ne d\'ependant pas du choix de la base $\B$, mais seulement de $u$, on l'appelle \emph{trace} de l'endomorphisme $u$ et on note $\Tr(u) = \Tr(M)$.

\Para{Propri\'et\'es}

\begin{enumerate}
\item La trace est une application lin\'eaire de $\LE$ dans $\K$.
\item Si $u$ et $v$ sont deux endomorphismes de $E$, alors $\Tr(u\circ v) = \Tr(v\circ u)$.
\end{enumerate}

% -----------------------------------------------------------------------------
\subsection{D\'eterminants}

\Para{Th\'eor\`eme}

Soit $n\in\Ns$.
Il existe une unique application $\Fn{d}{\MnK}{\K}$ v\'erifiant les propri\'et\'es suivantes:
\begin{enumerate}
\item
  $d(I_n) = 1$;
\item
  $d(A) = 0$ si deux colonnes de $A$ sont \'egales;
\item
  Si on fixe $n-1$ colonnes de $A$, l'application qui \`a la colonne restante associe $d(A)$ est lin\'eaire.

  Plus pr\'ecis\'ement, soit $A$, $B$ et $C$ des matrices de $\MnK$.
  pour $1\leq j\leq n$, on note $A_j$, $B_j$ et $C_j$ les $j$-\`eme colonnes de $A$, $B$ et $C$ respectivement.
  On suppose que
  \begin{itemize}
  \item $A_k = \lambda B_k + \mu C_k$ o\`u $k \in{} \ccro{1,n}$ et $(\lambda,\mu)\in \K^2$;
  \item pour tout $j \in{} \ccro{1,n} \setminus{} \acco{k}$, $A_j = B_j = C_j$.
  \end{itemize}
  Alors $d(A) = \lambda d(B) + \mu d(C)$.
\end{enumerate}

Cette application $d$ v\'erifie alors automatiquement les propri\'et\'es suivantes:
\begin{enumerate}[resume]
\item
  $d(B) = -d(A)$ si $B$ se d\'eduit de $A$ en \'echangeant deux colonnes;
\item
  $d(B) = d(A)$ si $B$ se d\'eduit de $A$ par une op\'eration du type $C_i \leftarrow C_i + \sum_{k\neq i} \alpha_k C_k$;
\item
  $d(B) = \lambda d(A)$ si $B$ se d\'eduit de $A$ par une op\'eration du type $C_i \leftarrow \lambda C_i$;
\item
  $d(A) = d(\T A)$ pour toute matrice $A\in\MnK$;
\item
  dans toutes les propri\'et\'es pr\'ec\'edentes, on peut remplacer \enquote{colonnes} par \enquote{lignes};
\item
  $d(AB) = d(A) d(B)$ pour toutes les matrices $(A,B)\in\MnK^2$;
\item
  $d(A) \neq{} 0$ si et seulement si $A$ est inversible.
\end{enumerate}

% -----------------------------------------------------------------------------
\subsubsection{D\'eveloppement par rapport \`a une ligne ou une colonne}

\Para{D\'efinition}

Soit $M \in{} \MnK$ o\`u $n \geq{} 2$.
Pour tous $(i,j) \in{} \ccro{1,n}^2$, on d\'efinit:
\begin{itemize}
\item
  le mineur $\mu_{i,j}$
  \'egal au d\'eterminant de la matrice $M$
  o\`u l'on enl\`eve la ligne $i$ et la colonne $j$;
\item
  le cofacteur $\Delta_{i,j} = (-1)^{i+j} \mu_{i,j}$;
\item
  la comatrice $\Com(M) \in{} \MnK$ form\'ee des cofacteurs.
\end{itemize}

\Para{Proposition}

Soit $M \in{} \MnK$ o\`u $n \geq{} 2$.

D\'eveloppement par rapport \`a la $i$-\`eme ligne:
\[ \det(M) = \sum_{j=1}^n a_{i,j} \Delta_{i,j}. \]

D\'eveloppement par rapport \`a la $j$-\`eme colonne:
\[ \det(M) = \sum_{i=1}^n a_{i,j} \Delta_{i,j}. \]

\Para{Proposition (hors-programme)}

Soit $M \in{} \MnK$ o\`u $n \geq{} 2$.
On a
\[ M \T{\Com(M)} = \T{\Com(M)} M = \det(M) I_n. \]
En particulier, si $M$ est inversible on a
\[ M^{-1} = \frac{1}{\det(M)} \, \T{\Com(M)}. \]

\Para{Remarque}

Cette derni\`ere formule n'est en g\'en\'eral pas tr\`es efficace
pour calculer l'inverse d'une matrice, sauf pour $n=2$:
\[ \begin{pmatrix} a & b \\ c & d \end{pmatrix}^{-1} = \frac{1}{ad-bc} \begin{pmatrix} d & -b \\ -c & a \end{pmatrix}. \]

% -----------------------------------------------------------------------------
\subsubsection{D\'eterminant de Vandermonde}

\Para{D\'efinitions}

Soit $\nUplet a1n \in \K^n$.
La \emph{matrice de Vandermonde} associ\'ee \`a $\nUplet a1n$ est
\[ M = \begin{pmatrix}
    1 &  1 &  1 &  \dots &  1  \\
    a_1 &  a_2 &  a_3 &  \dots &  a_n  \\
    a_1^2 &  a_2^2 &  a_3^2 &  \dots &  a_n^2  \\
    \vdots &  \vdots &  \vdots &   &  \vdots  \\
a_1^{n-1} &  a_2^{n-1} &  a_3^{n-1} &  \dots &  a_n^{n-1}  \end{pmatrix}. \]
Le \emph{d\'eterminant de Vandermonde} $V \nUplet a1n$ est le d\'eterminant de la matrice de Vandermonde ci-dessus.

\Para{Th\'eor\`eme}

Avec les m\^emes notations,
\[ V\nUplet a1n = \prod_{1\leq i < j\leq n} (a_j - a_i). \]
En particulier,
\begin{itemize}
\item $V(a) = 1$,
\item $V(a,b) = b-a$,
\item $V(a,b,c) = (b-a)(c-a)(c-b)$.
\end{itemize}

\Para{Corollaire}

La matrice de Vandermonde associ\'ee \`a $\nUplet a1n$ est inversible si et seulement si les nombres $\nUplet a1n$ sont deux \`a deux distincts.

% -----------------------------------------------------------------------------
\section{Matrice par blocs}

\Para{D\'efinition}

Soit $(s,t) \in(\Ns)^2$, $(n_1, \dots, n_s) \in(\Ns)^s$ et $(p_1, \dots, p_t) \in(\Ns)^t$ des entiers naturels non nuls.
On pose:
\begin{itemize}
\item pour tout $k \in\Dcro{0,s}$, $\sigma_k = \sum_{i=1}^k n_i$,
\item pour tout $l \in\Dcro{0,t}$, $\tau_l = \sum_{j=1}^l p_j$,
\item $n = n_1 + \dots + n_s = \sigma_s$,
\item $p = p_1 + \dots + p_t = \tau_t$.
\end{itemize}

Soit $A\in\mathrm{M}_{n,p}(\K) = \bigl( a_{ij} \bigr)_{\substack{1\leq i\leq n \\ 1\leq j\leq p}}$ une matrice.
Sa d\'ecomposition par blocs suivant le d\'ecoupage $\nUplet n1s$ pour les lignes et $\nUplet p1t$ pour les colonnes est
\begin{center}
  \begin{tikzpicture}[
    dots/.style={
      line width=1pt,
      line cap=round,
      dash pattern=on 0pt off 5pt,
      shorten >=.1cm,
    shorten <=.1cm}]
    \matrix[
    matrix of nodes,
    left delimiter=(,
    right delimiter=),
    ]{
      \node (A) {$A_{1,1}$}; &[1.1cm] \node (B) {$A_{1,t}$}; \\[1.1cm]
      \node (C) {$A_{s,1}$}; &        \node (D) {$A_{s,t}$}; \\
    };

    \draw [dots] (A.east)  -- (B.west);
    \draw [dots] (C.east)  -- (D.west);
    \draw [dots] (A.south) -- (C.north);
    \draw [dots] (B.south) -- (D.north);

    \draw [<->]  ([xshift=1cm]B.north east) -- node [right] {$n_1$} ([xshift=1cm]B.south east);
    \draw [<->]  ([xshift=1cm]D.north east) -- node [right] {$n_s$} ([xshift=1cm]D.south east);
    \draw [dots] ([xshift=1cm]B.south east) -- ([xshift=1cm]D.north east);
    \draw [<->]  ([xshift=1.75cm]B.north east) -- node [right] {$n$} ([xshift=1.75cm]D.south east);

    \draw [<->]  ([yshift=-0.5cm]C.south west) -- node [below] {$p_1$}([yshift=-0.5cm]C.south east);
    \draw [<->]  ([yshift=-0.5cm]D.south west) -- node [below] {$p_t$}([yshift=-0.5cm]D.south east);
    \draw [dots] ([yshift=-0.5cm]C.south east) -- ([yshift=-0.5cm]D.south west);
    \draw [<->]  ([yshift=-1.1cm]C.south west) -- node [below] {$p$} ([yshift=-1.1cm]D.south east);

    \node[xshift=-0.9cm] at ($(A.west)!0.5!(C.west)$) { $A =$ };
  \end{tikzpicture}
\end{center}
o\`u pour tout $k\in\Dcro{1,s}$ et $l\in\Dcro{1,t}$, on a
\[ A_{k,l} = \begin{pmatrix}
    a_{1+\sigma_{k-1},1+\tau_{l-1}}  &  \dots &  a_{1+\sigma_{k-1},\tau_l}  \\
    \vdots &   &  \vdots  \\
    a_{\sigma_k, 1+\tau_{l-1}} &  \dots &  a_{\sigma_k,\tau_l}
\end{pmatrix} \in\mathrm{M}_{n_k,q_l}(\K). \]

\Para{Proposition}[addition par blocs]

Soit $A$ et $B$ deux matrices de $\mathrm{M}_{n,p}(\K)$ exprim\'ees par blocs selon les m\^emes d\'ecoupages. Soit $(\lambda,\mu)\in \K^2$.
Alors la matrice $\lambda A + \mu B$ s'exprime par blocs selon les m\^emes d\'ecoupages que $A$ et $B$, et les blocs s'obtiennent en combinant les blocs situ\'es aux m\^emes places.

\Para{Th\'eor\`eme}[produit par blocs]

Soit $A\in\mathrm{M}_{n,p}(\K)$ et $B\in\mathrm{M}_{p,q}(\K)$.
On suppose que $A$ a une d\'ecompositon par blocs suivant le d\'ecoupage $\nUplet n1s$ pour les lignes et $\nUplet p1t$ pour les colonnes.
On suppose que $B$ a une d\'ecomposition par blocs suivant le d\'ecoupage $\nUplet p1t$ pour les lignes et $\nUplet q1u$ pour les colonnes.

Alors le produit $C = AB\in\mathrm{M}_{n,q}(\K)$ admet une d\'ecomposition par blocs suivant le d\'ecoupage $\nUplet n1s$ pour les lignes et $\nUplet q1u$ pour les colonnes
\[ C = \begin{pmatrix} C_{1,1} &  \dots &  C_{1,u}  \\  \vdots &   &  \vdots  \\  C_{s,1} &  \dots &  C_{s,u} \end{pmatrix} \]
o\`u pour tous $i\in\Dcro{1,s}$ et $j\in\Dcro{1,u}$,
\[ C_{i,j} = \sum_{k=1}^t A_{i,k} B_{k,j} \]

\Para{Remarque}

Le cas qui nous int\'eressera le plus fr\'equemment est celui des matrices carr\'ees ayant le m\^eme d\'ecoupage pour les lignes et pour les colonnes. Dans ce cas, les blocs diagonaux sont \'egalement des matrices carr\'ees.

\Para{Proposition}

Soit $E$ un espace vectoriel de dimension finie $n$, $F$ un sous-espace vectoriel de $E$ et $u\in\LE$.
Soit $\B' = \nUplet e1p$ une base de $F$ telle que $\B = \nUplet e1n$ soit une base de $E$.
Soit $A$ la matrice de l'endomorphisme $u$ dans la base $\B$;
on la d\'ecompose par blocs selon le d\'ecoupage $(p,n-p)$ pour les lignes et les colonnes:
\[ A = \begin{pmatrix} A_{1,1} & A_{1,2} \\ A_{2,1} & A_{2,2} \end{pmatrix}. \]
Alors $F$ est stable par $u$ si et seulement si ma matrice $A_{2,1}$ est nulle.
Dans ce cas, notons $v$ l'endomophisme induit par $u$ sur $F$.
La matrice de $v$ dans la base $\B'$ est $A_{1,1}$.

% -----------------------------------------------------------------------------
\subsection{D\'eterminant par blocs}

\Para{Attention}

Si $A$, $B$, $C$ et $D$ sont des matrices de $\MnK$ et si
$M = \begin{pmatrix} A & B \\ C & D \end{pmatrix} \in\mathrm{M}_{2n}(\K)$, alors en g\'en\'eral
\[ \det(M) \neq\det(AD-BC). \]

\Para{Th\'eor\`eme}[d\'eterminant triangulaire par blocs]

Soit $A\in\MnK$ une matrice carr\'ee admettant une d\'ecomposition par blocs suivant le d\'ecoupage $\nUplet n1s$ pour les lignes et pour les colonnes.
On note $A = \bigl( A_{i,j} \bigr)_{1\leq i,j\leq s}$ cette d\'ecomposition, de sorte que $A_{i,j}\in\mathrm{M}_{n_i,n_j}(\K)$.
On suppose que $A$ est triangulaire sup\'erieure par blocs, c.-\`a-d. que
\[ \forall(i,j)\in\Dcro{1,s}^2 \+ i > j \implies A_{i,j} = 0. \]

Alors le d\'eterminant de $A$ est \'egal au produit des d\'eterminants des blocs diagonaux:
\[ \det(A) = \prod_{k=1}^s \det(A_{k,k}). \]

Le r\'esultat est identique dans le cas des matrices triangulaires inf\'erieures par blocs.

% -----------------------------------------------------------------------------
\section{Exercices}

% -----------------------------------------------------------------------------
\par\pagebreak[1]\par
\paragraph{Exercice 1}%
\hfill{\tiny 4203}%
\begingroup~

Soit $(\K,+,\cdot)$ un corps.
Pour tout $x\in \K$, on note $-x$ le sym\'etrique de $x$ pour la loi $+$.
Montrer que les relations suivantes vraies pour tous $(x,y,z)\in \K^3$:
\begin{enumerate}
\item $0 x = 0$;
\item $(-1) x = -x$;
\item $(-x)y = x(-y) = -(xy)$;
\item $(x-y)z = xz - yz$.
\end{enumerate}
\endgroup

% -----------------------------------------------------------------------------
\par\pagebreak[1]\par
\paragraph{Exercice 2}%
\hfill{\tiny 2993}%
\begingroup~

Soit $(E,+,\cdot)$ un $\K$-espace vectoriel.
Pour tout $x\in E$, on note $-x$ le sym\'etrique de $x$ pour la loi $+$.
Pour $(\lambda,\mu)\in \K^2$ et $(x,y)\in E^2$, montrer que:
\begin{enumerate}
\item $\lambda x = 0_E$ si et seulement si $\lambda= 0_\K$ ou $x = 0_E$;
\item $(-\lambda) x = \lambda(-x) = -(\lambda x)$;
\item $(\lambda- \mu)x = \lambda x - \mu x$;
\item $\lambda(x-y) = \lambda x - \lambda y$.
\end{enumerate}
\endgroup

% -----------------------------------------------------------------------------
\par\pagebreak[1]\par
\paragraph{Exercice 3}%
\hfill{\tiny 6846}%
\begingroup~

Les ensembles suivants sont-ils des $\R$-espaces vectoriels?
\begin{enumerate}
\item L'ensemble des suites r\'eelles convergentes.
\item L'ensemble des suites r\'eelles convergentes vers $0$.
\item L'ensemble des suites r\'eelles convergentes vers $1$.
\item L'ensemble des suites r\'eelles born\'ees.
\item L'ensemble des suites r\'eelles croissantes.
\item L'ensemble des suites r\'eelles monotones.
\item L'ensemble des suites r\'eelles non convergentes.
\item L'ensemble des suites r\'eelles p\'eriodiques \`a partir d'un certain rang.
\item L'ensemble des fonctions lipschitziennes de $\R$ dans $\R$.
\item L'ensemble des fonctions paires de $\R$ dans $\R$.
\item L'ensemble des fonctions de $\R$ dans $\R$ qui prennent la valeur $\beta$ en $\alpha$.
\end{enumerate}
\endgroup

% -----------------------------------------------------------------------------
\par\pagebreak[1]\par
\paragraph{Exercice 4}%
\hfill{\tiny 2906}%
\begingroup~

Soit $E$, $F$, $G$ trois $\K$-espaces vectoriels, $f\in\mathscr{L}(E,F)$ et $g\in\mathscr{L}(F,G)$.

Montrer que $g\circ f = \tilde 0$ si et seulement si $\Ima f\subset\Ker g$.
\endgroup

% -----------------------------------------------------------------------------
\par\pagebreak[1]\par
\paragraph{Exercice 5}%
\hfill{\tiny 5827}%
\begingroup~

Soit $E$ un $\K$-espace vectoriel et $(f,g)\in\LE^2$.

Montrer que $f \bigl( \Ker(g\circ f) \bigr) = \Ker g\cap\Ima f$.
\endgroup

% -----------------------------------------------------------------------------
\par\pagebreak[1]\par
\paragraph{Exercice 6}%
\hfill{\tiny 0111}%
\begingroup~

Soit $E$ le $\R$-espace vectoriel des suites r\'eelles convergentes.
On note $F$ le sous-espace vectoriel des suites qui convergent vers $0$ et $G$ le sous-espace vectoriel des suites constantes.

Montrer que $F$ et $G$ sont suppl\'ementaires dans $E$.
\endgroup

% -----------------------------------------------------------------------------
\par\pagebreak[1]\par
\paragraph{Exercice 7}%
\hfill{\tiny 6490}%
\begingroup~

\begin{enumerate}
\item D\'eterminer une base des espaces vectoriels suivants:
  \begin{enumerate}
  \item $\Vect(e_1,e_2,e_3,e_4) \subset \R^3$
    o\`u $e_1 = (5,-3,-2)$, $e_2 = (17,-12,-5)$, $e_3 = (8,-12,4)$, $e_4 = (-91,46,45)$.
  \item $\Ensemble{P\in \R_3[X]}{P(1) = P'(1) = P''(1) = 0}$
  \item $\Ensemble{(x,y,z,t)\in \R^4}{ x+y+z+t = 2x+3y+4z+5t = 0 }$.
  \end{enumerate}
\item D\'eterminer un syst\`eme d'\'equations cart\'esiennes des sous-espaces vectoriels suivants:
  \begin{enumerate}
  \item $\Vect(e_1,e_2,e_3,e_4) \subset \R^3$
    o\`u $e_1 = (5,4,9)$, $e_2 = (17,-12,5)$, $e_3 = (8,-12,-4)$, $e_4 = (-91,46,-45)$.
  \item $\Vect\bigl( (1,2,3,4), (5,6,7,8) \bigr)\subset \R^4$.
  \end{enumerate}
\end{enumerate}
\endgroup

% -----------------------------------------------------------------------------
\par\pagebreak[1]\par
\paragraph{\href{https://psi.miomio.fr/exo/7222.pdf}{Exercice 8} (indice de nilpotence)}%
\hfill{\tiny 7222}%
\begingroup~

Soit $E$ un $\K$-espace vectoriel de dimension $n\in\Ns$ et $f\in\LE$ un endomorphisme nilpotent.
On rappelle qu'un endomorphisme $f$ est dit \emph{nilpotent}
s'il existe $m\in\Ns$ tel que $f^m = \tilde0$.

On note $q$ \emph{l'indice de nilpotence} de $f$, c.-\`a-d.
\[ q = \min \Ensemble{k\in\Ns}{f^k = \tilde 0}. \]
\begin{enumerate}
\item Montrer que $\Ker(f^{q-1})\neq E$.
\item Soit $x\in E\setminus\Ker(f^{q-1})$.
  Montrer que la famille
  \[ \bigl( x,f(x),\dots,f^{q-1}(x) \bigr) \] est libre.
\item En d\'eduire $q\leq n$.
\end{enumerate}
\endgroup

% -----------------------------------------------------------------------------
\par\pagebreak[1]\par
\paragraph{Exercice 9}%
\hfill{\tiny 8110}%
\begingroup~

Soit $E$ un $\K$-espace vectoriel de dimension finie et $\nUplet e1n$ une base de $E$.
On pose:
\[ \forall i\in\Dcro{1,n}\+ f_i = \sum_{\substack{j=1 \\ j\neq i}}^n e_j. \]
Que peut-on dire de la famille $\nUplet f1n$?
\endgroup

% -----------------------------------------------------------------------------
\par\pagebreak[1]\par
\paragraph{Exercice 10}%
\hfill{\tiny 7853}%
\begingroup~

Soit $E$ un $\K$-espace vectoriel et $F$, $G$, $H$ des sous-espaces vectoriels de $E$.
\begin{enumerate}
\item Comparer (au sens de l'inclusion) $F+(G\cap H)$ et $(F+G)\cap(F+H)$.
\item Comparer de m\^eme $F\cap(G+H)$ et $(F\cap G)+(F\cap H)$.
\item Montrer que, si $F\subset G$, on a $F+(G\cap H) = (F+G)\cap(F+H)$.
  Contre-exemple si $F \not\subset G$?
\item Montrer que, si $F\subset G$, $F+H=G+H$ et $F\cap H=G\cap H$, alors $F=G$.
\end{enumerate}
\endgroup

% -----------------------------------------------------------------------------
\par\pagebreak[1]\par
\paragraph{Exercice 11}%
\hfill{\tiny 4236}%
\begingroup~

Soit $E$ le $\R$-espace vectoriel des fonctions de classe $\CC\infty$ et $2\pi$-p\'eriodiques de $\R$ dans $\R$.
Soit $\Fonction\varphi E E f {f''}$
\begin{enumerate}
\item Montrer que $\Ker\varphi= \Ensemble{f\in E}{f \text{ est constante}}$.
\item Montrer que $\Ima\varphi= \Ensemble{f\in E}{\int_0^{2\pi} f = 0 }$.
\item En d\'eduire que $E = \Ker\varphi\oplus\Ima\varphi$.
\end{enumerate}
\endgroup

% -----------------------------------------------------------------------------
\par\pagebreak[1]\par
\paragraph{\href{https://psi.miomio.fr/exo/3706.pdf}{Exercice 12}}%
\hfill{\tiny 3706}%
\begingroup~

Soit $E$ un $\K$-espace vectoriel et $f\in\LE$ tel que $\forall x\in E$, la famille $(x, f(x))$ est li\'ee.

Montrer que $f$ est une homoth\'etie.
\endgroup

% -----------------------------------------------------------------------------
\par\pagebreak[1]\par
\paragraph{Exercice 13}%
\hfill{\tiny 6179}%
\begingroup~

Soit $E$ un $\K$-espace vectoriel et $A\subset\LE$.
On appelle \emph{centre} de $A$ l'ensemble
\[ \mathcal{Z}(A) = \Ensemble{f\in\LE} {\forall g\in A \+ f\circ g = g\circ f}. \]
D\'eterminer $\mathcal{Z}\bigl(\LE\bigr)$.
\endgroup

% -----------------------------------------------------------------------------
\par\pagebreak[1]\par
\paragraph{Exercice 14 (projecteur et projection)}%
\hfill{\tiny 6449}%
\begingroup~

Soit $E$ un $\K$-espace vectoriel.
\begin{itemize}
\item
  On appelle \emph{projecteur} de $E$ tout endomorphisme $u\in E$ tel que $u\circ u = u$.
\item
  Soit $F$ et $G$ sont deux sous-espaces vectoriels suppl\'ementaires de $E$.
  On appelle \emph{projection} sur $F$ parall\`element \`a $G$ l'application $\Fn vEE$ qui \`a tout vecteur $x\in E$ associe le vecteur $y$ tel que $x = y + z$ o\`u $y\in F$ et $z\in G$.
\end{itemize}

On va montrer l'\'equivalence entre ces deux notions.
\begin{enumerate}
\item
  Montrer que la projection $v$ sur $F$ parall\`element \`a $G$ est lin\'eaire, puis que c'est un projecteur.
  Montrer que $\Ima(v) = F$ et $\Ker(v) = G$.
\item
  Soit $u$ un projecteur.
  Montrer que $\Ima(u)$ et $\Ker(u)$ sont suppl\'ementaires,
  puis que $u$ est la projection sur $\Ima(u)$ parall\`element \`a $\Ker(u)$.
\end{enumerate}
\endgroup

% -----------------------------------------------------------------------------
\par\pagebreak[1]\par
\paragraph{Exercice 15 (\`a conna\^itre)}%
\hfill{\tiny 6973}%
\begingroup~

Soit $E$ un $\K$-espace vectoriel et $p$ un projecteur de $E$.
\begin{enumerate}
\item Montrer que $\Ima p = \Ensemble{x\in E}{p(x) = x} = \Ker(p - \Id_E)$.
\item Montrer que $E = \Ker p\oplus\Ima p$.
\end{enumerate}
\endgroup

% -----------------------------------------------------------------------------
\par\pagebreak[1]\par
\paragraph{Exercice 16}%
\hfill{\tiny 3524}%
\begingroup~

Soit $E$ un $\K$-espace vectoriel de dimension finie et $p$ un projecteur de $E$.
Montrer que $\Tr p = \Rang p$.
\endgroup

% -----------------------------------------------------------------------------
\par\pagebreak[1]\par
\paragraph{Exercice 17}%
\hfill{\tiny 6270}%
\begingroup~

Soit $E$ un $\K$-espace vectoriel, $p$ un projecteur de $E$ et $u\in\LE$.
Montrer que $p$ et $u$ commutent si et seulement si $\Ima p$ et $\Ker p$ sont stables par $u$.
\endgroup

% -----------------------------------------------------------------------------
\par\pagebreak[1]\par
\paragraph{Exercice 18}%
\hfill{\tiny 5844}%
\begingroup~

Soit $E$ un $\K$-espace vectoriel, $p$ et $q$ deux projecteurs de $E$.
\begin{enumerate}
\item Montrer que $p+q$ est un projecteur de $E$ si et seulement si $p\circ q = q\circ p = \tilde 0$.
\item Montrer que, dans ce cas, $\Ima(p+q) = \Ima p\oplus\Ima q$ et que $\Ker(p+q) = \Ker(p)\cap\Ker q$.
\end{enumerate}
\endgroup

% -----------------------------------------------------------------------------
\par\pagebreak[1]\par
\paragraph{Exercice 19}%
\hfill{\tiny 2483}%
\begingroup~

Soit $E$ un $\K$-espace vectoriel, $p$ et $q$ deux projecteurs de $E$ qui commutent.
\begin{enumerate}
\item Montrer que $p\circ q$ est un projecteur.
\item Montrer que, dans ce cas, $\Ima(p\circ q) = \Ima p\cap\Ima q$ et $\Ker(p\circ q) = \Ker p + \Ker q$.
\end{enumerate}
\endgroup

% -----------------------------------------------------------------------------
\par\pagebreak[1]\par
\paragraph{Exercice 20}%
\hfill{\tiny 2438}%
\begingroup~

Soit $E$ un $\K$-espace vectoriel, $p$ et $q$ deux projecteurs de $E$ tels que $p\circ q = \tilde 0$.
Soit $r = p + q - q\circ p$.
\begin{enumerate}
\item Montrer que $r$ est un projecteur.
\item Montrer que $\Ima r = \Ima p\oplus\Ima q$ et $\Ker r = \Ker p\cap\Ker q$.
\end{enumerate}
\endgroup

% -----------------------------------------------------------------------------
\par\pagebreak[1]\par
\paragraph{Exercice 21}%
\hfill{\tiny 5459}%
\begingroup~

Soit $E$ un $\K$-espace vectoriel de dimension finie et $u$ un endomorphisme de $E$.
On note $u^2$ l'endomorphisme $u\circ u$.
Montrer que les conditions suivantes sont \'equivalentes:
\begin{enumerate}
\item $\Ima u = \Ima u^2$
\item $\Rang u = \Rang u^2$
\item $\dim \Ker u = \dim \Ker u^2$
\item $\Ker u = \Ker u^2$
\item $\Ima u\cap\Ker u = \Acco{0}$
\item $E = \Ima u\oplus\Ker u$
\item $E = \Ima u + \Ker u$
\end{enumerate}
\endgroup

% -----------------------------------------------------------------------------
\par\pagebreak[1]\par
\paragraph{Exercice 22}%
\hfill{\tiny 2065}%
\begingroup~

Soit $E$ le $\R$-espace vectoriel des applications de $\R$ dans $\R$.

\'Etudier l'ind\'ependance des familles de fonctions suivantes:
\begin{enumerate}
\item $\bigl( x \mapsto \sin^n x \bigr)_{n\in\Dcro{0,N}}$ o\`u $N\in \N$ fix\'e
\item $\bigl( x \mapsto \ch^n x \bigr)_{n\in\Dcro{0,N}}$ o\`u $N\in \N$ fix\'e
\item $\bigl( x \mapsto \Abs{x-a} \bigr)_{a\in A}$ o\`u $A\subset \R$ est une partie finie de $\R$.
\end{enumerate}
\endgroup

% -----------------------------------------------------------------------------
\par\pagebreak[1]\par
\paragraph{\href{https://psi.miomio.fr/exo/4224.pdf}{Exercice 23}}%
\hfill{\tiny 4224}%
\begingroup~

Soit $E$ l'espace vectoriel des applications de $\R$ dans $\R$
et $F = \Ensemble{f\in E}{f(0)=f(1)=0}$.

Montrer que $F$ est un sous-espace vectoriel de $E$
et en donner un suppl\'ementaire.
\endgroup

% -----------------------------------------------------------------------------
\par\pagebreak[1]\par
\paragraph{Exercice 24 (sous-espace vectoriel engendr\'e par une partie)}%
\hfill{\tiny 5043}%
\begingroup~

Soit $E$ un $\K$-espace vectoriel et $X\subset E$ une partie non vide de $E$.
On note $V$ l'ensemble des vecteurs $v\in E$ tels qu'il existe $n\in \N^*$, $\nUplet x1n \in X^n$ et $\nUplet\alpha1n \in \K^n$ tels que
\[ v = \sum_{i=1}^n \alpha_i x_i. \]
\begin{enumerate}
\item Dans cette question seulement, on suppose que l'ensemble $X = \{y_1,\dots,y_p\}$ est fini. Montrer que $X = \Vect(y_1,\dots,y_p)$.
\item Montrer que $V$ est un sous-espace vectoriel de $E$ et que $X\subset V$.
\item Soit $F$ un sous-espace vectoriel de $E$ tel que $X\subset V$.
  Montrer que $V\subset F$.
\item En d\'eduire que $V$ est, au sens de l'inclusion, le plus petit sous-espace vectoriel de $E$ contenant la partie $X$.
\end{enumerate}

Le sous-espace vectoriel $V$ s'appelle le sous-espace vectoriel engendr\'e par $X$ et not\'e $V = \Vect(X)$.
\endgroup

% -----------------------------------------------------------------------------
\par\pagebreak[1]\par
\paragraph{\href{https://psi.miomio.fr/exo/2033.pdf}{Exercice 25} (l'union de sous-espaces vectoriels n'est pas un sous-espace vectoriel)}%
\hfill{\tiny 2033}%
\begingroup~

Soit $E$ un $\K$-espace vectoriel.
\begin{enumerate}
\item
  On suppose que $F$ et $G$ sont deux sous-espaces vectoriels de $E$
  tels que $H = F\cup G$ est \'egalement un sous-espace vectoriel.
  Montrer que $F\subset G$ ou $G\subset F$.
\item
  Plus g\'en\'eralement, si $\K$ est infini,
  en supposant que $\Uplet{F_1}{F_n}$ sont des sous-espaces vectoriels de $E$
  tels que $H = \bigcup_{i=1}^n F_i$ est \'egalement un sous-espace vectoriel,
  montrer qu'il existe un $j\in\ccro{1,n}$ tel que $H = F_j$.
  On pourra raisonner par r\'ecurrence sur $n$.
\end{enumerate}
\endgroup

% -----------------------------------------------------------------------------
\par\pagebreak[1]\par
\paragraph{\href{https://psi.miomio.fr/exo/7471.pdf}{Exercice 26} (suppl\'ementaire commun)}%
\hfill{\tiny 7471}%
\begingroup~

Soit $E$ un $\K$-espace vectoriel de dimension finie, $F$ et $G$ deux sous-espaces vectoriels de $E$ de m\^eme dimension $r$.
Montrer qu'il existe un sous-espace vectoriel $H$ tel que $E = F\oplus H = G\oplus H$.
On pourra proc\'eder par r\'ecurrence descendante sur $r$.
\endgroup

% -----------------------------------------------------------------------------
\par\pagebreak[1]\par
\paragraph{Exercice 27 (polyn\^omes de Lagrange)}%
\hfill{\tiny 9567}%
\begingroup~

Soit $\nUplet a0n \in \K^{n+1}$ des nombres deux \`a deux distincts.
\begin{enumerate}
\item Soit $\nUplet b0n \in \K^{n+1}$.
  Montrer qu'il existe un unique polyn\^ome $P\in \K_n[X]$ tel que
  \[ \forall i\in\Dcro{0,n} \+ P(a_i) = b_i. \]
  Il s'agit du polyn\^ome interpolateur de Lagrange.
  On pourra faire appara\^itre une matrice de Vandermonde.
\item On d\'efinit l'application
  \[ \Fonction{\varphi}{\K_n[X]}{\K^{n+1}}{P}{\bigl(P(a_0),\dots,P(a_n)\bigr).} \]
  \begin{enumerate}
  \item Montrer que $\varphi$ est un isomorphisme.
  \item On note $(e_0,\dots,e_n)$ la base canonique de $\K^{n+1}$.
    D\'eterminer explicitement l'unique polyn\^ome $L_i$ tel que $\varphi(L_i) = e_i$.
  \item En d\'eduire une expression du polyn\^ome interpolateur de Lagrange.
  \end{enumerate}
\end{enumerate}
\endgroup

% -----------------------------------------------------------------------------
\par\pagebreak[1]\par
\paragraph{Exercice 28}%
\hfill{\tiny 1021}%
\begingroup~

Soit $E$, $F$ et $G$ des $\K$-espaces vectoriels de dimensions finies, $\Fn uEF$ et $\Fn vFG$ des applications lin\'eaires.
\begin{enumerate}
\item Montrer que $\Ima(v\circ u)\subset\Ima(v)$. En d\'eduire que $\Rang(v\circ u)\leq\Rang(v)$.
\item Montrer que $\Ker(v\circ u)\supset\Ker(u)$. En d\'eduire que $\Rang(v\circ u)\leq\Rang(u)$.
\item On appelle $w$ la restriction de $v$ \`a $\Ima(u)$.
  \begin{enumerate}
  \item D\'eterminer l'image et le noyau de $w$.
  \item Appliquer le th\'eor\`eme du rang \`a $w$.
  \item En remarquant que $\Ker(w) \subset{} \Ker(v)$, montrer que
    $ \Rang(u) + \Rang(v)\leq\Rang(v\circ u) + \dim(F) $.
  \end{enumerate}
\item Montrer que
  $ \Rang(u) + \Rang(v) - \dim(F) \leq{} \Rang(v\circ u) \leq{} \min(\Rang u, \Rang v)$,
  et $ \dim \Ker(v\circ u)\leq\dim \Ker(u) + \dim \Ker(v) $.
\end{enumerate}
\endgroup

% -----------------------------------------------------------------------------
\par\pagebreak[1]\par
\paragraph{\href{https://psi.miomio.fr/exo/5898.pdf}{Exercice 29}}%
\hfill{\tiny 5898}%
\begingroup~

\begin{enumerate}
\item
  Soit $E$ et $F$ deux $\K$-espaces vectoriels de dimensions finies, $\Fn uEF$ et $\Fn vFE$ deux applications lin\'eaires telles que $u\circ v = \Id_F$.
  \begin{enumerate}
  \item
    On suppose $\dim E = \dim F$; montrer que $v\circ u = \Id_E$.
  \item
    Trouver un contre-exemple dans le cas o\`u $\dim E \neq{} \dim F$.
  \end{enumerate}
\item Soit $E$ un $\K$-espace vectoriel, $u$ et $v$ deux endomorphismes de $E$ tels que $u\circ v = \Id_E$.
  A-t-on n\'ecessairement $v\circ u = \Id_E$?
\end{enumerate}
\endgroup

% -----------------------------------------------------------------------------
\par\pagebreak[1]\par
\paragraph{\href{https://psi.miomio.fr/exo/9859.pdf}{Exercice 30}}%
\hfill{\tiny 9859}%
\begingroup~

Soit $E$ un $\K$-espace vectoriel de dimension finie,
$f$ et $g$ deux endomorphismes de $E$
tels que $f^2 + f\circ g = \Id_E$.
Montrer que $f$ et $g$ commutent.
\endgroup

% -----------------------------------------------------------------------------
\par\pagebreak[1]\par
\paragraph{Exercice 31}%
\hfill{\tiny 4591}%
\begingroup~

Soit $E$ un espace vectoriel de dimension finie et $f$, $g$ deux endomorphismes de $E$
tels que \[ E = \Ima f + \Ima g = \Ker f + \Ker g \]

Montrer que ces deux sommes sont directes.
\endgroup

% -----------------------------------------------------------------------------
\par\pagebreak[1]\par
\paragraph{Exercice 32}%
\hfill{\tiny 4446}%
\begingroup~

Soit $\nUplet f1n$ une famille de fonctions de $\R$ dans $\R$.
Montrer que $\nUplet f1n$ est libre
si et seulement s'il existe $\nUplet x1n\in \R^n$ tel que
$A = \bigPa{f_i(x_j)}_{1\leq i,j\leq n}$ soit de d\'eterminant non nul.
\endgroup

% -----------------------------------------------------------------------------
\par\pagebreak[1]\par
\paragraph{Exercice 33}%
\hfill{\tiny 6010}%
\begingroup~

Soit $(p,n)\in\Pa{\Ns}^2$ avec $p\neq n$,
$A\in\M{M}{n,p}{\R}$ et $B\in\M{M}{p,n}{\R}$.
Calculer $\det(AB) \times\det(BA)$.
\endgroup

% -----------------------------------------------------------------------------
\par\pagebreak[1]\par
\paragraph{Exercice 34}%
\hfill{\tiny 0612}%
\begingroup~

Soit $E$ un espace vectoriel de dimension $n$.
Soit $f$ et $g$ deux endomorphismes de $E$ tels que
$f\circ g = 0$ et $f+g\in\GLE$.
Montrer que $\Rang(f) + \Rang(g) = n$.
\endgroup

% -----------------------------------------------------------------------------
\par\pagebreak[1]\par
\paragraph{\href{https://psi.miomio.fr/exo/6582.pdf}{Exercice 35}}%
\hfill{\tiny 6582}%
\begingroup~

Existe-t-il une matrice $B \in\M M3\R$ telle que
\[ B^2 = \begin{pmatrix} 0 & 1 & 0 \\ 0 & 0 & 1 \\ 0 & 0 & 0 \end{pmatrix} ? \]
\endgroup

% -----------------------------------------------------------------------------
\par\pagebreak[1]\par
\paragraph{Exercice 36}%
\hfill{\tiny 8760}%
\begingroup~

Soit $f$ et $g$ deux endomorphismes d'un $\K$-espace vectoriel $E$
tels que $f\circ g = Id_E$.

\begin{enumerate}
\item Montrer que
  $\Ker(g\circ f) = \Ker(f)$,
  $\Ima(g\circ f) = \Ima(g)$ et
  $E = \Ker(f) \oplus{} \Ima(g)$.
\item Dans quels cas peut-on conclure $g = f^{-1}$?
\item Calculer $(g\circ f)\circ(g\circ f)$ et caract\'eriser $g\circ f$.
\end{enumerate}
\endgroup

% -----------------------------------------------------------------------------
\par\pagebreak[1]\par
\paragraph{Exercice 37}%
\hfill{\tiny 1419}%
\begingroup~

Soit $A = \begin{pmatrix} 1 & -1 & 0 & 0 \\ 0 & 1 & 0 & 0 \\ 0 & 0 & -1 & 1 \\ 0 & 0 & 0 & -1 \end{pmatrix}$.
Calculer $A^n$ pour tout $n \in \Z$.
\endgroup

% -----------------------------------------------------------------------------
\par\pagebreak[1]\par
\paragraph{\href{https://psi.miomio.fr/exo/8120.pdf}{Exercice 38}}%
\hfill{\tiny 8120}%
\begingroup~

Soit $E = \R_3[X]$ et $\B$ la base canonique.
Soit $f\in\LE$ d\'efini par $f(P) = P(X+1)$.
\begin{enumerate}
\item D\'eterminer $A = \Mat_\B f$.
\item Montrer que $A$ est inversible et calculer $A^{-1}$.
\item Reprendre les questions pr\'ec\'edentes pour $E = \R_n[X]$.
\end{enumerate}
\endgroup

% -----------------------------------------------------------------------------
\par\pagebreak[1]\par
\paragraph{Exercice 39}%
\hfill{\tiny 3425}%
\begingroup~

Soit $A = \begin{pmatrix} -2 & 1 & 1 \\ 8 & 1 & -5 \\ 4 & 3 & -3 \end{pmatrix}$
et $C = \begin{pmatrix} 1 & 2 & -1 \\ 2 & -1 & -1 \\ -5 & 0 & 3 \end{pmatrix}$.

Existe-t-il une matrice $B$ telle que $A=BC$?
\endgroup

% -----------------------------------------------------------------------------
\par\pagebreak[1]\par
\paragraph{\href{https://psi.miomio.fr/exo/8026.pdf}{Exercice 40}}%
\hfill{\tiny 8026}%
\begingroup~

Soit $E$ un $\K$-espace vectoriel de dimension finie tel que $f\circ f = -\Id_E$.
Montrer qu'il existe une base $\B$ telle que la matrice de $f$ dans cette base soit diagonale par blocs, de la forme $\Mat_\B f = \mathrm{diag}(A,A,\dots,A)$ o\`u $A = \begin{pmatrix} 0 & -1 \\ 1 & 0 \end{pmatrix}$.
\endgroup

% -----------------------------------------------------------------------------
\par\pagebreak[1]\par
\paragraph{Exercice 41 (matrices \`a diagonale dominante)}%
\hfill{\tiny 2448}%
\begingroup~

Soit $A\in\MnC$ telle que
\[ \forall i\in\Dcro{1,n} \+ \sum_{\substack{1\leq j\leq n \\ j\neq i}} \Abs{a_{i,j}} < \Abs{a_{i,i}}. \]
Montrer que $A$ est inversible. On pourra montrer que $\Ker A = \Acco{0}$.
\endgroup

% -----------------------------------------------------------------------------
\par\pagebreak[1]\par
\paragraph{Exercice 42}%
\hfill{\tiny 2264}%
\begingroup~

Soit $a\in \R$.
Pour $n\in \N^*$, on d\'efinit le d\'eterminant d'ordre $n$
\[ \Delta_n = \begin{vmatrix}
    a &  1 &  0 &  \cdots &  0  \\
    1 &  a &  1 &  \ddots &  \vdots  \\
    0 &  1 &  a &  \ddots &  0  \\
    \vdots &  \ddots &  \ddots &  \ddots &  1  \\
0 &  \cdots &  0 &  1 &  a \end{vmatrix} \]
\begin{enumerate}
\item D\'eterminer une relation de r\'ecurrence lin\'eaire d'ordre 2 v\'erifi\'ee par la suite $(\Delta_n)$.
\item On suppose $a = 2\cos\theta$ o\`u $\theta\in\intO{0,\pi}$.
  Montrer que \[ \Delta_n = \frac{\sin\bigl((n+1)\theta\bigr)}{\sin\theta}. \]
\end{enumerate}
\endgroup

% -----------------------------------------------------------------------------
\par\pagebreak[1]\par
\paragraph{Exercice 43}%
\hfill{\tiny 2500}%
\begingroup~

Soit $A$, $B$, $C$, $D$ des matrices carr\'ees de $\MnK$.
On suppose que $CD = DC$ et que $D$ est inversible.
\begin{enumerate}
\item
  En consid\'erant le produit par blocs
  \[ \begin{pmatrix} I_n & -BD^{-1} \\ 0 & I_n \end{pmatrix} \begin{pmatrix} A & B \\ C & D \end{pmatrix} \]
  montrer que
  \[ \det \begin{pmatrix} A & B \\ C & D \end{pmatrix} = \det(AD-BC). \]
\item Montrer que cette derni\`ere formule est encore vraie si l'on ne suppose plus que $D$ est inversible.
\item Trouver un contre-exemple si l'on ne suppose plus que $C$ et $D$ commutent.
\end{enumerate}
\endgroup

% -----------------------------------------------------------------------------
\par\pagebreak[1]\par
\paragraph{Exercice 44}%
\hfill{\tiny 5496}%
\begingroup~

Calculer $\begin{vmatrix} a & b & c \\ a^2 & b^2 & c^2 \\ a^3 & b^3 & c^3 \end{vmatrix}$.

En d\'eduire $\begin{vmatrix} a+b & b+c & c+a \\ a^2+b^2 & b^2+c^2 & c^2+a^2 \\ a^3+b^3 & b^3+c^3 & c^3+a^3 \end{vmatrix}$.
\endgroup

\end{document}
