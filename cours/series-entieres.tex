% autogenerated by ytex.rs

\documentclass{scrartcl}

\usepackage[francais]{babel}
\usepackage{geometry}
\usepackage{scrpage2}
\usepackage{lastpage}
\usepackage{ragged2e}
\usepackage{multicol}
\usepackage{etoolbox}
\usepackage{xparse}
\usepackage{enumitem}
% \usepackage{csquotes}
\usepackage{amsmath}
\usepackage{amsfonts}
\usepackage{amssymb}
\usepackage{mathrsfs}
\usepackage{stmaryrd}
\usepackage{dsfont}
\usepackage{eurosym}
% \usepackage{numprint}
% \usepackage[most]{tcolorbox}
\usepackage{tikz}
% \usepackage{tkz-tab}
\usepackage[unicode]{hyperref}
\usepackage[ocgcolorlinks]{ocgx2}

\let\ifTwoColumns\iftrue
\def\Classe{$\Psi$2019--2020}

% reproducible builds
% LuaTeX: \pdfvariable suppressoptionalinfo 1023 \relax
\pdfinfoomitdate=1
\pdftrailerid{}

\newif\ifDisplaystyle
\everymath\expandafter{\the\everymath\ifDisplaystyle\displaystyle\fi}
\newcommand\DS{\displaystyle}

\clearscrheadfoot
\pagestyle{scrheadings}
\thispagestyle{empty}
\ohead{\Classe}
\ihead{\thepage/\pageref*{LastPage}}

\setlist[itemize,1]{label=\textbullet}
\setlist[itemize,2]{label=\textbullet}

\ifTwoColumns
  \geometry{margin=1cm,top=2cm,bottom=3cm,foot=1cm}
  \setlist[enumerate]{leftmargin=*}
  \setlist[itemize]{leftmargin=*}
\else
  \geometry{margin=3cm}
\fi

\makeatletter
\let\@author=\relax
\let\@date=\relax
\renewcommand\maketitle{%
    \begin{center}%
        {\sffamily\huge\bfseries\@title}%
        \ifx\@author\relax\else\par\medskip{\itshape\Large\@author}\fi
        \ifx\@date\relax\else\par\bigskip{\large\@date}\fi
    \end{center}\bigskip
    \ifTwoColumns
        \par\begin{multicols*}{2}%
        \AtEndDocument{\end{multicols*}}%
        \setlength{\columnsep}{5mm}
    \fi
}
\makeatother

\newcounter{ParaNum}
\NewDocumentCommand\Para{smo}{%
  \IfBooleanF{#1}{\refstepcounter{ParaNum}}%
  \paragraph{\IfBooleanF{#1}{{\tiny\arabic{ParaNum}~}}#2\IfNoValueF{#3}{ (#3)}}}

\newcommand\I{i}
\newcommand\mi{i}
\def\me{e}

\def\do#1{\expandafter\undef\csname #1\endcsname}
\docsvlist{Ker,sec,csc,cot,sinh,cosh,tanh,coth,th}
\undef\do

\DeclareMathOperator\ch{ch}
\DeclareMathOperator\sh{sh}
\DeclareMathOperator\th{th}
\DeclareMathOperator\coth{coth}
\DeclareMathOperator\cotan{cotan}
\DeclareMathOperator\argch{argch}
\DeclareMathOperator\argsh{argsh}
\DeclareMathOperator\argth{argth}

\let\epsilon=\varepsilon
\let\phi=\varphi
\let\leq=\leqslant
\let\geq=\geqslant
\let\subsetneq=\varsubsetneq
\let\emptyset=\varnothing

\newcommand{\+}{,\;}

\undef\C
\newcommand\ninf{{n\infty}}
\newcommand\N{\mathbb{N}}
\newcommand\Z{\mathbb{Z}}
\newcommand\Q{\mathbb{Q}}
\newcommand\R{\mathbb{R}}
\newcommand\C{\mathbb{C}}
\newcommand\K{\mathbb{K}}
\newcommand\Ns{\N^*}
\newcommand\Zs{\Z^*}
\newcommand\Qs{\Q^*}
\newcommand\Rs{\R^*}
\newcommand\Cs{\C^*}
\newcommand\Ks{\K^*}
\newcommand\Rp{\R^+}
\newcommand\Rps{\R^+_*}
\newcommand\Rms{\R^-_*}
\newcommand{\Rpinf}{\Rp\cup\Acco{+\infty}}

\undef\B
\newcommand\B{\mathscr{B}}

\undef\P
\DeclareMathOperator\P{\mathbb{P}}
\DeclareMathOperator\E{\mathbb{E}}
\DeclareMathOperator\Var{\mathbb{V}}

\DeclareMathOperator*\PetitO{o}
\DeclareMathOperator*\GrandO{O}
\DeclareMathOperator*\Sim{\sim}
\DeclareMathOperator\Tr{tr}
\DeclareMathOperator\Ima{Im}
\DeclareMathOperator\Ker{Ker}
\DeclareMathOperator\Sp{Sp}
\DeclareMathOperator\Diag{diag}
\DeclareMathOperator\Rang{rang}
\DeclareMathOperator*\Coords{Coords}
\DeclareMathOperator*\Mat{Mat}
\DeclareMathOperator\Pass{Pass}
\DeclareMathOperator\Com{Com}
\DeclareMathOperator\Card{Card}
\DeclareMathOperator\Racines{Racines}
\DeclareMathOperator\Vect{Vect}
\DeclareMathOperator\Id{Id}

\newcommand\DerPart[2]{\frac{\partial #1}{\partial #2}}

\def\T#1{{#1}^T}

\def\pa#1{({#1})}
\def\Pa#1{\left({#1}\right)}
\def\bigPa#1{\bigl({#1}\bigr)}
\def\BigPa#1{\Bigl({#1}\Bigr)}
\def\biggPa#1{\biggl({#1}\biggr)}
\def\BiggPa#1{\Biggl({#1}\Biggr)}

\def\pafrac#1#2{\pa{\frac{#1}{#2}}}
\def\Pafrac#1#2{\Pa{\frac{#1}{#2}}}
\def\bigPafrac#1#2{\bigPa{\frac{#1}{#2}}}
\def\BigPafrac#1#2{\BigPa{\frac{#1}{#2}}}
\def\biggPafrac#1#2{\biggPa{\frac{#1}{#2}}}
\def\BiggPafrac#1#2{\BiggPa{\frac{#1}{#2}}}

\def\cro#1{[{#1}]}
\def\Cro#1{\left[{#1}\right]}
\def\bigCro#1{\bigl[{#1}\bigr]}
\def\BigCro#1{\Bigl[{#1}\Bigr]}
\def\biggCro#1{\biggl[{#1}\biggr]}
\def\BiggCro#1{\Biggl[{#1}\Biggr]}

\def\abs#1{\mathopen|{#1}\mathclose|}
\def\Abs#1{\left|{#1}\right|}
\def\bigAbs#1{\bigl|{#1}\bigr|}
\def\BigAbs#1{\Bigl|{#1}\Bigr|}
\def\biggAbs#1{\biggl|{#1}\biggr|}
\def\BiggAbs#1{\Biggl|{#1}\Biggr|}

\def\acco#1{\{{#1}\}}
\def\Acco#1{\left\{{#1}\right\}}
\def\bigAcco#1{\bigl\{{#1}\bigr\}}
\def\BigAcco#1{\Bigl\{{#1}\Bigr\}}
\def\biggAcco#1{\biggl\{{#1}\biggr\}}
\def\BiggAcco#1{\Biggl\{{#1}\Biggr\}}

\def\ccro#1{\llbracket{#1}\rrbracket}
\def\Dcro#1{\llbracket{#1}\rrbracket}

\def\floor#1{\lfloor#1\rfloor}
\def\Floor#1{\left\lfloor{#1}\right\rfloor}

\def\sEnsemble#1#2{\mathopen\{#1\mid#2\mathclose\}}
\def\bigEnsemble#1#2{\bigl\{#1\bigm|#2\bigr\}}
\def\BigEnsemble#1#2{\Bigl\{#1\Bigm|#2\Bigr\}}
\def\biggEnsemble#1#2{\biggl\{#1\biggm|#2\biggr\}}
\def\BiggEnsemble#1#2{\Biggl\{#1\Biggm|#2\Biggr\}}
\let\Ensemble=\bigEnsemble

\newcommand\IntO[1]{\left]#1\right[}
\newcommand\IntF[1]{\left[#1\right]}
\newcommand\IntOF[1]{\left]#1\right]}
\newcommand\IntFO[1]{\left[#1\right[}

\newcommand\intO[1]{\mathopen]#1\mathclose[}
\newcommand\intF[1]{\mathopen[#1\mathclose]}
\newcommand\intOF[1]{\mathopen]#1\mathclose]}
\newcommand\intFO[1]{\mathopen[#1\mathclose[}

\newcommand\Fn[3]{#1\colon#2\to#3}
\newcommand\CC[1]{\mathscr{C}^{#1}}
\newcommand\D{\mathop{}\!\mathrm{d}}

\newcommand\longto{\longrightarrow}

\undef\M
\newcommand\M[3]{\mathrm{#1}_{#2}\pa{#3}}
\newcommand\MnR{\M{M}{n}{\R}}
\newcommand\MnC{\M{M}{n}{\C}}
\newcommand\MnK{\M{M}{n}{\K}}
\newcommand\GLnR{\M{GL}{n}{\R}}
\newcommand\GLnC{\M{GL}{n}{\C}}
\newcommand\GLnK{\M{GL}{n}{\K}}
\newcommand\DnR{\M{D}{n}{\R}}
\newcommand\DnC{\M{D}{n}{\C}}
\newcommand\DnK{\M{D}{n}{\K}}
\newcommand\SnR{\M{S}{n}{\R}}
\newcommand\AnR{\M{A}{n}{\R}}
\newcommand\OnR{\M{O}{n}{\R}}
\newcommand\SnRp{\mathrm{S}_n^+(\R)}
\newcommand\SnRpp{\mathrm{S}_n^{++}(\R)}

\newcommand\LE{\mathscr{L}(E)}
\newcommand\GLE{\mathscr{GL}(E)}
\newcommand\SE{\mathscr{S}(E)}
\renewcommand\OE{\mathscr{O}(E)}

\newcommand\ImplD{$\Cro\Rightarrow$}
\newcommand\ImplR{$\Cro\Leftarrow$}
\newcommand\InclD{$\Cro\subset$}
\newcommand\InclR{$\Cro\supset$}
\newcommand\notInclD{$\Cro{\not\subset}$}
\newcommand\notInclR{$\Cro{\not\supset}$}

\newcommand\To[1]{\xrightarrow[#1]{}}
\newcommand\Toninf{\To{\ninf}}

\newcommand\Norm[1]{\|#1\|}
\newcommand\Norme{{\Norm{\cdot}}}

\newcommand\Int[1]{\mathring{#1}}
\newcommand\Adh[1]{\overline{#1}}

\newcommand\Uplet[2]{{#1},\dots,{#2}}
\newcommand\nUplet[3]{(\Uplet{{#1}_{#2}}{{#1}_{#3}})}

\newcommand\Fonction[5]{{#1}\left|\begin{aligned}{#2}&\;\longto\;{#3}\\{#4}&\;\longmapsto\;{#5}\end{aligned}\right.}

\DeclareMathOperator\orth{\bot}
\newcommand\Orth[1]{{#1}^\bot}
\newcommand\PS[2]{\langle#1,#2\rangle}

\newcommand{\Tribu}{\mathscr{T}}
\newcommand{\Part}{\mathcal{P}}
\newcommand{\Pro}{\bigPa{\Omega,\Tribu}}
\newcommand{\Prob}{\bigPa{\Omega,\Tribu,\P}}

\newcommand\DEMO{$\spadesuit$}
\newcommand\DUR{$\spadesuit$}

\newenvironment{psmallmatrix}{\left(\begin{smallmatrix}}{\end{smallmatrix}\right)}

% -----------------------------------------------------------------------------

\usepackage{tikz}

\newcommand{\sumni}{\sum_{n=0}^{+\infty}}
\newcommand{\Sanzn}{\sum_n a_n z^n}
\newcommand{\Sanxn}{\sum_n a_n x^n}
\newcommand{\Sbnzn}{\sum_n b_n z^n}
\newcommand{\Scnzn}{\sum_n c_n z^n}
\newcommand{\Ensembletq}[2]{\bigl\{#1\text{ tq }#2\bigr\}}

\begin{document}
\title{S\'eries enti\`eres}
\maketitle

% -----------------------------------------------------------------------------
\section{Rayon de convergence}

\Para{D\'efinition}

On appelle \emph{s\'erie enti\`ere} toute s\'erie de fonctions de la forme $\sum_n f_n$
o\`u $(a_n)_{n\in \C}$ est une suite num\'erique et
o\`u $\Fn{f_n}{\C}{\C}$ est d\'efinie par $f_n(z) = a_n z^n$.
La \emph{somme} de la s\'erie enti\`ere est la fonction \[ f \colon z \mapsto \sumni a_n z^n. \]

\Para{Lemme d'Abel}

Soit $\Sanzn$ une s\'erie enti\`ere et $z_0\in \C$ telle que la suite num\'erique $(a_n z_0^n)_{n\in \N}$ soit born\'ee.
Alors la s\'erie $\Sanzn$ converge absolument pour tout $z\in \C$ tel que $\Abs{z} < \Abs{z_0}$.

\Para{D\'efinition}

Soit $\Sanzn$ une s\'erie enti\`ere.
Le \emph{rayon de convergence} $R\in\Rpinf$
de la s\'erie enti\`ere $\Sanzn$ est d\'efini par
\begin{multline*}
  R = \sup \, \Bigl\{ \, r\geq0 \text{ tel que} \\
  \text{la suite de terme g\'en\'eral } \Abs{a_n} r^n \text{ est born\'ee} \, \Bigr\}.
\end{multline*}

\Para{Corollaire}

Soit $\Sanzn$ une s\'erie enti\`ere de rayon de convergence $R$ et $z_0 \in{} \C$.
\begin{itemize}
\item
  Si $\Abs{z_0} < R$,
  la s\'erie num\'erique $\sum_n a_n z_0^n$ converge absolument.
\item
  Si $\Abs{z_0} > R$,
  la s\'erie num\'erique $\sum_n a_n z_0^n$ diverge grossi\`erement;
  plus pr\'ecis\'ement, la suite $(a_n z_0^n)_{n\in \N}$ n'est pas born\'ee.
\item
  Si $\Abs{z_0} = R$, la s\'erie peut ou non converger.
\end{itemize}

\Para{D\'efinitions}

Soit $\Sanzn$ une s\'erie enti\`ere de rayon de convergence $R$.
\begin{itemize}
\item
  On appelle \emph{disque ouvert de convergence} de la s\'erie enti\`ere $\Sanzn$ l'ensemble
  \[ \Ensembletq{z\in \C}{\Abs{z}<R}. \]
\item
  On appelle \emph{cercle d'incertitude}
  (et parfois aussi, malheureusement, cercle de convergence)
  de la s\'erie enti\`ere $\Sanzn$ l'ensemble
  \[ \Ensembletq{z\in \C}{\Abs{z}=R}. \]
\end{itemize}

\begin{tikzpicture}[scale=2]
  \filldraw[thick,fill=gray] (0,0) circle (1) ;
  \draw[->] (-2,0) -- (2,0) ;
  \draw[->] (0,-1.5) -- (0,1.5) ;
  \draw (1,0) node {$\bullet$} node[above right] {$R$} ;
\end{tikzpicture}

% -----------------------------------------------------------------------------
\section{D\'etermination pratique du rayon de convergence}

\subsection{Avec le lemme d'Abel}

\Para{Proposition}

Soit $\Sanzn$ une s\'erie enti\`ere de rayon de convergence $R$.
Soit $z_0\in \C$.
\begin{itemize}
\item
  Si $(a_n z_0^n)$ est born\'ee, alors $R\geq\Abs{z_0}$.
\item
  Si $\sum_n \Abs{a_n z_0^n}$ diverge, alors $R\leq\Abs{z_0}$.
\end{itemize}

\subsection{Avec la r\`egle de d'Alembert}

On prend $r > 0$ et on applique le crit\`ere de d'Alembert pour d\'eterminer
pour quelles valeurs de $r$ la s\'erie $\sum_n a_n r^n$ converge.

\subsection{Avec les r\`egles de comparaison}

\Para{Th\'eor\`eme}

Soit $\Sanzn$ et $\Sbnzn$ deux s\'eries enti\`eres de rayons de convergence
respectivement $R_a$ et $R_b$.
\begin{enumerate}
\item
  S'il existe $\alpha\in \R$ tel que $\Abs{a_n} = \GrandO \left( n^\alpha\Abs{b_n} \right)$ quand $n\to+\infty$, alors $R_a\geq R_b$.
\item
  S'il existe $\alpha\in \R$ et $K>0$ tel que $\Abs{a_n} \sim K n^\alpha\Abs{b_n}$ quand $n\to+\infty$, alors $R_a = R_b$.
\end{enumerate}

\pagebreak

% -----------------------------------------------------------------------------
\section{Op\'erations sur les s\'eries enti\`eres}

\Para{Notations}
\begin{itemize}
\item
  \'Etant donn\'e une s\'erie enti\`ere $\Sanzn$, on note $R_a$ son rayon de convergence et $f_a$ sa somme.
\item
  De m\^eme pour $\Sbnzn$ et $\Scnzn$.
\end{itemize}

\subsection{Somme}

\Para{D\'efinition}

Soit $\Sanzn$ et $\Sbnzn$ deux s\'eries enti\`eres.
On d\'efinit la \emph{somme} de ces deux s\'eries enti\`eres
comme \'etant la s\'erie enti\`ere $\Scnzn$ o\`u
\[ \forall n\in \N, \quad c_n = a_n+b_n. \]

\Para{Proposition}

Dans ces conditions,
\begin{itemize}
\item
  $R_c\geq\min(R_a,R_b)$ avec \'egalit\'e si $R_a\neq R_b$,
\item
  pour tout $z\in \C$ tel que $\Abs{z} < \min(R_a,R_b)$, on a $f_c(z) = f_a(z) + f_b(z)$.
\end{itemize}

\subsection{Multiplication par un scalaire}

\Para{D\'efinition}

Soit $\lambda$ un scalaire et $\Sanzn$ une s\'erie enti\`ere.
On d\'efinit le \emph{produit} du scalaire et de la s\'erie enti\`ere
comme \'etant la s\'erie enti\`ere $\Sbnzn$ o\`u
\[ \forall n\in \N, \quad b_n = \lambda a_n. \]

\Para{Proposition}

Dans ces conditions,
\begin{itemize}
\item
  $R_b = R_a$ si $\lambda\neq0$, et $R_b = +\infty$ si $\lambda{} = 0$,
\item
  pour tout $z\in \C$ tel que $\Abs{z} < R_a$, on a $f_b(z) = \lambda f_a(z)$.
\end{itemize}

\subsection{Produit de Cauchy}

\Para{D\'efinition}

Soit $\Sanzn$ et $\Sbnzn$ deux s\'eries enti\`eres.
On d\'efinit le \emph{produit de Cauchy} de ces deux s\'eries enti\`eres
comme \'etant la s\'erie enti\`ere $\Scnzn$ o\`u
\[ \forall n\in \N, \quad c_n = \sum_{k=0}^n a_k b_{n-k}. \]

\Para{Proposition}

Dans ces conditions,
\begin{itemize}
\item
  $R_c \geq{} \min(R_a,R_b)$,
\item
  pour tout $z\in \C$ tel que $\Abs{z} < \min(R_a,R_b)$, on a $f_c(z) = f_a(z) f_b(z)$.
\end{itemize}

% -----------------------------------------------------------------------------
\section{Propri\'et\'es de la somme}

\Para{Lemme d'Abel (bis)}

Soit $\Sanzn$ une s\'erie enti\`ere de rayon de convergence $R$.
Si $0 \leq{} r < R$, la s\'erie de fonctions converge normalement
sur $\Ensembletq{z\in \C}{\Abs{z}\leq r}$,
et a fortiori sur $\intF{-r, r}$.

Autrement dit, la s\'erie enti\`ere converge normalement sur tout compact $K$ inclus dans le disque ouvert de convergence.

\Para{Attention!}

Il n'y a pas toujours convergence normale (ni m\^eme uniforme) sur $\Ensembletq{z\in \C}{\Abs{z}<R}$,
ni sur $\intO{-R, R}$.

\Para{Th\'eor\`eme}[continuit\'e sur le disque ouvert de convergence]

Soit $\Sanzn$ une s\'erie enti\`ere de rayon de convergence $R$ et de somme $f$.
Alors $f$ est continue sur le disque ouvert de convergence
\[ \Ensembletq{z\in \C}{\Abs{z}<R}. \]

\Para{Lemme}

Soit $p\in \Z$.
Les s\'eries enti\`eres $\sum_n a_n z^n$ et $\sum_n a_{n+p} \, z^n$
ont m\^eme rayon de convergence.

\Para{Proposition}

Soit $\Sanzn$ une s\'erie enti\`ere de rayon de convergence $R$.
Alors les deux s\'eries enti\`eres suivantes:
\begin{itemize}
\item
  la \og d\'eriv\'ee formelle\fg{} $\sum_n b_n z^n$ o\`u $b_n = (n+1)a_{n+1}$,
\item
  la \og primitive formelle\fg{} $\sum_n c_n z^n$ o\`u $c_n = \frac{a_{n-1}}{n}$
\end{itemize}
ont \'egalement un rayon de convergence \'egal \`a $R$.

\Para{Th\'eor\`eme}[primitivation terme \`a terme]

Soit $\Sanxn$ une s\'erie enti\`ere de rayon de convergence $R$ et de somme $f$.
Une primitive de $f$ sur $\intO{-R,R}$ est donn\'ee par
\[ F \colon x \mapsto \sumni a_n \frac{x^{n+1}}{n+1} = \sum_{n=1}^{+\infty} \frac{a_{n-1}}{n} x^n. \]

\Para{Proposition}[d\'erivation terme \`a terme]

Soit $\Sanxn$ une s\'erie enti\`ere de rayon de convergence $R$ et de somme $f$.
Alors $f$ est une fonction de classe $\CC1$ sur $\intO{-R,R}$
et $\forall x\in\intO{-R,R}$,
\[ f'(x) = \sum_{n=1}^{+\infty} n a_n x^{n-1} = \sumni (n+1) a_{n+1} \, x^n. \]

\Para{Th\'eor\`eme}[g\'en\'eralisation]

Soit $\Sanxn$ une s\'erie enti\`ere de rayon de convergence $R$ et de somme $f$.
Alors $f$ est une fonction de classe $\CC\infty$ sur $\intO{-R,R}$
et $\forall p\in \N$, $\forall x\in\intO{-R,R}$,
\[ \begin{aligned} f^{(p)}(x)
    &= \sum_{n=p}^{+\infty} a_n n(n-1) \cdots (n-p+1) x^{n-p} \\
    &= \sumni a_{n+p} \frac{(n+p)!}{n!} x^n.
\end{aligned} \]

\Para{Corollaire}

Soit $\Sanxn$ une s\'erie enti\`ere de rayon de convergence $R>0$ et de somme $f$.
Alors pour tout $n\in \N$, on a
\[ a_n = \frac{f^{(n)}(0)}{n!}. \]

% -----------------------------------------------------------------------------
\section{Fonctions d\'eveloppables en s\'eries enti\`eres}

\Para{Notation}
\begin{itemize}
\item
  $I$ d\'esigne un intervalle de $\R$ tel qu'il existe
  $\epsilon> 0$ tel que $\intO{-\epsilon,\epsilon}\subset I$.
\end{itemize}

\Para{D\'efinition}

Soit $\Fn fI\C$.
On dit que $f$ est \emph{d\'eveloppable en s\'erie enti\`ere} au voisinage de $0$
si et seulement s'il existe $\alpha> 0$
et une s\'erie enti\`ere $\Sanxn$ de rayon de convergence $R\geq \alpha$ tels que
$\intO{-\alpha,\alpha}\subset I$ et
\[ \forall x\in\intO{-\alpha,\alpha} \+ f(x) = \sumni a_n x^n. \]

\Para{Proposition}

Si $\Fn fI\C$ est d\'eveloppable en s\'erie enti\`ere au voisinage de $0$,
alors il existe $\alpha> 0$ tel que
$f$ soit de classe $\CC\infty$ sur $\intO{-\alpha,\alpha}$.

\Para{D\'efinition}

Soit $\Fn fI\C$ une fonction de classe $\CC\infty$.
On appelle \emph{s\'erie de Taylor} de $f$ au voisinage de $0$
la s\'erie enti\`ere
\[ \sum_n a_n z^n \quad \text{o\`u} \quad\forall n\in \N\+ a_n = \frac{f^{(n)}(0)}{n!}. \]

\Para{Th\'eor\`eme}[unicit\'e du d\'eveloppement en s\'erie enti\`ere]

Soit $\Fn fI\C$ d\'eveloppable en s\'erie enti\`ere au voisinage de $0$.
Alors tout d\'eveloppement en s\'erie enti\`ere de $f$ au voisinage de $0$
est \'egal \`a sa s\'erie de Taylor au voisinage de $0$.

\Para{Corollaire}

Soit $\Fn fI\C$ d\'eveloppable en s\'erie enti\`ere au voisinage de $0$.
Alors $f$ admet un d\'eveloppement limit\'e au voisinage de $0$ \`a tout ordre,
et ce d\'eveloppement limit\'e s'obtient en tronquant le d\'eveloppement en s\'erie enti\`ere.

\Para{Proposition}

Soit $\Fn fI\C$.
Alors $f$ est d\'eveloppable en s\'erie enti\`ere au voisinage de $0$
si et seulement si il existe $\alpha> 0$ tel que
\begin{itemize}
\item
  $f$ est de classe $\CC\infty$ sur $\intO{-\alpha,\alpha}$,
\item
  la s\'erie de Taylor de $f$ converge sur $\intO{-\alpha,\alpha}$,
\item
  $f$ soit \'egale \`a la somme de sa s\'erie de Taylor sur $\intO{-\alpha,\alpha}$.
\end{itemize}

% -----------------------------------------------------------------------------
\section{Fonctions usuelles}

\subsection{Exponentielle complexe}

\Para{Proposition}

La s\'erie enti\`ere $\sum_n \frac{z^n}{n!}$ a un rayon de convergence infini.
Notons $\varphi\colon z \mapsto \sumni \frac{z^n}{n!}$ sa somme.
On montre successivement:
\begin{enumerate}
\item
  $\forall(z_1,z_2)\in \C^2$, $\varphi(z_1+z_2) =\varphi(z_1)\varphi(z_2)$
\item
  $\forall x\in \R$, $\varphi(x) = \me^x$
\item
  $\forall \theta\in \R$, $\varphi(\I\theta) = \cos\theta+ \I \sin\theta$
\item
  $\forall(x,y)\in \R^2$, $\varphi(x+\I y) = \me^x(\cos y + \I \sin y)$
\end{enumerate}

\Para{D\'efinition}

On appelle \emph{exponentielle} la fonction d\'efinie par
\[ \forall z\in \C, \quad \exp(z) = \sumni \frac{z^n}{n!}. \]
Il s'agit d'une s\'erie enti\`ere de rayon de convergence infini.
On note fr\'equemment $\me^z$ au lieu de $\exp(z)$.

\Para{Proposition}

Soit $z\in \C$ fix\'e et $\Fonction{\varphi}{\R}{\C}{t}{\exp(tz)}$\\
Alors $\varphi$ est de classe $\CC\infty$ et
\[ \forall n\in \N{} \+ \forall t\in \R, \quad \varphi^{(n)}(t) = z^n \exp(tz). \]

\subsection{Trigonom\'etrie}

\Para{D\'efinitions}

On d\'efinit, pour tout $z\in \C$:
\begin{itemize}
\item
  $\cos z = \frac{\me^{\I z} + \me^{-\I z}}{2}$
\item
  $\sin z = \frac{\me^{\I z} - \me^{-\I z}}{2\I}$
\item
  $\ch z = \frac{\me^{z} + \me^{-z}}{2}$
\item
  $\sh z = \frac{\me^{z} - \me^{-z}}{2}$
\end{itemize}
Il s'agit de prolongement des fonctions cosinus, sinus, cosinus hyperbolique
et sinus hyperbolique, classiquement d\'efinies seulement sur $\R$.

\Para{Proposition}

\begin{itemize}
\item
  Les formules usuelles de trigonom\'etries restent valables,
  notamment $\forall(a,b)\in \C^2$:
  \begin{itemize}
  \item
    $\cos(a+b) = \cos(a) \cos(b) - \sin(a) \sin(b)$
  \item
    $\sin(a+b) = \sin(a) \cos(b) + \cos(a) \sin(b)$
  \end{itemize}
\item
  On peut passer de la trigonom\'etrie directe
  \`a la trigonom\'etrie hyperbolique sachant que $\forall z\in \C$:
  \begin{itemize}
  \item
    $\ch(\I z) = \cos(z)$
  \item
    $\sh(\I z) = \I\sin(z)$
  \item
    $\cos(\I z) = \ch(z)$
  \item
    $\sin(\I z) = \I\sh(z)$
  \end{itemize}
\end{itemize}

\Para{Proposition}

On a, pour tout $z\in \C$:
\begin{itemize}
\item
  $\DS \cos z  = \sum_{n=0}^{+\infty} \frac{(-1)^n}{(2n)!} \, z^{2n}$
\item
  $\DS \sin z  = \sum_{n=0}^{+\infty} \frac{(-1)^n}{(2n+1)!} \, z^{2n+1}$
\item
  $\DS \ch z = \sum_{n=0}^{+\infty} \frac{1}{(2n)!} \, z^{2n}$
\item
  $\DS \sh z = \sum_{n=0}^{+\infty} \frac{1}{(2n+1)!} \, z^{2n+1}$
\end{itemize}

Il s'agit de s\'eries enti\`eres de rayon de convergence infini.

\subsection{$x \mapsto (1+x)^\alpha$}

\Para{Proposition}

Soit $\alpha\in \C$ et $\Fn{f}{\intO{-1,+\infty}}{\R}$ d\'efinie par $f(x)=(1+x)^\alpha$.
Alors $f$ est d\'eveloppable en s\'erie enti\`ere au voisinage de $0$.
Plus pr\'ecis\'ement, on a
\[ \forall x\in\intO{-1,1}, \quad (1+x)^\alpha= \sum_{n=0}^{+\infty} a_n x^n \]
o\`u $a_0 = 1$ et \[ \forall n\in\Ns \+ a_n = \frac{1}{n!}\prod_{k=0}^{n-1} (\alpha-k). \]
On note parfois $a_n = \binom{\alpha}{n}$.
Il s'agit d'une s\'erie enti\`ere de rayon de convergence \'egal \`a~1
si $\alpha\notin \N$, et $+\infty$ si $\alpha\in \N$.

\Para{Remarque}

Si $\alpha\in \Z$, le d\'eveloppement est \'egalement valable dans $\C$
\[ \forall z\in \C\text{ tel que } \Abs{z}<1, \quad
(1+z)^\alpha=\sum_{n=0}^{+\infty} a_n z^n, \]
o\`u $(a_n)_{n\in \N}$ est d\'efini comme ci-dessus.

\subsection{Divers}

\Para{Proposition}
Pour tout $z \in{} \C$ tel que $\abs{z} < 1$, on a
\[ \frac{1}{1-z} = \sum_{n=0}^{+\infty} z^n, \]
et il s'agit d'une s\'erie enti\`ere de rayon de convergence 1.

\Para{Proposition}
On a,
\[
  \forall{} x \in{} \intO{-1,1} \+
  \arctan(x) = \sum_{n=0}^{+\infty} \frac{(-1)^n}{2n+1} \, x^{2n+1}.
\]
Le rayon de cette s\'erie enti\`ere vaut~1.

\Para{Remarque}
La formule ci-dessus est encore valable pour $x = \pm1$,
mais c'est plus d\'elicat \`a prouver,
cf. exercices~7 et~8.

\Para{Proposition}
Pour tout $x \in{} \intO{-1,1}$, on a
\[ \ln(1+x) = \sum_{n=1}^{+\infty} \frac{(-1)^{n-1}}{n} x^n, \]
et il s'agit d'une s\'erie enti\`ere de rayon de convergence 1.

\Para{Remarque}
La formule ci-dessus est encore valable pour $x = 1$,
ce que l'on peut montrer par la m\^eme technique que celle de l'exercice~8.

\subsection{Fractions rationnelles}

\Para{M\'ethode}

Pour obtenir un d\'eveloppement en s\'erie enti\`ere (au voisinage de $0$)
d'une fraction rationnelle, on peut:
\begin{itemize}
\item
  la d\'ecomposer en \'el\'ements simples $\frac{1}{(x-a)^n}$ o\`u $a\in \C^*$;
\item
  remarquer que $\frac{1}{(x-a)^n} = (-a)^{-n} \left[ 1+\left(-\frac xa\right) \right]^{-n}$;
\item
  se ramener \`a $(1+u)^\alpha$.
\end{itemize}

On obtient une s\'erie enti\`ere de rayon de convergence~$\Abs{a}$.

% -----------------------------------------------------------------------------
\section{Exercices}

%% \exo{1204}

% -----------------------------------------------------------------------------
\par\pagebreak[1]\par
\paragraph{Exercice 1}%
\hfill{\tiny 7052}%
\begingroup~

D\'eterminer le rayon de convergence de la s\'erie enti\`ere $\sum_n a_n z^n$ pour:
\begin{enumerate}
\item
  $a_n = n!$
\item
  $a_n = \ln n$
\item
  $a_n = \binom{3n}{n}$
\item
  $a_n = n^{\sqrt n}$
\item
  $a_n = \frac{n^n}{n!}$
\item
  $a_n$ vaut la somme des diviseurs de $n$
\item
  $a_n = \cos n$.
  On pourra commencer par montrer que $(\cos n)_{n\in \N}$ ne converge pas vers $0$.
\item
  $a_n = \frac{\sin n}{n}$
\item
  $\arctan\left(\frac1{n^\alpha}\right)$
\item
  $a_n = \frac{\alpha^n}{n} + \frac{\beta^n}{n^2}$
\item
  $a_n = \tan\left(\frac{n\pi}{7}\right)$
\item
  $a_{2n} = 0$,  $a_{2n+1} = \frac{(-1)^n}{\ch n}$
\item
  $a_{3n} = \frac{1}{n^2+1}$, $a_{3n+1}=\frac{1}{n!}$ et $a_{3n+2} = \alpha^n$
\item
  $a_0 > 0$, \\ $a_{n+1} = \ln(1+a_n)$
\item
  $a_n = \lfloor \me^n \rfloor$
\item
  $a_n = \frac{1}{n!} \left( 1+\frac1n \right)^{n^2}$
\item
  $a_n$ est la $n$-i\`eme d\'ecimale de $e$
\end{enumerate}
\endgroup

% -----------------------------------------------------------------------------
\par\pagebreak[1]\par
\paragraph{Exercice 2}%
\hfill{\tiny 6914}%
\begingroup~

D\'eterminer le rayon de convergence et la somme des s\'eries enti\`eres suivantes:
\begin{enumerate}
\item
  $\sum_{n\geq0} n^{(-1)^n} x^n$
\item
  $\sum_{n\geq0} n^2 x^n$
\item
  $\sum_{n\geq0} \left( \frac n3 - \left\lfloor \frac n3 \right\rfloor \right) x^n$
\item
  $\sum_{n\geq0} \frac{n^2+4n-1}{n!} \, x^n$
\item
  $\sum_{n\geq1} \left( \sum_{k=1}^n \frac1k \right) x^n$
\item
  $\sum_{n\geq1} \frac{x^n}{1+2+\cdots+n}$
\item
  $\sum_{n\geq0} \sin(n\theta) \, x^n$
\item
  $\sum_{n\geq1} \frac{\sin(n\theta)}{n} \, x^n$
\item
  $\sum_{n\geq0} \frac{\cos(n\theta)}{n!} \, x^n$
\item
  $\sum_{n\geq0} \left( \sum_{k=0}^n \frac{1}{k!} \right) x^n$
\item
  $\sum_{n\geq0} a_n x^n$ o\`u $(a_n)_{n\in \N}$ v\'erifie la r\'ecurrence $a_{n+3}=a_{n+2}+a_{n+1}-a_n$.
\end{enumerate}
\endgroup

% -----------------------------------------------------------------------------
\par\pagebreak[1]\par
\paragraph{Exercice 3}%
\hfill{\tiny 0512}%
\begingroup~

D\'evelopper en s\'erie enti\`ere, au voisinage de~$0$, les fonctions de $x$ suivantes:
\begin{enumerate}
\item
  $\frac{\ln(1+x)}{1+x^2}$
\item
  $\ln(x^2-5x+6)$
\item
  $\me^{-x} \sin x$
\item
  $\frac{1-x\ch\alpha}{1-2x\ch\alpha+x^2}$
\item
  $\arctan\left( \frac{1}{\sqrt3} + x \right)$
\item
  $\arcsin x$
\item
  $\frac{\me^x}{1-x}$
\item
  $\me^{x^2} \int_0^x \me^{t^2} \D t$
\end{enumerate}
\endgroup

% -----------------------------------------------------------------------------
\par\pagebreak[1]\par
\paragraph{Exercice 4}%
\hfill{\tiny 3808}%
\begingroup~

On suppose que les s\'eries enti\`eres $\sum_n \alpha_n z^n$ et $\sum_n \beta_n z^n$ ont m\^eme rayon $R$ de convergence
et que pour tout $n\in \N$, on ait $\Abs{\alpha_n} \leq{} \Abs{a_n} \leq{} \Abs{\beta_n}$.
Quel est alors le rayon de convergence de $\sum_n a_n z^n$?
\endgroup

% -----------------------------------------------------------------------------
\par\pagebreak[1]\par
\paragraph{Exercice 5}%
\hfill{\tiny 6370}%
\begingroup~

Soit $\sum{} a_n z^n$ une s\'erie enti\`ere de rayon de convergence~$R>0$.
Quel est le rayon des s\'eries enti\`eres suivantes:
\begin{enumerate}
\item
  $\sum_n \frac{a_n}{n!} z^n$
\item
  $\sum_n n^\alpha{} a_n z^n$
\item
  $\sum_n a_n z^{7n+3}$
\item
  $\sum_n a_n^\alpha{} z^n$ o\`u $\alpha\in\Rp$
\item
  $\sum_n \frac{a_n}{n^{42}+\cos n+1} \, z^n$
\item
  $\sum_n S_n z^n$ \\ o\`u $S_n = \sum_{k=0}^n a_k$
\end{enumerate}
\endgroup

% -----------------------------------------------------------------------------
\par\pagebreak[1]\par
\paragraph{Exercice 6}%
\hfill{\tiny 0437}%
\begingroup~

\begin{enumerate}
\item
  Soit $(u_n)$ une suite r\'eelle telle que
  $\forall n\geq n_0$, $u_{n+p}\leq \alpha u_n$.
  Montrer qu'il existe une constante $K$ telle que pour tout $n\in \N$ on ait l'in\'egalit\'e
  $u_n\leq K\alpha^{n/p}$.
\item
  Soit $p\in\Ns$ et $(a_n)_{n\in \N}$ une suite de complexes non nuls v\'erifiant
  $\lim_\ninf \left| \frac{a_{n+p}}{a_n} \right| = \ell$.
  D\'eterminer le rayon de convergence de la s\'erie enti\`ere $\sum_n a_n z^n$.
\end{enumerate}
\endgroup

% -----------------------------------------------------------------------------
\par\pagebreak[1]\par
\paragraph{Exercice 7 (th\'eor\`eme de continuit\'e au bord)}%
\hfill{\tiny 6361}%
\begingroup~

Soit $\sum{} a_n x^n$ une s\'erie enti\`ere de rayon de convergence $R$ et $f$ sa somme.
\begin{enumerate}
\item
  On suppose que $\sum_n \Abs{a_n} R^n$ converge.
  Montrer que $f$ est continue sur le disque ferm\'e de convergence $[-R,R]$.
\item
  \emph{Lemme}: Soit $\sum_n b_n$ une s\'erie num\'erique convergente;
  on ne la suppose pas n\'ecessairement absolument convergente.

  \begin{enumerate}
  \item
    On pose $g_n(x) = b_n x^n$.
    Montrer que la s\'erie de fonctions $\sum_n g_n$ converge uniform\'ement sur $[0,1]$.
  \item
    En d\'eduire que la fonction $g(x) = \sum_{n\geq0} b_n x^n$ est continue sur $[0,1]$.
  \end{enumerate}
\item
  \emph{Application:}

  \begin{enumerate}
  \item
    On suppose que $\sum_n a_n R^n$ converge.
    Montrer que $f$ est continue sur $[0,R]$.
  \item
    On suppose que $\sum_n a_n (-R)^n$ converge.
    Montrer que $f$ est continue sur $[-R,0]$.
  \end{enumerate}
\end{enumerate}
\endgroup

% -----------------------------------------------------------------------------
\par\pagebreak[1]\par
\paragraph{Exercice 8}%
\hfill{\tiny 7075}%
\begingroup~

\begin{enumerate}
\item
  Soit $f \colon x \mapsto \arctan x$.
  \begin{enumerate}
  \item
    D\'evelopper $f$ en s\'erie enti\`ere au voisinage de 0.
  \item
    A priori, sur quel intervalle a-t-on l'\'egalit\'e entre $f$ et son d\'eveloppement en s\'erie enti\`ere?
  \item
    En utilisant le th\'eor\`eme de continuit\'e au bord (exercice~7),
    montrer que ce d\'eveloppement est encore valable pour $x=1$.
  \item
    En d\'eduire la valeur de $\DS \sum_{n=0}^{+\infty} \frac{(-1)^n}{2n+1}$.
  \end{enumerate}
\item
  Avec la m\^eme m\'ethode, calculer
  \[ \sum_{n=1}^{+\infty} \frac{1}{n(2n+1)} \quad\text{et}\quad
  \sum_{n=1}^{+\infty} \frac{(-1)^n}{n(2n+1)}. \]
\end{enumerate}
\endgroup

% -----------------------------------------------------------------------------
\par\pagebreak[1]\par
\paragraph{Exercice 9}%
\hfill{\tiny 4424}%
\begingroup~

Soit $f$ l'unique fonction $\R\to\R$ solution de
\[ \begin{cases} y' - 2xy = 1 \\ y(0) = 0 \end{cases} \]
\begin{enumerate}
\item
  Calculer le d\'eveloppement en s\'erie enti\`ere de $f$ de deux fa\c cons diff\'erentes:

  \begin{enumerate}
  \item
    en r\'esolvant l'\'equation diff\'erentielle, puis en d\'eveloppant la solution en s\'erie enti\`ere,
  \item
    en supposant $f$ d\'eveloppable en s\'erie enti\`ere et en injectant ce d\'eveloppement dans l'\'equation diff\'erentielle pour obtenir une r\'ecurrence
  \end{enumerate}
\item
  En d\'eduire une expression simple de la somme
  \[ \sum_{k=0}^n \frac{(-1)^k}{2k+1} \binom{n}{k}. \]
\end{enumerate}
\endgroup

% -----------------------------------------------------------------------------
\par\pagebreak[1]\par
\paragraph{Exercice 10}%
\hfill{\tiny 5927}%
\begingroup~

Pour $x$ r\'eel, on pose $\DS f(x) = \sum_{n=1}^{+\infty} \frac{x^n}{\sqrt{n}}$.
\begin{enumerate}
\item
  D\'eterminer le rayon de convergence $R$ de la s\'erie enti\`ere d\'efinissant $f$.
\item
  \'Etudier la convergence de la s\'erie enti\`ere en~$1$ et en~$-1$.
\item
  \'Etablir la continuit\'e de $f$ en~$-1$.
\item
  D\'eterminer la limite de $f$ en~$1$.
\end{enumerate}
\endgroup

% -----------------------------------------------------------------------------
\par\pagebreak[1]\par
\paragraph{Exercice 11}%
\hfill{\tiny 2456}%
\begingroup~

Soit $f(z) = \sum_{n=0}^{+\infty} a_n z^n$ la somme d'une s\'erie enti\`ere de rayon de convergence infini.
\begin{enumerate}
\item
  Calculer $\DS I_n = \frac{1}{2\pi}\int_0^{2\pi} f(r\me^{\I\theta}) \me^{-n\I\theta} \D\theta$.
\item
  Soit $\DS M(r) = \sup_{\Abs{z}=r} \Abs{f(z)}$. Montrer que $\Abs{a_nr^n}\leq M(r)$.
\item
  En d\'eduire que, si $f$ est born\'ee sur $\C$, alors $f$ est constante;
  il s'agit de la version faible du th\'eor\`eme de Liouville.
\end{enumerate}
\endgroup

% -----------------------------------------------------------------------------
\par\pagebreak[1]\par
\paragraph{Exercice 12}%
\hfill{\tiny 9009}%
\begingroup~

Soit $(u_n)_{n\in \N}$ la suite d\'efinie par
$u_0=1$ et $\forall n\in \N$, \[ u_{n+1} = \sum_{k=0}^n u_k u_{n-k} \]
Soit $f(x) = \sum_{n=0}^{+\infty} u_n x^n$.
On note $R$ le rayon de convergence de cette s\'erie enti\`ere.
\begin{enumerate}
\item
  On suppose que $R > 0$.
  \begin{enumerate}
  \item
    Montrer que $\forall x\in\intO{-R,R} \setminus{} \acco{0}$, $f^2(x) = \frac{f(x)-1}{x}$.
  \item
    En d\'eduire une expression de $f$. \emph{Attention, c'est un peu subtil \`a cause du signe.}
  \end{enumerate}
\item
  Soit $\DS g(x) = \frac{1-\sqrt{1-4x}}{2x}$.
  \begin{enumerate}
  \item
    D\'evelopper $g$ en s\'erie enti\`ere. Quel est son rayon $R'$ de convergence?
  \item
    Montrer que $xg^2(x) = g(x) - 1$ pour tout $x\in\intO{-R',R'}$.
  \item
    En d\'eduire que le d\'eveloppement en s\'erie enti\`ere de $g$ est $\sum_{n\geq0} u_n x^n$.
  \end{enumerate}
\item
  Donner une expression simple de $u_n$.
\end{enumerate}

\emph{Remarque}: les $(u_n)$ s'appellent les \emph{nombres de Catalan}.
Ils poss\`edent de nombreuses interpr\'etations combinatoires.
\endgroup

% -----------------------------------------------------------------------------
\par\pagebreak[1]\par
\paragraph{Exercice 13}%
\hfill{\tiny 9904}%
\begingroup~

Pour $x\in \R$, on pose $\DS f(x) = \sum_{n=0}^{+\infty} \frac{\cos(2^n x)}{n!}$.
\begin{enumerate}
\item
  Montrer que $f$ est d\'efinie et de classe $\CC\infty$ sur $\R$.
\item
  Observer que le rayon de convergence de sa s\'erie de Taylor en~$0$ est nul.
\end{enumerate}
\endgroup

% -----------------------------------------------------------------------------
\par\pagebreak[1]\par
\paragraph{Exercice 14}%
\hfill{\tiny 2758}%
\begingroup~

Soit $\sum_n a_n z^n$ une s\'erie enti\`ere de rayon de convergence $R$.
Montrer que les propositions suivantes sont \'equivalentes:
\begin{enumerate}
\item
  $R > 0$;
\item
  $\exists{} K \geq{} 0$, $\exists{} q > 0$ tels que $\forall{} n \in{} \N$, $\abs{a_n} \leq{} K q^n$;
\item
  $\exists{} q > 0$ tels que $\forall{} n \in{} \N$, $\abs{a_n} \leq{} q^n$;
\item
  $\exists{} q > 0$ tel que $a_n = O(q^n)$ quand $n\to+\infty$;
\item
  $\exists{} q > 0$ tel que $a_n = o(q^n)$ quand $n\to+\infty$.
\end{enumerate}
\endgroup

\end{document}

% -----------------------------------------------------------------------------
% ------------------------------------------------------------------------------
% -----------------------------------------------------------------------------
