% autogenerated by ytex.rs

\documentclass{scrartcl}

\usepackage[francais]{babel}
\usepackage{geometry}
\usepackage{scrpage2}
\usepackage{lastpage}
\usepackage{multicol}
\usepackage{etoolbox}
\usepackage{xparse}
\usepackage{enumitem}
% \usepackage{csquotes}
\usepackage{amsmath}
\usepackage{amsfonts}
\usepackage{amssymb}
\usepackage{mathrsfs}
\usepackage{stmaryrd}
\usepackage{dsfont}
% \usepackage{eurosym}
% \usepackage{numprint}
% \usepackage[most]{tcolorbox}
% \usepackage{tikz}
% \usepackage{tkz-tab}
\usepackage[unicode]{hyperref}
\usepackage[ocgcolorlinks]{ocgx2}

\let\ifTwoColumns\iftrue
\def\Classe{$\Psi$2019--2020}

% reproducible builds
% LuaTeX: \pdfvariable suppressoptionalinfo 1023 \relax
\pdfinfoomitdate=1
\pdftrailerid{}

\newif\ifDisplaystyle
\everymath\expandafter{\the\everymath\ifDisplaystyle\displaystyle\fi}
\newcommand\DS{\displaystyle}

\clearscrheadfoot
\pagestyle{scrheadings}
\thispagestyle{empty}
\ohead{\Classe}
\ihead{\thepage/\pageref*{LastPage}}

\setlist[itemize,1]{label=\textbullet}
\setlist[itemize,2]{label=\textbullet}

\ifTwoColumns
  \geometry{margin=1cm,top=2cm,bottom=3cm,foot=1cm}
  \setlist[enumerate]{leftmargin=*}
  \setlist[itemize]{leftmargin=*}
\else
  \geometry{margin=3cm}
\fi

\makeatletter
\let\@author=\relax
\let\@date=\relax
\renewcommand\maketitle{%
    \begin{center}%
        {\sffamily\huge\bfseries\@title}%
        \ifx\@author\relax\else\par\medskip{\itshape\Large\@author}\fi
        \ifx\@date\relax\else\par\bigskip{\large\@date}\fi
    \end{center}\bigskip
    \ifTwoColumns
        \par\begin{multicols*}{2}%
        \AtEndDocument{\end{multicols*}}%
        \setlength{\columnsep}{5mm}
    \fi
}
\makeatother

\newcounter{ParaNum}
\NewDocumentCommand\Para{smo}{%
  \IfBooleanF{#1}{\refstepcounter{ParaNum}}%
  \paragraph{\IfBooleanF{#1}{{\tiny\arabic{ParaNum}~}}#2\IfNoValueF{#3}{ (#3)}}}

\newcommand\I{i}
\newcommand\mi{i}
\def\me{e}

\def\do#1{\expandafter\undef\csname #1\endcsname}
\docsvlist{Ker,sec,csc,cot,sinh,cosh,tanh,coth,th}
\undef\do

\DeclareMathOperator\ch{ch}
\DeclareMathOperator\sh{sh}
\DeclareMathOperator\th{th}
\DeclareMathOperator\coth{coth}
\DeclareMathOperator\cotan{cotan}
\DeclareMathOperator\argch{argch}
\DeclareMathOperator\argsh{argsh}
\DeclareMathOperator\argth{argth}

\let\epsilon=\varepsilon
\let\phi=\varphi
\let\leq=\leqslant
\let\geq=\geqslant
\let\subsetneq=\varsubsetneq
\let\emptyset=\varnothing

\newcommand{\+}{,\;}

\undef\C
\newcommand\ninf{{n\infty}}
\newcommand\N{\mathbb{N}}
\newcommand\Z{\mathbb{Z}}
\newcommand\Q{\mathbb{Q}}
\newcommand\R{\mathbb{R}}
\newcommand\C{\mathbb{C}}
\newcommand\K{\mathbb{K}}
\newcommand\Ns{\N^*}
\newcommand\Zs{\Z^*}
\newcommand\Qs{\Q^*}
\newcommand\Rs{\R^*}
\newcommand\Cs{\C^*}
\newcommand\Ks{\K^*}
\newcommand\Rp{\R^+}
\newcommand\Rps{\R^+_*}
\newcommand\Rms{\R^-_*}
\newcommand{\Rpinf}{\Rp\cup\Acco{+\infty}}

\undef\B
\newcommand\B{\mathscr{B}}

\undef\P
\DeclareMathOperator\P{\mathbb{P}}
\DeclareMathOperator\E{\mathbb{E}}
\DeclareMathOperator\Var{\mathbb{V}}

\DeclareMathOperator*\PetitO{o}
\DeclareMathOperator*\GrandO{O}
\DeclareMathOperator*\Sim{\sim}
\DeclareMathOperator\Tr{tr}
\DeclareMathOperator\Ima{Im}
\DeclareMathOperator\Ker{Ker}
\DeclareMathOperator\Sp{Sp}
\DeclareMathOperator\Diag{diag}
\DeclareMathOperator\Rang{rang}
\DeclareMathOperator*\Coords{Coords}
\DeclareMathOperator*\Mat{Mat}
\DeclareMathOperator\Pass{Pass}
\DeclareMathOperator\Com{Com}
\DeclareMathOperator\Card{Card}
\DeclareMathOperator\Racines{Racines}
\DeclareMathOperator\Vect{Vect}
\DeclareMathOperator\Id{Id}

\newcommand\DerPart[2]{\frac{\partial #1}{\partial #2}}

\def\T#1{{#1}^T}

\def\pa#1{({#1})}
\def\Pa#1{\left({#1}\right)}
\def\bigPa#1{\bigl({#1}\bigr)}
\def\BigPa#1{\Bigl({#1}\Bigr)}
\def\biggPa#1{\biggl({#1}\biggr)}
\def\BiggPa#1{\Biggl({#1}\Biggr)}

\def\pafrac#1#2{\pa{\frac{#1}{#2}}}
\def\Pafrac#1#2{\Pa{\frac{#1}{#2}}}
\def\bigPafrac#1#2{\bigPa{\frac{#1}{#2}}}
\def\BigPafrac#1#2{\BigPa{\frac{#1}{#2}}}
\def\biggPafrac#1#2{\biggPa{\frac{#1}{#2}}}
\def\BiggPafrac#1#2{\BiggPa{\frac{#1}{#2}}}

\def\cro#1{[{#1}]}
\def\Cro#1{\left[{#1}\right]}
\def\bigCro#1{\bigl[{#1}\bigr]}
\def\BigCro#1{\Bigl[{#1}\Bigr]}
\def\biggCro#1{\biggl[{#1}\biggr]}
\def\BiggCro#1{\Biggl[{#1}\Biggr]}

\def\abs#1{\mathopen|{#1}\mathclose|}
\def\Abs#1{\left|{#1}\right|}
\def\bigAbs#1{\bigl|{#1}\bigr|}
\def\BigAbs#1{\Bigl|{#1}\Bigr|}
\def\biggAbs#1{\biggl|{#1}\biggr|}
\def\BiggAbs#1{\Biggl|{#1}\Biggr|}

\def\acco#1{\{{#1}\}}
\def\Acco#1{\left\{{#1}\right\}}
\def\bigAcco#1{\bigl\{{#1}\bigr\}}
\def\BigAcco#1{\Bigl\{{#1}\Bigr\}}
\def\biggAcco#1{\biggl\{{#1}\biggr\}}
\def\BiggAcco#1{\Biggl\{{#1}\Biggr\}}

\def\ccro#1{\llbracket{#1}\rrbracket}
\def\Dcro#1{\llbracket{#1}\rrbracket}

\def\floor#1{\lfloor#1\rfloor}
\def\Floor#1{\left\lfloor{#1}\right\rfloor}

\def\sEnsemble#1#2{\mathopen\{#1\mid#2\mathclose\}}
\def\bigEnsemble#1#2{\bigl\{#1\bigm|#2\bigr\}}
\def\BigEnsemble#1#2{\Bigl\{#1\Bigm|#2\Bigr\}}
\def\biggEnsemble#1#2{\biggl\{#1\biggm|#2\biggr\}}
\def\BiggEnsemble#1#2{\Biggl\{#1\Biggm|#2\Biggr\}}
\let\Ensemble=\bigEnsemble

\newcommand\IntO[1]{\left]#1\right[}
\newcommand\IntF[1]{\left[#1\right]}
\newcommand\IntOF[1]{\left]#1\right]}
\newcommand\IntFO[1]{\left[#1\right[}

\newcommand\intO[1]{\mathopen]#1\mathclose[}
\newcommand\intF[1]{\mathopen[#1\mathclose]}
\newcommand\intOF[1]{\mathopen]#1\mathclose]}
\newcommand\intFO[1]{\mathopen[#1\mathclose[}

\newcommand\Fn[3]{#1\colon#2\to#3}
\newcommand\CC[1]{\mathscr{C}^{#1}}
\newcommand\D{\mathop{}\!\mathrm{d}}

\newcommand\longto{\longrightarrow}

\undef\M
\newcommand\M[3]{\mathrm{#1}_{#2}\pa{#3}}
\newcommand\MnR{\M{M}{n}{\R}}
\newcommand\MnC{\M{M}{n}{\C}}
\newcommand\MnK{\M{M}{n}{\K}}
\newcommand\GLnR{\M{GL}{n}{\R}}
\newcommand\GLnC{\M{GL}{n}{\C}}
\newcommand\GLnK{\M{GL}{n}{\K}}
\newcommand\DnR{\M{D}{n}{\R}}
\newcommand\DnC{\M{D}{n}{\C}}
\newcommand\DnK{\M{D}{n}{\K}}
\newcommand\SnR{\M{S}{n}{\R}}
\newcommand\AnR{\M{A}{n}{\R}}
\newcommand\OnR{\M{O}{n}{\R}}
\newcommand\SnRp{\mathrm{S}_n^+(\R)}
\newcommand\SnRpp{\mathrm{S}_n^{++}(\R)}

\newcommand\LE{\mathscr{L}(E)}
\newcommand\GLE{\mathscr{GL}(E)}
\newcommand\SE{\mathscr{S}(E)}
\renewcommand\OE{\mathscr{O}(E)}

\newcommand\ImplD{$\Cro\Rightarrow$}
\newcommand\ImplR{$\Cro\Leftarrow$}
\newcommand\InclD{$\Cro\subset$}
\newcommand\InclR{$\Cro\supset$}
\newcommand\notInclD{$\Cro{\not\subset}$}
\newcommand\notInclR{$\Cro{\not\supset}$}

\newcommand\To[1]{\xrightarrow[#1]{}}
\newcommand\Toninf{\To{\ninf}}

\newcommand\Norm[1]{\|#1\|}
\newcommand\Norme{{\Norm{\cdot}}}

\newcommand\Int[1]{\mathring{#1}}
\newcommand\Adh[1]{\overline{#1}}

\newcommand\Uplet[2]{{#1},\dots,{#2}}
\newcommand\nUplet[3]{(\Uplet{{#1}_{#2}}{{#1}_{#3}})}

\newcommand\Fonction[5]{{#1}\left|\begin{aligned}{#2}&\;\longto\;{#3}\\{#4}&\;\longmapsto\;{#5}\end{aligned}\right.}

\DeclareMathOperator\orth{\bot}
\newcommand\Orth[1]{{#1}^\bot}
\newcommand\PS[2]{\langle#1,#2\rangle}

\newcommand{\Tribu}{\mathscr{T}}
\newcommand{\Part}{\mathcal{P}}
\newcommand{\Pro}{\bigPa{\Omega,\Tribu}}
\newcommand{\Prob}{\bigPa{\Omega,\Tribu,\P}}

\newcommand\DEMO{$\spadesuit$}
\newcommand\DUR{$\spadesuit$}

\newenvironment{psmallmatrix}{\left(\begin{smallmatrix}}{\end{smallmatrix}\right)}

% -----------------------------------------------------------------------------

\begin{document}
\title{R\'evisions d'analyse r\'eelle}
\maketitle

Il s'agit essentiellement de r\'evisions de premi\`ere ann\'ee,
avec une petite g\'en\'eralisation: on consid\`ere, quand c'est possible,
des fonctions \`a valeurs vectorielles.

\Para*{Notations}

Dans tout le chapitre,
\begin{itemize}
\item $I$ et $J$ d\'esignent des intervalles de $\R$ (non vides et non r\'eduits \`a un point);
\item $E$, $F$ et $G$ d\'esignent des espaces vectoriels norm\'es \emph{de dimensions finies}.
\end{itemize}

% -----------------------------------------------------------------------------
\section{Continuit\'e}

\Para{Th\'eor\`eme}

Soit $f \colon \intF{a,b} \to\R$ continue.
Alors $f$ est born\'ee et ses bornes sont atteintes.

\Para{Th\'eor\`eme}[th\'eor\`eme des valeurs interm\'ediaires]

Soit $f \colon I \to \R$ continue.
Alors $f(I)$ est un intervalle de $\R$.

\Para{Corollaire}

Soit $f \colon \intF{a,b} \to \R$ continue
On suppose que $\gamma\in\bigl[f(a),f(b)\bigr] \cup\bigl[f(b),f(a)\bigr]$.
Alors $\exists c\in\intF{a,b}$, $f(c)=\gamma$.

% -----------------------------------------------------------------------------
\section{D\'erivation}

\subsection{D\'eriv\'ee}

\Para{D\'efinition}

Soit $f \colon I \to E$ et $a\in I$.
On dit que $f$ est \emph{d\'erivable} en $a$
si et seulement si
\[\lim_{\substack{x \to a \\ x\neq a}} \left( \frac{f(x) - f(a)}{x-a} \right) \text{ existe}.\]
Cette limite, si elle existe, s'appelle la \emph{d\'eriv\'ee} de $f$
en $a$ et se note $f'(a)$.

\Para{Proposition}
\begin{enumerate}
\item $f$ est d\'erivable en $a$ et $f'(a) =\ell$;
\item $f(x) = f(a) + (x-a)\ell+ o(x-a)$ quand $x \to a$;
\item $\forall \epsilon> 0$, $\exists \delta> 0$, $\forall x\in I\setminus\Acco{a}$,
  \[ \Abs{x-a}\leq \delta\implies \left\| \frac{f(x)-f(a)}{x-a} -\ell{} \right\|_E \leq{} \epsilon. \]
\end{enumerate}

\Para{Proposition}

Soit $f \colon I \to E$ et $a\in I$.
Si $f$ est d\'erivable en $a$, alors $f$ est continue en $a$.

\Para{Proposition}

Soit $f \colon I \to E$ d\'erivable et $u \colon E \to F$ une application \emph{lin\'eaire}.
Alors $u\circ f$ est d\'erivable et $(u\circ f)' = u\circ f'$.

\Para{Proposition}

Soit $f \colon I \to E$ et $g \colon I \to F$ deux fonctions d\'erivables.
Soit $B \colon E\times F \to G$ une application \emph{bilin\'eaire}.
On note $\Fn{h}{I}{G}$ telle que \[ h(t) = B(f(t),g(t)). \]
Alors $h$ est d\'erivable et \[ h'(t) = B(f'(t),g(t)) + B(f(t),g'(t)). \]

\Para{Exemple}

Soit $f$ et $g$ deux fonctions d\'erivables $I \to\R^3$.
On pose $\Fn{h}{I}{\R^3}$ o\`u $h(t) = f(t) \wedge g(t)$.
Alors $h$ est d\'erivable et $h'(t) = f'(t) \wedge g(t) + f(t) \wedge g'(t)$.

\Para{Th\'eor\`eme}[d\'eriv\'ee d'une compos\'ee]

Soit $f \colon I \to E$ et $g \colon J \to I$ deux fonctions d\'erivables.
Alors leur compos\'ee $f\circ g \colon J \to E$ est d\'erivable et
\[\forall t\in J\+ (f\circ g)'(t) = g'(t)\cdot f'(g(t)).\]

\Para{Th\'eor\`eme}[limite de la d\'eriv\'ee, HP]

Soit $f \colon I \to E$ une fonction continue et $a\in I$.
On suppose que $f$ est d\'erivable sur $I\setminus\Acco a$ et que
\[\ell= \lim_{\substack{x \to a \\ x\neq a}} f'(x) \text{ existe et est finie.}\]
Alors $f$ est d\'erivable en $a$ et $f'(a) = \ell$.

\Para{Th\'eor\`eme}[d\'eriv\'ee de la r\'eciproque]

Soit $f \colon I \to J$ une fonction d\'erivable et \emph{bijective}.
Soit $x\in I$ un point tel que $f'(x)\neq0$.

Alors $f^{-1}$ est d\'erivable au point $y = f(x)$ et
\[(f^{-1})'(y) = \frac{1}{f'(x)} = \frac{1}{f'\circ f^{-1} (y)}.\]

\subsection{Accroissements finis}

\Para{Th\'eor\`eme}[Rolle]

Soit $f \colon \intF{a,b} \to\R$ une fonction num\'erique.
On suppose:
\begin{itemize}
\item $f$ continue sur $\intF{a,b}$,
\item $f$ d\'erivable sur $\intO{a,b}$,
\item $f(a) = f(b)$.
\end{itemize}

Alors $\exists c\in\intO{a,b}$, $f'(c) = 0$.

\Para{Th\'eor\`eme}[\'egalit\'e des accroissements finis]

Soit $f \colon [a,b] \to \R$ une fonction num\'erique continue,
d\'erivable sur l'ouvert $\intO{a,b}$.
Alors $\exists c\in\intO{a,b} \+ f(b) - f(a) = f'(c) (b - a)$.

\Para{Corollaire}

Soit $f \colon [a,b] \to\R$ une fonction num\'erique continue, d\'erivable sur l'ouvert $\intO{a,b}$.
Si $\forall x\in\intO{a,b} \+ f'(x)\geq0$ (resp. $f'(x) > 0$),
alors $f$ est croissante (resp. strictement croissante) sur $[a,b]$.

\Para{Th\'eor\`eme}[in\'egalit\'e des accroissements finis]

Soit $f \colon \intF{a,b} \to E$ une fonction continue, d\'erivable sur l'ouvert $\intO{a,b}$.
On suppose que $\forall x\in\intO{a,b} \+ \Norm{f'(x)}\leq M$.
Alors $\Norm{f(b)-f(a)}\leq M(b-a)$.

\subsection{Fonctions de classe $\CC p$}

\Para{D\'efinition}

Soit $f \colon I \to E$.
On d\'efinit $f^{(p)}$ pour $p\in \N$ par r\'ecurrence:
$f^{(0)} = f$, et $f^{(p+1)}$ est la d\'eriv\'ee de $f^{(p)}$,
d\'efinie sur l'ensemble des points o\`u $f^{(p)}$ est d\'erivable.

\Para{D\'efinition}

Soit $f \colon I \to E$.
On dit que $f$ est de classe $\CC p$
si et seulement si $f$ est $p$ fois d\'erivable sur $I$
et $f^{(p)}$ est continue sur $I$.

\Para{Th\'eor\`eme}

Soit $f \colon I \to E$ et $g \colon J \to I$ de classe $\CC p$.
Alors $f\circ g \colon J \to E$ est de classe $\CC p$.
Autrement dit, la compos\'ee de deux fonctions de classe $\CC p$ l'est \'egalement.

\Para{Th\'eor\`eme}[formule de Leibniz]

Soit $f$ et $g$ deux fonctions de $I$ dans $\K$ de classe $\CC p$.
Alors leur produit $fg$ est \'egalement de classe $\CC p$ et
\[(fg)^{(p)} = \sum_{n=0}^p \binom pn f^{(n)} g^{(p-n)}.\]

\Para{Proposition}[g\'en\'eralisation]

Soit $f \colon I \to E$ et $g \colon I \to F$ deux fonctions de classe $\CC p$.
Soit $B \colon E\times F \to G$ une application bilin\'eaire.
On pose \[ \Fonction{h}{I}{G}{x}{B \bigl( f(x),g(x) \bigr).} \]
Alors $h$ est de classe $\CC p$ et
\[ h^{(p)}(x) = \sum_{n=0}^p \binom pn \; B\bigl( f^{(n)}(x), g^{(p-n)}(x) \bigr). \]

\Para{Th\'eor\`eme}[prolongement $\CC k$]

Soit $f \colon I\setminus\{a\} \to E$ une fonction de classe $\CC k$.
On suppose que $f^{(i)}$ admet une limite finie en $a$ pour tout $i\in\Dcro{0,k}$.
Alors $f$ admet une prolongement de classe $\CC k$ sur $I$.

\Para{D\'efinition}

Soit $f \colon I \to J$ et $p\in\Ns$.
On dit que $f$ est un $\CC p$-diff\'eomorphisme si et seulement si
les trois conditions suivantes sont v\'erifi\'ees:
\begin{itemize}
\item $f$ est bijective;
\item $f$ est de classe $\CC p$;
\item $f^{-1}$ est de classe $\CC p$.
\end{itemize}

\Para{Proposition}[caract\'erisation des diff\'eomorphismes]

Soit $f \colon I \to J$ et $p\in\Ns$.
Alors $f$ est un $\CC p$-diff\'eomorphisme si et seulement si les trois conditions suivantes sont v\'erifi\'ees:
\begin{itemize}
\item $f$ est surjective;
\item $f$ est de classe $\CC p$ sur $I$;
\item $\forall x\in I$, $f'(x)\neq0$.
\end{itemize}

\subsection{Formules de Taylor}

\Para{Th\'eor\`eme}[formule de Taylor avec reste int\'egral]

Soit $\Fn{f}{I}{E}$ une fonction de classe $\CC{n+1}$.
Alors, pour tous $(a,x)\in I^2$, on a
\[f(x) = \underbrace{\sum_{k=0}^n \frac{(x-a)^k}{k!} f^{(k)}(a)}_{\text{partie principale}} + \underbrace{\vphantom{\sum_{k=0}}\int_a^x \frac{(x-t)^n}{n!} f^{(n+1)}(t) \D t}_{\text{reste int\'egral}}.\]

\Para{Th\'eor\`eme}[formule de Taylor-Young]

Soit $\Fn{f}{I}{E}$ une fonction de classe $\CC{n}$ sur un voisinage de $a\in I$.
Alors, on a
\[f(x) = \sum_{k=0}^n \frac{(x-a)^k}{k!} f^{(k)}(a) + \PetitO_{x \to a}\Bigl( (x-a)^n \Bigr).\]

\Para{Th\'eor\`eme}[in\'egalit\'e de Taylor-Lagrange, HP]

Soit $\Fn{f}{I}{E}$ une fonction de classe $\CC{n+1}$.

On suppose que
\[\forall t\in I\+ \Norm{f^{(n+1)}(t)}\leq M.\]
Alors, pour tous $(a,x)\in I^2$, on a
\[\left\| f(x) - \sum_{k=0}^n \frac{(x-a)^k}{k!} f^{(k)}(a) \right\| \leq M \frac{\Abs{x-a}^{n+1}}{(n+1)!}.\]

\Para{Th\'eor\`eme}[\'egalit\'e de Taylor-Lagrange, HP]

Soit $\Fn{f}{I}{\R}$ une fonction de classe $\CC{n+1}$.
Alors, pour tous $(a,x)\in I^2$, il existe $\theta\in\intO{0,1}$ tel que
\[f(x) = \sum_{k=0}^n \frac{(x-a)^k}{k!} f^{(k)}(a) + \frac{(x-a)^{n+1}}{(n+1)!} f^{(n+1)}(\xi)\]
o\`u $\xi= a + (x-a)\theta$ est compris entre $a$ et $x$.

\subsection{Fonctions convexes (HP)}

\Para{D\'efinition}

Soit $f \colon I \to \R$.

On dit que $f$ est \emph{convexe} si et seulement si
$\forall(x,y)\in I^2$, $\forall \lambda\in[0,1]$,
\[f \bigl( \lambda x + (1-\lambda)y \bigr) \leq \lambda f(x) + (1-\lambda)f(y)\]
On dit que $f$ est \emph{concave} si et seulement si $(-f)$ est convexe.

\Para{Proposition}

Soit $f \colon I \to\R$.
\begin{itemize}
\item $f$ est convexe si et seulement si tout arc de la courbe repr\'esentative de $f$
  est situ\'e au dessous de la corde correspondante.
\item Si $f$ est d\'erivable et convexe, alors toute tangente \`a la courbe repr\'esentative de $f$
  est situ\'e au dessous de celle-ci.
\item On suppose $f$ est d\'erivable.
  Alors $f$ est convexe si et seulement si $f'$ est croissante.
\item On suppose $f$ deux fois d\'erivable.
  Alors $f$ est convexe si et seulement si $f''$ est positive.
\end{itemize}

\Para{Th\'eor\`eme}[in\'egalit\'e de Jensen]

Soit $f \colon I \to\R$ une fonction convexe et $\nUplet x1n\in I^n$.
Soit $\nUplet\lambda1n\in \R_+^n$ tels que $\sum_{k=1}^n\lambda_k = 1$.
Alors
\[f \left( \sum_{k=1}^n \lambda_k x_k \right) \leq\sum_{k=1}^n \lambda_k f(x_k).\]

% -----------------------------------------------------------------------------
\section{Int\'egration}

\subsection{Subdivisions}

\Para{D\'efinition}

Soit $\intF{a,b}\subset \R$ un segment.
Une \emph{subdivision} de $[a,b]$ est un $(n+1)$-uplet
$\nUplet\sigma0n\in \R^{n+1}$ tel que
\[a =\sigma_0 <\sigma_1 < \cdots <\sigma_n = b\]

\Para{D\'efinitions}

Soit $f \colon \intF{a,b} \to E$.
\begin{itemize}
\item $f$ est dite \emph{en escaliers} sur $\intF{a,b}$
  si et seulement si il existe une subdivision $\nUplet\sigma0n$ de $\intF{a,b}$ telle que
  pour tout $k\in\Dcro{0,n-1}$,
  $f$ est constante sur $\intO{\sigma_k,\sigma_{k+1}}$.
\item $f$ est dite \emph{continue par morceaux} sur $\intF{a,b}$
  si et seulement si il existe une subdivision $\nUplet\sigma0n$ de $\intF{a,b}$ telle que
  pour tout $k\in\Dcro{0,n-1}$,
  la restriction de $f$ \`a $\intO{\sigma_k,\sigma_{k+1}}$ est continue
  \emph{et se prolonge en une fonction continue sur $\intF{\sigma_k,\sigma_{k+1}}$}.
\item Plus g\'en\'eralement,
  $f$ est dire \emph{de classe $\CC k$ par morceaux} sur $\intF{a,b}$
  si et seulement si il existe une subdivision $\nUplet\sigma0n$ de $\intF{a,b}$ telle que
  pour tout $k\in\Dcro{0,n-1}$,
  la restriction de $f$ \`a $\intO{\sigma_k,\sigma_{k+1}}$ est de classe $\CC k$
  \emph{et se prolonge en une fonction de classe $\CC k$ sur $\intF{\sigma_k,\sigma_{k+1}}$}.
\end{itemize}

\Para{Th\'eor\`eme}

Soit $f \colon [a,b] \to E$ continue par morceaux et $\epsilon> 0$.
Alors il existe $\varphi\colon [a,b] \to E$ en escaliers
tel que $\forall x\in[a,b]\+ \Norm{f(x) -\varphi(x)}\leq \epsilon$.

\subsection{Int\'egration sur un segment}

\Para{Remarque}

La construction de l'int\'egrale n'est pas l'objet principal d'\'etude,
je renvoie donc au cours de premi\`ere ann\'ee.
On pourra toutefois noter que les propri\'et\'es suivantes
\emph{caract\'erisent} l'int\'egrale des fonctions continues par morceaux
sur un segment:
\begin{itemize}
\item Int\'egrale d'une fonction constante,
\item Relation de Chasles,
\item In\'egalit\'e de la moyenne,
\item Croissance.
\end{itemize}

\subsection{Propri\'et\'es de l'int\'egrale}

\Para{Proposition}[int\'egrale d'une fonction constante]

Soit $f \colon I \to E$ une fonction, $(a,b)\in I^2$ o\`u $a\leq b$.
On suppose que
\[\exists C\in E\+\forall x\in\intO{a,b}\+ f(x) = C.\]
Alors $f$ est continue par morceaux sur $\intF{a,b}$ et \[\int_a^b f = (b-a)C.\]

\Para{Proposition}[lin\'earit\'e]

Soit $f, g \colon I \to E$ deux fonctions continues par morceaux, $(a,b)\in I^2$ et $(\lambda,\mu)\in \K^2$.
Alors $\lambda f +\mu g \colon I \to E$ est continue par morceaux sur $I$ et
\[\int_a^b (\lambda f +\mu g) =\lambda\int_a^b f +\mu\int_a^b g.\]

\Para{Proposition}[relation de Chasles]

Soit $f \colon I \to E$ une fonction continue par morceaux et $(a,b,c)\in I^3$.
Alors \[\int_a^b f =\int_a^c f +\int_c^b f.\]

\Para{Proposition}[in\'egalit\'e de la moyenne]

Soit $f \colon I \to E$ une fonction continue par morceaux,
$(a,b)\in I^2$ o\`u $a\leq b$.
Alors \[\left\| \int_a^b f(t) \D t \right\| \leq\int_a^b \Norm{f(t)} \D t.\]

\Para{Remarque}

Si l'on enl\`eve l'hypoth\`ese $a\leq b$, la conclusion s'\'ecrit alors
\[\left\| \int_a^b f(t) \D t \right\| \leq\left| \int_a^b \Norm{f(t)} \D t \right|.\]

\Para{Proposition}[positivit\'e]

Soit $f \colon I \to \Rp$ une fonction continue par morceaux,
$(a,b)\in I^2$ o\`u $a\leq b$.
Alors \[\int_a^b f\geq0.\]

\Para{Proposition}[croissance]

Soit $f, g \colon I \to\R$ deux fonctions continues par morceaux,
$(a,b)\in I^2$ o\`u $a\leq b$.
On suppose que
\[\forall x\in[a,b]\+ f(x)\leq g(x).\]
Alors
\[\int_a^b f\leq\int_a^b g.\]

\Para{Th\'eor\`eme}[stricte positivit\'e]

Soit $f \colon I \to \Rp$ une fonction \emph{continue},
$(a,b)\in I^2$ o\`u $a\leq b$.
On suppose que
\[\int_a^b f = 0.\]
Alors $f$ est identiquement nulle sur $\intF{a,b}$.

\Para{Th\'eor\`eme}[in\'egalit\'e de Cauchy-Schwarz]

Soit $f, g \colon I \to\K$ deux fonctions continues par morceaux
et $(a,b)\in I^2$.
Alors
\[\left|\int_a^b f(t) g(t) \D t \right|\leq{}
  \sqrt{\int_a^b \Abs{f(t)}^2 \D t}
\sqrt{\int_a^b \Abs{g(t)}^2 \D t}.\]
Si l'on suppose en outre que $f$ et $g$ sont continues,
alors il y a \'egalit\'e si et seulement si la famille $(f,g)$ est li\'ee.

\Para{D\'efinition}

Soit $f \colon [a,b] \to E$ continue par morceaux.
La \emph{valeur moyenne} de $f$ sur $[a,b]$ est $\frac{1}{b-a}\int_a^b f$.

\subsection{Sommes de Riemann}

\Para{Th\'eor\`eme}

Soit $f \colon [a,b] \to E$ continue par morceaux.
Soit $(\sigma_n)_{n\in \N}$ une suite de subdivisions de $[a,b]$.
On note $\sigma_n$ la subdivision
\[a = x_{n,1} < x_{n,2} < \dots < x_{n,m_n} = b.\]
On note $h_n$ le pas de la subdivision $\sigma_n$, c.-\`a-d.
\[h_n = \max_{1\leq k<m_n} \BigPa{ x_{n,k+1} - x_{n,k} }.\]
On suppose que $h_n \to 0$.
Pour tout $1 \leq{} k < m_n$,
on choisit \'egalement un point $y_{n,k}$
dans l'intervalle $\intF{x_{n,k}, x_{n,{k+1}}}$
Alors
\[ \sum_{k=1}^{m_n-1} (x_{n,k+1} - x_{n,k}) f(y_{n,k}) \Toninf \int_a^b f(t) \D t. \]

\Para{Corollaire}

Soit $f \colon [0,1] \to E$ continue par morceaux.
Alors
\[\frac{1}{n+1} \sum_{k=0}^{n} f\left(\frac{k}{n}\right) \Toninf \int_0^1 f.\]

\subsection{Primitives}

\Para{D\'efinition}

Soit $f \colon I \to E$. Une \emph{primitive} $F$ de $f$ sur $I$
est une fonction d\'erivable sur $I$ telle que $F' = f$.

\Para{Proposition}

Soit $f \colon I \to E$ une fonction admettant deux primitives $F$ et $G$.
Alors \[\exists C\in E\+\forall x\in I\+ F(x) - G(x) = C.\]

Autrement dit, \emph{sur un intervalle}, deux primitives d'une
m\^eme fonctions diff\`erent d'une constante.

\Para{Th\'eor\`eme}

Soit $f \colon I \to E$ une fonction continue par morceaux et $a\in I$.
On pose \[\Fonction{\Phi}{I}{E}{x}{\int_a^x f(t) \D t.}\]
Alors $\Phi$ est continue sur $I$.
De plus, si $f$ est continue en $x\in I$, alors
$\Phi$ est d\'erivable en $x$ et $\Phi'(x) = f(x)$.

\Para{Corollaire}

Toute fonction continue sur un intervalle admet des primitives sur
cet intervalle.

% -----------------------------------------------------------------------------
\section{Exercices}

% -----------------------------------------------------------------------------
\par\pagebreak[1]\par
\paragraph{\href{https://psi.miomio.fr/exo/8761.pdf}{Exercice 1}}%
\hfill{\tiny 8761}%
\begingroup~

Trouver une fonction de $\R$ dans $\R$ d\'erivable mais dont la d\'eriv\'ee n'est pas continue.
\endgroup

% -----------------------------------------------------------------------------
\par\pagebreak[1]\par
\paragraph{Exercice 2}%
\hfill{\tiny 2830}%
\begingroup~

Soit $f \colon\R\to \Rps$ une fonction continue.
\begin{enumerate}
\item Montrer qu'il existe une infinit\'e de fonctions $g$ de $\R$ dans $\R$ telles que $\Abs{g} = f$.
\item Montrer qu'il existe exactement deux fonctions continues $g$ de $\R$ dans $\R$ telles que $\Abs{g} = f$.
\item Le r\'esultat pr\'ec\'edent est-il encore valable si $f$ est \`a valeurs dans $\Rp$?
\end{enumerate}
\endgroup

% -----------------------------------------------------------------------------
\par\pagebreak[1]\par
\paragraph{\href{https://psi.miomio.fr/exo/2312.pdf}{Exercice 3}}%
\hfill{\tiny 2312}%
\begingroup~

Soit $\Fn{f}{\intF{0,1}}{\R}$ continue telle que
\[ \int_0^1 f(t) \D t = \frac12. \]

Montrer que $f$ admet un point fixe.
\endgroup

% -----------------------------------------------------------------------------
\par\pagebreak[1]\par
\paragraph{Exercice 4}%
\hfill{\tiny 9928}%
\begingroup~

Montrer que la fonction $f \colon \Rp \to\R$ d\'efinie par $f(x) = \cos(\sqrt x)$
est de classe $\CC1$ sur $\Rp$.
\endgroup

% -----------------------------------------------------------------------------
\par\pagebreak[1]\par
\paragraph{\href{https://psi.miomio.fr/exo/0462.pdf}{Exercice 5}}%
\hfill{\tiny 0462}%
\begingroup~

\newcommand\hpi{\frac\pi2}
Soit $f \colon \Rp \to \R$ d\'erivable telle que $f(0) = \lim_{+\infty} f = 0$.
Montrer qu'il existe $x > 0$ tel que $f'(x) = 0$.
\endgroup

% -----------------------------------------------------------------------------
\par\pagebreak[1]\par
\paragraph{Exercice 6}%
\hfill{\tiny 7668}%
\begingroup~

Soit $f \colon \Rp \to\R$ de classe $\CC1$.
\begin{enumerate}
\item On suppose $\lim_{+\infty} f' = \ell\in \R$.
  Montrer que \[\frac{f(x)}{x} \To{x\to+\infty} \ell.\]
\item La r\'eciproque est-elle vraie?
\end{enumerate}
\endgroup

% -----------------------------------------------------------------------------
\par\pagebreak[1]\par
\paragraph{\href{https://psi.miomio.fr/exo/9967.pdf}{Exercice 7}}%
\hfill{\tiny 9967}%
\begingroup~

Soit $f \colon \intF{0,1} \to \R$ d\'efinie par
$f(x) = \frac{1}{x}$ si $x > 0$ et $f(0) = 0$.
Montrer que $f$ \emph{n'est pas} continue par morceaux sur $\intF{0,1}$.
\endgroup

% -----------------------------------------------------------------------------
\par\pagebreak[1]\par
\paragraph{\href{https://psi.miomio.fr/exo/1767.pdf}{Exercice 8}}%
\hfill{\tiny 1767}%
\begingroup~

Soit $(a,b)\in \R^2$ tels que $a < b$ et $\Fn{f}{\intF{a,b}}{\intF{a,b}}$.
\begin{enumerate}
\item
  Dans cette question, on suppose $f$ est continue.
  Montrer que $f$ admet un point fixe, c.-\`a-d.
  qu'il existe un $x\in\intF{a,b}$ tel que $f(x) = x$.
\item
  Dans cette question, on suppose $f$ croissante,
  mais pas n\'ecessairement continue.
  \begin{enumerate}
  \item
    Montrer que $E = \Ensemble{x\in\intF{a,b}}{f(x) \geq{} x}$
    admet une borne sup\'erieure~$\sigma$.
  \item
    Montrer que $f(\sigma) \geq{} \sigma$.
  \item
    Montrer que $f(\sigma) \leq{} \sigma$.
  \item
    En d\'eduire que $f$ admet un point fixe.
  \end{enumerate}
\item
  Dans cette question, on suppose $f$ d\'ecroissante,
  mais pas n\'ecessairement continue.
  La fonction $f$ admet-elle n\'ecessairement un point fixe?
\end{enumerate}
\endgroup

% -----------------------------------------------------------------------------
\par\pagebreak[1]\par
\paragraph{Exercice 9}%
\hfill{\tiny 1714}%
\begingroup~

D\'eveloppement limit\'e en $0^+$ \`a l'ordre~3 de
\[ f(x) = \frac{\arcsin \sqrt x}{\sqrt{x(1-x)}}. \]
\endgroup

% -----------------------------------------------------------------------------
\par\pagebreak[1]\par
\paragraph{\href{https://psi.miomio.fr/exo/0041.pdf}{Exercice 10}}%
\hfill{\tiny 0041}%
\begingroup~

Soit $P\in \R[X]$ un polyn\^ome r\'eel scind\'e.
Montrer que $P'$ est \'egalement scind\'e;
on pourra commencer par traiter le cas o\`u $P$ est scind\'e \`a racines simples.
\endgroup

% -----------------------------------------------------------------------------
\par\pagebreak[1]\par
\paragraph{Exercice 11}%
\hfill{\tiny 4730}%
\begingroup~

Soit $f \colon \R{} \to \R$ convexe telle que sa courbe repr\'esentative $\mathcal{C}_f$
admette une asympote $\Delta$ d'\'equation $y=ax+b$ en $+\infty$.
Montrer que $\mathcal{C}_f$ est situ\'ee au dessus de $\Delta$.
\endgroup

% -----------------------------------------------------------------------------
\par\pagebreak[1]\par
\paragraph{Exercice 12}%
\hfill{\tiny 2236}%
\begingroup~

\begin{enumerate}
\item Montrer que $\forall n\in \N$, $\forall x\in \R$, on a
  \[\left| e^x - \sum_{k=0}^n \frac{x^k}{k!} \right| \leq\frac{\Abs{x}^{n+1} e^{\Abs x}}{(n+1)!}\]
  On pourra utiliser les formules de Taylor.
\item En d\'eduire que \[\forall x\in \R\+ e^x = \sum_{k=0}^{+\infty} \frac{x^k}{k!}.\]
\end{enumerate}
\endgroup

% -----------------------------------------------------------------------------
\par\pagebreak[1]\par
\paragraph{Exercice 13}%
\hfill{\tiny 7697}%
\begingroup~

Soit $f \in\CC\infty(\R,\R)$ une fonction telle que $f(0) = 0$.
On d\'efinit $g$ sur $\R^*$ par $g(x) = \frac{f(x)}{x}$.

Montrer que $g$ est prolongeable par continuit\'e
en une fonction $\CC\infty$ sur $\R$.
\endgroup

% -----------------------------------------------------------------------------
\par\pagebreak[1]\par
\paragraph{\href{https://psi.miomio.fr/exo/0441.pdf}{Exercice 14}}%
\hfill{\tiny 0441}%
\begingroup~

Trouver toutes les applications continues
$\Fn{f}{\R\setminus\Acco{-1}}{\R}$ telles que
\[ \forall x \in \R\setminus\Acco{-1,-2} \+
f(x) = f\Pafrac{x}{2} + \frac{x}{(x+1)(x+2)}. \]
\endgroup

% -----------------------------------------------------------------------------
\par\pagebreak[1]\par
\paragraph{Exercice 15}%
\hfill{\tiny 5913}%
\begingroup~

D\'eterminer les fonctions $f \colon\R\to\R$ d\'erivables telles que
pour tout $x\in \R$ on ait $(f'(x))^2 = 1$.
\endgroup

\end{document}
