% autogenerated by ytex.rs

\documentclass{scrartcl}

\usepackage[francais]{babel}
\usepackage{geometry}
\usepackage{scrpage2}
\usepackage{lastpage}
\usepackage{multicol}
\usepackage{etoolbox}
\usepackage{xparse}
\usepackage{enumitem}
\usepackage{csquotes}
\usepackage{amsmath}
\usepackage{amsfonts}
\usepackage{amssymb}
\usepackage{mathrsfs}
\usepackage{stmaryrd}
\usepackage{dsfont}
% \usepackage{eurosym}
% \usepackage{numprint}
% \usepackage[most]{tcolorbox}
\usepackage{tikz}
% \usepackage{tkz-tab}
\usepackage[unicode]{hyperref}
\usepackage[ocgcolorlinks]{ocgx2}

\let\ifTwoColumns\iftrue
\def\Classe{$\Psi$2019--2020}

% reproducible builds
% LuaTeX: \pdfvariable suppressoptionalinfo 1023 \relax
\pdfinfoomitdate=1
\pdftrailerid{}

\newif\ifDisplaystyle
\everymath\expandafter{\the\everymath\ifDisplaystyle\displaystyle\fi}
\newcommand\DS{\displaystyle}

\clearscrheadfoot
\pagestyle{scrheadings}
\thispagestyle{empty}
\ohead{\Classe}
\ihead{\thepage/\pageref*{LastPage}}

\setlist[itemize,1]{label=\textbullet}
\setlist[itemize,2]{label=\textbullet}

\ifTwoColumns
  \geometry{margin=1cm,top=2cm,bottom=3cm,foot=1cm}
  \setlist[enumerate]{leftmargin=*}
  \setlist[itemize]{leftmargin=*}
\else
  \geometry{margin=3cm}
\fi

\makeatletter
\let\@author=\relax
\let\@date=\relax
\renewcommand\maketitle{%
    \begin{center}%
        {\sffamily\huge\bfseries\@title}%
        \ifx\@author\relax\else\par\medskip{\itshape\Large\@author}\fi
        \ifx\@date\relax\else\par\bigskip{\large\@date}\fi
    \end{center}\bigskip
    \ifTwoColumns
        \par\begin{multicols*}{2}%
        \AtEndDocument{\end{multicols*}}%
        \setlength{\columnsep}{5mm}
    \fi
}
\makeatother

\newcounter{ParaNum}
\NewDocumentCommand\Para{smo}{%
  \IfBooleanF{#1}{\refstepcounter{ParaNum}}%
  \paragraph{\IfBooleanF{#1}{{\tiny\arabic{ParaNum}~}}#2\IfNoValueF{#3}{ (#3)}}}

\newcommand\I{i}
\newcommand\mi{i}
\def\me{e}

\def\do#1{\expandafter\undef\csname #1\endcsname}
\docsvlist{Ker,sec,csc,cot,sinh,cosh,tanh,coth,th}
\undef\do

\DeclareMathOperator\ch{ch}
\DeclareMathOperator\sh{sh}
\DeclareMathOperator\th{th}
\DeclareMathOperator\coth{coth}
\DeclareMathOperator\cotan{cotan}
\DeclareMathOperator\argch{argch}
\DeclareMathOperator\argsh{argsh}
\DeclareMathOperator\argth{argth}

\let\epsilon=\varepsilon
\let\phi=\varphi
\let\leq=\leqslant
\let\geq=\geqslant
\let\subsetneq=\varsubsetneq
\let\emptyset=\varnothing

\newcommand{\+}{,\;}

\undef\C
\newcommand\ninf{{n\infty}}
\newcommand\N{\mathbb{N}}
\newcommand\Z{\mathbb{Z}}
\newcommand\Q{\mathbb{Q}}
\newcommand\R{\mathbb{R}}
\newcommand\C{\mathbb{C}}
\newcommand\K{\mathbb{K}}
\newcommand\Ns{\N^*}
\newcommand\Zs{\Z^*}
\newcommand\Qs{\Q^*}
\newcommand\Rs{\R^*}
\newcommand\Cs{\C^*}
\newcommand\Ks{\K^*}
\newcommand\Rp{\R^+}
\newcommand\Rps{\R^+_*}
\newcommand\Rms{\R^-_*}
\newcommand{\Rpinf}{\Rp\cup\Acco{+\infty}}

\undef\B
\newcommand\B{\mathscr{B}}

\undef\P
\DeclareMathOperator\P{\mathbb{P}}
\DeclareMathOperator\E{\mathbb{E}}
\DeclareMathOperator\Var{\mathbb{V}}

\DeclareMathOperator*\PetitO{o}
\DeclareMathOperator*\GrandO{O}
\DeclareMathOperator*\Sim{\sim}
\DeclareMathOperator\Tr{tr}
\DeclareMathOperator\Ima{Im}
\DeclareMathOperator\Ker{Ker}
\DeclareMathOperator\Sp{Sp}
\DeclareMathOperator\Diag{diag}
\DeclareMathOperator\Rang{rang}
\DeclareMathOperator*\Coords{Coords}
\DeclareMathOperator*\Mat{Mat}
\DeclareMathOperator\Pass{Pass}
\DeclareMathOperator\Com{Com}
\DeclareMathOperator\Card{Card}
\DeclareMathOperator\Racines{Racines}
\DeclareMathOperator\Vect{Vect}
\DeclareMathOperator\Id{Id}

\newcommand\DerPart[2]{\frac{\partial #1}{\partial #2}}

\def\T#1{{#1}^T}

\def\pa#1{({#1})}
\def\Pa#1{\left({#1}\right)}
\def\bigPa#1{\bigl({#1}\bigr)}
\def\BigPa#1{\Bigl({#1}\Bigr)}
\def\biggPa#1{\biggl({#1}\biggr)}
\def\BiggPa#1{\Biggl({#1}\Biggr)}

\def\pafrac#1#2{\pa{\frac{#1}{#2}}}
\def\Pafrac#1#2{\Pa{\frac{#1}{#2}}}
\def\bigPafrac#1#2{\bigPa{\frac{#1}{#2}}}
\def\BigPafrac#1#2{\BigPa{\frac{#1}{#2}}}
\def\biggPafrac#1#2{\biggPa{\frac{#1}{#2}}}
\def\BiggPafrac#1#2{\BiggPa{\frac{#1}{#2}}}

\def\cro#1{[{#1}]}
\def\Cro#1{\left[{#1}\right]}
\def\bigCro#1{\bigl[{#1}\bigr]}
\def\BigCro#1{\Bigl[{#1}\Bigr]}
\def\biggCro#1{\biggl[{#1}\biggr]}
\def\BiggCro#1{\Biggl[{#1}\Biggr]}

\def\abs#1{\mathopen|{#1}\mathclose|}
\def\Abs#1{\left|{#1}\right|}
\def\bigAbs#1{\bigl|{#1}\bigr|}
\def\BigAbs#1{\Bigl|{#1}\Bigr|}
\def\biggAbs#1{\biggl|{#1}\biggr|}
\def\BiggAbs#1{\Biggl|{#1}\Biggr|}

\def\acco#1{\{{#1}\}}
\def\Acco#1{\left\{{#1}\right\}}
\def\bigAcco#1{\bigl\{{#1}\bigr\}}
\def\BigAcco#1{\Bigl\{{#1}\Bigr\}}
\def\biggAcco#1{\biggl\{{#1}\biggr\}}
\def\BiggAcco#1{\Biggl\{{#1}\Biggr\}}

\def\ccro#1{\llbracket{#1}\rrbracket}
\def\Dcro#1{\llbracket{#1}\rrbracket}

\def\floor#1{\lfloor#1\rfloor}
\def\Floor#1{\left\lfloor{#1}\right\rfloor}

\def\sEnsemble#1#2{\mathopen\{#1\mid#2\mathclose\}}
\def\bigEnsemble#1#2{\bigl\{#1\bigm|#2\bigr\}}
\def\BigEnsemble#1#2{\Bigl\{#1\Bigm|#2\Bigr\}}
\def\biggEnsemble#1#2{\biggl\{#1\biggm|#2\biggr\}}
\def\BiggEnsemble#1#2{\Biggl\{#1\Biggm|#2\Biggr\}}
\let\Ensemble=\bigEnsemble

\newcommand\IntO[1]{\left]#1\right[}
\newcommand\IntF[1]{\left[#1\right]}
\newcommand\IntOF[1]{\left]#1\right]}
\newcommand\IntFO[1]{\left[#1\right[}

\newcommand\intO[1]{\mathopen]#1\mathclose[}
\newcommand\intF[1]{\mathopen[#1\mathclose]}
\newcommand\intOF[1]{\mathopen]#1\mathclose]}
\newcommand\intFO[1]{\mathopen[#1\mathclose[}

\newcommand\Fn[3]{#1\colon#2\to#3}
\newcommand\CC[1]{\mathscr{C}^{#1}}
\newcommand\D{\mathop{}\!\mathrm{d}}

\newcommand\longto{\longrightarrow}

\undef\M
\newcommand\M[3]{\mathrm{#1}_{#2}\pa{#3}}
\newcommand\MnR{\M{M}{n}{\R}}
\newcommand\MnC{\M{M}{n}{\C}}
\newcommand\MnK{\M{M}{n}{\K}}
\newcommand\GLnR{\M{GL}{n}{\R}}
\newcommand\GLnC{\M{GL}{n}{\C}}
\newcommand\GLnK{\M{GL}{n}{\K}}
\newcommand\DnR{\M{D}{n}{\R}}
\newcommand\DnC{\M{D}{n}{\C}}
\newcommand\DnK{\M{D}{n}{\K}}
\newcommand\SnR{\M{S}{n}{\R}}
\newcommand\AnR{\M{A}{n}{\R}}
\newcommand\OnR{\M{O}{n}{\R}}
\newcommand\SnRp{\mathrm{S}_n^+(\R)}
\newcommand\SnRpp{\mathrm{S}_n^{++}(\R)}

\newcommand\LE{\mathscr{L}(E)}
\newcommand\GLE{\mathscr{GL}(E)}
\newcommand\SE{\mathscr{S}(E)}
\renewcommand\OE{\mathscr{O}(E)}

\newcommand\ImplD{$\Cro\Rightarrow$}
\newcommand\ImplR{$\Cro\Leftarrow$}
\newcommand\InclD{$\Cro\subset$}
\newcommand\InclR{$\Cro\supset$}
\newcommand\notInclD{$\Cro{\not\subset}$}
\newcommand\notInclR{$\Cro{\not\supset}$}

\newcommand\To[1]{\xrightarrow[#1]{}}
\newcommand\Toninf{\To{\ninf}}

\newcommand\Norm[1]{\|#1\|}
\newcommand\Norme{{\Norm{\cdot}}}

\newcommand\Int[1]{\mathring{#1}}
\newcommand\Adh[1]{\overline{#1}}

\newcommand\Uplet[2]{{#1},\dots,{#2}}
\newcommand\nUplet[3]{(\Uplet{{#1}_{#2}}{{#1}_{#3}})}

\newcommand\Fonction[5]{{#1}\left|\begin{aligned}{#2}&\;\longto\;{#3}\\{#4}&\;\longmapsto\;{#5}\end{aligned}\right.}

\DeclareMathOperator\orth{\bot}
\newcommand\Orth[1]{{#1}^\bot}
\newcommand\PS[2]{\langle#1,#2\rangle}

\newcommand{\Tribu}{\mathscr{T}}
\newcommand{\Part}{\mathcal{P}}
\newcommand{\Pro}{\bigPa{\Omega,\Tribu}}
\newcommand{\Prob}{\bigPa{\Omega,\Tribu,\P}}

\newcommand\DEMO{$\spadesuit$}
\newcommand\DUR{$\spadesuit$}

\newenvironment{psmallmatrix}{\left(\begin{smallmatrix}}{\end{smallmatrix}\right)}

% -----------------------------------------------------------------------------

\newcommand\Rbar{\overline\R}
\newcommand\hpi{{\frac\pi2}}

\begin{document}
\title{Les infinis: $\infty$, $\omega$, $\aleph$, ...}
\maketitle

\section{En Analyse}

\subsection{Limites: \enquote{$\infty$} n'est qu'une notation pratique}

L'affirmation
\[ \tag{$1$} \lim_{n \to +\infty} u_n = \ell{} \]
n'est qu'une abr\'eviation pour
\[ \tag{$1'$} \forall \epsilon>0 \+ \exists n_0\in \N{} \+ \forall n\geq n_0 \+ \Abs{u_n-\ell} \leq{} \epsilon. \]

De m\^eme,
\[ \tag{$2$} \lim_{x \to 0^+} f(x) = +\infty{} \]
signifie \emph{par d\'efinition}
\[ \tag{$2'$} \forall A\in \R{} \+ \exists \eta>0 \+ \forall x\in\intO{0,\eta} \+ f(x) > A. \]

Dernier exemple,
\[ \tag{$3$} \lim_{x \to +\infty} f(x) = \frac{\pi}{2} \]
est d\'efini comme
\[ \tag{$3'$} \forall \epsilon>0 \+ \exists A\in \R{} \+ \forall x>A \+ \Abs{f(x)-\frac\pi2} \leq{} \epsilon. \]

\subsection{Droite r\'eelle achev\'ee}

Soit $\alpha$ et $\omega$ qui n'appartiennent pas \`a $\R$,
et notons $\Rbar = \R{} \cup{} \acco{\alpha,\omega}$.

Consid\'erons l'application
\[
  \Fonction{f}{\IntF{-\hpi,\hpi}}{\Rbar}{x}{\begin{cases}
      \hfil \alpha{} & \text{si } x = -\hpi, \\
      \tan x & \text{si } -\hpi < x < \hpi, \\
      \hfil \omega{} & \text{si } x = \hpi.
  \end{cases}}
\]

\begin{center}
  \begin{tikzpicture}
    \begin{scope}
      \clip (-4,0) rectangle (4, -2) ;
      \draw[thick] (0,0) circle (1.5cm);
    \end{scope}
    \node at (-1.5,0) [above] {$-\hpi$} ;
    \node at (1.5,0) [above] {$\hpi$} ;

    \node at (0,-1.5) [above left] {$0$} ;
    \node at (0,-2) [below] {$0$} ;
    \filldraw (0,-1.5) circle (1.2pt) ;
    \draw[dotted] (0,0) -- (0,-1.5) ;
    \draw[->,>=stealth] (0,-1.5) -- (0,-2) ;

    \node at (1.0607,-1.0607) [above] {$x$} ;
    \node at (2,-2) [below] {$f(x)$} ;
    \filldraw (1.0607,-1.0607) circle (1.2pt) ;
    \draw[dotted] (0,0) -- (1.0607,-1.0607) ;
    \draw[->,>=stealth] (1.0607, -1.0607) -- (2, -2) ;

    \node at (-1.31956,-0.65) [left] {$y$} ;
    \node at (-3.7,-2) [below] {$f(y)$} ;
    \filldraw (-1.31956,-0.71328) circle (1.2pt) ;
    \draw[dotted] (0,0) -- (-1.31956,-0.71328)  ;
    \draw[->,>=stealth] (-1.31956,-0.71328) -- (-3.7,-2) ;

    \draw[thick] (-4,-2) -- (4,-2) ;
    \node at (4,-2) [right] {$\R$} ;
  \end{tikzpicture}
\end{center}

Il s'agit d'une bijection.
De plus, sa restriction
\[ \Fonction{g}{\IntO{-\hpi,\hpi}}{\R}{x}{f(x)} \]
est un \emph{hom\'eomorphisme}, c.-\`a-d. une bijection continue dont la r\'eciproque est \'egalement continue.

Il est possible de d\'efinir une \emph{structure m\'etrique} de sorte que $f$ soit un hom\'eomorphisme.

Ainsi, du point de vue topologique, $\Rbar$ est identifiable au segment $\intF{-\hpi,\hpi}$.

De m\^eme, si l'on \'etend la relation d'ordre $\leq$ sur $\R$ par $\forall{} x \in{} \R$, $\alpha{} \leq{} x$ et $x \leq{} \omega$,
l'ensemble $\Rbar$ devient un ensemble (totalement) ordonn\'e.
On constate que $f$ et $f^{-1}$ sont croissantes, donc $\Rbar$ et $\intF{-\hpi,\hpi}$ sont \'egalement identifiables en tant qu'ensembles ordonn\'es.

En passant de $\R$ \`a $\Rbar$, on conserve donc les notions d'ordre et de topologie, mais $\Rbar$ n'est pas un corps.

On note usuellement \enquote{$-\infty$} et \enquote{$+\infty$}
\`a la place de $\alpha$ et $\omega$.

Certaines fonctions r\'eelles $\R{} \to \R$ s'\'etendent en des fonctions $\Rbar \to \Rbar$, par exemple $f(x) = x^3$.

\subsection{Compactifi\'e d'Alexandrov de la droite r\'eelle}

Soit $\omega$ qui n'appartient pas \`a $\R$.

\begin{center}
  \begin{tikzpicture}
    \draw[thick] (0,0) circle (1.5cm);
    \node at (0,1.5) [above] {$N$};
    \filldraw (0,1.5) circle (1.5pt) ;
    \draw[dotted,->,>=stealth] (0,1.5) -- (0,-1.5) ;
    \draw[thick] (-4,-1.5) -- (4,-1.5) ;
    \node at (4,-2) [right] {$\R$} ;

    \filldraw (-1.4676,0.31005) circle (1.2pt) ;
    \draw[dotted] (0,1.5) -- (-1.4676,0.31005) ;
    \draw[->,>=stealth] (-1.4676,0.31005) -- (-3.7,-1.5) ;
    \filldraw (-1.5,-2.7555e-16) circle (1.2pt) ;
    \draw[dotted] (0,1.5) -- (-1.5,-2.7555e-16) ;
    \draw[->,>=stealth] (-1.5,-2.7555e-16) -- (-3,-1.5) ;
    \filldraw (-1.3846,-0.57692) circle (1.2pt) ;
    \draw[dotted] (0,1.5) -- (-1.3846,-0.57692) ;
    \draw[->,>=stealth] (-1.3846,-0.57692) -- (-2,-1.5) ;
    \filldraw (-0.9,-1.2) circle (1.2pt) ;
    \draw[dotted] (0,1.5) -- (-0.9,-1.2) ;
    \draw[->,>=stealth] (-0.9,-1.2) -- (-1,-1.5) ;
    \filldraw (0.9,-1.2) circle (1.2pt) ;
    \draw[dotted] (0,1.5) -- (0.9,-1.2) ;
    \draw[->,>=stealth] (0.9,-1.2) -- (1,-1.5) ;
    \filldraw (1.3846,-0.57692) circle (1.2pt) ;
    \draw[dotted] (0,1.5) -- (1.3846,-0.57692) ;
    \draw[->,>=stealth] (1.3846,-0.57692) -- (2,-1.5) ;
    \filldraw (1.5,9.1849e-17) circle (1.2pt) ;
    \draw[dotted] (0,1.5) -- (1.5,9.1849e-17) ;
    \draw[->,>=stealth] (1.5,9.1849e-17) -- (3,-1.5) ;
    \filldraw (1.4676,0.31005) circle (1.2pt) ;
    \draw[dotted] (0,1.5) -- (1.4676,0.31005) ;
    \draw[->,>=stealth] (1.4676,0.31005) -- (3.7,-1.5) ;
  \end{tikzpicture}
\end{center}

Notons $S^1$ le cercle unit\'e. L'application
\[ \Fonction{f}{S^1}{\R{} \cup{} \acco{\omega}}
  {\begin{pmatrix} \sin\theta \\ \cos\theta \end{pmatrix}}{
    \begin{cases}
      2\cotan(\theta/2) & \text{si } \theta{} \neq{} 0 \pmod{2\pi} \\
      \hfil \omega{} & \text{si } \theta{} = 0 \pmod{2\pi}
    \end{cases}
} \]
est un hom\'eomorphisme.

On note usuellement \enquote{$\infty$} \`a la place de $\omega$.

Toutes les fractions rationnelles s'\'etendent \`a des fonctions $\R{} \cup{} \acco{\infty} \to \R{} \cup{} \acco{\infty}$.

\subsection{Sph\`ere de Riemann}

On peut adjoindre un point $\omega$ \`a $\C$, et on obtient un espace hom\'eomorphe \`a la sph\`ere $S^2$ de $\R^3$ via la projection st\'er\'eographique.

\section{En Alg\`ebre}

Et non!

\section{En Th\'eorie des ensembles}

On se place dans le cadre de la th\'eorie axiomatique ZFC, i.e. Zermelo-Fraenkel avec l'axiome du choix.

\Para{D\'efinition}
Soit $X$ et $Y$ deux ensembles.
On dit que $X$ et $Y$ sont \emph{\'equipotents}
si et seulement s'il existe une application $\Fn{\phi}{X}{Y}$ bijective.

\Para{Proposition}
Il s'agit d'une \enquote{relation} d'\'equivalence.

\Para{Remarque}
Il n'existe pas d'ensemble de tous les ensembles, d'o\`u les guillemets.

\subsection{Ensembles finis}

\Para{D\'efinition}

Soit $X$ un ensemble.
On dit que $X$ est \emph{fini} s'il existe un entier naturel $n\in \N$
tel que $X$ est \'equipotent \`a $\Dcro{1,n}$.

\Para{Lemme}

Soit $(n,p)\in \N^2$.
Si $\Dcro{1,n}$ et $\Dcro{1,p}$ sont \'equipotents, alors $n=p$.
Voir l'exercice~3.

\Para{D\'efinition}

Soit $X$ un ensemble fini. On note $\Card(X)$ l'unique entier $n\in \N$
tel qu'il existe une application bijective $\Fn{\varphi}{\Dcro{1,n}}{X}$.

\subsection{Ensembles infinis}

\Para{D\'efinition}

Soit $X$ un ensemble.
On dit que $X$ est \emph{infini} si et seulement si $X$ n'est pas fini.

\Para{Th\'eor\`eme}

Soit $X$ un ensemble. Les conditions suivantes sont \'equivalentes:
\begin{enumerate}
\item
  $X$ est infini;
\item
  il existe une suite $(x_n)_{n\in \N}$ \`a valeurs dans $X$ dont les \'el\'ements sont deux \`a deux distincts;
\item
  il existe une injection $\N{} \to X$;
\item
  il existe une partie stricte $Y \subsetneq X$ telle que $X$ et $Y$ sont \'equipotents.
\end{enumerate}

\Para{Th\'eor\`eme}

Soit $X$ un ensemble.
Alors $X$ et $\mathcal{P}(X)$ ne sont pas \'equipotents.

\Para{Corollaire}

Certains ensembles infinis sont plus gros que d'autres!

\subsection{D\'enombrabilit\'e}

\Para{D\'efinitions}

Soit $X$ un ensemble.
\begin{itemize}
\item
  On dit que $X$ est \emph{d\'enombrable} si $X$ est \'equipotent \`a $\N$.
\item
  On dit que $X$ est \emph{au plus d\'enombrable} si $X$ est fini ou d\'enombrable.
\end{itemize}

\Para{Lemme}

Soit $A$ une partie de $\N$. Alors
\begin{itemize}
\item
  si $A$ est major\'ee, alors $A$ est finie;
\item
  si $A$ est n'est pas major\'ee, alors $A$ est d\'enombrable.
\end{itemize}

\Para{Th\'eor\`eme}

Soit $X$ un ensemble.
\begin{itemize}
\item
  Si $X$ est d\'enombrable, alors il existe une suite $(x_n)$ injective telle que $X = \Ensemble{x_n}{n\in \N}$.
\item
  Si $X$ est au plus d\'enombrable et non vide, alors il existe une suite $(x_n)$ telle que $X = \Ensemble{x_n}{n\in \N}$.
\item
  Si $X = \Ensemble{x_n}{n\in \N}$ o\`u $(x_n)$ est une suite, alors $X$ est au plus d\'enombrable et non vide.
\end{itemize}

\Para{Th\'eor\`eme}
\begin{itemize}
\item
  $\N$, $\Z$ et $\Q$ sont d\'enombrables.
\item
  Si $X$ est d\'enombrable et $k\geq1$, alors $X^k$ est d\'enombrable.
\item
  $\mathcal{P}(\N)$, $\R$ et $\{0,1\}^\N$ sont infinis non-d\'enombrables.
\end{itemize}

\Para{Proposition}

Soit $(X_i)_{i\in I}$ est une famille d'ensembles. On suppose que
\begin{itemize}
\item
  $I$ est au plus d\'enombrable et
\item
  pour tout $i\in I$, $X_i$ est au plus d\'enombrable.
\end{itemize}

Alors l'union $\bigcup_{i\in I} X_i$ est au plus d\'enombrable.

\subsection{Pour aller (un petit peu) plus loin}

\Para{Th\'eor\`eme}[Cantor-Bernstein]
Soit $X$ et $Y$ deux ensembles.
On suppose qu'il existe une injection $\Fn{f}{X}{Y}$
et une injection $\Fn{g}{Y}{X}$.
Alors $X$ et $Y$ sont \'equipotents.

\Para{Th\'eor\`eme}
Soit $X$ et $Y$ deux ensembles.
Alors il existe une injection de $X \to Y$,
ou une injection $Y \to X$.

\Para{Corollaire}
Pour deux ensembles quelconques $X$ et $Y$,
notons $X \preceq Y$ si et seulement s'il existe une injection $X \to Y$
et $X \sim Y$ si et seulement si $X$ et $Y$ sont \'equipotents.

Alors pour tous ensembles $X$, $Y$ et $Z$, on a:
\begin{itemize}
\item
  $X \preceq X$;
\item
  si $X \preceq Y$ et $Y \preceq Z$, alors $X \preceq Z$;
\item
  si $X \preceq Y$ et $Y \preceq X$, alors $X \sim Y$.
\item
  $X \preceq Y$ ou $Y \preceq X$;
\item
  $X \prec \mathcal{P}(X)$, c.-\`a-d. $X \preceq \mathcal{P}(X)$ mais $\mathcal{P}(X) \not\preceq X$;
\item
  $X$ est infini si et seulement si $\N{} \preceq X$.
\item
  $\N{} \prec \R$.
\end{itemize}

\section{Exercices}

% -----------------------------------------------------------------------------
\par\pagebreak[1]\par
\paragraph{Exercice 1}%
\hfill{\tiny 4736}%
\begingroup~

Soit $X$ et $Y$ deux ensembles.
Montrer qu'on a \'equivalence entre:
\begin{enumerate}
\item
  il existe une injection $X \to Y$;
\item
  il existe une surjection $Y \to X$.
\end{enumerate}
\endgroup

% -----------------------------------------------------------------------------
\par\pagebreak[1]\par
\paragraph{\href{https://psi.miomio.fr/exo/7455.pdf}{Exercice 2}}%
\hfill{\tiny 7455}%
\begingroup~

Montrer que $\Q[X]$ et $\Q(X)$ sont d\'enombrables.
\endgroup

% -----------------------------------------------------------------------------
\par\pagebreak[1]\par
\paragraph{\href{https://psi.miomio.fr/exo/2588.pdf}{Exercice 3}}%
\hfill{\tiny 2588}%
\begingroup~

Soit $H_p$ la propri\'et\'e
\og pour tout $n\in \N$ et pour toute application $\Fn{\varphi}{\ccro{1,n}}{\ccro{1,p}}$ injective, on a $n\leq p$\fg.
\begin{enumerate}
\item
  Montrer $H_0$ et $H_1$.
\item
  Montrer que $H$ est h\'er\'editaire.
\item
  Soit $(n,p)\in \N^2$.
  En d\'eduire que si $\Fn{\varphi}{\ccro{1,n}}{\ccro{1,p}}$ est bijective, alors $n=p$.
\end{enumerate}
\endgroup

% -----------------------------------------------------------------------------
\par\pagebreak[1]\par
\paragraph{Exercice 4}%
\hfill{\tiny 0406}%
\begingroup~

Soit $X$ et $Y$ deux ensembles finis.

\begin{enumerate}
\item
  Montrer que s'il existe une injection $X \to Y$, alors $\Card(X) \leq{} \Card(Y)$.
\item
  Montrer que s'il existe une surjection $X \to Y$, alors $\Card(X) \geq{} \Card(Y)$.
\item
  Montrer que s'il existe une bijection $X \to Y$, alors $\Card(X) = \Card(Y)$.
\end{enumerate}
\endgroup

% -----------------------------------------------------------------------------
\par\pagebreak[1]\par
\paragraph{Exercice 5}%
\hfill{\tiny 4253}%
\begingroup~

Existe-t-il un ensemble $X$ tel que $\N{} \prec X \prec \R$?
\endgroup

\end{document}
