% autogenerated by ytex.rs

\documentclass{scrartcl}

\usepackage[francais]{babel}
\usepackage{geometry}
\usepackage{scrpage2}
\usepackage{lastpage}
\usepackage{ragged2e}
\usepackage{multicol}
\usepackage{etoolbox}
\usepackage{xparse}
\usepackage{enumitem}
\usepackage{csquotes}
\usepackage{amsmath}
\usepackage{amsfonts}
\usepackage{amssymb}
\usepackage{mathrsfs}
\usepackage{stmaryrd}
\usepackage{dsfont}
\usepackage{eurosym}
\usepackage{numprint}
\usepackage[most]{tcolorbox}
\usepackage{tikz}
\usepackage{tkz-tab}
\usepackage[unicode]{hyperref}
\usepackage[ocgcolorlinks]{ocgx2}

\let\ifTwoColumns\iftrue
\def\Classe{$\Psi$2019--2020}

% reproducible builds
% LuaTeX: \pdfvariable suppressoptionalinfo 1023 \relax
\pdfinfoomitdate=1
\pdftrailerid{}

\newif\ifDisplaystyle
\everymath\expandafter{\the\everymath\ifDisplaystyle\displaystyle\fi}
\newcommand\DS{\displaystyle}

\clearscrheadfoot
\pagestyle{scrheadings}
\thispagestyle{empty}
\ohead{\Classe}
\ihead{\thepage/\pageref*{LastPage}}

\setlist[itemize,1]{label=\textbullet}
\setlist[itemize,2]{label=\textbullet}

\ifTwoColumns
  \geometry{margin=1cm,top=2cm,bottom=3cm,foot=1cm}
  \setlist[enumerate]{leftmargin=*}
  \setlist[itemize]{leftmargin=*}
\else
  \geometry{margin=3cm}
\fi

\makeatletter
\let\@author=\relax
\let\@date=\relax
\renewcommand\maketitle{%
    \begin{center}%
        {\sffamily\huge\bfseries\@title}%
        \ifx\@author\relax\else\par\medskip{\itshape\Large\@author}\fi
        \ifx\@date\relax\else\par\bigskip{\large\@date}\fi
    \end{center}\bigskip
    \ifTwoColumns
        \par\begin{multicols*}{2}%
        \AtEndDocument{\end{multicols*}}%
        \setlength{\columnsep}{5mm}
    \fi
}
\makeatother

\newcounter{ParaNum}
\NewDocumentCommand\Para{smo}{%
  \IfBooleanF{#1}{\refstepcounter{ParaNum}}%
  \paragraph{\IfBooleanF{#1}{{\tiny\arabic{ParaNum}~}}#2\IfNoValueF{#3}{ (#3)}}}

\newcommand\I{i}
\newcommand\mi{i}
\def\me{e}

\def\do#1{\expandafter\undef\csname #1\endcsname}
\docsvlist{Ker,sec,csc,cot,sinh,cosh,tanh,coth,th}
\undef\do

\DeclareMathOperator\ch{ch}
\DeclareMathOperator\sh{sh}
\DeclareMathOperator\th{th}
\DeclareMathOperator\coth{coth}
\DeclareMathOperator\cotan{cotan}
\DeclareMathOperator\argch{argch}
\DeclareMathOperator\argsh{argsh}
\DeclareMathOperator\argth{argth}

\let\epsilon=\varepsilon
\let\phi=\varphi
\let\leq=\leqslant
\let\geq=\geqslant
\let\subsetneq=\varsubsetneq
\let\emptyset=\varnothing

\newcommand{\+}{,\;}

\undef\C
\newcommand\ninf{{n\infty}}
\newcommand\N{\mathbb{N}}
\newcommand\Z{\mathbb{Z}}
\newcommand\Q{\mathbb{Q}}
\newcommand\R{\mathbb{R}}
\newcommand\C{\mathbb{C}}
\newcommand\K{\mathbb{K}}
\newcommand\Ns{\N^*}
\newcommand\Zs{\Z^*}
\newcommand\Qs{\Q^*}
\newcommand\Rs{\R^*}
\newcommand\Cs{\C^*}
\newcommand\Ks{\K^*}
\newcommand\Rp{\R^+}
\newcommand\Rps{\R^+_*}
\newcommand\Rms{\R^-_*}
\newcommand{\Rpinf}{\Rp\cup\Acco{+\infty}}

\undef\B
\newcommand\B{\mathscr{B}}

\undef\P
\DeclareMathOperator\P{\mathbb{P}}
\DeclareMathOperator\E{\mathbb{E}}
\DeclareMathOperator\Var{\mathbb{V}}

\DeclareMathOperator*\PetitO{o}
\DeclareMathOperator*\GrandO{O}
\DeclareMathOperator*\Sim{\sim}
\DeclareMathOperator\Tr{tr}
\DeclareMathOperator\Ima{Im}
\DeclareMathOperator\Ker{Ker}
\DeclareMathOperator\Sp{Sp}
\DeclareMathOperator\Diag{diag}
\DeclareMathOperator\Rang{rang}
\DeclareMathOperator*\Coords{Coords}
\DeclareMathOperator*\Mat{Mat}
\DeclareMathOperator\Pass{Pass}
\DeclareMathOperator\Com{Com}
\DeclareMathOperator\Card{Card}
\DeclareMathOperator\Racines{Racines}
\DeclareMathOperator\Vect{Vect}
\DeclareMathOperator\Id{Id}

\newcommand\DerPart[2]{\frac{\partial #1}{\partial #2}}

\def\T#1{{#1}^T}

\def\pa#1{({#1})}
\def\Pa#1{\left({#1}\right)}
\def\bigPa#1{\bigl({#1}\bigr)}
\def\BigPa#1{\Bigl({#1}\Bigr)}
\def\biggPa#1{\biggl({#1}\biggr)}
\def\BiggPa#1{\Biggl({#1}\Biggr)}

\def\pafrac#1#2{\pa{\frac{#1}{#2}}}
\def\Pafrac#1#2{\Pa{\frac{#1}{#2}}}
\def\bigPafrac#1#2{\bigPa{\frac{#1}{#2}}}
\def\BigPafrac#1#2{\BigPa{\frac{#1}{#2}}}
\def\biggPafrac#1#2{\biggPa{\frac{#1}{#2}}}
\def\BiggPafrac#1#2{\BiggPa{\frac{#1}{#2}}}

\def\cro#1{[{#1}]}
\def\Cro#1{\left[{#1}\right]}
\def\bigCro#1{\bigl[{#1}\bigr]}
\def\BigCro#1{\Bigl[{#1}\Bigr]}
\def\biggCro#1{\biggl[{#1}\biggr]}
\def\BiggCro#1{\Biggl[{#1}\Biggr]}

\def\abs#1{\mathopen|{#1}\mathclose|}
\def\Abs#1{\left|{#1}\right|}
\def\bigAbs#1{\bigl|{#1}\bigr|}
\def\BigAbs#1{\Bigl|{#1}\Bigr|}
\def\biggAbs#1{\biggl|{#1}\biggr|}
\def\BiggAbs#1{\Biggl|{#1}\Biggr|}

\def\acco#1{\{{#1}\}}
\def\Acco#1{\left\{{#1}\right\}}
\def\bigAcco#1{\bigl\{{#1}\bigr\}}
\def\BigAcco#1{\Bigl\{{#1}\Bigr\}}
\def\biggAcco#1{\biggl\{{#1}\biggr\}}
\def\BiggAcco#1{\Biggl\{{#1}\Biggr\}}

\def\ccro#1{\llbracket{#1}\rrbracket}
\def\Dcro#1{\llbracket{#1}\rrbracket}

\def\floor#1{\lfloor#1\rfloor}
\def\Floor#1{\left\lfloor{#1}\right\rfloor}

\def\sEnsemble#1#2{\mathopen\{#1\mid#2\mathclose\}}
\def\bigEnsemble#1#2{\bigl\{#1\bigm|#2\bigr\}}
\def\BigEnsemble#1#2{\Bigl\{#1\Bigm|#2\Bigr\}}
\def\biggEnsemble#1#2{\biggl\{#1\biggm|#2\biggr\}}
\def\BiggEnsemble#1#2{\Biggl\{#1\Biggm|#2\Biggr\}}
\let\Ensemble=\bigEnsemble

\newcommand\IntO[1]{\left]#1\right[}
\newcommand\IntF[1]{\left[#1\right]}
\newcommand\IntOF[1]{\left]#1\right]}
\newcommand\IntFO[1]{\left[#1\right[}

\newcommand\intO[1]{\mathopen]#1\mathclose[}
\newcommand\intF[1]{\mathopen[#1\mathclose]}
\newcommand\intOF[1]{\mathopen]#1\mathclose]}
\newcommand\intFO[1]{\mathopen[#1\mathclose[}

\newcommand\Fn[3]{#1\colon#2\to#3}
\newcommand\CC[1]{\mathscr{C}^{#1}}
\newcommand\D{\mathop{}\!\mathrm{d}}

\newcommand\longto{\longrightarrow}

\undef\M
\newcommand\M[3]{\mathrm{#1}_{#2}\pa{#3}}
\newcommand\MnR{\M{M}{n}{\R}}
\newcommand\MnC{\M{M}{n}{\C}}
\newcommand\MnK{\M{M}{n}{\K}}
\newcommand\GLnR{\M{GL}{n}{\R}}
\newcommand\GLnC{\M{GL}{n}{\C}}
\newcommand\GLnK{\M{GL}{n}{\K}}
\newcommand\DnR{\M{D}{n}{\R}}
\newcommand\DnC{\M{D}{n}{\C}}
\newcommand\DnK{\M{D}{n}{\K}}
\newcommand\SnR{\M{S}{n}{\R}}
\newcommand\AnR{\M{A}{n}{\R}}
\newcommand\OnR{\M{O}{n}{\R}}
\newcommand\SnRp{\mathrm{S}_n^+(\R)}
\newcommand\SnRpp{\mathrm{S}_n^{++}(\R)}

\newcommand\LE{\mathscr{L}(E)}
\newcommand\GLE{\mathscr{GL}(E)}
\newcommand\SE{\mathscr{S}(E)}
\renewcommand\OE{\mathscr{O}(E)}

\newcommand\ImplD{$\Cro\Rightarrow$}
\newcommand\ImplR{$\Cro\Leftarrow$}
\newcommand\InclD{$\Cro\subset$}
\newcommand\InclR{$\Cro\supset$}
\newcommand\notInclD{$\Cro{\not\subset}$}
\newcommand\notInclR{$\Cro{\not\supset}$}

\newcommand\To[1]{\xrightarrow[#1]{}}
\newcommand\Toninf{\To{\ninf}}

\newcommand\Norm[1]{\|#1\|}
\newcommand\Norme{{\Norm{\cdot}}}

\newcommand\Int[1]{\mathring{#1}}
\newcommand\Adh[1]{\overline{#1}}

\newcommand\Uplet[2]{{#1},\dots,{#2}}
\newcommand\nUplet[3]{(\Uplet{{#1}_{#2}}{{#1}_{#3}})}

\newcommand\Fonction[5]{{#1}\left|\begin{aligned}{#2}&\;\longto\;{#3}\\{#4}&\;\longmapsto\;{#5}\end{aligned}\right.}

\DeclareMathOperator\orth{\bot}
\newcommand\Orth[1]{{#1}^\bot}
\newcommand\PS[2]{\langle#1,#2\rangle}

\newcommand{\Tribu}{\mathscr{T}}
\newcommand{\Part}{\mathcal{P}}
\newcommand{\Pro}{\bigPa{\Omega,\Tribu}}
\newcommand{\Prob}{\bigPa{\Omega,\Tribu,\P}}

\newcommand\DEMO{$\spadesuit$}
\newcommand\DUR{$\spadesuit$}

\newenvironment{psmallmatrix}{\left(\begin{smallmatrix}}{\end{smallmatrix}\right)}


% -----------------------------------------------------------------------------

\usepackage{tikz}
\usepackage{tkz-tab}

\newcommand{\gammaI}{\Fn{\gamma}{I}}
\newcommand{\gammaIE}{\gammaI\EE}
\newcommand{\deltaJ}{\Fn{\delta}{J}}
\newcommand{\deltaJE}{\deltaJ\EE}
\newcommand{\EE}{\mathscr{E}}
\renewcommand\Vec{\overrightarrow}

\tikzset{
  Axe/.style = {},
  Vecs/.style = { thick, <->, >=stealth },
  Vec/.style = { thick, ->, >=stealth },
  Curve/.style = { very thick },
  Curve name/.style = {},
  gamma p pos/.style = { below },
  gamma q pos/.style = { left },
}

\begin{document}
\title{Courbes param\'etr\'ees}
\maketitle

\Para{Notations}
\begin{itemize}
\item $I$ et $J$ d\'esignent des intervalles non vides et non r\'eduits \`a un point de $\R$;
\item $\EE$ d\'esigne un espace euclidien de dimension~2.
\end{itemize}

% -----------------------------------------------------------------------------
\section{G\'en\'eralit\'es}

\Para{D\'efinition}
On appelle \emph{arc param\'etr\'e} de $\EE$ toute application $\gammaIE$ de classe $\CC1$.
On parle parfois de \emph{courbe param\'etr\'ee} ou d'\emph{arc orient\'e} \`a la place d'arc param\'etr\'e.

\Para{D\'efinition}[Interpr\'etation cin\'ematique]
Un \emph{mouvement ponctuel} est un arc param\'etr\'e $\gamma$ dans lequel la variable $t$ est le temps.
\begin{itemize}
\item $\gamma(t)$ est la \emph{position} du point mobile \`a l'instant~$t$;
\item $\gamma'(t)$ est la \emph{vitesse} du point mobile \`a l'instant~$t$;
\item $\gamma''(t)$ est l'\emph{acc\'el\'eration} du point mobile \`a l'instant~$t$;
\item $\Gamma=\gamma(I)$ s'appelle la \emph{trajectoire} du point mobile.
\end{itemize}

\begin{center}
  \begin{tikzpicture}
    \draw[Curve] (-1.5,-1) .. controls +(60:1) and +(-25:-1) .. (1,1) .. controls +(-25:1) and +(0,0) .. (2,-1) ;
    \draw[Vec] (1,1) -- +(-25:2) ;
    \filldraw (1,1) circle (1pt) node [below,xshift=-5pt] { $\gamma(t)$ } ;
    \draw (2.25,0.8) node { $\gamma'(t)$ } ;
    \draw[Curve name] (-1.5,-0.5) node { $\Gamma$ } ;
  \end{tikzpicture}
\end{center}

% -----------------------------------------------------------------------------
\section{Sym\'etries}

\Para{Proposition}
Soit $\Fn{\gamma}{I}{\EE}$ une courbe param\'etr\'ee.
Soit $\Fn{\theta}{I}{I}$ et $\Fn{s}{\EE}{\EE}$ deux applications.
Si $\gamma\circ \theta=s\circ \gamma$, alors la trajectoire $\Gamma$ de $\gamma$ est stable par~$s$.

\Para{Corollaire}
En particulier, si $I$ est sym\'etrique par rapport \`a z\'ero et si
\[ \gamma(-t) = s\bigPa{\gamma(t)}, \]
Alors la trajectoire de $\gamma$ est stable par $s$; on peut alors r\'eduire l'\'etude \`a $I\cap \R^+$.

\Para{Exemples de sym\'etries}
Soit $\Fn{\gamma}{\R}{\R^2}$ un arc param\'etr\'e.
On note $\gamma(t) = \bigPa{x(t),y(t)}$.
\begin{enumerate}
\item Si $x(-t) = x(t)$ et $y(-t) = y(t)$,
  alors il suffit d'\'etudier $\gamma$ sur $\R^+$.
\item Si $x(-t) = x(t)$ et $y(-t) = -y(t)$,
  alors la trajectoire de $\gamma$ est sym\'etrique
  par rapport \`a l'axe des abscisses et
  il suffit d'\'etudier $\gamma$ sur $\R^+$.
\item Si $x(\theta(t)) = -y(t)$ et $y(\theta(t)) = x(t)$,
  alors la trajectoire de $\gamma$ est invariante
  par rotation d'angle $\frac\pi2$.
\item Si
  \[ \left\{ \begin{aligned}
        x(t+17) &= -\frac12 x(t) - \frac{\sqrt3}{2} y(t) \\
        y(t+17) &= \frac{\sqrt3}2 x(t) - \frac1{2} y(t)
  \end{aligned} \right. \]
  alors la trajectoire de $\gamma$ est stable
  par la rotation d'angle $2\pi/3$ et il suffit d'\'etudier
  $\gamma$ sur $\intF{0,17}$.
\item etc.
\end{enumerate}

% -----------------------------------------------------------------------------
\section{Variations}

L'outil principal pour le trac\'e de l'allure de $\gamma$ est le tableau regroupant les variations de $x$ et de $y$.
Par exemple, pour la stropho\"ide droite (cf. exercice~1):
\begin{center}
  \begin{tikzpicture}
    \tkzTabInit
    {$t$/1, $x'$/1, $x$/1.5, $y$/1.5, $y'$/1}
    {$0$, $\sqrt{\sqrt{5}-2}$, $+\infty$}
    \tkzTabLine{z,,-}
    \tkzTabVar{+/$1$,R,-/$-1$}
    \tkzTabVar{-/$0$,+/ ,-/$-\infty$}
    \tkzTabLine{$1$, +, z, -}
  \end{tikzpicture}
\end{center}

% -----------------------------------------------------------------------------
\section{\'Etude locale}

% -----------------------------------------------------------------------------
\subsection{Tangentes}

\Para{D\'efinition}[demi-tangente]
Soit $\gammaIE$ un arc param\'etr\'e.
Pour $t \in I$, on note $M(t) = \gamma(t)$.
Soit $t_0 \in I$. On note $A = A(t_0) = M(t_0)$.
On dit que \emph{$\gamma$ admet une demi-tangente en $A(t_0^+)$} si et seulement si la limite
\[
  \vec u = \lim_{t \to t_0^+} \frac{\Vec{AM(t)}}{\Norm{AM(t)}}
  = \lim_{t \to t_0^+} \frac{1}{\Norm{\gamma(t)-\gamma(t_0)}} \BigPa{ \gamma(t) - \gamma(t_0) }
\]
existe.
Dans ce cas, la \emph{demi-tangente \`a $\gamma$ en $A(t_0^+)$} est par d\'efinition la droite passant par $A$ de vecteur directeur $\vec u \neq0$.

On d\'efinit de fa\c con analogue la demi-tangente en $A(t_0^-)$.

\Para{D\'efinition}[tangente]
Avec les m\^emes notations que pr\'ec\'edemment, si deux limites sont \emph{\'egales ou oppos\'ees}, alors les demi-tangentes sont \'egales et l'on dit que $\gamma$ admet une tangente en $A(t_0)$.

% -----------------------------------------------------------------------------
\subsection{Au voisinage d'un point r\'egulier}

\Para{D\'efinitions}
Soit $\gammaIE$ un arc param\'etr\'e.
\begin{itemize}
\item On dit que $t_0 \in I$ est un \emph{point r\'egulier} de $\gamma$ si et seulement si $\gamma'(t_0)\neq0$.
  On dit que $t_0$ est un \emph{point stationnaire} dans le cas contraire.
\item On dit que $\gamma$ est un \emph{arc r\'egulier} si et seulement si tous les points $t \in I$ sont des points r\'eguliers de $\gamma$.
\end{itemize}

\Para{Proposition}
Soit $\gammaIE$ un arc param\'etr\'e et $t_0 \in I$ un point r\'egulier de $\gamma$.
Alors $\gamma$ admet une tangente en $t_0$;
il s'agit de la droite passant par $\gamma(t_0)$ et de vecteur directeur $\gamma'(t_0)$.

% -----------------------------------------------------------------------------
\subsection{Au voisinage d'un point stationnaire}

Dans ce cas, l'allure locale est plus complexe.
La proposition suivante permet d'\'etudier l'allure au voisinage d'un point quelconque.

\Para{Proposition}
Soit $\Fn{\gamma}{I}{\EE}$ un arc param\'etr\'e de classe $\CC k$ et $t\in I$.
On note
\begin{itemize}
\item $p$ le plus petit entier tel que
  \[ \left\{ \begin{array}{l} 1\leq p<k, \\ \gamma^{(p)}(t)\neq0; \end{array} \right. \]
\item $q$ le plus petit entier tel que
  \[ \left\{ \begin{array}{l} p<q\leq k, \\ \text{la famille $\bigPa{ \gamma^{(p)}(t), \gamma^{(q)}(t)}$ est libre}.  \end{array} \right. \]
\item $\vec u = \gamma^{(p)(t)}$ et $\vec v = \gamma^{(q)}(t)$.
\end{itemize}
On suppose que $p$ et $q$ existent.

La famille $(\vec u,\vec v)$ est alors une base du plan $\EE$, et d'apr\`es Taylor-Young, quand $h\to0$, on a
\[ \gamma(t+h) = \gamma(t) + \frac{h^p}{p!} \bigPa{1+o(1)} \, \vec u + \frac{h^q}{q!} \bigPa{1+o(1)} \, \vec v. \]

Cette formule montre que l'allure locale de $\gamma$ au voisinage de $t$ d\'epend essentiellement de la parit\'e de $p$ et~$q$.

% <noindent>
\newenvironment{myfig}[2]{
  \begin{center}
    \begin{tikzpicture}[scale=1.5]
      %\draw[step=0.25,gray!30,very thin] (-1,-1) grid (2.25,2);
      \useasboundingbox (-1,-1) -- (2.25,2);
      \draw[Axe] (-1,0) -- (2,0);
      \draw (60:-1) -- (60:2);
      \draw[Vecs] (60:1) -- (0,0) -- (1,0);

      \draw (1,0) node[gamma p pos] {$\vec u$};
      \draw (60:1) node[gamma q pos] {$\vec v$};
      \draw (0,1.5) node[align=center] {$p$ #1\\$q$ #2};
}{\end{tikzpicture}\end{center}}
% </noindent>

\begin{enumerate}
\item Point \`a allure normale:
  \begin{myfig}{impair}{pair}
    \draw[Curve] (2,0.5) .. controls +(-1,-0.5) and +(0.5,0) .. (0,0) .. controls +(-1,0) and +(0,0) .. (-1,1);
    \draw[Curve name] (2,0.5) node[above] {$\Gamma$};
  \end{myfig}

\item Point d'inflexion:
  \begin{myfig}{impair}{impair}
    \draw[Curve] (2,0.5) .. controls +(-1,-0.5) and +(0.5,0) .. (0,0) .. controls +(-1,0) and +(60:0.1) .. (-1,-0.5);
    \draw[Curve name] (2,0.5) node[above] {$\Gamma$};
  \end{myfig}

  \pagebreak[3]
\item Point de rebroussement de premi\`ere esp\`ece:
  \begingroup
    \tikzset{gamma p pos/.style = {below left=1ex}}
    \begin{myfig}{pair}{impair}
      \draw[Curve] (2,0.5) .. controls +(-1,-0.5) and +(0.5,0) .. (0,0) .. controls +(1,0) and +(-0.5,0.5) .. (2,-0.75);
      \draw[Curve name] (2,0.5) node[above] {$\Gamma$};
    \end{myfig}
  \endgroup

\item Point de rebroussement de seconde esp\`ece:
  \begin{myfig}{pair}{pair}
    \draw[Curve] (2,0.5) .. controls +(-1,-0.5) and +(0.5,0) .. (0,0) .. controls +(1,0) and +(0,0) .. (2,1);
    \draw[Curve name] (2,1) node[above] {$\Gamma$};
  \end{myfig}
\end{enumerate}

% -----------------------------------------------------------------------------
\section{Branches infinies}

\Para{D\'efinition}
Soit $\gammaI{\EE}$ un arc param\'etr\'e.
On dit que $\gamma$ admet une \emph{branche infinie} lorsque $t$ tend vers $t_0$ si et seulement si
\[ \Norm{\gamma(t)} \To{t\to t_0} +\infty. \]

\Para{D\'efinition}
Soit $\gammaI{\EE}$ un arc param\'etr\'e admettant une \emph{branche infinie} lorsque $t$ tend vers $t_0$ si et seulement si
On dit que $\gamma$ admet une \emph{direction asymptotique}
lorsque $t$ tend vers $t_0$ si et seulement si la limite
\[ \vec u = \lim_{t \to t_0} \frac{\gamma(t)}{\Norm{\gamma(t)}} \]
existe.
On dit que $\vec u$ est la \emph{direction asymptotique} de $\gamma$ lorsque $t$ tend vers $t_0$.

\Para{D\'efinition}[asymtote et branche parabolique]
Avec les m\^emes notations, supposons que $\gamma$ admette $\vec u$ comme direction asymptotique lorsque $t$ tend vers $t_0$.
Soit $\vec v$ un vecteur tel que $\B=(\vec u, \vec v)$ soit libre.
Notons $\bigPa{X(t), Y(t)}$ les coordonn\'ees de $\gamma(t)$ dans la base $\B$.

\begin{itemize}
\item Si \[ \left\{ \begin{array}{l} X(t) \To{t\to t_0} \pm\infty\\ Y(t) \To{t\to t_0} b\in \R, \end{array} \right. \]
  on dit que la droite d'\'equation $Y=b$ est \emph{asymptote} \`a $\gamma$ lorsque $t$ tend vers $t_0$.
\item Dans le cas contraire, on dit que $\gamma$ admet une \emph{branche parabolique} de direction $\vec u$ lorsque $t$ tend vers $t_0$.
\end{itemize}

\Para{Remarque}
Ceci est ind\'ependant du choix du vecteur $\vec v$ non colin\'eaire \`a $\vec u$.

% -----------------------------------------------------------------------------
\section{Changement de param\'etrage}

Cette partie n'est pas clairement au programme.

\Para{D\'efinition}
Soit $\Fn{\phi}{I}{J}$ une application o\`u $I$ et $J$ sont
des intervalles de $\R$.
On dit que $\phi$ est un \emph{$\CC k$-diff\'eomorphisme} ($k \geq1$)
lorsque les conditions suivantes sont v\'erifi\'ees:
\begin{enumerate}
\item $\phi$ est bijective;
\item $\phi$ est de classe $\CC k$ sur~$I$;
\item $\phi^{-1}$ est de classe $\CC k$ sur~$J$.
\end{enumerate}

\Para{Th\'eor\`eme}[caract\'erisation des diff\'eomorphismes]
Soit $\Fn{\phi}{I}{J}$ une application o\`u $I$ et $J$ sont
des intervalles de $\R$.
Alors $\phi$ est un $\CC k$-diff\'eomorphisme ($k \geq1$) si et seulement si:
\begin{enumerate}
\item $\phi$ est surjective;
\item $\phi$ est de classe $\CC k$ sur $I$;
\item $\forall t \in I \+ \phi'(t) \neq0$.
\end{enumerate}

\Para{D\'efinition}
Soit $\gammaIE$ un arc param\'etr\'e de classe $\CC k$, $k \geq1$.
On appelle \emph{changement de param\'etrage} (de classe $\CC k$) de $\gamma$ tout $\CC k$-diff\'eomorphisme $\Fn{\phi}{J}{I}$ o\`u $J$ est un intervalle de $I$.

\Para{D\'efinition}
On dit que deux arcs param\'etr\'es $\gammaIE$ et $\deltaJE$ de classe $\CC k$ ($k \geq1$) sont \emph{\'equivalents} si et seulement s'il existe un changement de param\'etrage (de classe $\CC k$) $\phi$ de $\gamma$ tel que $\delta=\gamma\circ \phi$.

\Para{Exemple}
Soit $\EE = \R^2$. On d\'efinit les arcs $\gamma$ et $\delta$ par
\[ \Fonction{\gamma} {\R} {\R^2} {t} {\Pa{\frac{1-t^2}{1+t^2},\frac{2t}{1+t^2}},} \]
\[ \Fonction{\delta} {\intO{-\pi,\pi}} {\R^2} {\theta} {\pa{\cos\theta,\sin\theta}.} \]
Alors les arcs $\gamma$ et $\delta$ sont \'equivalents via le changement de param\'etrage $t = \tan(\theta/2)$.

\Para{Proposition}
Il s'agit d'une relation d'\'equivalence, c.-\`a-d.
\begin{itemize}
\item tout arc param\'etr\'e est \'equivalent \`a lui-m\^eme;
\item si $\gamma$ est \'equivalent \`a $\delta$, alors $\delta$ est \'equivalent \`a $\gamma$;
\item deux arcs param\'etr\'es \'equivalents \`a un m\^eme troisi\`eme sont \'equivalents.
\end{itemize}

\Para{D\'efinition}
Soit $\gammaIE$ un arc param\'etr\'e de classe $\CC k$, $k \geq1$.
On appelle \emph{param\'etrage admissible} de $\gamma$ toute arc param\'etr\'e $\deltaJE$ \'equivalent \`a $\gamma$.

\Para{D\'efinition}
Soit $\gammaIE$ et $\deltaJE$ deux arcs param\'etr\'es \'equivalents et $\Fn{\phi}{J}{I}$ un diff\'eomorphisme tel que $\delta=\gamma\circ \phi$.
Comme $\phi$ est une application injective entre deux intervalles de $\R$, elle est soit strictement croissante, soit strictement d\'ecroissante.
\begin{itemize}
\item Si $\phi$ est croissante, on dit que $\gamma$ et $\delta$ sont \emph{orient\'es dans le m\^eme sens}.
\item Si $\phi$ est d\'ecroissante, on dit que $\gamma$ et $\delta$ sont \emph{orient\'es dans le sens contraire}.
\end{itemize}

% -----------------------------------------------------------------------------
\subsection{Longueur d'une courbe param\'etr\'ee}

\Para{D\'efinition}
Soit $\Fn{\gamma}{\intF{a,b}}{\EE}$ un arc param\'etr\'e r\'egulier.
La \emph{longueur} de $\gamma$ est par d\'efinition
\[ L = \int_a^b \Norm{\gamma'(t)} \D t. \]
Si $\B$ est une base orthonormale de $\EE$ et si
\[ \Coords_\B \gamma(t) = \bigPa{ x(t), y(t) }, \]
alors
\[ L = \int_a^b \sqrt{x'(t)^2 + y'(t)^2} \D t. \]

% -----------------------------------------------------------------------------
\section{Exercices}

\subsection{\'Etudes de courbes param\'etr\'ees}

% -----------------------------------------------------------------------------
\par\pagebreak[1]\par
\paragraph{Exercice 1}%
\hfill{\tiny 5849}%
\begingroup~

\def\AccoDeux#1#2{\begin{cases} #1 \\ #2 \end{cases}}
Tracer les courbes suivantes.
On fera l'\'etude la plus compl\`ete possible des courbes suivantes:
les points r\'eguliers, doubles, les asymptotes, les points d'inflexion;
et pour les plus motiv\'es: la courbure, la longueur de la courbe, l'aire des boucles, etc.
\begin{enumerate}
\item Astro\"ide:
  $\AccoDeux{x = a\cos^3(t)}{y=a\sin^3(t)}$
\item Stropho\"ide droite:
  $\AccoDeux{x = \frac{1-t^2}{1+t^2}}{y = t\frac{1-t^2}{1+t^2}}$
\item Lemniscate de Benoulli:
  $\AccoDeux{x = \frac{t}{1+t^4}}{y = \frac{t^3}{1+t^4}}$
\item Folium De Descartes:
  $\AccoDeux{x = \frac{t}{1+t^3}}{y = \frac{t^2}{1+t^3}}$
\item Delto\"ide:
  $\AccoDeux{x=2\cos t+\cos2t}{y=2\sin t-\sin2t}$
\item C\oe ur de R\'emi:
  $\AccoDeux{x=\sin^3 t}{y=\cos t - \cos^4 t}$
\end{enumerate}

% -----------------------------------------------------------------------------
\endgroup

\subsection{Autres exercices}

% -----------------------------------------------------------------------------
\par\pagebreak[1]\par
\paragraph{Exercice 2}%
\hfill{\tiny 5141}%
\begingroup~

Soit $\mathscr{P}$ la parabole d'\'equation $y^2 = 2px$.
Trouver la longueur minimale d'une corde de $\mathscr{P}$ normale \`a $\mathscr{P}$ en l'une de ses extr\'emit\'es.
\endgroup

% -----------------------------------------------------------------------------
\par\pagebreak[1]\par
\paragraph{Exercice 3}%
\hfill{\tiny 8286}%
\begingroup~

Reconna\^itre la transformation
\[ \left\{ \begin{alignedat}{5}
      x' &{}={}&  3x &{}+{}& 4y &{}+{}& 2z &{}-{}& 4 \\
      y' &{}={}& -2x &{}-{}& 3y &{}-{}& 2z &{}+{}& 4 \\
      z' &{}={}&  4x &{}+{}& 8y &{}+{}& 5z &{}-{}& 8
\end{alignedat} \right. \]
\endgroup

% -----------------------------------------------------------------------------
\par\pagebreak[1]\par
\paragraph{Exercice 4}%
\hfill{\tiny 3541}%
\begingroup~

\def\AccoDeux#1#2{\begin{cases} #1 \\ #2 \end{cases}}
Montrer que les droites
\[ \mathcal{D}  \AccoDeux{x = 2z + 1}{y = z - 1}
  \qquad \text{et} \qquad
\mathcal{D}' \AccoDeux{x = z + 2}{y = 3z - 3} \]
sont coplanaires, et d\'eterminer leur plan.
\endgroup

% -----------------------------------------------------------------------------
\par\pagebreak[1]\par
\paragraph{Exercice 5}%
\hfill{\tiny 5005}%
\begingroup~

Soit $\Fn{f}{\R^2}{\R^2}$ une fonction telle que pour tous points $A$ et $B$ du plan $\R^2$,
la famille $(\Vec{AB}, \Vec{f(A)f(B)})$ soit li\'ee.

Montrer que $f$ est soit une homoth\'etie, soit une translation, soit une application constante.
\endgroup

% -----------------------------------------------------------------------------
\par\pagebreak[1]\par
\paragraph{Exercice 6}%
\hfill{\tiny 7703}%
\begingroup~

Soit $A$, $B$, $C$ et $D$ quatre points de l'espace, et $\lambda$, $\mu$ deux r\'eels.
On d\'efinit les points $M$, $N$, $P$ et $Q$ par les relations:
$\Vec{AM} = \lambda\Vec{AB}$,
$\Vec{DN} = \lambda\Vec{DC}$,
$\Vec{AP} = \mu\Vec{AD}$ et
$\Vec{BQ} = \mu\Vec{BC}$.
Montrer que les droites $(MN)$ et $(PQ)$ sont coplanaires.
\endgroup

% -----------------------------------------------------------------------------
\par\pagebreak[1]\par
\paragraph{Exercice 7 (distance d'un point \`a une droite)}%
\hfill{\tiny 6864}%
\begingroup~

Soit $\mathcal{D}$ la droite du plan d'\'equation $ax+by+c=0$ et $M_0$ le point de coordonn\'ees $(x_0,y_0)$.
Montrer que la distance de $M_0$ \`a $\mathcal{D}$ est donn\'ee par
\[ d(M_0,\mathcal{D}) = \frac{\Abs{ax_0+by_0+c}}{\sqrt{a^2+b^2}}. \]

M\^eme question dans l'espace avec le plan $\mathcal{P}$ d'\'equation $ax+by+cz+d = 0$
et le point $M_0(x_0,y_0,z_0)$.
\endgroup

% -----------------------------------------------------------------------------
\par\pagebreak[1]\par
\paragraph{Exercice 8}%
\hfill{\tiny 9317}%
\begingroup~

\def\AccoDeux#1#2{\begin{cases} #1 \\ #2 \end{cases}}
Soit $\mathcal{D}$ et $\mathcal{D}'$ deux droites de l'espace non parall\`eles.
Montrer qu'il existe une unique droite $\Delta$ qui soit perpendiculaire $\mathcal{D}$ et \`a $\mathcal{D}'$;
$\Delta$ s'appelle la \emph{perpendiculaire commune} \`a $\mathcal{D}$ et $\mathcal{D}'$.

La calculer pour $\mathcal{D}  \AccoDeux{x=a}{y=b}$ et $\mathcal{D}' \AccoDeux{x+cy+z=0}{cx-y+z=0}$.
\endgroup

% -----------------------------------------------------------------------------
\par\pagebreak[1]\par
\paragraph{Exercice 9}%
\hfill{\tiny 8792}%
\begingroup~

\def\AccoDeux#1#2{\begin{cases} #1 \\ #2 \end{cases}}
Former un syst\`eme d'\'equations cart\'esiennes de la sym\'etrique $\mathcal{D}'$
de la droite $\mathcal{D} \AccoDeux{x = 3z+1}{y = -z+2}$
par rapport \`a la droite $\Delta\AccoDeux{x = y}{y = z}$
\endgroup

% -----------------------------------------------------------------------------
\par\pagebreak[1]\par
\paragraph{Exercice 10}%
\hfill{\tiny 3856}%
\begingroup~

D\'eterminer l'image du cercle unit\'e de $\C$ par l'application $f$ d\'efinie par $f(z) = 1/(1-z)^2$.
\endgroup

\end{document}
